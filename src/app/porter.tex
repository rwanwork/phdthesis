\newchapter{Porter Stemming Algorithm}{chap:porter}

The Porter stemming algorithm was originally described 
by \citet{porter80:program} and later reprinted in 1997.  
The algorithm is used
in the word-based and punctuation-aligned \repair variants
to stem 
words to their root form by removing word suffixes.  
\chapref{chap:prepair} provides details of
the different methods of selecting phrases for \repair 
and how the word-based pre-processor, \prepair, 
incorporates the algorithm.
This appendix provides information specific to the
stemming algorithm, with emphasis on the modifiers for
stemming reversal.

The version of the algorithm employed by \prepair 
transforms an English word 
into its root form by applying a sequence of six independent
steps for suffix removal.  Step 0 is the first step and
consists of a single action.
Step 1 is composed of multiple
sub-steps (1a to 1c), unlike Steps 2, 3, and 4.
And step 5 consists of 
two sub-steps (5a and 5b), applied one after the other.  
Each step consists of rules which 
either remove characters from the end of the word, or change
characters at the end of the word, or do a combination of both.
A rule is applied if a certain set of conditions is met.  
Some possible conditions include a match on the suffix,
or a minimum number of consonant and vowel 
combinations before the stem.  

Some changes to the traditional implementation of the 
Porter stemming algorithm were necessary for 
the implementation found in \prepair.  
The differences are as follows:

\begin{enumerate}
\item If a word ends in ``logi'', then the final ``i'' is
removed. (New rule for step 2)
\item If a word ends in ``bli'', then the final ``i'' is
changed to an ``e''.  (Updated rule for step 2.)
\item Words of length 3 characters or less, or 
which are \sgml tags that end with a ``\tg'' character are 
not processed.
\item Seven contractions which include the apostrophe are 
removed in step 0.
\end{enumerate}

The first two changes were suggested by 
\citeauthor{porter80:program} on the web page created for 
the algorithm, located at 
\myurl{http://www.tartarus.org/~martin/PorterStemmer/}.  
The last two changes were added specifically for \prepair.
The third change prevents short words from being
stemmed further and improves overall efficiency.  In an \sgml
document, word tokens which end with a ``\tg'' character
would not be stemmed according to the algorithm.  The 
modification to the algorithm allows \prepair to identify 
tags immediately, without attempting to stem them.  
The last change introduces a step 0 to the stemming
mechanism in \prepair for the removal of certain suffixes
which include an apostrophe.  The seven contractions
identified by step 0 are some of the most common ones in
English \citep[pg.~217]{fm93:book}.  In particular, they
include contractions where a verb occurs with the 
word ``not'', or when a pronoun is followed by an 
auxiliary verb.  In other cases not covered by these rules,
the apostrophe is treated as a non-word during parsing.
For example, ``o'clock'' contains two word tokens,
with the first one being the letter ``o'' by itself.
As an example of the apostrophe handling, 
the word ``king's'' is first treated as a single word token.
Step 0 removes the apostrophe
and the ``s'', since the ``s'' is too short to be useful for 
phrase browsing.  The alternative would result in
``king'' and ``s'' being processed as two separate word tokens,
to the detriment of any subsequent phrase searching.

In this investigation, stemming is required for compression,
and must be reversible.
To achieve reversibility, a modifier is stored along with the
stemmed word to indicate which rules must be applied to invert the
stemming process.  While the stemming process requires 
knowledge about the length of the word token and the 
characters which precede the stem, the inverse is more mechanical
and applies the steps from 5b to 0 without checking the word.
Moreover, in the algorithm, several rules within a step
remove the same suffix, but based on different conditions on the
characters preceding the suffix.  Since the inverse of all of
these rules add the same suffix back, they can be combined
into a single modifier.
The first rule of every step indicates
that the step was not used.  Only one rule is 
applied per step.

\tabref{tab:porter-bits} lists the steps in the order 
in which they are applied 
when decoding.  Since each step is 
independent of other steps, a modifier for a word token is
23 bits in length.  The table
also lists the number of rules per step.  In a few cases,
the number of possible rules that can be represented by 
the bits allocated exceeds the actual number of rules.  For
example, 5 bits for step 4 can account for 32 rules.  
While the size of the modifier could be reduced by merging
steps, the steps were separated due to encoding and decoding
efficiency reasons.  \chapref{chap:review} takes a closer
look at the distribution of stemming modifiers when 
coding the stream of stemming modifiers is examined.

\tab{ccc}
{\multirow{2}*{Step} & Number   & Number  \\
                         & of Rules & of Bits \\}
{
5b     & \D2        & \D1       \\
5a     & \D2        & \D1       \\
\D4    & 20         & \D5       \\
\D3    & \D8        & \D3       \\
\D2    & 15         & \D4       \\
1c     & \D2        & \D1       \\
1b     & \D7       & \D3       \\
1a     & \D3        & \D2       \\
0      & \D8        & \D3       \\
\hline
Total: & 67         & 23      \\
}
{
The steps in \prepair's version of the Porter 
stemming algorithm.  The steps are listed in the order they
are used when decoding.  The second column lists the number
of rules in each step.  The number of bits required 
to express which rule was used in that step is shown in the third 
column.  In total, a 23-bit modifier is required for each word 
token to indicate how to reverse the stemming.}
{Steps in \prepair's version of the Porter stemming algorithm}
{tab:porter-bits}

The remainder of this appendix shows the decoding rules for
each step.  Each rule applies changes to the current word, 
\stem.  
In the following tables, most rules appear in the 
following form:  \stemchange{e}{ional}.  
This means that the letter ``e'' which ends the word is
replaced with ``ional''.  
Other rules may simply append the suffix on to the word,
without any prior changes to the word.

As an example of how stemming is reversed, suppose the initial
stemmed word token is ``commun'' and its corresponding modifier 
is:  { 0, (2, 0, 0), 7, 1, 5, (0, 0) }.  Step 0 is represented
by the first ``0'', and parentheses
have been inserted around the composite steps 1 and 5 
to improve legibility. \tabref{tab:porter-example} demonstrates 
how the modifier is used to recover the original word,
``communications'', starting from step 5b.

\tab{ccl}
{
Step & Value & Word (\stem) \\
}
{
     &      & commun \\
5b   & 0    & commun \\
5a   & 0    & commun \\
4    & 5    & communic \\
3    & 1    & communicate \\
2    & 7    & communication \\
1c   & 0    & communication \\
1b   & 0    & communication \\
1a   & 2    & communications \\
0    & 0    & communications \\
}{Example of transforming the word ``commun'' back to the
original word ``communications'' using its corresponding
modifier, starting from step 5b.  No changes are
applied to the word whenever rule 0 is found.}
{Example of reversing stemming}
{tab:porter-example}

\fig{
\begin{tabularx}{14cm}{XX}
  \begin{tabular}[t]{cp{4cm}}
  \hline
  Value & \multicolumn{1}{c}{Rule} \\
  \hline
  0 & No change to \stem \\
  1 & Double final letter in \stem \\
  \hline
  \end{tabular}

  & 

  \begin{tabular}[t]{cp{4cm}}
  \hline
  Value & \multicolumn{1}{c}{Rule} \\
  \hline
  0 & No change to \stem \\
  1 & \stemadd{e} \\
  \hline
  \end{tabular}                         \\

  \multicolumn{1}{c}{(b) Step 5b requires 1 bit.} & \multicolumn{1}{c}{(a) Step 5a requires 1 bit.} \\
\end{tabularx}
}{Porter stemming algorithm -- step 5.}{Porter stemming algorithm -- step 5}
{fig:porter-5}

\fig{
\begin{tabularx}{14cm}{XX}
  \begin{tabular}[t]{cp{4cm}}
  \hline
  Value & \multicolumn{1}{c}{Rule} \\
  \hline
  0  & No change to \stem \\
  1  & \stemadd{al} \\
  2  & \stemadd{ance} \\
  3  & \stemadd{ence} \\
  4  & \stemadd{er} \\
  5  & \stemadd{ic} \\
  6  & \stemadd{able} \\
  7  & \stemadd{ible} \\
  8  & \stemadd{ant} \\
  9  & \stemadd{ement} \\
  \hline
  \end{tabular}

  &

  \begin{tabular}[t]{cp{4cm}}
  \hline
  Value & \multicolumn{1}{c}{Rule} \\
  \hline
  10 & \stemadd{ment} \\
  11 & \stemadd{ent} \\
  12 & \stemadd{ion} \\
  13 & \stemadd{ou} \\
  14 & \stemadd{ism} \\
  15 & \stemadd{ate} \\
  16 & \stemadd{iti} \\
  17 & \stemadd{ous} \\
  18 & \stemadd{ive} \\
  19 & \stemadd{ize} \\
  \hline
  \end{tabular}               \\
  \multicolumn{2}{c}{Step 4 requires 5 bits.} \\
\end{tabularx}
}{Porter stemming algorithm -- step 4.}{Porter stemming algorithm -- step 4}
{fig:porter-4}


\fig{
\begin{tabularx}{14cm}{XX}
  \begin{tabular}[t]{cp{4cm}}
  \hline
  Value & \multicolumn{1}{c}{Rule} \\
  \hline
  0 & No change to \stem \\
  1 & \stemadd{ate} \\
  2 & \stemadd{ative} \\
  3 & \stemadd{ize} \\
  4 & \stemadd{iti} \\
  5 & \stemadd{al} \\
  6 & \stemadd{ful} \\
  7 & \stemadd{ness} \\
  \hline
  \end{tabular}

  &

  \begin{tabular}[t]{cp{4cm}}
  \hline
  Value & \multicolumn{1}{c}{Rule} \\
  \hline
  \D0 & No change to \stem \\
  \D1 & \stemchange{e}{ional} \\
  \D2 & \stemadd{al} \\
  \D3 & \stemchange{e}{i} \\
  \D4 & \stemadd{r} \\
  \D5 & \stemadd{li} \\
  \D6 & \stemchange{e}{ation} \\
  \D7 & \stemchange{e}{ion} \\
  \D8 & \stemchange{e}{or} \\
  \D9 & \stemadd{ism} \\
  10  & \stemadd{ness} \\
  11  & \stemadd{iti} \\
  12  & \stemchange{e}{iti} \\
  13  & \stemchange{le}{iliti} \\
  14  & \stemadd{i}                    \\
  \hline
  \end{tabular}               \\
  \multicolumn{1}{c}{(a) Step 3 requires 3 bits.} & \multicolumn{1}{c}{(b) Step 2 requires 4 bits.} \\
\end{tabularx}
}{Porter stemming algorithm -- steps 3 and 2.}{Porter stemming algorithm -- steps 3 and 2}
{fig:porter-32}

\fig{
\begin{tabularx}{14cm}{XX}
  \begin{tabular}[t]{cp{4cm}}
  \hline
  Value & \multicolumn{1}{c}{Rule} \\
  \hline
  0  & No change to \stem \\
  1  & \stemchange{i}{y} \\
  \hline
  \end{tabular}

  &

  \begin{tabular}[t]{cp{4cm}}
  \hline
  Value & \multicolumn{1}{c}{Rule} \\
  \hline
  0  & No change to \stem \\
  1  & \stemadd{ed} \\
  2  & \stemadd{d} \\
  3  & Double the final letter in \stem\ and then apply \stemadd{ed} \\
  4  & \stemadd{ing} \\
  5  & \stemchange{e}{ing} \\
  6  & Double the final letter in \stem\ and then apply \stemadd{ing} \\
  \hline
  \end{tabular}                       \\

  \multicolumn{1}{c}{(c) Step 1c requires 1 bit.} & \multicolumn{1}{c}{(b) Step 1b requires 3 bits.} \\

  \begin{tabular}[t]{cp{4cm}}
  \hline
  Value & \multicolumn{1}{c}{Rule} \\
  \hline
  0  & No change to \stem \\
  1  & \stemadd{es} \\
  2  & \stemadd{s} \\
  \hline
  \end{tabular}

  & \\
  \multicolumn{1}{c}{(a) Step 1a requires 2 bits.} & \\
\end{tabularx}
}{Porter stemming algorithm -- step 1.}{Porter stemming algorithm -- step 1}
{fig:porter-1}


\fig{
  \begin{tabularx}{7cm}{X}
  \begin{tabular}[t]{cp{4cm}}
  \hline
  Value & \multicolumn{1}{c}{Rule} \\
  \hline
  0  &  No change to \stem \\
  1  &  \stemadd{'d} \\
  2  &  \stemadd{'m} \\
  3  &  \stemadd{'s} \\
  4  &  \stemadd{'t} \\
  5  &  \stemadd{'ll} \\
  6  &  \stemadd{'re} \\
  7  &  \stemadd{'ve} \\
  \hline
  \end{tabular}                       \\
  \multicolumn{1}{c}{Step 0 requires 3 bits.} \\
  \end{tabularx}
}{Porter stemming algorithm -- step 0.}{Porter stemming algorithm -- step 0}
{fig:porter-0}

