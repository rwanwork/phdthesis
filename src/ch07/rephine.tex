\newchapter{Phrase Browsing}{chap:rephine}

The main focus of the previous chapters has been 
the problem of
transforming a message into a more economical form which
is also suitable for phrase browsing.  Decisions had to
be made with respect to the level of word-based pre-processing
necessary and the efficiency of coders.  All of these issues
were considered in the context of the sender of the compressed message.  In
this chapter, that compressed message is viewed from the point
of view of its receiver, in the form of a phrase browser.

The phrase browser is a tool which hides the complexity
and size of the compressed message from the user.  The browser
described in this chapter not only allows a user to gain an
understanding of a document through its phrases, but it also
provides a searching facility.  As alluded to in
\chapref{chap:review}, after phrase browsing ends and a phrase
of interest has been found, the contexts in 
which the phrase appears may be needed.  The phrase browser
searches for these locations and also allows the user
to refine the search result by incorporating case, stemming,
and non-word information.  Continuing
with the naming scheme adopted by this thesis, the phrase 
browser is called \rephine, in that both the phrase browsing
and the subsequent searching are opportunities for the user
to continually refine the initial query.

The remainder of this chapter is structured as follows.
The steps embodied by the phrase browser are further elaborated
in \secref{sec:rephine-steps}.  In order to efficiently browse 
the phrase hierarchy, a graph data structure is
required.  This data structure is described in 
\secref{sec:rephine-datastruct}.  A demonstration of
\rephine is given in \secref{sec:rephine-example}.
Then, \rephine is compared and contrasted with related 
work and traditional IR techniques 
in \secref{sec:rephine-evaluate}, including an assessment of
the quality of
phrases made available to the user.

\newsection{Browsing and Searching Steps}{sec:rephine-steps}

The \rephine system encompasses both phrase browsing
and context searching.  The precise steps that are available
to the user in order to achieve these aims depend heavily
on the seven streams produced during the compression process.
Recall that the seven streams were the phrase hierarchy, the
word lexicon, the non-word lexicon, the reduced word sequence, 
the case folding modifiers, the stemming 
modifiers, and the non-word modifiers.

Implementation 
details about \rephine are covered in the next section.
In this section, only a general overview of \rephine is provided,
starting with \figref{fig:rephine-alg}.  Each rectangular
box represents an action performed within \rephine.  Boxes
with dashed borders are actions performed by
\rephine, whereas solid boxes represent
actions performed by the system in direct response to the
user's request.  A shaded cylinder represents one of
the seven streams available to \rephine, with dashed lines
symbolising file access.
The sequence of steps commences at
the top left-hand corner of the figure.  An explicit end
to the sequence of steps is not shown because the user
can stop phrase browsing at any time.

\fig{\newchapter{Phrase Browsing}{chap:rephine}

The main focus of the previous chapters has been 
the problem of
transforming a message into a more economical form which
is also suitable for phrase browsing.  Decisions had to
be made with respect to the level of word-based pre-processing
necessary and the efficiency of coders.  All of these issues
were considered in the context of the sender of the compressed message.  In
this chapter, that compressed message is viewed from the point
of view of its receiver, in the form of a phrase browser.

The phrase browser is a tool which hides the complexity
and size of the compressed message from the user.  The browser
described in this chapter not only allows a user to gain an
understanding of a document through its phrases, but it also
provides a searching facility.  As alluded to in
\chapref{chap:review}, after phrase browsing ends and a phrase
of interest has been found, the contexts in 
which the phrase appears may be needed.  The phrase browser
searches for these locations and also allows the user
to refine the search result by incorporating case, stemming,
and non-word information.  Continuing
with the naming scheme adopted by this thesis, the phrase 
browser is called \rephine, in that both the phrase browsing
and the subsequent searching are opportunities for the user
to continually refine the initial query.

The remainder of this chapter is structured as follows.
The steps embodied by the phrase browser are further elaborated
in \secref{sec:rephine-steps}.  In order to efficiently browse 
the phrase hierarchy, a graph data structure is
required.  This data structure is described in 
\secref{sec:rephine-datastruct}.  A demonstration of
\rephine is given in \secref{sec:rephine-example}.
Then, \rephine is compared and contrasted with related 
work and traditional IR techniques 
in \secref{sec:rephine-evaluate}, including an assessment of
the quality of
phrases made available to the user.

\newsection{Browsing and Searching Steps}{sec:rephine-steps}

The \rephine system encompasses both phrase browsing
and context searching.  The precise steps that are available
to the user in order to achieve these aims depend heavily
on the seven streams produced during the compression process.
Recall that the seven streams were the phrase hierarchy, the
word lexicon, the non-word lexicon, the reduced word sequence, 
the case folding modifiers, the stemming 
modifiers, and the non-word modifiers.

Implementation 
details about \rephine are covered in the next section.
In this section, only a general overview of \rephine is provided,
starting with \figref{fig:rephine-alg}.  Each rectangular
box represents an action performed within \rephine.  Boxes
with dashed borders are actions performed by
\rephine, whereas solid boxes represent
actions performed by the system in direct response to the
user's request.  A shaded cylinder represents one of
the seven streams available to \rephine, with dashed lines
symbolising file access.
The sequence of steps commences at
the top left-hand corner of the figure.  An explicit end
to the sequence of steps is not shown because the user
can stop phrase browsing at any time.

\fig{\newchapter{Phrase Browsing}{chap:rephine}

The main focus of the previous chapters has been 
the problem of
transforming a message into a more economical form which
is also suitable for phrase browsing.  Decisions had to
be made with respect to the level of word-based pre-processing
necessary and the efficiency of coders.  All of these issues
were considered in the context of the sender of the compressed message.  In
this chapter, that compressed message is viewed from the point
of view of its receiver, in the form of a phrase browser.

The phrase browser is a tool which hides the complexity
and size of the compressed message from the user.  The browser
described in this chapter not only allows a user to gain an
understanding of a document through its phrases, but it also
provides a searching facility.  As alluded to in
\chapref{chap:review}, after phrase browsing ends and a phrase
of interest has been found, the contexts in 
which the phrase appears may be needed.  The phrase browser
searches for these locations and also allows the user
to refine the search result by incorporating case, stemming,
and non-word information.  Continuing
with the naming scheme adopted by this thesis, the phrase 
browser is called \rephine, in that both the phrase browsing
and the subsequent searching are opportunities for the user
to continually refine the initial query.

The remainder of this chapter is structured as follows.
The steps embodied by the phrase browser are further elaborated
in \secref{sec:rephine-steps}.  In order to efficiently browse 
the phrase hierarchy, a graph data structure is
required.  This data structure is described in 
\secref{sec:rephine-datastruct}.  A demonstration of
\rephine is given in \secref{sec:rephine-example}.
Then, \rephine is compared and contrasted with related 
work and traditional IR techniques 
in \secref{sec:rephine-evaluate}, including an assessment of
the quality of
phrases made available to the user.

\newsection{Browsing and Searching Steps}{sec:rephine-steps}

The \rephine system encompasses both phrase browsing
and context searching.  The precise steps that are available
to the user in order to achieve these aims depend heavily
on the seven streams produced during the compression process.
Recall that the seven streams were the phrase hierarchy, the
word lexicon, the non-word lexicon, the reduced word sequence, 
the case folding modifiers, the stemming 
modifiers, and the non-word modifiers.

Implementation 
details about \rephine are covered in the next section.
In this section, only a general overview of \rephine is provided,
starting with \figref{fig:rephine-alg}.  Each rectangular
box represents an action performed within \rephine.  Boxes
with dashed borders are actions performed by
\rephine, whereas solid boxes represent
actions performed by the system in direct response to the
user's request.  A shaded cylinder represents one of
the seven streams available to \rephine, with dashed lines
symbolising file access.
The sequence of steps commences at
the top left-hand corner of the figure.  An explicit end
to the sequence of steps is not shown because the user
can stop phrase browsing at any time.

\fig{\newchapter{Phrase Browsing}{chap:rephine}

The main focus of the previous chapters has been 
the problem of
transforming a message into a more economical form which
is also suitable for phrase browsing.  Decisions had to
be made with respect to the level of word-based pre-processing
necessary and the efficiency of coders.  All of these issues
were considered in the context of the sender of the compressed message.  In
this chapter, that compressed message is viewed from the point
of view of its receiver, in the form of a phrase browser.

The phrase browser is a tool which hides the complexity
and size of the compressed message from the user.  The browser
described in this chapter not only allows a user to gain an
understanding of a document through its phrases, but it also
provides a searching facility.  As alluded to in
\chapref{chap:review}, after phrase browsing ends and a phrase
of interest has been found, the contexts in 
which the phrase appears may be needed.  The phrase browser
searches for these locations and also allows the user
to refine the search result by incorporating case, stemming,
and non-word information.  Continuing
with the naming scheme adopted by this thesis, the phrase 
browser is called \rephine, in that both the phrase browsing
and the subsequent searching are opportunities for the user
to continually refine the initial query.

The remainder of this chapter is structured as follows.
The steps embodied by the phrase browser are further elaborated
in \secref{sec:rephine-steps}.  In order to efficiently browse 
the phrase hierarchy, a graph data structure is
required.  This data structure is described in 
\secref{sec:rephine-datastruct}.  A demonstration of
\rephine is given in \secref{sec:rephine-example}.
Then, \rephine is compared and contrasted with related 
work and traditional IR techniques 
in \secref{sec:rephine-evaluate}, including an assessment of
the quality of
phrases made available to the user.

\newsection{Browsing and Searching Steps}{sec:rephine-steps}

The \rephine system encompasses both phrase browsing
and context searching.  The precise steps that are available
to the user in order to achieve these aims depend heavily
on the seven streams produced during the compression process.
Recall that the seven streams were the phrase hierarchy, the
word lexicon, the non-word lexicon, the reduced word sequence, 
the case folding modifiers, the stemming 
modifiers, and the non-word modifiers.

Implementation 
details about \rephine are covered in the next section.
In this section, only a general overview of \rephine is provided,
starting with \figref{fig:rephine-alg}.  Each rectangular
box represents an action performed within \rephine.  Boxes
with dashed borders are actions performed by
\rephine, whereas solid boxes represent
actions performed by the system in direct response to the
user's request.  A shaded cylinder represents one of
the seven streams available to \rephine, with dashed lines
symbolising file access.
The sequence of steps commences at
the top left-hand corner of the figure.  An explicit end
to the sequence of steps is not shown because the user
can stop phrase browsing at any time.

\fig{\input{./rephine.pstex_t}}
{The sequence of steps performed by \rephine during phrase
browsing.  Dashed boxes indicate actions 
performed by the system, while solid boxes require
input from the user.  Shaded cylinders represent one of
the seven streams accessible by \rephine, with dashed lines
from them representing file access.  The user can exit
the system at any time.}{Sequence of steps
performed by \rephine during phrase browsing}{fig:rephine-alg}

In the first step of the figure, \rephine decodes the phrase
hierarchy, the word lexicon, and the non-word lexicon, and
maintains all three within memory.  Next, a directed
graph (DG) is built using the symbols of the phrase
hierarchy, to permit efficient searching.  Details
about the structure of the graph is given in the next section.
The third step is interactive phrase browsing.  
In browsing the phrase
hierarchy, the user interactively traverses the DG
until a symbol of interest, $\alpha$ say, has been isolated.

In the fourth step, the symbol of interest is used to identify all
other symbols in the DG which contain $\alpha$.
When that search completes, a set of symbols $\Criteria$ has been 
established.  Next, $\Criteria$ is input to the fifth
step, in which the \reviewb decoder is used to locate every
context in which any symbol in $\Criteria$ occurs.  The \reviewb
decoder returns a result set, $\Result$.  Suppose a symbol
$\beta$ contains the symbol $\alpha$ as a component.  In order
to avoid reporting the same location multiple times, when an
occurrence of $\beta$ has been found in the reduced word sequence, 
only $\beta$ is reported.
The size of $\Result$ is denoted as $|\Result|$, and members
are enumerated from $i=1$ to $i=|\Result|$.

The first result's context, $\Result_1$, is
displayed to the user.  At this point, the user may choose
any combination of five actions.  These actions are:  return to
phrase browsing; toggle case folding modifiers, stemming modifiers, 
or non-word insertion for $\Result_1$; or examine
the next or previous result.  It is assumed that the result
set is a circular list, so that the previous result to $\Result_1$
is $\Result_{|\Result|}$.  The three actions related to the modifiers
initiate seek operations to the corresponding positions in 
their respective compressed
streams.  After the modifiers have been retrieved, $\Result_i$
is re-displayed to the user.  The application of 
modifiers can be toggled in any order.  The first time
modifiers are required, access to the streams are performed.
Assuming requested modifiers are buffered in memory, subsequent 
requests do not need to access the file again.
Finally, if the contexts that were obtained were inadequate in
satisfying the user's need, any part of a context can be used as
input into the phrase browsing part of \rephine.

Later in this chapter, the entire phrase browsing process is detailed 
through a brief example which demonstrates the choices available
to the user.  But first, in the next section, the directed graph
required for phrase browsing is explained.

\newsection{Phrase Hierarchy as a Directed Graph}{sec:rephine-datastruct}

Once the phrase hierarchy has been decoded, a directed
graph (DG) is 
constructed.  Every symbol in the phrase hierarchy is assigned
to a separate node in the DG.  Each node contains six
directed edges connecting it to other nodes, as shown in 
\figref{fig:rephine-graph}.
In this figure, the symbols are based on character-based
\repair instead of punctuation-aligned \repair in order to simplify 
the discussion.  The current node is represented by the
symbol ``w{\textbar}o'', with the vertical bar (\textbar)
separating the symbol into its two components.
Two of the six pointers
are {\emph {child pointers}}, and refer to a symbol's structure
These pointers were described in 
\chapref{chap:remerge} as part of phase 1 of \remerge,
but were called components at that time.
The remaining four pointers are collectively called
{\emph {navigational pointers}}.
Navigational pointers are added for the purpose of
browsing and have not been required in the previous discussion.

\fig{
\includegraphics*{./graph.eps}
}{The six pointers used by \rephine to link nodes representing
symbols in the phrase hierarchy into a directed graph.  The symbols in
the example are based on character-based \repair, for illustrative
purposes only.  The node of interest is in the centre, with the
symbol ``w{\textbar}o''.  A vertical bar (\textbar) separates
the expanded symbols into left and right components.
Names of each pair of pointers are provided to the right of the
figure.
}{The six pointers used by a \rephine graph}
{fig:rephine-graph}

The purpose of each pointer is illustrated in the figure.
Pointers appear in pairs and are designated as being 
``left'' or ``right''.  The four navigational
pointers are in two pairs -- the {\emph {sibling pointers}}, 
and the {\emph {parent pointers}}.  

In the example figure, the current symbol is in
the centre, with its six pointers listed in full.  All of the other
nodes in the graph
have six pointers each as well, but the outgoing edges are 
omitted in the figure for simplicity.  The centre symbol 
is ``w{\textbar}o'', with left and right
components (or children) of ``w'' and ``o'', respectively.
The left sibling symbol of ``w{\textbar}o'' is another symbol which 
also has ``w'' as a left child.  A chain is formed from the 
centre node, with all symbols with ``w'' as a 
left child being linked together with their
left sibling pointers.  Within a chain, the nodes are unsorted,
so any node can serve as the front of the list.
Each chain ends with a final
{\emph {empty node}}.  The child pointers of all primitives
also lead to empty nodes.
Similarly, the right sibling pointer from 
``w{\textbar}o'', leads to another symbol with ``o'' as a right child.  

One important operation is to
obtain the entire chain when the current symbol is any symbol
along that chain.
The child and parent pointers help solve this problem.
Parent pointers directly lead to symbols which are a single 
{\emph {extension}} away from the current symbol.  An extension
occurs when a symbol is juxtaposed
to the current symbol on either side, arriving at another
symbol in the phrase hierarchy.  For example, in 
\figref{fig:rephine-graph}, the right parent pointer from
``w{\textbar}o'' allows a right extension to the symbol 
``wo{\textbar}od''.
\figref{fig:rephine-expandgraph} shows \figref{fig:rephine-graph}
at a larger scale to demonstrate how an entire chain of symbols
can be obtained from anywhere in the list.  The ``w{\textbar}o''
is at the head of the list of right siblings, 
and the current symbol is $\alpha$.
Both of these symbols have the same right component.
Every symbol along that chain
has a right child link to the primitive ``o''.  The front of
the chain can be reached by taking the right child pointer from
$\alpha$, followed by the left parent link of ``o'',
as shown by the dashed lines in the figure.
This is possible provided that
graph construction ensures that every node's parent pointer
leads to the beginning of a chain of parents.  If a symbol
does not have a parent, the parent is represented as
an empty node.

\fig{
\input{./expandgraph.pstex_t}
}{The directed graph structure created by \rephine to link symbols
together, at a larger scale than \figref{fig:rephine-graph}.  The
node of interest is ``w{\textbar}o'' in the centre.  The dashed
arrows shows how an entire chain of symbols can be obtained from
an arbitrary symbol $\alpha$ within the chain.  Also shown are two
siblings of the symbol ``wo{\textbar}od''.
}{The directed graph structure of \rephine, at a larger scale}
{fig:rephine-expandgraph}

The graph structure created by \rephine
resembles the data structure employed by \mka \citep{wolff75:bjp}
for the recursive pairing of symbols for discovering sentence
structure.  Each
symbol in the dictionary created by \mka has major and minor links.
A major forwards link is the same as a left child pointer in
\rephine's graph, while a major backwards link is identical
to a right child pointer.  Minor links operate in a similar
fashion to the sibling pointers of \rephine.

Equipped with this graph, five operations are available to
the user.  Assuming the user's currently selected symbol
is ``w{\textbar}o'' of \figref{fig:rephine-graph} and 
\figref{fig:rephine-expandgraph}, the
five operations are as follows.  First, the user can decompose
the symbol into its two components through the child pointers.
Second, if the right parent pointer from ``w{\textbar}o'' 
is used, followed by all left sibling pointers until the empty
node is reached, then all right extensions of ``w{\textbar}o'' can
be retrieved.  In \figref{fig:rephine-expandgraph}, this operation
would obtain the set 
of two symbols, \{``wo{\textbar}od'', ``wo{\textbar}uld''\}.
Likewise, a third operation is to extend ``w{\textbar}o''
to the left.  The union of these last two sets results in a
fourth function, extending ``w{\textbar}o'' in both directions.  

The fifth and final operation is reserved for decoding, instead
of phrase browsing.  Decoding requires identification of
all symbols which contain the current symbol $\alpha$,
either directly or indirectly.  The
located symbols are denoted in 
\figref{fig:rephine-alg} as $\Criteria$.  In order to find all
symbols in the phrase hierarchy graph which contain $\alpha$, 
a breadth-first search
with the navigational pointers is sufficient.  Let $\beta$
represent a symbol which is an ancestor of $\alpha$.  If $\beta$
has only one occurrence of $\alpha$, 
then all of its parents does as well.  But $\alpha$
must appear in either its left or its right component, which
can be determined based on how $\beta$ was reached.  So, only
one of the sibling pointers need to be followed.  If $\beta$
has multiple occurrences of $\alpha$, with at least one in each
component, then during the breadth-first search, $\beta$ is reached
twice.  Care must be taken to ensure that when $\beta$ is reached
the second time, only the other sibling pointer is traversed, and not the
parent pointers.  If this strategy is followed, then 
the nature of the navigational pointers
ensures that all symbols which include ``w{\textbar}o'' are included.

Construction of the DG requires $O(|\Sigma| + |\rho|)$ time,
for a phrase hierarchy of $|\Sigma|$ primitives and $|\rho|$
phrases.  Since the chiastic slide and interpolative coding have been
used to code the phrase hierarchy, the phrase hierarchy is already
in increasing generation order, and within each generation,
in increasing chiastic slide value order.  As no symbol is in
the same generation as either of its two components, constructing
the graph from the first primitive symbol in sequential order
results in graph construction resembling a bottom-up 
approach.
If the phrase hierarchy is stored in memory as an array, 
and if each of the six graph
pointers are represented as array indexes, then locating a
symbol's components requires $O(1)$ time.  
For each phrase, an insertion
is performed into the appropriate parent pointers of 
each of its two components, so that each node is added
at the front of its two chains of siblings in $O(1)$ time.
That is, a single linear pass through the phrase hierarchy
is sufficient to construct the graph.

The memory space required for the phrase hierarchy is linear in
the number of symbols in the phrase hierarchy.
The graph structure adds an additional 6 words of memory 
per symbol.  Also, the phrase hierarchy may
have every symbol expanded into primitives, or symbol
expansion may be deferred until the symbol is required.  
The choice made depends which is more important:
memory space or response time.  Experiments demonstrating
the advantages and disadvantages of phrase
hierarchy expansion were reported in \tabref{tab:repair-despair}
on \pgref{tab:repair-despair}.  In
the implementation of \rephine described in this chapter,
a buffer of recently decoded symbols is maintained and consulted
before decoding a symbol.  This approach was also employed in
\repair's decoder, \despair.

\newsection{Phrase Browsing Example}{sec:rephine-example}

The \rephine system performs two main tasks:  phrase browsing
and decoding symbol contexts for presentation.  Examples of both tasks
are presented in Figures \ref{fig:rephine-scr-browse} and 
\ref{fig:rephine-scr-decode}.  The 
aspects of \rephine related to the graphical user
interface were 
implemented with version 2.0.2 of the \gtk 
library\footnote{Available from \myurl{http://www.gtk.org/}.}.

The data for the example
is \news, processed with punctuation-aligned
\repair exactly as was described in the experiments of
\chapref{chap:remerge}.  The sequence of word tokens in
those experiments were partitioned into blocks of 20,971,520
symbols each, and then merged using method E of \remerge 
(\secref{sec:remerge-expt}).
Modifier and sequence coding were done with indexed \shuff
and \reviewb respectively.

In the previous section, the choices available to the user
for phrase browsing from the current symbol were discussed.
The user also needs to be given a starting point to browse
from.  In \rephine, the user is first presented with the
document's entire lexicon, as shown in
\subfigref{fig:rephine-scr-browse}{a}.  The extensive list
can be daunting, and one solution
is to filter words, discussed later
in the chapter.  In \subfigref{fig:rephine-scr-browse}{a}, the user
has selected the word ``prime''.  
The first symbol is extended to
the right in \subfigref{fig:rephine-scr-browse}{b}, and the
phrase ``prime minist'' has been selected.  While phrases in this 
window are ordered according to their locations in the DG, 
\citet{jp99:dl} and \citet{pwcb00:acmdl} describe systems %% Second cite due to witten00:cpm
where phrases can be ordered according to the number of documents
in which they appear, or their frequency in the collection.
While details about the implementation of these two features are
not available, it is expected that either the additional information
is stored by the system, or calculated at run-time.  Also note
the variations on the phrase ``prime minist'' due to the different
spellings of the second word.  While stemming and case folding
have reduced the size of the phrase hierarchy, it is still 
possible that two seemingly equivalent words stem differently.
The chosen phrase is extended to the left and the phrase 
``canadian prime minist'' has been highlighted in 
\subfigref{fig:rephine-scr-browse}{c}.  As this window shows,
unfortunately, the words ``Canadian'' and ``Canada's'' stem
to different words.  The identified phrase of 
\subfigref{fig:rephine-scr-browse}{c} is extended in both
directions, resulting in \subfigref{fig:rephine-scr-browse}{d}
and its 7 phrases.  Of these 7 phrases, 3 are right extensions
and 4 are left extensions to the phrase ``canadian prime minist''.
The selected phrase is broken down into its two components through
its child pointers, as shown in 
\subfigref{fig:rephine-scr-browse}{e}.  This phrase browsing
example commenced with the word ``prime'' and, in four steps,
arrived at the name of a Canadian prime minister.  In \rephine,
the refinement of a query term is done interactively with the
help of the phrase hierarchy's graph.

%%  A double-page figure
%%  Snapshot created using the command:
%%  import -frame -window "Main browsing session" ~/thesis/t/ch07/scrshots/snapshot1.eps
%%  Then, export to PS with xv (greyscale).
\fig{
    \setlength{\tabcolsep}{3pt}
    \begin{tabular}{ccc}
      \multicolumn{2}{c}{\includegraphics*[scale=0.6]{./scrshots/browse-0.ps}} &
      \includegraphics*[scale=0.6]{./scrshots/browse-1.ps} \\ 
      \multicolumn{2}{c}{(a) Initial window} & (b) Extend right \\ \\
      \includegraphics*[scale=0.6]{./scrshots/browse-2.ps} &
      \includegraphics*[scale=0.6]{./scrshots/browse-3.ps} &
      \includegraphics*[scale=0.6]{./scrshots/browse-4.ps} \\
      (c) Extend left & (d) Extend both ways & (e) Decompose \\
    \end{tabular}
%  }
%  \end{leftfullpage}
}{A sample phrase browsing session of the \news document.}
{Sample phrase browsing session of \rephine (\news)}{fig:rephine-scr-browse}
%\end{figure}

A sample decoding session is illustrated in 
\figref{fig:rephine-scr-decode}.  The selected phrase of
\subfigref{fig:rephine-scr-browse}{e} is the starting point
for the decoding.  The first of 17 occurrences
of the phrase ``pierr elliott trudeau''
is given in \subfigref{fig:rephine-scr-decode}{a}.  The results
are presented to the user in the order in which they occur in
the original document.  In the first window, the phrase exists
in a longer phrase, which has been marked.  A fixed context
of 10 symbols in either direction of the highlighted symbol
has been selected for \rephine.  In the absence of detailed information
about the
non-words, a space character is inserted between each word
token, with linefeeds added as appropriate.  
Even though the words in this window 
have been stemmed and case folded, reasonable guesses can
be made with most of the words.  In 
\subfigref{fig:rephine-scr-decode}{b}, the user has skipped
ahead to the 15th result.  As this section of text may be
of interest, the case folding is first reversed in 
\subfigref{fig:rephine-scr-decode}{c}, and then the stemming
modifiers are applied in 
\subfigref{fig:rephine-scr-decode}{d}.  The corresponding modifiers are
located by the system and only the small number of required
modifiers are transmitted to the \rephine window.  Modifiers can 
be toggled in any order, and allow the user to progressively
change the displayed context.  Finally, in 
\subfigref{fig:rephine-scr-decode}{e},
the non-words are inserted into the text and the window is 
enlarged for improved display.  The text exists in the 
original \news document as shown.  

\fig{
    \setlength{\tabcolsep}{3pt}
    \begin{tabular}{ccc}
      \includegraphics*[scale=0.6]{./scrshots/decode-0.ps} &
      \includegraphics*[scale=0.6]{./scrshots/decode-1.ps} &
      \includegraphics*[scale=0.6]{./scrshots/decode-2.ps} \\ 
      (a) Result 1 of 17 & (b) Result 15 of 17 & (c) Reverse case folding\\ \\
      \includegraphics*[scale=0.6]{./scrshots/decode-3.ps} &
      \multicolumn{2}{c}{\includegraphics*[scale=0.6]{./scrshots/decode-4.ps}} \\
      (d) Reverse stemming & \multicolumn{2}{c}{(e) Add non-words} \\
    \end{tabular}
}{A sample decoding session of the \news document, 
continuing from \subfigref{fig:rephine-scr-browse}{e}.}
{Sample decoding session of \rephine (\news)}{fig:rephine-scr-decode}

At this stage, the user's information need may be satisfied.  
Alternatively, the 
user may decide to select the phrase ``Progressive 
Conservative'' as the new starting point for phrase browsing.
Then, the starting point for the browsing session becomes
the phrase ``progress conserv''.

\rephine applies the modifiers only when requested by the user.
This approach takes advantage of the fact that the modifier
streams are encoded separately.  At least two other ways of
handling modifiers are possible, which have not been implemented
for \rephine.  If the cost of decoding and transmitting the 
modifiers is acceptable, the user can be given the option to 
reverse all modifiers by clicking one button.  This feature
can be added by making a small change to the \rephine 
interface.  Another option is to display initial word
tokens with default stems and case information applied.  
That is, the most frequent stemming and
case folding modifier is recorded for each word in the lexicon
during pre-processing with
\prepair.  The additional disk space required is $2|\Sigma|$ 
integers, uncompressed,
for a compressed document collection of $|\Sigma|$ primitives.
When phrases are first displayed (for example,
in \subfigref{fig:rephine-scr-decode}{a}), these default modifiers
can be applied immediately without any intervention from the user.
While the phrases presented may occasionally be
linguistically incorrect, they should be valid most of
the time.
These default modifiers can be transmitted as needed, or once
at the beginning of the browsing session.  A similar idea was
adopted by \citet{pwcb00:acmdl}.  
The aim of both of these alternatives is to make the interface
more user-friendly by minimising the amount of exposure users
have to case folded, stemmed words.

\newsection{Evaluating \rephine}{sec:rephine-evaluate}

Information retrieval systems are generally evaluated
in terms of efficiency and effectiveness.  Efficiency is
measured as the amount of time required to return a response
to the user.  Effectiveness is an assessment of the extent to which
the set of results satisfies the user's information need.

Suppose an IR system indexes the words in a 
collection of $n$ documents, and that for some set of queries,
the number of documents that are relevant is known.
The effectiveness of the
IR system for the query is determined by the number of
documents the system returns which are relevant.
Two standard methods of quantifying
effectiveness are {\emph {precision}} and {\emph {recall}}.
These and other methods of judging retrieval effectiveness 
are described by various sources, including \citet{korfhage97:book},
\citet{br99:book}, and \citet{wmb99:book}.

Precision and recall are expressed on a scale from 0 to 1.
Precision is the fraction of returned results that are relevant, 
while recall is the proportion of all relevant
documents that are returned.  If a user poses a query for which
the set of relevant documents in the collection is $r$,
and the set of documents returned by the system is $a$, then
precision and recall are defined as follows:
\neweqn{
\textrm{Precision} = \frac{|r \cap a|}{|a|} \\
\textrm{Recall} = \frac{|r \cap a|}{|r|}
}{eqn:rephine-prec-rec}

Ideally, the effectiveness of \rephine should
be evaluated in a similar manner.  However, significant 
differences between \rephine and traditional IR systems prevent 
any straightforward comparison.

Instead, \rephine is evaluated in three steps.  In the
first step, an overview of past work related to phrase browsing
is summarised.  
Second, \rephine is analysed in detail in the context of related
work and traditional query methods.  Some possible improvements
to \rephine are also proposed.
Finally, the phrase hierarchy used by 
\rephine is compared to noun phrases formed
through natural language processing techniques.

\newsubsection{Related Work}{subsec:rephine-related}

Previous work in the area of phrase browsing can be classified based
on the method employed for drawing phrases from a document.
Phrases can be obtained through
the semantic meaning of the words using natural language processing
techniques, or statistically, by applying a mechanism similar to
\repair.

Most work with natural language processing (NLP) techniques
are related to tagging words according to their parts-of-speech (that is,
noun, verb, etc.)\ and forming phrases according to the rules of
grammar.  A significant amount of work has concentrated on 
{\emph {noun phrases}}.
A noun phrase is a linguistic unit.  In it,
a noun is both required and designated as the most semantically
important word and, as a result, is assigned as
the phrase's head.
\figref{fig:rephine-np} shows an example of a noun
phrase, divided into its various parts.

\fig{
\includegraphics*{./nounphrase.eps}
}{Definition of a noun phrase, adapted from 
\citet[pg.~209]{greenbaum96:book}.}{Definition of a noun phrase}
{fig:rephine-np}

While some retrieval systems treat a query as being
made up of individual words, occasionally, a query should be
seen as a set of one or more noun phrases.  If a user posed
the query ``University of Melbourne'', ideally, documents 
about other universities or the city of Melbourne are not
of interest.  In fact, the seemingly unimportant word ``of''
is crucial for this phrase, since ``University in Melbourne'' has
a completely different meaning.  \citet{kirsch98:sigirforum}
provided some insight into the Internet search engine Infoseek 
and noted that most users submitted queries which were noun phrases
and that queries were, on average, 2.2 words in
length.  

Several researchers have made use of noun phrases for their
phrase-based IR systems.
The \clarit system \citep{ez96:acl} 
identified both simplex and complex noun phrases  
for the purpose of improved
precision and recall in information retrieval, without the need
for training data.  The difference between a simplex noun phrase and
a complex noun phrase is that a simplex noun phrase omits all prepositional
phrases (the postmodifier in \figref{fig:rephine-np}).
\citet{av97:sigir} developed the \paraphrase system
which partitioned documents into clusters and summarised each 
cluster with noun phrases for the purpose of browsing.  A 
variant of the Brill tagger \citep{brill94:aaai} labelled
the words according to their parts-of-speech.  \linkit
by \citet{wek01:jcdl} identified noun phrases, which were subsequently
displayed by the \intellindex interface.  A part-of-speech
tagger was used to determine simplex noun phrases, which were then grouped
according to their heads.  As a result, phrases such as 
``the university bookstore'' was associated with the
phrase ``a bookstore''.

Extraction of keyphrases (document keywords in the form of 
phrases) with machine learning techniques was investigated by 
\citet{fpwgn99:ijcai} and embedded within a system dubbed
\kea.  Several interfaces have been built on top of \kea,
including \kniles, \phrasier, and \keyphind.  \kniles and
\phrasier uses the 
phrases identified by \kea to link documents together via
hyperlinks \citep{jones99:interact, jp99:dl}.  Browsing
phrases identified by \kea are supported by the \keyphind
system \citep{gpwnf99:dss}.  \keyphind removes all but twelve
phrases for each document in a collection and constructs three
types of indexes with the phrases:  
a word-to-phrase index, a phrase-to-document
index, and a document-to-phrase index.  A user study was conducted
which compared the usability of \keyphind to a traditional IR system
based on ranked queries.  The results of the study showed that
\keyphind was preferred for certain browsing tasks, such as collection
evaluation.  A more recent user study \citep{jp01:jcdl}
with \kea and keyphrases looked at computer science papers and
compared the phrases selected by \kea with those specified by
the authors of each paper.  The study's conclusion was that the
phrases selected by \kea were almost as good as ones chosen by
the authors.

Phrase selection through statistical methods is more relevant to
\rephine than through semantic methods.  
The \phind system of \citet{nwp97:acmdl} selected 
phrases using the \sequitur compression mechanism, which is similar
to \repair and was discussed in detail in \chapref{chap:repair}.  Like
word-based \repair described here, \sequitur was modified to 
operate on word tokens
in order to ensure word alignment.  Punctuation marks were removed and
words were case folded to lower case.  Word stems were handled by 
the phrase
browser so that if the word ``book'' was chosen, a search for the word
with different endings was performed.  Follow-up work on a browsing
system built on top of \sequitur was described by \citet{nwp99:ijdl}.
In order to browse large document collections, a disk-based browsing
system is proposed.  The phrase hierarchy is placed on disk with
an inverted index.  Smaller auxiliary data structures are
kept in main memory for accessing the phrase hierarchy.

\citet{wza99:adc} describe an index structure called a {\emph {nextword index}}
which provides explicit support for phrase querying.
This multi-level data structure is an index to pairs of words 
that appear in the document collection.  The word lexicon forms the
top-level of the index.  Each word in the lexicon 
has a nextword list that, together, indicates the word pairs that exist
in the collection.  Every word pair has a position
list as the third and final level in the nextword index.  The position
list specifies the documents and the document positions where
the pairs of words occur.  Subsequent extensions to this work include
compaction techniques for improving storage requirements
\citep{bwz01:spire}, and a description of how nextword indexes can be
used for mono-directional phrase browsing \citep{bwz01:acsc}.  
The nextword index for a transformed version of \wsjb 
occupied 56\% of the collection size.  When a query phrase with
more than two words is processed, it must be partitioned into pairs of
words for look-up into the index.  The position lists obtained
using the word pairs need to be combined in order to find locations
of the original, longer phrase.  
\citeauthor{bwz01:acsc} [\citeyear{bwz01:acsc, bwz02:sigir}] 
showed how the word
pairs should be selected in order to optimise querying time, and then
implemented a retrieval system which combined a
nextword index with a word-level IR system.
It was shown that the addition of the nextword index provided
more efficient retrieval when the first word is a common word.

Several methods combine semantics with statistics for phrase 
selection.  \citet{wolff80:ls} tagged words according to 
their parts-of-speech, and replaced frequently
occurring pairs of word classifications with a recursive pairing
mechanism similar to \repair.  \citet{pwcb00:acmdl} improved on
\phind by combining \sequitur with \kea.  A Brill tagger was combined with 
\kea and applied to the documents in order to identify boundaries
between noun phrases.  Next, \sequitur was applied to combine
words into phrases, with constraints enforced to prevent any crossing
over the boundaries.

A separate, yet related area to retrieval with phrases
is that of locality-based indexing
(or proximity indexing \citep[pg.~101]{br99:book}).  Locality-based
indexing creates an index at the granularity of individual word
tokens instead of documents.  When multiple terms are included in
the query, the results are chosen not only because of query
frequency, but also the distance between query terms within 
the documents.  Some past works with locality-based retrieval
include those of \citet{bsms95:trec}, \citet{ccb95:trec},
\citet{ht95:trec}, and \citet{dm99:sigir}.  

\newsubsection{Analysing \rephine}{subsec:rephine-analyse}

The \rephine system of this chapter can be analysed in
terms of its efficiency, effectiveness, and similarity and
differences with other work.
With respect to the efficiency of \rephine, the time
required by decompression has already been presented.
Experiments in \chapref{chap:remerge}
showed that the total decoding time for \news 
was 236.9 seconds, when \shuff was used as the coder
(\tabref{tab:remerge-pa-time} on 
\pgref{tab:remerge-pa-time}).  For the purpose of \rephine,
less time is required since only the phrase hierarchy and
the two lexicons are fully decoded.  Decoding the
two lexicons required 0.8 seconds (\tabref{tab:remerge-pa-otherstats} on
\pgref{tab:remerge-pa-otherstats}).  The total time required for phrase 
hierarchy decoding, including graph construction by \rephine was
12.3 seconds, 
averaged over three trials on a \vipe.  Experiments
in \chapref{chap:review} showed that 
searching for symbols in the reduced word sequence
encoded with \reviewb is fast, but depends on the number of 
symbols searched for (see Figures \ref{fig:review-ws-freq}
and \ref{fig:review-ws-len} on pages
\pageref{fig:review-ws-freq} and \pageref{fig:review-ws-len}, 
respectively).  The values in these graphs
provide a good estimate of the amount of time required
to search a sequence coded with \reviewb.  Finally, locating 
100 positions in a stream of modifiers with indexed Huffman
coding requires no more than 2.2 elapsed seconds
if a block size of 32,768 bytes was chosen, shown in
\tabref{tab:review-modtime} on \pgref{tab:review-modtime}.

In summary, 13.1 CPU seconds is required by \rephine to decode
the phrase hierarchy and the two lexicons.  Approximately
10 elapsed seconds is necessary for \reviewb to search 
the compressed sequence for 100 different symbols.  The
modifiers at a certain location in a stream can be found in
less than 0.1 seconds.

It is more 
difficult to assess the effectiveness of \rephine, 
or phrase browsing in general.  The main feature
of phrase browsing is the 
user-driven navigation between symbols in the phrase hierarchy.  This
interface experience cannot be easily quantified.
If the searching phase of \rephine is assessed using
precision as the evaluation metric, then problems similar to those encountered with 
Boolean queries need to be addressed, since every result returned contains
the symbol being sought, and precision is 1.0.  

On the other hand, the proportion of relevant results (recall) can 
be measured, but in a slightly different manner than in the traditional
definition.  Unlike with a Boolean query or an exhaustive string search, 
recall is not guaranteed to
be 1.0, because of the way in which the pairing process of \repair breaks up
phrases.  For example, suppose part of a message contains the
five symbols:  $\alpha$$\beta$$\delta$$\beta$$\delta$.
Suppose also that the frequency of $\alpha$$\beta$ throughout the message
is higher than the pair $\beta$$\delta$, and that the replacements
made are \tc{1} $\rightarrow$ $\alpha$$\beta$ followed by
\tc{2} $\rightarrow$ $\beta$$\delta$.  With these replacements, the segment
$\alpha$$\beta$$\delta$$\beta$$\delta$
reduces to \tc{1}$\delta$\tc{2}.  As a consequence, if the symbol \tc{2} is selected
during phrase browsing, then only the second occurrence of it
in this segment is found.  The first occurrence
has been split by the replacement with symbol \tc{1}, and is not reported 
to the user.  

Experiments were conducted with the first \mib{20} of \news (\wsja), compressed
with punctuation-aligned \repair, to evaluate the average recall.  Unlike
results from \tabref{tab:prepair-stats} on \pgref{tab:prepair-stats}, 
phase 1 of \remerge has been applied,
removing 16 duplicates in the phrase hierarchy.  For each symbol in the phrase
hierarchy, the frequency of the symbol and all other higher
generation symbols that contained it were compared with the actual
frequency in the word sequence.  The results from these
experiments are presented in \tabref{tab:rephine-recall}.

\tab{ccccc}
{
Symbol      & Number of symbols & \multicolumn{3}{c}{Distribution of recall values} \\ \cline{3-5}
length      & in phrase hierarchy & Median      & Mean   & Standard   \\
            & ($\Ph$) & recall       & recall & deviation  \\
}
{
%%  39,638 changed artificially to 39,637 for 0-length word, total symbols actually 231,546
%%  Upper range of #3 is 2.500
%%  total symbols 16 less than ch4 due to Remerge phase 1
{\D\D}1     &  \D39,637 & 1.000 & 1.000 & 0.000\\
{\D\D}2     &  \D79,693 & 0.750 & 0.706 & 0.295\\
{\D\D}3     &  \D60,825 & 1.000 & 0.794 & 0.260\\
{\D\D}4     &  \D25,245 & 1.000 & 0.862 & 0.225\\
{\D\D}5     &  \D10,932 & 1.000 & 0.917 & 0.184\\
{\D\D}6     & \D\D4,998 & 1.000 & 0.938 & 0.163\\
{\D\D}7     & \D\D2,895 & 1.000 & 0.954 & 0.141\\
{\D\D}8     & \D\D1,686 & 1.000 & 0.954 & 0.144\\
{\D\D}9     & \D\D1,180 & 1.000 & 0.957 & 0.136\\
10+         & \D\D4,454 & 1.000 & 0.955 & 0.136\\[1.0ex]
Overall     &   231,545 & 0.975 & 0.904 & 0.168 \\
}{Recall of the symbols in the \wsja phrase hierarchy.
Median recall values for each symbol length are in the third column,
followed by the mean, and the standard deviation.
}{Recall of symbols in the phrase hierarchy (\wsja)}{tab:rephine-recall}

In \tabref{tab:rephine-recall}, the 231,545 distinct symbols formed from
\wsja were separated into ten symbol groupings based on their expanded
length, measured in word tokens.  Since the zero-length
word token for separating a long non-word token is of no interest,
it does not appear in the table.  

The second column of the table shows
the size of each grouping.  The majority of symbols were 2 or 3 word tokens
in length.  The remaining three columns indicate some gross statistics
for the distribution of recall values for these symbols:  the median,
the mean, and the standard deviation.
Generally, results
were worse for symbols which represented 2 to 4 word tokens.
For the remaining categories, the average recall was at least 0.9
and their corresponding medians meant that at least half of the 
symbols had a recall of 1.000.  This result is further reinforced
by the standard deviation ranging from 0.000 to 0.184.  The average
recall and standard deviations were less satisfactory for symbols lengths of 2
to 4.  When the symbol length is 2, the median recall was only 
0.750.

The results from the table can be interpreted as follows.
Short phrases are more difficult to accurately find than primitives and
long phrases.  For example, in \wsja, the phrase ``prime 
ministr'', can be easily split depending on the words immediately
before and after it.  Note, however, that the most common phrase
that includes each word is exactly the one most likely to have
a high recall.  And, as phrases get longer, since their 
underlying components are more frequent, \repair does not 
separate them.

One anomaly is not shown in the experiments with \rephine.  Even
though precision across the set of symbols in the phrase hierarchy
is always 1.0, not every symbol can be 
searched.  In particular, while \rephine is able to locate
every symbol chosen through phrase browsing,
many combinations of words do not exist in the phrase hierarchy
and cannot be searched for.
In \subfigref{fig:rephine-scr-browse}{c}, the phrase 
``prime minist'' was extended to the left, and one phrase found
was ``canadian prime minister''.  However, the
phrase ``canadian prime'' does not exist in the phrase hierarchy,
and therefore, cannot be searched for by \rephine.  
The problem is the symbol pairing order employed by \repair 
(and \remerge) makes no attempt to place every multi-symbol combination
into the phrase hierarchy.  This point is expanded further later in this chapter.

Phrases for the dictionary are selected by \repair and \remerge 
based solely on symbol frequency, and in this sense, \rephine is
closely related to \sequitur's phrase browser, \phind.
\citet{nwp97:acmdl} identified the phrase boundary problem
with the phrase hierarchies produced by \sequitur; a problem
which also affects the phrase hierarchies of \repair.  The phrase
boundary problem refers to two symbols which expand to identical
strings, but have differing compositions.
For example,
it is possible that ``canadian prime \textbar\ minist'' and
``canadian \textbar\ prime minist'' become two separate symbols
in the phrase hierarchy, creating no problems for compression,
but confusing the browsing process.  
On the other hand, phase 1 of \remerge explicitly addresses 
this problem when phrase hierarchies are merged.  Recall
that phase 1 expands each symbol into a string of primitives
for comparison.  Symbols that represent duplicate strings are
removed.

Another drawback with \rephine is apparent in
the first screen of \subfigref{fig:rephine-scr-browse}{a},
which presents a list of the word tokens 
in the document collection.  Such an exhaustive list is similar
to a Boolean or ranked query system which lists every
word found in the collection, and is analogous
to a book index containing every word found in the book.
The advantage of \rephine for phrase browsing is not the
word lexicon,
but the relationships formed between words in order
to combine into longer phrases.  Nevertheless, such an opening
window is distracting to users.  Subsequent lists also
suffer from the same problem, but to a lesser degree.
The words in the first window and phrases in the remaining
windows of \figref{fig:rephine-scr-browse} could be
filtered so that unhelpful symbols are removed.  

Case folding and stemming by \prepair reduce the size of the
word lexicon and phrase hierarchy, while also improving browsing 
by providing one word which represents several similar ones.
\chapref{chap:prepair} showed the effect stemming and case folding had 
on the the word lexicon of \wsja 
(\tabref{tab:prepair-lexicon-size} on \pgref{tab:prepair-lexicon-size}).
\tabref{tab:rephine-cfstem} updates those results for
\news.  Recall that the rules for stemming
only apply to words in lower case.  In the third row of
the table, if the part of a word 
in upper case is examined by the stemming algorithm, and
no case folding is applied first, then the word would be
left unchanged.  For example, ``Searching'' stems to ``Search'',
while ``SearchinG'' does not.  The third
column of the table shows the number of primitives
in the phrase hierarchy, while the last column indicates
the number of phrases.  Case folding and stemming has an
effect on the size of the word lexicon, confirming the
results for the smaller \wsja file.  Moreover, 
the average lengths of words in the lexicon 
are slightly longer than the values listed in 
\tabref{tab:prepair-lexicon-size}, as indicated in the fourth
column of the table.

\tab{ccccccccc}
{
Case & \multirow{3}*{Stem} & Unique      & Average & \multicolumn{4}{c}{Percentage of primitive extensions} & Unique \\ \cline{5-8}
fold      &      & primitives  & length  & Left & Right & \multirow{2}*{Both} & \multirow{2}*{None} & phrases \\
          &      &             & (chars) & only & only  &      &      &         \\
}
{
No        & No   & 533,205    & 8.00 & 10.2\% & 11.3\% & 18.5\%  & 60.0\% & 8,464,282 \\
Yes       & No   & 448,278    & 8.10 & \D9.6\%& 11.4\% & 18.8\%  & 60.2\% & 8,269,436 \\
No        & Yes  & 401,448    & 7.34 & 10.3\% & 11.7\% & 18.2\%  & 59.8\% & 8,147,217 \\
Yes       & Yes  & 336,739    & 7.46 & \D9.7\%& 11.8\% & 18.1\%  & 60.4\% & 7,940,620 \\
}
{The effect case folding and stemming has on the phrase 
hierarchy created from punctuation-aligned \repair for
\news with method E of \remerge.  The last row corresponds to
the application of both case folding and stemming of word tokens, 
and reflects the arrangement used for the experiments with 
\news in this chapter.}{The effect 
case folding and stemming has on the phrase hierarchy (\news)}
{tab:rephine-cfstem}

In the fifth, sixth, seventh, and eighth columns of the table, 
the set of primitives has been divided into four
groups based on whether or not each one can be extended. 
In all cases, about 60\% of the symbols do not participate as
a component of a higher generation symbol.  However,
simply eliminating these symbols from the opening 
window of \rephine is too extreme.  If a word token 
fails to become part of a phrase, it does not mean that the 
word is not interesting or is infrequent.

\citet{nwp97:acmdl} noticed a similar problem with the
phrase browser \phind.  The
solution adopted was to filter words and phrases based on
word categories.  Words were labelled as {\emph {rare}}, 
{\emph {interesting}}, or {\emph {common}} based on
a set of frequency thresholds.  When the phrase hierarchy
is explored, an expanded phrase is shown only if it
differs from the previous phrase by at least one interesting word.
For example, if ``book'' is an interesting word, but ``the'' 
is a common word, then the expansion of ``book''  
to ``the book'' would not be displayed.  The reason is that adding a common word to 
the original word does not add any value for the user.  
However, ``book'' can be expanded directly to ``the new book'', 
provided the word ``new'' is also an interesting word.
The drawback of this heuristic is that a common word
may still be of interest to users, and a rare word does not
necessarily represent a typographical error.

At the centre of the problem with \sequitur, 
and \repair as well, is that selecting
phrases based on frequency generates too many phrases,
with many differing by only a common word like ``the''.  
\citet{pwcb00:acmdl} proposed the solution of combining
\sequitur with \kea, since \kea was known to select
too few phrases by itself.  Likewise, the combination of
\repair with a mechanism similar to \kea may improve the number of
phrases presented by \rephine.
A more detailed look at the phrases created by \repair
for \rephine is provided next.

\newsubsection{Comparing \rephine with Noun Phrases}{subsec:rephine-nlp}

Another part of \rephine that is worth evaluating is the usefulness of 
the phrases presented to the user.  The quality of the
phrases is not related to precision or recall because 
the primary concern is the type of phrases available for
browsing, and not the result from searching.

\citet{wolff80:ls} produced phrases through a recursive
pairing mechanism called \mka.  In \mka, pairs are chosen based on
the part-of-speech categories assigned to the word tokens.
The quality of the phrases was compared to the phrases
selected by a human linguist.  In a similar fashion, the
phrases made available by \repair can be compared 
with the ones identified by a reference program employing 
natural language processing (NLP) techniques.

As with other IR systems based on NLP techniques, the basis
for these experiments is that a user is more interested in a
complete noun phrase such as ``prime minister'' than 
a broken phrase like ``Canadian prime''.  The aim of the experiments is to
report on the probability that a symbol in the phrase hierarchy
is a noun phrase, and the probability that noun phrases
exist in the phrase hierarchy.
Noun phrases are identified through an NLP
technique called \linkgram \citep{st91:tech} embodied in a
program by the same name, 
available from \myurl{http://www.link.cs.cmu.edu/link/}.  The
latest version of the software is 4.1, dated 
August 2000.

The \linkgram system processes a sentence at a time.  
In the first step, individual words are assigned to part-of-speech 
categories by consulting a static dictionary.
The categories of novel words are guessed through
a technique the authors call ``morpho-guessing''.  Words have links to
both sides for combining with other words in the sentence.
For example, the two words ``the book'' can merge because of
a right-facing ``D$+$'' link on the word ``the''  
and a corresponding left-facing ``D$-$'' link on the noun
``book''.  The ``D'' in both cases symbolise a determiner
such as ``the'' and ``an''.
An edge is drawn between these two words to indicate
that they are compatible with each other.  When a grammatical
sentence is correctly parsed, a valid {\emph {linkage}} is said
to be formed.  Several requirements 
must be satisfied in order to obtain a linkage.  One of these
requirements is that the graph formed must be connected, 
with all of the words in the sentence reachable.

Subsequent work on the system \citep{gls95:tech} permits 
{\emph {null links}} and post-processing of valid linkages with a
phrase parser for identifying constituents such as noun phrases and
verb phrases.  If a sentence cannot be successfully parsed, then 
words in the sentence are removed incrementally until a successful
parsing can be found.  When the sentence is finally parsed, each
removed word introduces a null link into the linkage.  When a linkage
for a sentence has been established, the post-processing phrase
parser combines words to form constituents of one or more words.
On-line documentation for the phrase parser\footnote{Available
at \myurl{http://www.link.cs.cmu.edu/link/ph-explanation.html}.}
reports that the constituents determined by the parser were
75\% correct on the Penn Treebank 
text\footnote{The Penn Treebank project is at 
\myurl{http://www.cis.upenn.edu/~treebank/}.}, 
which is newspaper text already annotated linguistically.

A sample parsing of a headline from \wsja 
is shown in \figref{fig:rephine-linkgram}.
In the figure the headline is shown in part (a).  The \linkgram
system considers it a valid sentence since two linkages are 
identified, as shown in parts (b) and (c).  In 
\subfigref{fig:rephine-linkgram}{a}, the ``G'' link is used to
pair ``South'' with ``Korea''.  A ``G'' link connects proper
nouns together, identified by the upper case letters at the beginning
of each of these words.  Similarly, a ``G'' link is inserted
between ``Current'' and ``Account''.  The second linkage case folds
``South'' so that the first ``G'' link is converted to an ``A''
link.  An ``A'' link connects a adjective with the noun it follows.
The ``Ss'' link pairs a subject with a verb, while ``Ost'' edges
connect a transitive verb to a singular object.  So, \linkgram
incorrectly treats ``'s'' as a verb in both linkages.
Linkages are scored so that ones with less null links or a lower edge
length total are displayed first.
The constituents for the linkage in (b) is
presented in \subfigref{fig:rephine-linkgram}{d}.  At the beginning
of each constituent is its classification.
The abbreviation ``VP'' denotes a verb phrase, 
while ``NP'' is a noun phrase.

\fig{
\begin{tabular}{c}
{\texttt {South Korea's Current Account}} \\[0.5ex]
(a) News headline \\[1.5ex]
\includegraphics*{./linkage-1.eps} \\[0.5ex]
(b) First valid linkage \\[1.5ex]
\includegraphics*{./linkage-2.eps} \\[0.5ex]
(c) Second valid linkage \\[1.5ex]
\parbox{5cm}{\input{./parser-ex.tex}} \\
(d) Constituents derived from the first linkage \\
\end{tabular}
}{An example parsing of a headline from \wsja using the \linkgram
system.  The headline is shown along the top as part (a).  Two
linkages are determined by \linkgram, which are shown in (b) and (c).
The linkage from (b) is post-processed for constituent identification,
yielding the output in (d).  Noun phrases are marked with ``NP''.}{Example
parsing of a headline with \linkgram (\wsja)}
{fig:rephine-linkgram}

As with previous work in IR and NLP, the focus of the 
phrase comparison is with noun phrases.  
The number of unique phrases that \linkgram found in \wsja was
compared with \rephine's phrase hierarchy.  Since only grammatical
sentences can produce linkages, the test data of news
articles were filtered so that sequences of words that are not 
expected to give many noun phrases were omitted.  First, \sgml tags
were removed and multiple consecutive whitespaces collapsed into
a single space.  Sentences were given to \linkgram one at
a time.  Then, the constituent analysis was determined
by the phrase parser.

As with the earlier experiments to measure recall, the \wsja document was
used as test data.  The computing resources of the \linkgram program were
quite high.  It required just over 5 hours
to process the first \mib{1} of \wsja
on a \vipe test machine.  This time included
the above steps, starting from the \sgml tag filtering, and ending
with the identification of constituents.  Note that
the amount of work performed by \linkgram is more than what is
necessary to identify noun phrases.  Other systems, such as the
\keyphind browser of \citet{gpwnf99:dss}, simply categorise words
according to their parts-of-speech, and then form noun phrases based on a set
of fixed rules.  Even so, \citeauthor{gpwnf99:dss}\ required 4 days to process
\gib{1} of text on a Pentium 233 MHz machine running Linux.
In contrast, \linkgram must first successfully find a linkage for a
sentence, and only when this is done, can it isolate the noun phrases.

After \linkgram completes, constituents which are noun phrases,
and are not composed of other phrases, were
recorded.  The noun phrases were case folded
and then stemmed with the Porter stemming algorithm employed
by \prepair.  In order to comply with the 16 character per word
limitation imposed by \prepair, spaces were inserted into long
words.  Finally, a list of all of the distinct noun phrases in
\wsja was prepared.

\figref{fig:rephine-sgmlsample} illustrates how a news article
in \wsja was parsed, with linefeeds altered to improve display.
Text that was stripped from the article is shown shaded.  All \sgml tags
are removed, as well as text within most tags.  Only text
between opening and closing {\tl}HL{\tg} and {\tl}TEXT{\tg} tags
were retained, since the headline and the body of the article has
the best chance of providing complete sentences.  The noun phrases
found by \linkgram for this example are shown in boxes.  The headline used
as an example in 
\figref{fig:rephine-linkgram} was taken from the article shown in
\figref{fig:rephine-sgmlsample}.  

\fig{
\begin{tabular}{l}
\input{./wsja_sample_parse.sgml}
\end{tabular}
}
{An example article from the \wsja document, with noun
phrases indicated with outlined boxes.  Shaded boxes
represent parts of the article that were removed prior to
applying \linkgram.  The article has been formatted with
linefeeds for improved display.}{Example article
with noun phrases identified (\wsja)}{fig:rephine-sgmlsample}

The \linkgram system successfully found 147,294 grammatical
sentences in \wsja.  Within these sentences, 744,721 noun
phrases were identified, of which 216,661 were unique
after stemming and case folding.  These unique
phrases were divided into groups based on their lengths, in words,
and compared with the phrases available to \rephine, which were
also grouped according to length.  The phrases for \rephine are
identical to the ones reported in \tabref{tab:rephine-recall}.
Information about these
two sets of phrases are shown in \tabref{tab:rephine-nlpph-stats}.
The sizes of the sets of phrases are listed by lengths in the
second and third columns.
Note that noun phrases are only found if a valid linkage has been determined.
In the fourth column of
this table, the number of phrases common to the phrase
hierarchy and in the set of noun phrases has been tabulated.
The probability of a symbol in $\Ph$ being a noun phrase is given in fifth
column.  The reverse, the
probability of randomly selecting a noun phrase which is in
the phrase hierarchy, is listed in the last column of the table.  

\tab{cccccc}
{
Phrase      & Number of        & Number of    & Size of                & Probability & Probability \\
length      & phrases in       & unique noun  & $\Ph \cap {\text{NP}}$ & $\Ph$ has & noun phrase \\
            & hierarchy ($\Ph$)& phrases (NP) &                        & noun phrase & in \Ph \\
}
{
{\D\D}1     & \D39,637  &    13,172 &        13,172 & 0.332 &    1.000 \\
{\D\D}2     & \D79,693  &    78,527 &        21,692 & 0.272 &    0.276 \\
{\D\D}3     & \D60,825  &    72,492 &       \D6,350 & 0.104 &    0.088 \\
{\D\D}4     & \D25,245  &    32,419 &       \D1,578 & 0.063 &    0.049 \\
{\D\D}5     & \D10,932  &    12,304 &       \D\C397 & 0.036 &    0.032 \\
{\D\D}6     & \D\D4,998 &    \D4,939 &      \D\C144 & 0.029 &   0.029  \\
{\D\D}7     & \D\D2,895 &    \D1,696 &     \D\C\D32 & 0.011 &   0.019  \\
{\D\D}8     & \D\D1,686 &    \D\C662 &    \D\C\D\D8 & 0.005 &   0.012  \\
{\D\D}9     & \D\D1,180 &    \D\C247 &    \D\C\D\D2 & 0.002 &   0.008  \\
10+         & \D\D4,454 &    \D\C203 &    \D\C\D\D1 & 0.000 &   0.005  \\[1.0ex]
Total       & 231,545   &    216,661 &       43,376 & 0.187 &   0.200 \\
}{Quantifying the quality of symbols in the phrase hierarchy of
\wsja by calculating the intersection of noun phrases, as identified by 
\linkgram, with the phrase hierarchy.  Symbols have been
divided into 10 groups, based on length (in words).  The second and
third columns provide insight into the number of symbols (or 
noun phrases).  The intersection of these two columns are reported
in the fourth column.  The last two columns indicate the proportion of
phrases in $\Ph$ which are noun phrases (fourth column divided
by the second column), and the probability a noun phrase also being in
$\Ph$ (fourth column divided by the third column), 
respectively.}{Comparing the phrase hierarchy against the
noun phrases (\wsja)}{tab:rephine-nlpph-stats}

As \tabref{tab:rephine-nlpph-stats} shows, the majority of phrases in
the phrase hierarchy and noun phrases from \linkgram are
2 or 3 words in length.  The overlap between the two sets
is around 28\% for phrases of length two for both measures, 
a somewhat disappointing result.  Worse, as the phrases
increase in length, there is even less correlation with the phrase 
hierarchy, as the last two columns show.  This difference is partly due to the methods used by \repair
and \linkgram to build phrases.  The \linkgram system operated on sentences
individually, so phrases were guaranteed to be aligned on sentence boundaries.
Also, \sgml markup did not appear in any of the sentences, because of the
preliminary filtering.  On the other
hand, while punctuation-aligned \repair encourages alignment with punctuation,
phrases can still span sentences, since they are not processed individually.
Furthermore, since \repair was applied on the original version of \wsja,
any phrase that contained markup would not have a corresponding phrase
from \linkgram.  The analysis in \tabref{tab:rephine-nlpph-stats}
is based solely on the existence of phrases, and takes no account of the
frequency.  Because \repair finds common phrases, it is likely 
that the scores would
be considerably better if the probabilities were assessed over the 
sequence instead 
of over the phrase hierarchy.

Making phrases available in the phrase hierarchy is only half of \rephine's
purpose since matching contexts must also be located in
the reduced sequence.  Using the same methodology as established for
\tabref{tab:rephine-recall}, the recall of the phrases that exist
in both the phrase hierarchy and the list of noun phrases is presented
in \tabref{tab:rephine-recall-nlp}.  This time, rather than
presenting recall values for all symbols in the phrase
hierarchy, only the ones that were also noun phrases identified
by \linkgram are considered.  The second column of this
table lists the number of noun phrases that were also 
in the phrase hierarchy, copied from the fourth column of 
\tabref{tab:rephine-nlpph-stats}.  In \tabref{tab:rephine-recall-nlp},
as with the earlier experiment, 
the median recall value for each phrase length group is reported.
The mean and standard deviation of
each set of recall values is also shown.

\tab{ccccc}
{
Phrase      & Size of                & \multicolumn{3}{c}{Distribution of recall values} \\ \cline{3-5}
length      & $\Ph \cap {\text{NP}}$ & Median       & Mean   & Standard   \\
            &                        & recall       & recall & deviation  \\
}
{
{\D\D}1     &   13,172 & 1.000 & 1.000 & 0.000\\
{\D\D}2     &   21,692 & 0.800 & 0.720 & 0.298\\
{\D\D}3     &  \C6,350 & 1.000 & 0.787 &  0.271\\

{\D\D}4     &  \D1,578 & 1.000 & 0.845 & 0.242\\

{\D\D}5     &  \D\C397 & 1.000 & 0.831 &  0.264\\
{\D\D}6     &  \D\C144 & 1.000 & 0.849 & 0.250\\

{\D\D}7     & \D\C\D32 & 1.000 & 0.853 & 0.258\\
{\D\D}8     &\D\C\D\D8 & 1.000 & 1.000 &  0.000\\
{\D\D}9     &\D\C\D\D2 & 1.000 & 1.000 & 0.000\\
10+         &\D\C\D\D1 & 1.000 & 1.000 & 0.000\\[1.0ex]
Overall     &   43,376 & 0.980 & 0.889 &  0.158\\
}{Recall effectiveness of the noun phrases in the phrase hierarchy
for \wsja.  The median has been calculated
based on phrase length (in words)
in the third column, with the mean, and the
standard deviation indicated in the last two columns.}
{Recall of noun phrases in the phrase hierarchy (\wsja)}
{tab:rephine-recall-nlp}

The results from \tabref{tab:rephine-recall-nlp} mirror those
of \tabref{tab:rephine-recall}, but note the small size of the
sample for phrases of length 7 or more.  Again, the average 
recall is above 0.7, and more than half
of the symbols had perfect recall.  The standard deviation 
is wider and affects more groupings than in 
\tabref{tab:rephine-recall}.  

\newsection{Tools for Retrieval}{sec:rephine-summary}

The \rephine system allows users to perform two inter-related
tasks.  The first of these is phrase browsing.  Starting
from a list of the distinct words found in the document, the user can
either extend the current symbol by adding symbols to the left or 
to the right, or decompose
the current symbol into its two constituents.  A graph structure
with one symbol from the phrase hierarchy at each node
was established for the purpose of phrase browsing.  Each node
has six pointers for navigation and defining its structure.

Once the user has found a phrase of interest, the graph
structure is consulted again to determine the complete set of phrases
containing the chosen one.  This set of symbols, $\Criteria$,
is the basis for the search in the reduced word sequence to determine 
all of the locations at which any symbol in $\Criteria$ appears, along
with a suitable context for each.  The second job of \rephine
is to interactively display this set of results one by one to the user.
Each result consists of word tokens which have been stemmed and
case folded.  The non-words that follow each word token are
also initially unavailable.  Each presented result can be refined by the 
user by requesting modifiers from the three modifier streams, in any
combination.  If at this stage, another symbol of interest has been
isolated, it can be passed back to the phrase browsing part of 
\rephine.

In this chapter, \rephine has been demonstrated with a document collection 
\mib{1,000} in size on a \vipe.  Before considering larger collections, note that
only three of the seven streams available to \rephine must reside
in memory.  The four remaining streams are compressed in blocks so that
only parts of them are read from disk into memory.  The phrase hierarchy
and the two lexicons occupy 1.8\% of disk space compared to the size of
\news, and all of them must be accessible by \rephine in decoded form.
The complexity and size of the phrase hierarchy ensures that it requires 
the most memory during browsing after it has been fully decoded.

Based on these observations, several
approaches may be selected in order to browse larger collections.
First, \secref{sec:remerge-append} showed how \remerge can be extended
so that documents are appended to a compressed representation while
keeping the phrase hierarchy static.  Experiments demonstrated this
technique with character-based \repair, and an analysis of the changes
necessary for punctuation-aligned \repair was discussed.  In particular,
a method of handling novel words is required.  Second, either the maximum
generation or the maximum symbol length can be limited so that the
phrase hierarchy only contains moderately long phrases.  Third, while all
word tokens must exist in the phrase hierarchy as primitives, additional
heuristics, such as those offered by \kea and \linkgram, 
may be used when building the phrase hierarchy.  Instead of reducing
the number of phrases shown by \rephine, this approach would prevent
certain symbols from being added to the phrase hierarchy.  Finally, any
combination of these three methods can be used in order for larger document
collections to be compressed by \repair and \remerge, and subsequently
browsed.

The \rephine implementation has not reached
the stage where it could sensibly be made the 
subject of a user study.  As with \sequitur, too many symbols 
exist in \repair's phrase hierarchy, and some filtering is
necessary to improve usability.  The removal of symbols
through NLP techniques has been trialled by other IR systems, and
it is likely that \rephine would be more feasible if similar 
techniques could be employed.  An extension of this idea is to 
rank symbols using some metric so that those that are expected to
be the most useful are presented at the top.  Since there are
reasons why users would want frequent and
and infrequent phrases, simply sorting symbols
by frequency would not help.

User studies have been conducted to quantify the usefulness
of phrase browsing retrieval systems \citep{gpwnf99:dss, jp01:jcdl}, 
with some results being in favour of phrase browsing, and some less
conclusive.  \citet{wm01:sigirforum} concluded that
more work is still required in the area of phrase browsing
evaluation with users.  But even if the suggested changes 
were adopted for \rephine, it is expected that
a comparative user study would not favour \rephine.  While providing
users with a phrase browser was one of the aims of this development, the
necessary tension between browsing and compression inevitably means
that \rephine is not as careful as a dedicated phrase
searching mechanism.  One necessary improvement appears to be
achieving higher recall.
It is possible that the solution lies in
closely regulating word-based \repair's phrase selection 
heuristic.  As word-aligned \repair showed, preventing
certain pairs of symbols from joining does not necessarily
degrade compression effectiveness (see 
\figref{fig:prepair-bargraph-all} on 
\pgref{fig:prepair-bargraph-all}).

The benefits to
compression are easier to quantify than the benefits of phrase
browsing.  Since the latter quantity cannot be measured,
it is also difficult to assess when a reasonable compromise
between the two has been achieved.
Instead, the goal of this thesis has been to 
identify areas where compression and retrieval effectiveness and
efficiency are lost, so that one can be adjusted in favour
of another, as necessary.

\rephine is built on top of a compression mechanism, a structure which
limits its ability to browse and retrieve data.  
Throughout this thesis, the amount of compression effectiveness
lost in favour of faster retrieval times has been highlighted as
a means of compromising between the two competing sets of constraints.  In this
chapter, the current drawbacks with \rephine
represent the amount of
retrieval functionality lost in favour of supporting the
lossless compression of a document collection
with the underlying mechanism.
Employing NLP techniques with \rephine
may be the ideal solution for compensating
for \rephine's loss in retrieval effectiveness.  As mentioned
above, at the same time, NLP techniques may also reduce the
number of symbols that reside in memory during browsing.

Regardless of the improvements made to \rephine, phrase
browsing in general is one of many IR techniques.  This chapter
has also provided a glimpse into other querying techniques, 
including ranked queries,
proximity queries, and Boolean queries.  The system a user 
chooses depends on factors such as how familiar the user is
with the document, whether query terms include phrases, and whether
the order and the positions of the words in a phrase matter.
As \citet{gpwnf99:dss} pointed out, no single IR method 
is perfect and the most viable solution 
might be to provide ``several search tools that can 
make up for each others' shortfalls''.  

This chapter concludes the discussion in this thesis
about the balance between compression and phrase browsing.
Before summarising the thesis in \chapref{chap:summary},
the next chapter discusses a separate topic related to the
compression of small \html files.

}
{The sequence of steps performed by \rephine during phrase
browsing.  Dashed boxes indicate actions 
performed by the system, while solid boxes require
input from the user.  Shaded cylinders represent one of
the seven streams accessible by \rephine, with dashed lines
from them representing file access.  The user can exit
the system at any time.}{Sequence of steps
performed by \rephine during phrase browsing}{fig:rephine-alg}

In the first step of the figure, \rephine decodes the phrase
hierarchy, the word lexicon, and the non-word lexicon, and
maintains all three within memory.  Next, a directed
graph (DG) is built using the symbols of the phrase
hierarchy, to permit efficient searching.  Details
about the structure of the graph is given in the next section.
The third step is interactive phrase browsing.  
In browsing the phrase
hierarchy, the user interactively traverses the DG
until a symbol of interest, $\alpha$ say, has been isolated.

In the fourth step, the symbol of interest is used to identify all
other symbols in the DG which contain $\alpha$.
When that search completes, a set of symbols $\Criteria$ has been 
established.  Next, $\Criteria$ is input to the fifth
step, in which the \reviewb decoder is used to locate every
context in which any symbol in $\Criteria$ occurs.  The \reviewb
decoder returns a result set, $\Result$.  Suppose a symbol
$\beta$ contains the symbol $\alpha$ as a component.  In order
to avoid reporting the same location multiple times, when an
occurrence of $\beta$ has been found in the reduced word sequence, 
only $\beta$ is reported.
The size of $\Result$ is denoted as $|\Result|$, and members
are enumerated from $i=1$ to $i=|\Result|$.

The first result's context, $\Result_1$, is
displayed to the user.  At this point, the user may choose
any combination of five actions.  These actions are:  return to
phrase browsing; toggle case folding modifiers, stemming modifiers, 
or non-word insertion for $\Result_1$; or examine
the next or previous result.  It is assumed that the result
set is a circular list, so that the previous result to $\Result_1$
is $\Result_{|\Result|}$.  The three actions related to the modifiers
initiate seek operations to the corresponding positions in 
their respective compressed
streams.  After the modifiers have been retrieved, $\Result_i$
is re-displayed to the user.  The application of 
modifiers can be toggled in any order.  The first time
modifiers are required, access to the streams are performed.
Assuming requested modifiers are buffered in memory, subsequent 
requests do not need to access the file again.
Finally, if the contexts that were obtained were inadequate in
satisfying the user's need, any part of a context can be used as
input into the phrase browsing part of \rephine.

Later in this chapter, the entire phrase browsing process is detailed 
through a brief example which demonstrates the choices available
to the user.  But first, in the next section, the directed graph
required for phrase browsing is explained.

\newsection{Phrase Hierarchy as a Directed Graph}{sec:rephine-datastruct}

Once the phrase hierarchy has been decoded, a directed
graph (DG) is 
constructed.  Every symbol in the phrase hierarchy is assigned
to a separate node in the DG.  Each node contains six
directed edges connecting it to other nodes, as shown in 
\figref{fig:rephine-graph}.
In this figure, the symbols are based on character-based
\repair instead of punctuation-aligned \repair in order to simplify 
the discussion.  The current node is represented by the
symbol ``w{\textbar}o'', with the vertical bar (\textbar)
separating the symbol into its two components.
Two of the six pointers
are {\emph {child pointers}}, and refer to a symbol's structure
These pointers were described in 
\chapref{chap:remerge} as part of phase 1 of \remerge,
but were called components at that time.
The remaining four pointers are collectively called
{\emph {navigational pointers}}.
Navigational pointers are added for the purpose of
browsing and have not been required in the previous discussion.

\fig{
\includegraphics*{./graph.eps}
}{The six pointers used by \rephine to link nodes representing
symbols in the phrase hierarchy into a directed graph.  The symbols in
the example are based on character-based \repair, for illustrative
purposes only.  The node of interest is in the centre, with the
symbol ``w{\textbar}o''.  A vertical bar (\textbar) separates
the expanded symbols into left and right components.
Names of each pair of pointers are provided to the right of the
figure.
}{The six pointers used by a \rephine graph}
{fig:rephine-graph}

The purpose of each pointer is illustrated in the figure.
Pointers appear in pairs and are designated as being 
``left'' or ``right''.  The four navigational
pointers are in two pairs -- the {\emph {sibling pointers}}, 
and the {\emph {parent pointers}}.  

In the example figure, the current symbol is in
the centre, with its six pointers listed in full.  All of the other
nodes in the graph
have six pointers each as well, but the outgoing edges are 
omitted in the figure for simplicity.  The centre symbol 
is ``w{\textbar}o'', with left and right
components (or children) of ``w'' and ``o'', respectively.
The left sibling symbol of ``w{\textbar}o'' is another symbol which 
also has ``w'' as a left child.  A chain is formed from the 
centre node, with all symbols with ``w'' as a 
left child being linked together with their
left sibling pointers.  Within a chain, the nodes are unsorted,
so any node can serve as the front of the list.
Each chain ends with a final
{\emph {empty node}}.  The child pointers of all primitives
also lead to empty nodes.
Similarly, the right sibling pointer from 
``w{\textbar}o'', leads to another symbol with ``o'' as a right child.  

One important operation is to
obtain the entire chain when the current symbol is any symbol
along that chain.
The child and parent pointers help solve this problem.
Parent pointers directly lead to symbols which are a single 
{\emph {extension}} away from the current symbol.  An extension
occurs when a symbol is juxtaposed
to the current symbol on either side, arriving at another
symbol in the phrase hierarchy.  For example, in 
\figref{fig:rephine-graph}, the right parent pointer from
``w{\textbar}o'' allows a right extension to the symbol 
``wo{\textbar}od''.
\figref{fig:rephine-expandgraph} shows \figref{fig:rephine-graph}
at a larger scale to demonstrate how an entire chain of symbols
can be obtained from anywhere in the list.  The ``w{\textbar}o''
is at the head of the list of right siblings, 
and the current symbol is $\alpha$.
Both of these symbols have the same right component.
Every symbol along that chain
has a right child link to the primitive ``o''.  The front of
the chain can be reached by taking the right child pointer from
$\alpha$, followed by the left parent link of ``o'',
as shown by the dashed lines in the figure.
This is possible provided that
graph construction ensures that every node's parent pointer
leads to the beginning of a chain of parents.  If a symbol
does not have a parent, the parent is represented as
an empty node.

\fig{
\input{./expandgraph.pstex_t}
}{The directed graph structure created by \rephine to link symbols
together, at a larger scale than \figref{fig:rephine-graph}.  The
node of interest is ``w{\textbar}o'' in the centre.  The dashed
arrows shows how an entire chain of symbols can be obtained from
an arbitrary symbol $\alpha$ within the chain.  Also shown are two
siblings of the symbol ``wo{\textbar}od''.
}{The directed graph structure of \rephine, at a larger scale}
{fig:rephine-expandgraph}

The graph structure created by \rephine
resembles the data structure employed by \mka \citep{wolff75:bjp}
for the recursive pairing of symbols for discovering sentence
structure.  Each
symbol in the dictionary created by \mka has major and minor links.
A major forwards link is the same as a left child pointer in
\rephine's graph, while a major backwards link is identical
to a right child pointer.  Minor links operate in a similar
fashion to the sibling pointers of \rephine.

Equipped with this graph, five operations are available to
the user.  Assuming the user's currently selected symbol
is ``w{\textbar}o'' of \figref{fig:rephine-graph} and 
\figref{fig:rephine-expandgraph}, the
five operations are as follows.  First, the user can decompose
the symbol into its two components through the child pointers.
Second, if the right parent pointer from ``w{\textbar}o'' 
is used, followed by all left sibling pointers until the empty
node is reached, then all right extensions of ``w{\textbar}o'' can
be retrieved.  In \figref{fig:rephine-expandgraph}, this operation
would obtain the set 
of two symbols, \{``wo{\textbar}od'', ``wo{\textbar}uld''\}.
Likewise, a third operation is to extend ``w{\textbar}o''
to the left.  The union of these last two sets results in a
fourth function, extending ``w{\textbar}o'' in both directions.  

The fifth and final operation is reserved for decoding, instead
of phrase browsing.  Decoding requires identification of
all symbols which contain the current symbol $\alpha$,
either directly or indirectly.  The
located symbols are denoted in 
\figref{fig:rephine-alg} as $\Criteria$.  In order to find all
symbols in the phrase hierarchy graph which contain $\alpha$, 
a breadth-first search
with the navigational pointers is sufficient.  Let $\beta$
represent a symbol which is an ancestor of $\alpha$.  If $\beta$
has only one occurrence of $\alpha$, 
then all of its parents does as well.  But $\alpha$
must appear in either its left or its right component, which
can be determined based on how $\beta$ was reached.  So, only
one of the sibling pointers need to be followed.  If $\beta$
has multiple occurrences of $\alpha$, with at least one in each
component, then during the breadth-first search, $\beta$ is reached
twice.  Care must be taken to ensure that when $\beta$ is reached
the second time, only the other sibling pointer is traversed, and not the
parent pointers.  If this strategy is followed, then 
the nature of the navigational pointers
ensures that all symbols which include ``w{\textbar}o'' are included.

Construction of the DG requires $O(|\Sigma| + |\rho|)$ time,
for a phrase hierarchy of $|\Sigma|$ primitives and $|\rho|$
phrases.  Since the chiastic slide and interpolative coding have been
used to code the phrase hierarchy, the phrase hierarchy is already
in increasing generation order, and within each generation,
in increasing chiastic slide value order.  As no symbol is in
the same generation as either of its two components, constructing
the graph from the first primitive symbol in sequential order
results in graph construction resembling a bottom-up 
approach.
If the phrase hierarchy is stored in memory as an array, 
and if each of the six graph
pointers are represented as array indexes, then locating a
symbol's components requires $O(1)$ time.  
For each phrase, an insertion
is performed into the appropriate parent pointers of 
each of its two components, so that each node is added
at the front of its two chains of siblings in $O(1)$ time.
That is, a single linear pass through the phrase hierarchy
is sufficient to construct the graph.

The memory space required for the phrase hierarchy is linear in
the number of symbols in the phrase hierarchy.
The graph structure adds an additional 6 words of memory 
per symbol.  Also, the phrase hierarchy may
have every symbol expanded into primitives, or symbol
expansion may be deferred until the symbol is required.  
The choice made depends which is more important:
memory space or response time.  Experiments demonstrating
the advantages and disadvantages of phrase
hierarchy expansion were reported in \tabref{tab:repair-despair}
on \pgref{tab:repair-despair}.  In
the implementation of \rephine described in this chapter,
a buffer of recently decoded symbols is maintained and consulted
before decoding a symbol.  This approach was also employed in
\repair's decoder, \despair.

\newsection{Phrase Browsing Example}{sec:rephine-example}

The \rephine system performs two main tasks:  phrase browsing
and decoding symbol contexts for presentation.  Examples of both tasks
are presented in Figures \ref{fig:rephine-scr-browse} and 
\ref{fig:rephine-scr-decode}.  The 
aspects of \rephine related to the graphical user
interface were 
implemented with version 2.0.2 of the \gtk 
library\footnote{Available from \myurl{http://www.gtk.org/}.}.

The data for the example
is \news, processed with punctuation-aligned
\repair exactly as was described in the experiments of
\chapref{chap:remerge}.  The sequence of word tokens in
those experiments were partitioned into blocks of 20,971,520
symbols each, and then merged using method E of \remerge 
(\secref{sec:remerge-expt}).
Modifier and sequence coding were done with indexed \shuff
and \reviewb respectively.

In the previous section, the choices available to the user
for phrase browsing from the current symbol were discussed.
The user also needs to be given a starting point to browse
from.  In \rephine, the user is first presented with the
document's entire lexicon, as shown in
\subfigref{fig:rephine-scr-browse}{a}.  The extensive list
can be daunting, and one solution
is to filter words, discussed later
in the chapter.  In \subfigref{fig:rephine-scr-browse}{a}, the user
has selected the word ``prime''.  
The first symbol is extended to
the right in \subfigref{fig:rephine-scr-browse}{b}, and the
phrase ``prime minist'' has been selected.  While phrases in this 
window are ordered according to their locations in the DG, 
\citet{jp99:dl} and \citet{pwcb00:acmdl} describe systems %% Second cite due to witten00:cpm
where phrases can be ordered according to the number of documents
in which they appear, or their frequency in the collection.
While details about the implementation of these two features are
not available, it is expected that either the additional information
is stored by the system, or calculated at run-time.  Also note
the variations on the phrase ``prime minist'' due to the different
spellings of the second word.  While stemming and case folding
have reduced the size of the phrase hierarchy, it is still 
possible that two seemingly equivalent words stem differently.
The chosen phrase is extended to the left and the phrase 
``canadian prime minist'' has been highlighted in 
\subfigref{fig:rephine-scr-browse}{c}.  As this window shows,
unfortunately, the words ``Canadian'' and ``Canada's'' stem
to different words.  The identified phrase of 
\subfigref{fig:rephine-scr-browse}{c} is extended in both
directions, resulting in \subfigref{fig:rephine-scr-browse}{d}
and its 7 phrases.  Of these 7 phrases, 3 are right extensions
and 4 are left extensions to the phrase ``canadian prime minist''.
The selected phrase is broken down into its two components through
its child pointers, as shown in 
\subfigref{fig:rephine-scr-browse}{e}.  This phrase browsing
example commenced with the word ``prime'' and, in four steps,
arrived at the name of a Canadian prime minister.  In \rephine,
the refinement of a query term is done interactively with the
help of the phrase hierarchy's graph.

%%  A double-page figure
%%  Snapshot created using the command:
%%  import -frame -window "Main browsing session" ~/thesis/t/ch07/scrshots/snapshot1.eps
%%  Then, export to PS with xv (greyscale).
\fig{
    \setlength{\tabcolsep}{3pt}
    \begin{tabular}{ccc}
      \multicolumn{2}{c}{\includegraphics*[scale=0.6]{./scrshots/browse-0.ps}} &
      \includegraphics*[scale=0.6]{./scrshots/browse-1.ps} \\ 
      \multicolumn{2}{c}{(a) Initial window} & (b) Extend right \\ \\
      \includegraphics*[scale=0.6]{./scrshots/browse-2.ps} &
      \includegraphics*[scale=0.6]{./scrshots/browse-3.ps} &
      \includegraphics*[scale=0.6]{./scrshots/browse-4.ps} \\
      (c) Extend left & (d) Extend both ways & (e) Decompose \\
    \end{tabular}
%  }
%  \end{leftfullpage}
}{A sample phrase browsing session of the \news document.}
{Sample phrase browsing session of \rephine (\news)}{fig:rephine-scr-browse}
%\end{figure}

A sample decoding session is illustrated in 
\figref{fig:rephine-scr-decode}.  The selected phrase of
\subfigref{fig:rephine-scr-browse}{e} is the starting point
for the decoding.  The first of 17 occurrences
of the phrase ``pierr elliott trudeau''
is given in \subfigref{fig:rephine-scr-decode}{a}.  The results
are presented to the user in the order in which they occur in
the original document.  In the first window, the phrase exists
in a longer phrase, which has been marked.  A fixed context
of 10 symbols in either direction of the highlighted symbol
has been selected for \rephine.  In the absence of detailed information
about the
non-words, a space character is inserted between each word
token, with linefeeds added as appropriate.  
Even though the words in this window 
have been stemmed and case folded, reasonable guesses can
be made with most of the words.  In 
\subfigref{fig:rephine-scr-decode}{b}, the user has skipped
ahead to the 15th result.  As this section of text may be
of interest, the case folding is first reversed in 
\subfigref{fig:rephine-scr-decode}{c}, and then the stemming
modifiers are applied in 
\subfigref{fig:rephine-scr-decode}{d}.  The corresponding modifiers are
located by the system and only the small number of required
modifiers are transmitted to the \rephine window.  Modifiers can 
be toggled in any order, and allow the user to progressively
change the displayed context.  Finally, in 
\subfigref{fig:rephine-scr-decode}{e},
the non-words are inserted into the text and the window is 
enlarged for improved display.  The text exists in the 
original \news document as shown.  

\fig{
    \setlength{\tabcolsep}{3pt}
    \begin{tabular}{ccc}
      \includegraphics*[scale=0.6]{./scrshots/decode-0.ps} &
      \includegraphics*[scale=0.6]{./scrshots/decode-1.ps} &
      \includegraphics*[scale=0.6]{./scrshots/decode-2.ps} \\ 
      (a) Result 1 of 17 & (b) Result 15 of 17 & (c) Reverse case folding\\ \\
      \includegraphics*[scale=0.6]{./scrshots/decode-3.ps} &
      \multicolumn{2}{c}{\includegraphics*[scale=0.6]{./scrshots/decode-4.ps}} \\
      (d) Reverse stemming & \multicolumn{2}{c}{(e) Add non-words} \\
    \end{tabular}
}{A sample decoding session of the \news document, 
continuing from \subfigref{fig:rephine-scr-browse}{e}.}
{Sample decoding session of \rephine (\news)}{fig:rephine-scr-decode}

At this stage, the user's information need may be satisfied.  
Alternatively, the 
user may decide to select the phrase ``Progressive 
Conservative'' as the new starting point for phrase browsing.
Then, the starting point for the browsing session becomes
the phrase ``progress conserv''.

\rephine applies the modifiers only when requested by the user.
This approach takes advantage of the fact that the modifier
streams are encoded separately.  At least two other ways of
handling modifiers are possible, which have not been implemented
for \rephine.  If the cost of decoding and transmitting the 
modifiers is acceptable, the user can be given the option to 
reverse all modifiers by clicking one button.  This feature
can be added by making a small change to the \rephine 
interface.  Another option is to display initial word
tokens with default stems and case information applied.  
That is, the most frequent stemming and
case folding modifier is recorded for each word in the lexicon
during pre-processing with
\prepair.  The additional disk space required is $2|\Sigma|$ 
integers, uncompressed,
for a compressed document collection of $|\Sigma|$ primitives.
When phrases are first displayed (for example,
in \subfigref{fig:rephine-scr-decode}{a}), these default modifiers
can be applied immediately without any intervention from the user.
While the phrases presented may occasionally be
linguistically incorrect, they should be valid most of
the time.
These default modifiers can be transmitted as needed, or once
at the beginning of the browsing session.  A similar idea was
adopted by \citet{pwcb00:acmdl}.  
The aim of both of these alternatives is to make the interface
more user-friendly by minimising the amount of exposure users
have to case folded, stemmed words.

\newsection{Evaluating \rephine}{sec:rephine-evaluate}

Information retrieval systems are generally evaluated
in terms of efficiency and effectiveness.  Efficiency is
measured as the amount of time required to return a response
to the user.  Effectiveness is an assessment of the extent to which
the set of results satisfies the user's information need.

Suppose an IR system indexes the words in a 
collection of $n$ documents, and that for some set of queries,
the number of documents that are relevant is known.
The effectiveness of the
IR system for the query is determined by the number of
documents the system returns which are relevant.
Two standard methods of quantifying
effectiveness are {\emph {precision}} and {\emph {recall}}.
These and other methods of judging retrieval effectiveness 
are described by various sources, including \citet{korfhage97:book},
\citet{br99:book}, and \citet{wmb99:book}.

Precision and recall are expressed on a scale from 0 to 1.
Precision is the fraction of returned results that are relevant, 
while recall is the proportion of all relevant
documents that are returned.  If a user poses a query for which
the set of relevant documents in the collection is $r$,
and the set of documents returned by the system is $a$, then
precision and recall are defined as follows:
\neweqn{
\textrm{Precision} = \frac{|r \cap a|}{|a|} \\
\textrm{Recall} = \frac{|r \cap a|}{|r|}
}{eqn:rephine-prec-rec}

Ideally, the effectiveness of \rephine should
be evaluated in a similar manner.  However, significant 
differences between \rephine and traditional IR systems prevent 
any straightforward comparison.

Instead, \rephine is evaluated in three steps.  In the
first step, an overview of past work related to phrase browsing
is summarised.  
Second, \rephine is analysed in detail in the context of related
work and traditional query methods.  Some possible improvements
to \rephine are also proposed.
Finally, the phrase hierarchy used by 
\rephine is compared to noun phrases formed
through natural language processing techniques.

\newsubsection{Related Work}{subsec:rephine-related}

Previous work in the area of phrase browsing can be classified based
on the method employed for drawing phrases from a document.
Phrases can be obtained through
the semantic meaning of the words using natural language processing
techniques, or statistically, by applying a mechanism similar to
\repair.

Most work with natural language processing (NLP) techniques
are related to tagging words according to their parts-of-speech (that is,
noun, verb, etc.)\ and forming phrases according to the rules of
grammar.  A significant amount of work has concentrated on 
{\emph {noun phrases}}.
A noun phrase is a linguistic unit.  In it,
a noun is both required and designated as the most semantically
important word and, as a result, is assigned as
the phrase's head.
\figref{fig:rephine-np} shows an example of a noun
phrase, divided into its various parts.

\fig{
\includegraphics*{./nounphrase.eps}
}{Definition of a noun phrase, adapted from 
\citet[pg.~209]{greenbaum96:book}.}{Definition of a noun phrase}
{fig:rephine-np}

While some retrieval systems treat a query as being
made up of individual words, occasionally, a query should be
seen as a set of one or more noun phrases.  If a user posed
the query ``University of Melbourne'', ideally, documents 
about other universities or the city of Melbourne are not
of interest.  In fact, the seemingly unimportant word ``of''
is crucial for this phrase, since ``University in Melbourne'' has
a completely different meaning.  \citet{kirsch98:sigirforum}
provided some insight into the Internet search engine Infoseek 
and noted that most users submitted queries which were noun phrases
and that queries were, on average, 2.2 words in
length.  

Several researchers have made use of noun phrases for their
phrase-based IR systems.
The \clarit system \citep{ez96:acl} 
identified both simplex and complex noun phrases  
for the purpose of improved
precision and recall in information retrieval, without the need
for training data.  The difference between a simplex noun phrase and
a complex noun phrase is that a simplex noun phrase omits all prepositional
phrases (the postmodifier in \figref{fig:rephine-np}).
\citet{av97:sigir} developed the \paraphrase system
which partitioned documents into clusters and summarised each 
cluster with noun phrases for the purpose of browsing.  A 
variant of the Brill tagger \citep{brill94:aaai} labelled
the words according to their parts-of-speech.  \linkit
by \citet{wek01:jcdl} identified noun phrases, which were subsequently
displayed by the \intellindex interface.  A part-of-speech
tagger was used to determine simplex noun phrases, which were then grouped
according to their heads.  As a result, phrases such as 
``the university bookstore'' was associated with the
phrase ``a bookstore''.

Extraction of keyphrases (document keywords in the form of 
phrases) with machine learning techniques was investigated by 
\citet{fpwgn99:ijcai} and embedded within a system dubbed
\kea.  Several interfaces have been built on top of \kea,
including \kniles, \phrasier, and \keyphind.  \kniles and
\phrasier uses the 
phrases identified by \kea to link documents together via
hyperlinks \citep{jones99:interact, jp99:dl}.  Browsing
phrases identified by \kea are supported by the \keyphind
system \citep{gpwnf99:dss}.  \keyphind removes all but twelve
phrases for each document in a collection and constructs three
types of indexes with the phrases:  
a word-to-phrase index, a phrase-to-document
index, and a document-to-phrase index.  A user study was conducted
which compared the usability of \keyphind to a traditional IR system
based on ranked queries.  The results of the study showed that
\keyphind was preferred for certain browsing tasks, such as collection
evaluation.  A more recent user study \citep{jp01:jcdl}
with \kea and keyphrases looked at computer science papers and
compared the phrases selected by \kea with those specified by
the authors of each paper.  The study's conclusion was that the
phrases selected by \kea were almost as good as ones chosen by
the authors.

Phrase selection through statistical methods is more relevant to
\rephine than through semantic methods.  
The \phind system of \citet{nwp97:acmdl} selected 
phrases using the \sequitur compression mechanism, which is similar
to \repair and was discussed in detail in \chapref{chap:repair}.  Like
word-based \repair described here, \sequitur was modified to 
operate on word tokens
in order to ensure word alignment.  Punctuation marks were removed and
words were case folded to lower case.  Word stems were handled by 
the phrase
browser so that if the word ``book'' was chosen, a search for the word
with different endings was performed.  Follow-up work on a browsing
system built on top of \sequitur was described by \citet{nwp99:ijdl}.
In order to browse large document collections, a disk-based browsing
system is proposed.  The phrase hierarchy is placed on disk with
an inverted index.  Smaller auxiliary data structures are
kept in main memory for accessing the phrase hierarchy.

\citet{wza99:adc} describe an index structure called a {\emph {nextword index}}
which provides explicit support for phrase querying.
This multi-level data structure is an index to pairs of words 
that appear in the document collection.  The word lexicon forms the
top-level of the index.  Each word in the lexicon 
has a nextword list that, together, indicates the word pairs that exist
in the collection.  Every word pair has a position
list as the third and final level in the nextword index.  The position
list specifies the documents and the document positions where
the pairs of words occur.  Subsequent extensions to this work include
compaction techniques for improving storage requirements
\citep{bwz01:spire}, and a description of how nextword indexes can be
used for mono-directional phrase browsing \citep{bwz01:acsc}.  
The nextword index for a transformed version of \wsjb 
occupied 56\% of the collection size.  When a query phrase with
more than two words is processed, it must be partitioned into pairs of
words for look-up into the index.  The position lists obtained
using the word pairs need to be combined in order to find locations
of the original, longer phrase.  
\citeauthor{bwz01:acsc} [\citeyear{bwz01:acsc, bwz02:sigir}] 
showed how the word
pairs should be selected in order to optimise querying time, and then
implemented a retrieval system which combined a
nextword index with a word-level IR system.
It was shown that the addition of the nextword index provided
more efficient retrieval when the first word is a common word.

Several methods combine semantics with statistics for phrase 
selection.  \citet{wolff80:ls} tagged words according to 
their parts-of-speech, and replaced frequently
occurring pairs of word classifications with a recursive pairing
mechanism similar to \repair.  \citet{pwcb00:acmdl} improved on
\phind by combining \sequitur with \kea.  A Brill tagger was combined with 
\kea and applied to the documents in order to identify boundaries
between noun phrases.  Next, \sequitur was applied to combine
words into phrases, with constraints enforced to prevent any crossing
over the boundaries.

A separate, yet related area to retrieval with phrases
is that of locality-based indexing
(or proximity indexing \citep[pg.~101]{br99:book}).  Locality-based
indexing creates an index at the granularity of individual word
tokens instead of documents.  When multiple terms are included in
the query, the results are chosen not only because of query
frequency, but also the distance between query terms within 
the documents.  Some past works with locality-based retrieval
include those of \citet{bsms95:trec}, \citet{ccb95:trec},
\citet{ht95:trec}, and \citet{dm99:sigir}.  

\newsubsection{Analysing \rephine}{subsec:rephine-analyse}

The \rephine system of this chapter can be analysed in
terms of its efficiency, effectiveness, and similarity and
differences with other work.
With respect to the efficiency of \rephine, the time
required by decompression has already been presented.
Experiments in \chapref{chap:remerge}
showed that the total decoding time for \news 
was 236.9 seconds, when \shuff was used as the coder
(\tabref{tab:remerge-pa-time} on 
\pgref{tab:remerge-pa-time}).  For the purpose of \rephine,
less time is required since only the phrase hierarchy and
the two lexicons are fully decoded.  Decoding the
two lexicons required 0.8 seconds (\tabref{tab:remerge-pa-otherstats} on
\pgref{tab:remerge-pa-otherstats}).  The total time required for phrase 
hierarchy decoding, including graph construction by \rephine was
12.3 seconds, 
averaged over three trials on a \vipe.  Experiments
in \chapref{chap:review} showed that 
searching for symbols in the reduced word sequence
encoded with \reviewb is fast, but depends on the number of 
symbols searched for (see Figures \ref{fig:review-ws-freq}
and \ref{fig:review-ws-len} on pages
\pageref{fig:review-ws-freq} and \pageref{fig:review-ws-len}, 
respectively).  The values in these graphs
provide a good estimate of the amount of time required
to search a sequence coded with \reviewb.  Finally, locating 
100 positions in a stream of modifiers with indexed Huffman
coding requires no more than 2.2 elapsed seconds
if a block size of 32,768 bytes was chosen, shown in
\tabref{tab:review-modtime} on \pgref{tab:review-modtime}.

In summary, 13.1 CPU seconds is required by \rephine to decode
the phrase hierarchy and the two lexicons.  Approximately
10 elapsed seconds is necessary for \reviewb to search 
the compressed sequence for 100 different symbols.  The
modifiers at a certain location in a stream can be found in
less than 0.1 seconds.

It is more 
difficult to assess the effectiveness of \rephine, 
or phrase browsing in general.  The main feature
of phrase browsing is the 
user-driven navigation between symbols in the phrase hierarchy.  This
interface experience cannot be easily quantified.
If the searching phase of \rephine is assessed using
precision as the evaluation metric, then problems similar to those encountered with 
Boolean queries need to be addressed, since every result returned contains
the symbol being sought, and precision is 1.0.  

On the other hand, the proportion of relevant results (recall) can 
be measured, but in a slightly different manner than in the traditional
definition.  Unlike with a Boolean query or an exhaustive string search, 
recall is not guaranteed to
be 1.0, because of the way in which the pairing process of \repair breaks up
phrases.  For example, suppose part of a message contains the
five symbols:  $\alpha$$\beta$$\delta$$\beta$$\delta$.
Suppose also that the frequency of $\alpha$$\beta$ throughout the message
is higher than the pair $\beta$$\delta$, and that the replacements
made are \tc{1} $\rightarrow$ $\alpha$$\beta$ followed by
\tc{2} $\rightarrow$ $\beta$$\delta$.  With these replacements, the segment
$\alpha$$\beta$$\delta$$\beta$$\delta$
reduces to \tc{1}$\delta$\tc{2}.  As a consequence, if the symbol \tc{2} is selected
during phrase browsing, then only the second occurrence of it
in this segment is found.  The first occurrence
has been split by the replacement with symbol \tc{1}, and is not reported 
to the user.  

Experiments were conducted with the first \mib{20} of \news (\wsja), compressed
with punctuation-aligned \repair, to evaluate the average recall.  Unlike
results from \tabref{tab:prepair-stats} on \pgref{tab:prepair-stats}, 
phase 1 of \remerge has been applied,
removing 16 duplicates in the phrase hierarchy.  For each symbol in the phrase
hierarchy, the frequency of the symbol and all other higher
generation symbols that contained it were compared with the actual
frequency in the word sequence.  The results from these
experiments are presented in \tabref{tab:rephine-recall}.

\tab{ccccc}
{
Symbol      & Number of symbols & \multicolumn{3}{c}{Distribution of recall values} \\ \cline{3-5}
length      & in phrase hierarchy & Median      & Mean   & Standard   \\
            & ($\Ph$) & recall       & recall & deviation  \\
}
{
%%  39,638 changed artificially to 39,637 for 0-length word, total symbols actually 231,546
%%  Upper range of #3 is 2.500
%%  total symbols 16 less than ch4 due to Remerge phase 1
{\D\D}1     &  \D39,637 & 1.000 & 1.000 & 0.000\\
{\D\D}2     &  \D79,693 & 0.750 & 0.706 & 0.295\\
{\D\D}3     &  \D60,825 & 1.000 & 0.794 & 0.260\\
{\D\D}4     &  \D25,245 & 1.000 & 0.862 & 0.225\\
{\D\D}5     &  \D10,932 & 1.000 & 0.917 & 0.184\\
{\D\D}6     & \D\D4,998 & 1.000 & 0.938 & 0.163\\
{\D\D}7     & \D\D2,895 & 1.000 & 0.954 & 0.141\\
{\D\D}8     & \D\D1,686 & 1.000 & 0.954 & 0.144\\
{\D\D}9     & \D\D1,180 & 1.000 & 0.957 & 0.136\\
10+         & \D\D4,454 & 1.000 & 0.955 & 0.136\\[1.0ex]
Overall     &   231,545 & 0.975 & 0.904 & 0.168 \\
}{Recall of the symbols in the \wsja phrase hierarchy.
Median recall values for each symbol length are in the third column,
followed by the mean, and the standard deviation.
}{Recall of symbols in the phrase hierarchy (\wsja)}{tab:rephine-recall}

In \tabref{tab:rephine-recall}, the 231,545 distinct symbols formed from
\wsja were separated into ten symbol groupings based on their expanded
length, measured in word tokens.  Since the zero-length
word token for separating a long non-word token is of no interest,
it does not appear in the table.  

The second column of the table shows
the size of each grouping.  The majority of symbols were 2 or 3 word tokens
in length.  The remaining three columns indicate some gross statistics
for the distribution of recall values for these symbols:  the median,
the mean, and the standard deviation.
Generally, results
were worse for symbols which represented 2 to 4 word tokens.
For the remaining categories, the average recall was at least 0.9
and their corresponding medians meant that at least half of the 
symbols had a recall of 1.000.  This result is further reinforced
by the standard deviation ranging from 0.000 to 0.184.  The average
recall and standard deviations were less satisfactory for symbols lengths of 2
to 4.  When the symbol length is 2, the median recall was only 
0.750.

The results from the table can be interpreted as follows.
Short phrases are more difficult to accurately find than primitives and
long phrases.  For example, in \wsja, the phrase ``prime 
ministr'', can be easily split depending on the words immediately
before and after it.  Note, however, that the most common phrase
that includes each word is exactly the one most likely to have
a high recall.  And, as phrases get longer, since their 
underlying components are more frequent, \repair does not 
separate them.

One anomaly is not shown in the experiments with \rephine.  Even
though precision across the set of symbols in the phrase hierarchy
is always 1.0, not every symbol can be 
searched.  In particular, while \rephine is able to locate
every symbol chosen through phrase browsing,
many combinations of words do not exist in the phrase hierarchy
and cannot be searched for.
In \subfigref{fig:rephine-scr-browse}{c}, the phrase 
``prime minist'' was extended to the left, and one phrase found
was ``canadian prime minister''.  However, the
phrase ``canadian prime'' does not exist in the phrase hierarchy,
and therefore, cannot be searched for by \rephine.  
The problem is the symbol pairing order employed by \repair 
(and \remerge) makes no attempt to place every multi-symbol combination
into the phrase hierarchy.  This point is expanded further later in this chapter.

Phrases for the dictionary are selected by \repair and \remerge 
based solely on symbol frequency, and in this sense, \rephine is
closely related to \sequitur's phrase browser, \phind.
\citet{nwp97:acmdl} identified the phrase boundary problem
with the phrase hierarchies produced by \sequitur; a problem
which also affects the phrase hierarchies of \repair.  The phrase
boundary problem refers to two symbols which expand to identical
strings, but have differing compositions.
For example,
it is possible that ``canadian prime \textbar\ minist'' and
``canadian \textbar\ prime minist'' become two separate symbols
in the phrase hierarchy, creating no problems for compression,
but confusing the browsing process.  
On the other hand, phase 1 of \remerge explicitly addresses 
this problem when phrase hierarchies are merged.  Recall
that phase 1 expands each symbol into a string of primitives
for comparison.  Symbols that represent duplicate strings are
removed.

Another drawback with \rephine is apparent in
the first screen of \subfigref{fig:rephine-scr-browse}{a},
which presents a list of the word tokens 
in the document collection.  Such an exhaustive list is similar
to a Boolean or ranked query system which lists every
word found in the collection, and is analogous
to a book index containing every word found in the book.
The advantage of \rephine for phrase browsing is not the
word lexicon,
but the relationships formed between words in order
to combine into longer phrases.  Nevertheless, such an opening
window is distracting to users.  Subsequent lists also
suffer from the same problem, but to a lesser degree.
The words in the first window and phrases in the remaining
windows of \figref{fig:rephine-scr-browse} could be
filtered so that unhelpful symbols are removed.  

Case folding and stemming by \prepair reduce the size of the
word lexicon and phrase hierarchy, while also improving browsing 
by providing one word which represents several similar ones.
\chapref{chap:prepair} showed the effect stemming and case folding had 
on the the word lexicon of \wsja 
(\tabref{tab:prepair-lexicon-size} on \pgref{tab:prepair-lexicon-size}).
\tabref{tab:rephine-cfstem} updates those results for
\news.  Recall that the rules for stemming
only apply to words in lower case.  In the third row of
the table, if the part of a word 
in upper case is examined by the stemming algorithm, and
no case folding is applied first, then the word would be
left unchanged.  For example, ``Searching'' stems to ``Search'',
while ``SearchinG'' does not.  The third
column of the table shows the number of primitives
in the phrase hierarchy, while the last column indicates
the number of phrases.  Case folding and stemming has an
effect on the size of the word lexicon, confirming the
results for the smaller \wsja file.  Moreover, 
the average lengths of words in the lexicon 
are slightly longer than the values listed in 
\tabref{tab:prepair-lexicon-size}, as indicated in the fourth
column of the table.

\tab{ccccccccc}
{
Case & \multirow{3}*{Stem} & Unique      & Average & \multicolumn{4}{c}{Percentage of primitive extensions} & Unique \\ \cline{5-8}
fold      &      & primitives  & length  & Left & Right & \multirow{2}*{Both} & \multirow{2}*{None} & phrases \\
          &      &             & (chars) & only & only  &      &      &         \\
}
{
No        & No   & 533,205    & 8.00 & 10.2\% & 11.3\% & 18.5\%  & 60.0\% & 8,464,282 \\
Yes       & No   & 448,278    & 8.10 & \D9.6\%& 11.4\% & 18.8\%  & 60.2\% & 8,269,436 \\
No        & Yes  & 401,448    & 7.34 & 10.3\% & 11.7\% & 18.2\%  & 59.8\% & 8,147,217 \\
Yes       & Yes  & 336,739    & 7.46 & \D9.7\%& 11.8\% & 18.1\%  & 60.4\% & 7,940,620 \\
}
{The effect case folding and stemming has on the phrase 
hierarchy created from punctuation-aligned \repair for
\news with method E of \remerge.  The last row corresponds to
the application of both case folding and stemming of word tokens, 
and reflects the arrangement used for the experiments with 
\news in this chapter.}{The effect 
case folding and stemming has on the phrase hierarchy (\news)}
{tab:rephine-cfstem}

In the fifth, sixth, seventh, and eighth columns of the table, 
the set of primitives has been divided into four
groups based on whether or not each one can be extended. 
In all cases, about 60\% of the symbols do not participate as
a component of a higher generation symbol.  However,
simply eliminating these symbols from the opening 
window of \rephine is too extreme.  If a word token 
fails to become part of a phrase, it does not mean that the 
word is not interesting or is infrequent.

\citet{nwp97:acmdl} noticed a similar problem with the
phrase browser \phind.  The
solution adopted was to filter words and phrases based on
word categories.  Words were labelled as {\emph {rare}}, 
{\emph {interesting}}, or {\emph {common}} based on
a set of frequency thresholds.  When the phrase hierarchy
is explored, an expanded phrase is shown only if it
differs from the previous phrase by at least one interesting word.
For example, if ``book'' is an interesting word, but ``the'' 
is a common word, then the expansion of ``book''  
to ``the book'' would not be displayed.  The reason is that adding a common word to 
the original word does not add any value for the user.  
However, ``book'' can be expanded directly to ``the new book'', 
provided the word ``new'' is also an interesting word.
The drawback of this heuristic is that a common word
may still be of interest to users, and a rare word does not
necessarily represent a typographical error.

At the centre of the problem with \sequitur, 
and \repair as well, is that selecting
phrases based on frequency generates too many phrases,
with many differing by only a common word like ``the''.  
\citet{pwcb00:acmdl} proposed the solution of combining
\sequitur with \kea, since \kea was known to select
too few phrases by itself.  Likewise, the combination of
\repair with a mechanism similar to \kea may improve the number of
phrases presented by \rephine.
A more detailed look at the phrases created by \repair
for \rephine is provided next.

\newsubsection{Comparing \rephine with Noun Phrases}{subsec:rephine-nlp}

Another part of \rephine that is worth evaluating is the usefulness of 
the phrases presented to the user.  The quality of the
phrases is not related to precision or recall because 
the primary concern is the type of phrases available for
browsing, and not the result from searching.

\citet{wolff80:ls} produced phrases through a recursive
pairing mechanism called \mka.  In \mka, pairs are chosen based on
the part-of-speech categories assigned to the word tokens.
The quality of the phrases was compared to the phrases
selected by a human linguist.  In a similar fashion, the
phrases made available by \repair can be compared 
with the ones identified by a reference program employing 
natural language processing (NLP) techniques.

As with other IR systems based on NLP techniques, the basis
for these experiments is that a user is more interested in a
complete noun phrase such as ``prime minister'' than 
a broken phrase like ``Canadian prime''.  The aim of the experiments is to
report on the probability that a symbol in the phrase hierarchy
is a noun phrase, and the probability that noun phrases
exist in the phrase hierarchy.
Noun phrases are identified through an NLP
technique called \linkgram \citep{st91:tech} embodied in a
program by the same name, 
available from \myurl{http://www.link.cs.cmu.edu/link/}.  The
latest version of the software is 4.1, dated 
August 2000.

The \linkgram system processes a sentence at a time.  
In the first step, individual words are assigned to part-of-speech 
categories by consulting a static dictionary.
The categories of novel words are guessed through
a technique the authors call ``morpho-guessing''.  Words have links to
both sides for combining with other words in the sentence.
For example, the two words ``the book'' can merge because of
a right-facing ``D$+$'' link on the word ``the''  
and a corresponding left-facing ``D$-$'' link on the noun
``book''.  The ``D'' in both cases symbolise a determiner
such as ``the'' and ``an''.
An edge is drawn between these two words to indicate
that they are compatible with each other.  When a grammatical
sentence is correctly parsed, a valid {\emph {linkage}} is said
to be formed.  Several requirements 
must be satisfied in order to obtain a linkage.  One of these
requirements is that the graph formed must be connected, 
with all of the words in the sentence reachable.

Subsequent work on the system \citep{gls95:tech} permits 
{\emph {null links}} and post-processing of valid linkages with a
phrase parser for identifying constituents such as noun phrases and
verb phrases.  If a sentence cannot be successfully parsed, then 
words in the sentence are removed incrementally until a successful
parsing can be found.  When the sentence is finally parsed, each
removed word introduces a null link into the linkage.  When a linkage
for a sentence has been established, the post-processing phrase
parser combines words to form constituents of one or more words.
On-line documentation for the phrase parser\footnote{Available
at \myurl{http://www.link.cs.cmu.edu/link/ph-explanation.html}.}
reports that the constituents determined by the parser were
75\% correct on the Penn Treebank 
text\footnote{The Penn Treebank project is at 
\myurl{http://www.cis.upenn.edu/~treebank/}.}, 
which is newspaper text already annotated linguistically.

A sample parsing of a headline from \wsja 
is shown in \figref{fig:rephine-linkgram}.
In the figure the headline is shown in part (a).  The \linkgram
system considers it a valid sentence since two linkages are 
identified, as shown in parts (b) and (c).  In 
\subfigref{fig:rephine-linkgram}{a}, the ``G'' link is used to
pair ``South'' with ``Korea''.  A ``G'' link connects proper
nouns together, identified by the upper case letters at the beginning
of each of these words.  Similarly, a ``G'' link is inserted
between ``Current'' and ``Account''.  The second linkage case folds
``South'' so that the first ``G'' link is converted to an ``A''
link.  An ``A'' link connects a adjective with the noun it follows.
The ``Ss'' link pairs a subject with a verb, while ``Ost'' edges
connect a transitive verb to a singular object.  So, \linkgram
incorrectly treats ``'s'' as a verb in both linkages.
Linkages are scored so that ones with less null links or a lower edge
length total are displayed first.
The constituents for the linkage in (b) is
presented in \subfigref{fig:rephine-linkgram}{d}.  At the beginning
of each constituent is its classification.
The abbreviation ``VP'' denotes a verb phrase, 
while ``NP'' is a noun phrase.

\fig{
\begin{tabular}{c}
{\texttt {South Korea's Current Account}} \\[0.5ex]
(a) News headline \\[1.5ex]
\includegraphics*{./linkage-1.eps} \\[0.5ex]
(b) First valid linkage \\[1.5ex]
\includegraphics*{./linkage-2.eps} \\[0.5ex]
(c) Second valid linkage \\[1.5ex]
\parbox{5cm}{\begin{tabbing}
xxx\=xxx\=xxx\=xxx\= \kill
(S (NP {\texttt {South Korea}}) \\
 \>(VP {\texttt {'s}} (NP {\texttt {Current Account}})))
\end{tabbing}
} \\
(d) Constituents derived from the first linkage \\
\end{tabular}
}{An example parsing of a headline from \wsja using the \linkgram
system.  The headline is shown along the top as part (a).  Two
linkages are determined by \linkgram, which are shown in (b) and (c).
The linkage from (b) is post-processed for constituent identification,
yielding the output in (d).  Noun phrases are marked with ``NP''.}{Example
parsing of a headline with \linkgram (\wsja)}
{fig:rephine-linkgram}

As with previous work in IR and NLP, the focus of the 
phrase comparison is with noun phrases.  
The number of unique phrases that \linkgram found in \wsja was
compared with \rephine's phrase hierarchy.  Since only grammatical
sentences can produce linkages, the test data of news
articles were filtered so that sequences of words that are not 
expected to give many noun phrases were omitted.  First, \sgml tags
were removed and multiple consecutive whitespaces collapsed into
a single space.  Sentences were given to \linkgram one at
a time.  Then, the constituent analysis was determined
by the phrase parser.

As with the earlier experiments to measure recall, the \wsja document was
used as test data.  The computing resources of the \linkgram program were
quite high.  It required just over 5 hours
to process the first \mib{1} of \wsja
on a \vipe test machine.  This time included
the above steps, starting from the \sgml tag filtering, and ending
with the identification of constituents.  Note that
the amount of work performed by \linkgram is more than what is
necessary to identify noun phrases.  Other systems, such as the
\keyphind browser of \citet{gpwnf99:dss}, simply categorise words
according to their parts-of-speech, and then form noun phrases based on a set
of fixed rules.  Even so, \citeauthor{gpwnf99:dss}\ required 4 days to process
\gib{1} of text on a Pentium 233 MHz machine running Linux.
In contrast, \linkgram must first successfully find a linkage for a
sentence, and only when this is done, can it isolate the noun phrases.

After \linkgram completes, constituents which are noun phrases,
and are not composed of other phrases, were
recorded.  The noun phrases were case folded
and then stemmed with the Porter stemming algorithm employed
by \prepair.  In order to comply with the 16 character per word
limitation imposed by \prepair, spaces were inserted into long
words.  Finally, a list of all of the distinct noun phrases in
\wsja was prepared.

\figref{fig:rephine-sgmlsample} illustrates how a news article
in \wsja was parsed, with linefeeds altered to improve display.
Text that was stripped from the article is shown shaded.  All \sgml tags
are removed, as well as text within most tags.  Only text
between opening and closing {\tl}HL{\tg} and {\tl}TEXT{\tg} tags
were retained, since the headline and the body of the article has
the best chance of providing complete sentences.  The noun phrases
found by \linkgram for this example are shown in boxes.  The headline used
as an example in 
\figref{fig:rephine-linkgram} was taken from the article shown in
\figref{fig:rephine-sgmlsample}.  

\fig{
\begin{tabular}{l}
\input{./wsja_sample_parse.sgml}
\end{tabular}
}
{An example article from the \wsja document, with noun
phrases indicated with outlined boxes.  Shaded boxes
represent parts of the article that were removed prior to
applying \linkgram.  The article has been formatted with
linefeeds for improved display.}{Example article
with noun phrases identified (\wsja)}{fig:rephine-sgmlsample}

The \linkgram system successfully found 147,294 grammatical
sentences in \wsja.  Within these sentences, 744,721 noun
phrases were identified, of which 216,661 were unique
after stemming and case folding.  These unique
phrases were divided into groups based on their lengths, in words,
and compared with the phrases available to \rephine, which were
also grouped according to length.  The phrases for \rephine are
identical to the ones reported in \tabref{tab:rephine-recall}.
Information about these
two sets of phrases are shown in \tabref{tab:rephine-nlpph-stats}.
The sizes of the sets of phrases are listed by lengths in the
second and third columns.
Note that noun phrases are only found if a valid linkage has been determined.
In the fourth column of
this table, the number of phrases common to the phrase
hierarchy and in the set of noun phrases has been tabulated.
The probability of a symbol in $\Ph$ being a noun phrase is given in fifth
column.  The reverse, the
probability of randomly selecting a noun phrase which is in
the phrase hierarchy, is listed in the last column of the table.  

\tab{cccccc}
{
Phrase      & Number of        & Number of    & Size of                & Probability & Probability \\
length      & phrases in       & unique noun  & $\Ph \cap {\text{NP}}$ & $\Ph$ has & noun phrase \\
            & hierarchy ($\Ph$)& phrases (NP) &                        & noun phrase & in \Ph \\
}
{
{\D\D}1     & \D39,637  &    13,172 &        13,172 & 0.332 &    1.000 \\
{\D\D}2     & \D79,693  &    78,527 &        21,692 & 0.272 &    0.276 \\
{\D\D}3     & \D60,825  &    72,492 &       \D6,350 & 0.104 &    0.088 \\
{\D\D}4     & \D25,245  &    32,419 &       \D1,578 & 0.063 &    0.049 \\
{\D\D}5     & \D10,932  &    12,304 &       \D\C397 & 0.036 &    0.032 \\
{\D\D}6     & \D\D4,998 &    \D4,939 &      \D\C144 & 0.029 &   0.029  \\
{\D\D}7     & \D\D2,895 &    \D1,696 &     \D\C\D32 & 0.011 &   0.019  \\
{\D\D}8     & \D\D1,686 &    \D\C662 &    \D\C\D\D8 & 0.005 &   0.012  \\
{\D\D}9     & \D\D1,180 &    \D\C247 &    \D\C\D\D2 & 0.002 &   0.008  \\
10+         & \D\D4,454 &    \D\C203 &    \D\C\D\D1 & 0.000 &   0.005  \\[1.0ex]
Total       & 231,545   &    216,661 &       43,376 & 0.187 &   0.200 \\
}{Quantifying the quality of symbols in the phrase hierarchy of
\wsja by calculating the intersection of noun phrases, as identified by 
\linkgram, with the phrase hierarchy.  Symbols have been
divided into 10 groups, based on length (in words).  The second and
third columns provide insight into the number of symbols (or 
noun phrases).  The intersection of these two columns are reported
in the fourth column.  The last two columns indicate the proportion of
phrases in $\Ph$ which are noun phrases (fourth column divided
by the second column), and the probability a noun phrase also being in
$\Ph$ (fourth column divided by the third column), 
respectively.}{Comparing the phrase hierarchy against the
noun phrases (\wsja)}{tab:rephine-nlpph-stats}

As \tabref{tab:rephine-nlpph-stats} shows, the majority of phrases in
the phrase hierarchy and noun phrases from \linkgram are
2 or 3 words in length.  The overlap between the two sets
is around 28\% for phrases of length two for both measures, 
a somewhat disappointing result.  Worse, as the phrases
increase in length, there is even less correlation with the phrase 
hierarchy, as the last two columns show.  This difference is partly due to the methods used by \repair
and \linkgram to build phrases.  The \linkgram system operated on sentences
individually, so phrases were guaranteed to be aligned on sentence boundaries.
Also, \sgml markup did not appear in any of the sentences, because of the
preliminary filtering.  On the other
hand, while punctuation-aligned \repair encourages alignment with punctuation,
phrases can still span sentences, since they are not processed individually.
Furthermore, since \repair was applied on the original version of \wsja,
any phrase that contained markup would not have a corresponding phrase
from \linkgram.  The analysis in \tabref{tab:rephine-nlpph-stats}
is based solely on the existence of phrases, and takes no account of the
frequency.  Because \repair finds common phrases, it is likely 
that the scores would
be considerably better if the probabilities were assessed over the 
sequence instead 
of over the phrase hierarchy.

Making phrases available in the phrase hierarchy is only half of \rephine's
purpose since matching contexts must also be located in
the reduced sequence.  Using the same methodology as established for
\tabref{tab:rephine-recall}, the recall of the phrases that exist
in both the phrase hierarchy and the list of noun phrases is presented
in \tabref{tab:rephine-recall-nlp}.  This time, rather than
presenting recall values for all symbols in the phrase
hierarchy, only the ones that were also noun phrases identified
by \linkgram are considered.  The second column of this
table lists the number of noun phrases that were also 
in the phrase hierarchy, copied from the fourth column of 
\tabref{tab:rephine-nlpph-stats}.  In \tabref{tab:rephine-recall-nlp},
as with the earlier experiment, 
the median recall value for each phrase length group is reported.
The mean and standard deviation of
each set of recall values is also shown.

\tab{ccccc}
{
Phrase      & Size of                & \multicolumn{3}{c}{Distribution of recall values} \\ \cline{3-5}
length      & $\Ph \cap {\text{NP}}$ & Median       & Mean   & Standard   \\
            &                        & recall       & recall & deviation  \\
}
{
{\D\D}1     &   13,172 & 1.000 & 1.000 & 0.000\\
{\D\D}2     &   21,692 & 0.800 & 0.720 & 0.298\\
{\D\D}3     &  \C6,350 & 1.000 & 0.787 &  0.271\\

{\D\D}4     &  \D1,578 & 1.000 & 0.845 & 0.242\\

{\D\D}5     &  \D\C397 & 1.000 & 0.831 &  0.264\\
{\D\D}6     &  \D\C144 & 1.000 & 0.849 & 0.250\\

{\D\D}7     & \D\C\D32 & 1.000 & 0.853 & 0.258\\
{\D\D}8     &\D\C\D\D8 & 1.000 & 1.000 &  0.000\\
{\D\D}9     &\D\C\D\D2 & 1.000 & 1.000 & 0.000\\
10+         &\D\C\D\D1 & 1.000 & 1.000 & 0.000\\[1.0ex]
Overall     &   43,376 & 0.980 & 0.889 &  0.158\\
}{Recall effectiveness of the noun phrases in the phrase hierarchy
for \wsja.  The median has been calculated
based on phrase length (in words)
in the third column, with the mean, and the
standard deviation indicated in the last two columns.}
{Recall of noun phrases in the phrase hierarchy (\wsja)}
{tab:rephine-recall-nlp}

The results from \tabref{tab:rephine-recall-nlp} mirror those
of \tabref{tab:rephine-recall}, but note the small size of the
sample for phrases of length 7 or more.  Again, the average 
recall is above 0.7, and more than half
of the symbols had perfect recall.  The standard deviation 
is wider and affects more groupings than in 
\tabref{tab:rephine-recall}.  

\newsection{Tools for Retrieval}{sec:rephine-summary}

The \rephine system allows users to perform two inter-related
tasks.  The first of these is phrase browsing.  Starting
from a list of the distinct words found in the document, the user can
either extend the current symbol by adding symbols to the left or 
to the right, or decompose
the current symbol into its two constituents.  A graph structure
with one symbol from the phrase hierarchy at each node
was established for the purpose of phrase browsing.  Each node
has six pointers for navigation and defining its structure.

Once the user has found a phrase of interest, the graph
structure is consulted again to determine the complete set of phrases
containing the chosen one.  This set of symbols, $\Criteria$,
is the basis for the search in the reduced word sequence to determine 
all of the locations at which any symbol in $\Criteria$ appears, along
with a suitable context for each.  The second job of \rephine
is to interactively display this set of results one by one to the user.
Each result consists of word tokens which have been stemmed and
case folded.  The non-words that follow each word token are
also initially unavailable.  Each presented result can be refined by the 
user by requesting modifiers from the three modifier streams, in any
combination.  If at this stage, another symbol of interest has been
isolated, it can be passed back to the phrase browsing part of 
\rephine.

In this chapter, \rephine has been demonstrated with a document collection 
\mib{1,000} in size on a \vipe.  Before considering larger collections, note that
only three of the seven streams available to \rephine must reside
in memory.  The four remaining streams are compressed in blocks so that
only parts of them are read from disk into memory.  The phrase hierarchy
and the two lexicons occupy 1.8\% of disk space compared to the size of
\news, and all of them must be accessible by \rephine in decoded form.
The complexity and size of the phrase hierarchy ensures that it requires 
the most memory during browsing after it has been fully decoded.

Based on these observations, several
approaches may be selected in order to browse larger collections.
First, \secref{sec:remerge-append} showed how \remerge can be extended
so that documents are appended to a compressed representation while
keeping the phrase hierarchy static.  Experiments demonstrated this
technique with character-based \repair, and an analysis of the changes
necessary for punctuation-aligned \repair was discussed.  In particular,
a method of handling novel words is required.  Second, either the maximum
generation or the maximum symbol length can be limited so that the
phrase hierarchy only contains moderately long phrases.  Third, while all
word tokens must exist in the phrase hierarchy as primitives, additional
heuristics, such as those offered by \kea and \linkgram, 
may be used when building the phrase hierarchy.  Instead of reducing
the number of phrases shown by \rephine, this approach would prevent
certain symbols from being added to the phrase hierarchy.  Finally, any
combination of these three methods can be used in order for larger document
collections to be compressed by \repair and \remerge, and subsequently
browsed.

The \rephine implementation has not reached
the stage where it could sensibly be made the 
subject of a user study.  As with \sequitur, too many symbols 
exist in \repair's phrase hierarchy, and some filtering is
necessary to improve usability.  The removal of symbols
through NLP techniques has been trialled by other IR systems, and
it is likely that \rephine would be more feasible if similar 
techniques could be employed.  An extension of this idea is to 
rank symbols using some metric so that those that are expected to
be the most useful are presented at the top.  Since there are
reasons why users would want frequent and
and infrequent phrases, simply sorting symbols
by frequency would not help.

User studies have been conducted to quantify the usefulness
of phrase browsing retrieval systems \citep{gpwnf99:dss, jp01:jcdl}, 
with some results being in favour of phrase browsing, and some less
conclusive.  \citet{wm01:sigirforum} concluded that
more work is still required in the area of phrase browsing
evaluation with users.  But even if the suggested changes 
were adopted for \rephine, it is expected that
a comparative user study would not favour \rephine.  While providing
users with a phrase browser was one of the aims of this development, the
necessary tension between browsing and compression inevitably means
that \rephine is not as careful as a dedicated phrase
searching mechanism.  One necessary improvement appears to be
achieving higher recall.
It is possible that the solution lies in
closely regulating word-based \repair's phrase selection 
heuristic.  As word-aligned \repair showed, preventing
certain pairs of symbols from joining does not necessarily
degrade compression effectiveness (see 
\figref{fig:prepair-bargraph-all} on 
\pgref{fig:prepair-bargraph-all}).

The benefits to
compression are easier to quantify than the benefits of phrase
browsing.  Since the latter quantity cannot be measured,
it is also difficult to assess when a reasonable compromise
between the two has been achieved.
Instead, the goal of this thesis has been to 
identify areas where compression and retrieval effectiveness and
efficiency are lost, so that one can be adjusted in favour
of another, as necessary.

\rephine is built on top of a compression mechanism, a structure which
limits its ability to browse and retrieve data.  
Throughout this thesis, the amount of compression effectiveness
lost in favour of faster retrieval times has been highlighted as
a means of compromising between the two competing sets of constraints.  In this
chapter, the current drawbacks with \rephine
represent the amount of
retrieval functionality lost in favour of supporting the
lossless compression of a document collection
with the underlying mechanism.
Employing NLP techniques with \rephine
may be the ideal solution for compensating
for \rephine's loss in retrieval effectiveness.  As mentioned
above, at the same time, NLP techniques may also reduce the
number of symbols that reside in memory during browsing.

Regardless of the improvements made to \rephine, phrase
browsing in general is one of many IR techniques.  This chapter
has also provided a glimpse into other querying techniques, 
including ranked queries,
proximity queries, and Boolean queries.  The system a user 
chooses depends on factors such as how familiar the user is
with the document, whether query terms include phrases, and whether
the order and the positions of the words in a phrase matter.
As \citet{gpwnf99:dss} pointed out, no single IR method 
is perfect and the most viable solution 
might be to provide ``several search tools that can 
make up for each others' shortfalls''.  

This chapter concludes the discussion in this thesis
about the balance between compression and phrase browsing.
Before summarising the thesis in \chapref{chap:summary},
the next chapter discusses a separate topic related to the
compression of small \html files.

}
{The sequence of steps performed by \rephine during phrase
browsing.  Dashed boxes indicate actions 
performed by the system, while solid boxes require
input from the user.  Shaded cylinders represent one of
the seven streams accessible by \rephine, with dashed lines
from them representing file access.  The user can exit
the system at any time.}{Sequence of steps
performed by \rephine during phrase browsing}{fig:rephine-alg}

In the first step of the figure, \rephine decodes the phrase
hierarchy, the word lexicon, and the non-word lexicon, and
maintains all three within memory.  Next, a directed
graph (DG) is built using the symbols of the phrase
hierarchy, to permit efficient searching.  Details
about the structure of the graph is given in the next section.
The third step is interactive phrase browsing.  
In browsing the phrase
hierarchy, the user interactively traverses the DG
until a symbol of interest, $\alpha$ say, has been isolated.

In the fourth step, the symbol of interest is used to identify all
other symbols in the DG which contain $\alpha$.
When that search completes, a set of symbols $\Criteria$ has been 
established.  Next, $\Criteria$ is input to the fifth
step, in which the \reviewb decoder is used to locate every
context in which any symbol in $\Criteria$ occurs.  The \reviewb
decoder returns a result set, $\Result$.  Suppose a symbol
$\beta$ contains the symbol $\alpha$ as a component.  In order
to avoid reporting the same location multiple times, when an
occurrence of $\beta$ has been found in the reduced word sequence, 
only $\beta$ is reported.
The size of $\Result$ is denoted as $|\Result|$, and members
are enumerated from $i=1$ to $i=|\Result|$.

The first result's context, $\Result_1$, is
displayed to the user.  At this point, the user may choose
any combination of five actions.  These actions are:  return to
phrase browsing; toggle case folding modifiers, stemming modifiers, 
or non-word insertion for $\Result_1$; or examine
the next or previous result.  It is assumed that the result
set is a circular list, so that the previous result to $\Result_1$
is $\Result_{|\Result|}$.  The three actions related to the modifiers
initiate seek operations to the corresponding positions in 
their respective compressed
streams.  After the modifiers have been retrieved, $\Result_i$
is re-displayed to the user.  The application of 
modifiers can be toggled in any order.  The first time
modifiers are required, access to the streams are performed.
Assuming requested modifiers are buffered in memory, subsequent 
requests do not need to access the file again.
Finally, if the contexts that were obtained were inadequate in
satisfying the user's need, any part of a context can be used as
input into the phrase browsing part of \rephine.

Later in this chapter, the entire phrase browsing process is detailed 
through a brief example which demonstrates the choices available
to the user.  But first, in the next section, the directed graph
required for phrase browsing is explained.

\newsection{Phrase Hierarchy as a Directed Graph}{sec:rephine-datastruct}

Once the phrase hierarchy has been decoded, a directed
graph (DG) is 
constructed.  Every symbol in the phrase hierarchy is assigned
to a separate node in the DG.  Each node contains six
directed edges connecting it to other nodes, as shown in 
\figref{fig:rephine-graph}.
In this figure, the symbols are based on character-based
\repair instead of punctuation-aligned \repair in order to simplify 
the discussion.  The current node is represented by the
symbol ``w{\textbar}o'', with the vertical bar (\textbar)
separating the symbol into its two components.
Two of the six pointers
are {\emph {child pointers}}, and refer to a symbol's structure
These pointers were described in 
\chapref{chap:remerge} as part of phase 1 of \remerge,
but were called components at that time.
The remaining four pointers are collectively called
{\emph {navigational pointers}}.
Navigational pointers are added for the purpose of
browsing and have not been required in the previous discussion.

\fig{
\includegraphics*{./graph.eps}
}{The six pointers used by \rephine to link nodes representing
symbols in the phrase hierarchy into a directed graph.  The symbols in
the example are based on character-based \repair, for illustrative
purposes only.  The node of interest is in the centre, with the
symbol ``w{\textbar}o''.  A vertical bar (\textbar) separates
the expanded symbols into left and right components.
Names of each pair of pointers are provided to the right of the
figure.
}{The six pointers used by a \rephine graph}
{fig:rephine-graph}

The purpose of each pointer is illustrated in the figure.
Pointers appear in pairs and are designated as being 
``left'' or ``right''.  The four navigational
pointers are in two pairs -- the {\emph {sibling pointers}}, 
and the {\emph {parent pointers}}.  

In the example figure, the current symbol is in
the centre, with its six pointers listed in full.  All of the other
nodes in the graph
have six pointers each as well, but the outgoing edges are 
omitted in the figure for simplicity.  The centre symbol 
is ``w{\textbar}o'', with left and right
components (or children) of ``w'' and ``o'', respectively.
The left sibling symbol of ``w{\textbar}o'' is another symbol which 
also has ``w'' as a left child.  A chain is formed from the 
centre node, with all symbols with ``w'' as a 
left child being linked together with their
left sibling pointers.  Within a chain, the nodes are unsorted,
so any node can serve as the front of the list.
Each chain ends with a final
{\emph {empty node}}.  The child pointers of all primitives
also lead to empty nodes.
Similarly, the right sibling pointer from 
``w{\textbar}o'', leads to another symbol with ``o'' as a right child.  

One important operation is to
obtain the entire chain when the current symbol is any symbol
along that chain.
The child and parent pointers help solve this problem.
Parent pointers directly lead to symbols which are a single 
{\emph {extension}} away from the current symbol.  An extension
occurs when a symbol is juxtaposed
to the current symbol on either side, arriving at another
symbol in the phrase hierarchy.  For example, in 
\figref{fig:rephine-graph}, the right parent pointer from
``w{\textbar}o'' allows a right extension to the symbol 
``wo{\textbar}od''.
\figref{fig:rephine-expandgraph} shows \figref{fig:rephine-graph}
at a larger scale to demonstrate how an entire chain of symbols
can be obtained from anywhere in the list.  The ``w{\textbar}o''
is at the head of the list of right siblings, 
and the current symbol is $\alpha$.
Both of these symbols have the same right component.
Every symbol along that chain
has a right child link to the primitive ``o''.  The front of
the chain can be reached by taking the right child pointer from
$\alpha$, followed by the left parent link of ``o'',
as shown by the dashed lines in the figure.
This is possible provided that
graph construction ensures that every node's parent pointer
leads to the beginning of a chain of parents.  If a symbol
does not have a parent, the parent is represented as
an empty node.

\fig{
\input{./expandgraph.pstex_t}
}{The directed graph structure created by \rephine to link symbols
together, at a larger scale than \figref{fig:rephine-graph}.  The
node of interest is ``w{\textbar}o'' in the centre.  The dashed
arrows shows how an entire chain of symbols can be obtained from
an arbitrary symbol $\alpha$ within the chain.  Also shown are two
siblings of the symbol ``wo{\textbar}od''.
}{The directed graph structure of \rephine, at a larger scale}
{fig:rephine-expandgraph}

The graph structure created by \rephine
resembles the data structure employed by \mka \citep{wolff75:bjp}
for the recursive pairing of symbols for discovering sentence
structure.  Each
symbol in the dictionary created by \mka has major and minor links.
A major forwards link is the same as a left child pointer in
\rephine's graph, while a major backwards link is identical
to a right child pointer.  Minor links operate in a similar
fashion to the sibling pointers of \rephine.

Equipped with this graph, five operations are available to
the user.  Assuming the user's currently selected symbol
is ``w{\textbar}o'' of \figref{fig:rephine-graph} and 
\figref{fig:rephine-expandgraph}, the
five operations are as follows.  First, the user can decompose
the symbol into its two components through the child pointers.
Second, if the right parent pointer from ``w{\textbar}o'' 
is used, followed by all left sibling pointers until the empty
node is reached, then all right extensions of ``w{\textbar}o'' can
be retrieved.  In \figref{fig:rephine-expandgraph}, this operation
would obtain the set 
of two symbols, \{``wo{\textbar}od'', ``wo{\textbar}uld''\}.
Likewise, a third operation is to extend ``w{\textbar}o''
to the left.  The union of these last two sets results in a
fourth function, extending ``w{\textbar}o'' in both directions.  

The fifth and final operation is reserved for decoding, instead
of phrase browsing.  Decoding requires identification of
all symbols which contain the current symbol $\alpha$,
either directly or indirectly.  The
located symbols are denoted in 
\figref{fig:rephine-alg} as $\Criteria$.  In order to find all
symbols in the phrase hierarchy graph which contain $\alpha$, 
a breadth-first search
with the navigational pointers is sufficient.  Let $\beta$
represent a symbol which is an ancestor of $\alpha$.  If $\beta$
has only one occurrence of $\alpha$, 
then all of its parents does as well.  But $\alpha$
must appear in either its left or its right component, which
can be determined based on how $\beta$ was reached.  So, only
one of the sibling pointers need to be followed.  If $\beta$
has multiple occurrences of $\alpha$, with at least one in each
component, then during the breadth-first search, $\beta$ is reached
twice.  Care must be taken to ensure that when $\beta$ is reached
the second time, only the other sibling pointer is traversed, and not the
parent pointers.  If this strategy is followed, then 
the nature of the navigational pointers
ensures that all symbols which include ``w{\textbar}o'' are included.

Construction of the DG requires $O(|\Sigma| + |\rho|)$ time,
for a phrase hierarchy of $|\Sigma|$ primitives and $|\rho|$
phrases.  Since the chiastic slide and interpolative coding have been
used to code the phrase hierarchy, the phrase hierarchy is already
in increasing generation order, and within each generation,
in increasing chiastic slide value order.  As no symbol is in
the same generation as either of its two components, constructing
the graph from the first primitive symbol in sequential order
results in graph construction resembling a bottom-up 
approach.
If the phrase hierarchy is stored in memory as an array, 
and if each of the six graph
pointers are represented as array indexes, then locating a
symbol's components requires $O(1)$ time.  
For each phrase, an insertion
is performed into the appropriate parent pointers of 
each of its two components, so that each node is added
at the front of its two chains of siblings in $O(1)$ time.
That is, a single linear pass through the phrase hierarchy
is sufficient to construct the graph.

The memory space required for the phrase hierarchy is linear in
the number of symbols in the phrase hierarchy.
The graph structure adds an additional 6 words of memory 
per symbol.  Also, the phrase hierarchy may
have every symbol expanded into primitives, or symbol
expansion may be deferred until the symbol is required.  
The choice made depends which is more important:
memory space or response time.  Experiments demonstrating
the advantages and disadvantages of phrase
hierarchy expansion were reported in \tabref{tab:repair-despair}
on \pgref{tab:repair-despair}.  In
the implementation of \rephine described in this chapter,
a buffer of recently decoded symbols is maintained and consulted
before decoding a symbol.  This approach was also employed in
\repair's decoder, \despair.

\newsection{Phrase Browsing Example}{sec:rephine-example}

The \rephine system performs two main tasks:  phrase browsing
and decoding symbol contexts for presentation.  Examples of both tasks
are presented in Figures \ref{fig:rephine-scr-browse} and 
\ref{fig:rephine-scr-decode}.  The 
aspects of \rephine related to the graphical user
interface were 
implemented with version 2.0.2 of the \gtk 
library\footnote{Available from \myurl{http://www.gtk.org/}.}.

The data for the example
is \news, processed with punctuation-aligned
\repair exactly as was described in the experiments of
\chapref{chap:remerge}.  The sequence of word tokens in
those experiments were partitioned into blocks of 20,971,520
symbols each, and then merged using method E of \remerge 
(\secref{sec:remerge-expt}).
Modifier and sequence coding were done with indexed \shuff
and \reviewb respectively.

In the previous section, the choices available to the user
for phrase browsing from the current symbol were discussed.
The user also needs to be given a starting point to browse
from.  In \rephine, the user is first presented with the
document's entire lexicon, as shown in
\subfigref{fig:rephine-scr-browse}{a}.  The extensive list
can be daunting, and one solution
is to filter words, discussed later
in the chapter.  In \subfigref{fig:rephine-scr-browse}{a}, the user
has selected the word ``prime''.  
The first symbol is extended to
the right in \subfigref{fig:rephine-scr-browse}{b}, and the
phrase ``prime minist'' has been selected.  While phrases in this 
window are ordered according to their locations in the DG, 
\citet{jp99:dl} and \citet{pwcb00:acmdl} describe systems %% Second cite due to witten00:cpm
where phrases can be ordered according to the number of documents
in which they appear, or their frequency in the collection.
While details about the implementation of these two features are
not available, it is expected that either the additional information
is stored by the system, or calculated at run-time.  Also note
the variations on the phrase ``prime minist'' due to the different
spellings of the second word.  While stemming and case folding
have reduced the size of the phrase hierarchy, it is still 
possible that two seemingly equivalent words stem differently.
The chosen phrase is extended to the left and the phrase 
``canadian prime minist'' has been highlighted in 
\subfigref{fig:rephine-scr-browse}{c}.  As this window shows,
unfortunately, the words ``Canadian'' and ``Canada's'' stem
to different words.  The identified phrase of 
\subfigref{fig:rephine-scr-browse}{c} is extended in both
directions, resulting in \subfigref{fig:rephine-scr-browse}{d}
and its 7 phrases.  Of these 7 phrases, 3 are right extensions
and 4 are left extensions to the phrase ``canadian prime minist''.
The selected phrase is broken down into its two components through
its child pointers, as shown in 
\subfigref{fig:rephine-scr-browse}{e}.  This phrase browsing
example commenced with the word ``prime'' and, in four steps,
arrived at the name of a Canadian prime minister.  In \rephine,
the refinement of a query term is done interactively with the
help of the phrase hierarchy's graph.

%%  A double-page figure
%%  Snapshot created using the command:
%%  import -frame -window "Main browsing session" ~/thesis/t/ch07/scrshots/snapshot1.eps
%%  Then, export to PS with xv (greyscale).
\fig{
    \setlength{\tabcolsep}{3pt}
    \begin{tabular}{ccc}
      \multicolumn{2}{c}{\includegraphics*[scale=0.6]{./scrshots/browse-0.ps}} &
      \includegraphics*[scale=0.6]{./scrshots/browse-1.ps} \\ 
      \multicolumn{2}{c}{(a) Initial window} & (b) Extend right \\ \\
      \includegraphics*[scale=0.6]{./scrshots/browse-2.ps} &
      \includegraphics*[scale=0.6]{./scrshots/browse-3.ps} &
      \includegraphics*[scale=0.6]{./scrshots/browse-4.ps} \\
      (c) Extend left & (d) Extend both ways & (e) Decompose \\
    \end{tabular}
%  }
%  \end{leftfullpage}
}{A sample phrase browsing session of the \news document.}
{Sample phrase browsing session of \rephine (\news)}{fig:rephine-scr-browse}
%\end{figure}

A sample decoding session is illustrated in 
\figref{fig:rephine-scr-decode}.  The selected phrase of
\subfigref{fig:rephine-scr-browse}{e} is the starting point
for the decoding.  The first of 17 occurrences
of the phrase ``pierr elliott trudeau''
is given in \subfigref{fig:rephine-scr-decode}{a}.  The results
are presented to the user in the order in which they occur in
the original document.  In the first window, the phrase exists
in a longer phrase, which has been marked.  A fixed context
of 10 symbols in either direction of the highlighted symbol
has been selected for \rephine.  In the absence of detailed information
about the
non-words, a space character is inserted between each word
token, with linefeeds added as appropriate.  
Even though the words in this window 
have been stemmed and case folded, reasonable guesses can
be made with most of the words.  In 
\subfigref{fig:rephine-scr-decode}{b}, the user has skipped
ahead to the 15th result.  As this section of text may be
of interest, the case folding is first reversed in 
\subfigref{fig:rephine-scr-decode}{c}, and then the stemming
modifiers are applied in 
\subfigref{fig:rephine-scr-decode}{d}.  The corresponding modifiers are
located by the system and only the small number of required
modifiers are transmitted to the \rephine window.  Modifiers can 
be toggled in any order, and allow the user to progressively
change the displayed context.  Finally, in 
\subfigref{fig:rephine-scr-decode}{e},
the non-words are inserted into the text and the window is 
enlarged for improved display.  The text exists in the 
original \news document as shown.  

\fig{
    \setlength{\tabcolsep}{3pt}
    \begin{tabular}{ccc}
      \includegraphics*[scale=0.6]{./scrshots/decode-0.ps} &
      \includegraphics*[scale=0.6]{./scrshots/decode-1.ps} &
      \includegraphics*[scale=0.6]{./scrshots/decode-2.ps} \\ 
      (a) Result 1 of 17 & (b) Result 15 of 17 & (c) Reverse case folding\\ \\
      \includegraphics*[scale=0.6]{./scrshots/decode-3.ps} &
      \multicolumn{2}{c}{\includegraphics*[scale=0.6]{./scrshots/decode-4.ps}} \\
      (d) Reverse stemming & \multicolumn{2}{c}{(e) Add non-words} \\
    \end{tabular}
}{A sample decoding session of the \news document, 
continuing from \subfigref{fig:rephine-scr-browse}{e}.}
{Sample decoding session of \rephine (\news)}{fig:rephine-scr-decode}

At this stage, the user's information need may be satisfied.  
Alternatively, the 
user may decide to select the phrase ``Progressive 
Conservative'' as the new starting point for phrase browsing.
Then, the starting point for the browsing session becomes
the phrase ``progress conserv''.

\rephine applies the modifiers only when requested by the user.
This approach takes advantage of the fact that the modifier
streams are encoded separately.  At least two other ways of
handling modifiers are possible, which have not been implemented
for \rephine.  If the cost of decoding and transmitting the 
modifiers is acceptable, the user can be given the option to 
reverse all modifiers by clicking one button.  This feature
can be added by making a small change to the \rephine 
interface.  Another option is to display initial word
tokens with default stems and case information applied.  
That is, the most frequent stemming and
case folding modifier is recorded for each word in the lexicon
during pre-processing with
\prepair.  The additional disk space required is $2|\Sigma|$ 
integers, uncompressed,
for a compressed document collection of $|\Sigma|$ primitives.
When phrases are first displayed (for example,
in \subfigref{fig:rephine-scr-decode}{a}), these default modifiers
can be applied immediately without any intervention from the user.
While the phrases presented may occasionally be
linguistically incorrect, they should be valid most of
the time.
These default modifiers can be transmitted as needed, or once
at the beginning of the browsing session.  A similar idea was
adopted by \citet{pwcb00:acmdl}.  
The aim of both of these alternatives is to make the interface
more user-friendly by minimising the amount of exposure users
have to case folded, stemmed words.

\newsection{Evaluating \rephine}{sec:rephine-evaluate}

Information retrieval systems are generally evaluated
in terms of efficiency and effectiveness.  Efficiency is
measured as the amount of time required to return a response
to the user.  Effectiveness is an assessment of the extent to which
the set of results satisfies the user's information need.

Suppose an IR system indexes the words in a 
collection of $n$ documents, and that for some set of queries,
the number of documents that are relevant is known.
The effectiveness of the
IR system for the query is determined by the number of
documents the system returns which are relevant.
Two standard methods of quantifying
effectiveness are {\emph {precision}} and {\emph {recall}}.
These and other methods of judging retrieval effectiveness 
are described by various sources, including \citet{korfhage97:book},
\citet{br99:book}, and \citet{wmb99:book}.

Precision and recall are expressed on a scale from 0 to 1.
Precision is the fraction of returned results that are relevant, 
while recall is the proportion of all relevant
documents that are returned.  If a user poses a query for which
the set of relevant documents in the collection is $r$,
and the set of documents returned by the system is $a$, then
precision and recall are defined as follows:
\neweqn{
\textrm{Precision} = \frac{|r \cap a|}{|a|} \\
\textrm{Recall} = \frac{|r \cap a|}{|r|}
}{eqn:rephine-prec-rec}

Ideally, the effectiveness of \rephine should
be evaluated in a similar manner.  However, significant 
differences between \rephine and traditional IR systems prevent 
any straightforward comparison.

Instead, \rephine is evaluated in three steps.  In the
first step, an overview of past work related to phrase browsing
is summarised.  
Second, \rephine is analysed in detail in the context of related
work and traditional query methods.  Some possible improvements
to \rephine are also proposed.
Finally, the phrase hierarchy used by 
\rephine is compared to noun phrases formed
through natural language processing techniques.

\newsubsection{Related Work}{subsec:rephine-related}

Previous work in the area of phrase browsing can be classified based
on the method employed for drawing phrases from a document.
Phrases can be obtained through
the semantic meaning of the words using natural language processing
techniques, or statistically, by applying a mechanism similar to
\repair.

Most work with natural language processing (NLP) techniques
are related to tagging words according to their parts-of-speech (that is,
noun, verb, etc.)\ and forming phrases according to the rules of
grammar.  A significant amount of work has concentrated on 
{\emph {noun phrases}}.
A noun phrase is a linguistic unit.  In it,
a noun is both required and designated as the most semantically
important word and, as a result, is assigned as
the phrase's head.
\figref{fig:rephine-np} shows an example of a noun
phrase, divided into its various parts.

\fig{
\includegraphics*{./nounphrase.eps}
}{Definition of a noun phrase, adapted from 
\citet[pg.~209]{greenbaum96:book}.}{Definition of a noun phrase}
{fig:rephine-np}

While some retrieval systems treat a query as being
made up of individual words, occasionally, a query should be
seen as a set of one or more noun phrases.  If a user posed
the query ``University of Melbourne'', ideally, documents 
about other universities or the city of Melbourne are not
of interest.  In fact, the seemingly unimportant word ``of''
is crucial for this phrase, since ``University in Melbourne'' has
a completely different meaning.  \citet{kirsch98:sigirforum}
provided some insight into the Internet search engine Infoseek 
and noted that most users submitted queries which were noun phrases
and that queries were, on average, 2.2 words in
length.  

Several researchers have made use of noun phrases for their
phrase-based IR systems.
The \clarit system \citep{ez96:acl} 
identified both simplex and complex noun phrases  
for the purpose of improved
precision and recall in information retrieval, without the need
for training data.  The difference between a simplex noun phrase and
a complex noun phrase is that a simplex noun phrase omits all prepositional
phrases (the postmodifier in \figref{fig:rephine-np}).
\citet{av97:sigir} developed the \paraphrase system
which partitioned documents into clusters and summarised each 
cluster with noun phrases for the purpose of browsing.  A 
variant of the Brill tagger \citep{brill94:aaai} labelled
the words according to their parts-of-speech.  \linkit
by \citet{wek01:jcdl} identified noun phrases, which were subsequently
displayed by the \intellindex interface.  A part-of-speech
tagger was used to determine simplex noun phrases, which were then grouped
according to their heads.  As a result, phrases such as 
``the university bookstore'' was associated with the
phrase ``a bookstore''.

Extraction of keyphrases (document keywords in the form of 
phrases) with machine learning techniques was investigated by 
\citet{fpwgn99:ijcai} and embedded within a system dubbed
\kea.  Several interfaces have been built on top of \kea,
including \kniles, \phrasier, and \keyphind.  \kniles and
\phrasier uses the 
phrases identified by \kea to link documents together via
hyperlinks \citep{jones99:interact, jp99:dl}.  Browsing
phrases identified by \kea are supported by the \keyphind
system \citep{gpwnf99:dss}.  \keyphind removes all but twelve
phrases for each document in a collection and constructs three
types of indexes with the phrases:  
a word-to-phrase index, a phrase-to-document
index, and a document-to-phrase index.  A user study was conducted
which compared the usability of \keyphind to a traditional IR system
based on ranked queries.  The results of the study showed that
\keyphind was preferred for certain browsing tasks, such as collection
evaluation.  A more recent user study \citep{jp01:jcdl}
with \kea and keyphrases looked at computer science papers and
compared the phrases selected by \kea with those specified by
the authors of each paper.  The study's conclusion was that the
phrases selected by \kea were almost as good as ones chosen by
the authors.

Phrase selection through statistical methods is more relevant to
\rephine than through semantic methods.  
The \phind system of \citet{nwp97:acmdl} selected 
phrases using the \sequitur compression mechanism, which is similar
to \repair and was discussed in detail in \chapref{chap:repair}.  Like
word-based \repair described here, \sequitur was modified to 
operate on word tokens
in order to ensure word alignment.  Punctuation marks were removed and
words were case folded to lower case.  Word stems were handled by 
the phrase
browser so that if the word ``book'' was chosen, a search for the word
with different endings was performed.  Follow-up work on a browsing
system built on top of \sequitur was described by \citet{nwp99:ijdl}.
In order to browse large document collections, a disk-based browsing
system is proposed.  The phrase hierarchy is placed on disk with
an inverted index.  Smaller auxiliary data structures are
kept in main memory for accessing the phrase hierarchy.

\citet{wza99:adc} describe an index structure called a {\emph {nextword index}}
which provides explicit support for phrase querying.
This multi-level data structure is an index to pairs of words 
that appear in the document collection.  The word lexicon forms the
top-level of the index.  Each word in the lexicon 
has a nextword list that, together, indicates the word pairs that exist
in the collection.  Every word pair has a position
list as the third and final level in the nextword index.  The position
list specifies the documents and the document positions where
the pairs of words occur.  Subsequent extensions to this work include
compaction techniques for improving storage requirements
\citep{bwz01:spire}, and a description of how nextword indexes can be
used for mono-directional phrase browsing \citep{bwz01:acsc}.  
The nextword index for a transformed version of \wsjb 
occupied 56\% of the collection size.  When a query phrase with
more than two words is processed, it must be partitioned into pairs of
words for look-up into the index.  The position lists obtained
using the word pairs need to be combined in order to find locations
of the original, longer phrase.  
\citeauthor{bwz01:acsc} [\citeyear{bwz01:acsc, bwz02:sigir}] 
showed how the word
pairs should be selected in order to optimise querying time, and then
implemented a retrieval system which combined a
nextword index with a word-level IR system.
It was shown that the addition of the nextword index provided
more efficient retrieval when the first word is a common word.

Several methods combine semantics with statistics for phrase 
selection.  \citet{wolff80:ls} tagged words according to 
their parts-of-speech, and replaced frequently
occurring pairs of word classifications with a recursive pairing
mechanism similar to \repair.  \citet{pwcb00:acmdl} improved on
\phind by combining \sequitur with \kea.  A Brill tagger was combined with 
\kea and applied to the documents in order to identify boundaries
between noun phrases.  Next, \sequitur was applied to combine
words into phrases, with constraints enforced to prevent any crossing
over the boundaries.

A separate, yet related area to retrieval with phrases
is that of locality-based indexing
(or proximity indexing \citep[pg.~101]{br99:book}).  Locality-based
indexing creates an index at the granularity of individual word
tokens instead of documents.  When multiple terms are included in
the query, the results are chosen not only because of query
frequency, but also the distance between query terms within 
the documents.  Some past works with locality-based retrieval
include those of \citet{bsms95:trec}, \citet{ccb95:trec},
\citet{ht95:trec}, and \citet{dm99:sigir}.  

\newsubsection{Analysing \rephine}{subsec:rephine-analyse}

The \rephine system of this chapter can be analysed in
terms of its efficiency, effectiveness, and similarity and
differences with other work.
With respect to the efficiency of \rephine, the time
required by decompression has already been presented.
Experiments in \chapref{chap:remerge}
showed that the total decoding time for \news 
was 236.9 seconds, when \shuff was used as the coder
(\tabref{tab:remerge-pa-time} on 
\pgref{tab:remerge-pa-time}).  For the purpose of \rephine,
less time is required since only the phrase hierarchy and
the two lexicons are fully decoded.  Decoding the
two lexicons required 0.8 seconds (\tabref{tab:remerge-pa-otherstats} on
\pgref{tab:remerge-pa-otherstats}).  The total time required for phrase 
hierarchy decoding, including graph construction by \rephine was
12.3 seconds, 
averaged over three trials on a \vipe.  Experiments
in \chapref{chap:review} showed that 
searching for symbols in the reduced word sequence
encoded with \reviewb is fast, but depends on the number of 
symbols searched for (see Figures \ref{fig:review-ws-freq}
and \ref{fig:review-ws-len} on pages
\pageref{fig:review-ws-freq} and \pageref{fig:review-ws-len}, 
respectively).  The values in these graphs
provide a good estimate of the amount of time required
to search a sequence coded with \reviewb.  Finally, locating 
100 positions in a stream of modifiers with indexed Huffman
coding requires no more than 2.2 elapsed seconds
if a block size of 32,768 bytes was chosen, shown in
\tabref{tab:review-modtime} on \pgref{tab:review-modtime}.

In summary, 13.1 CPU seconds is required by \rephine to decode
the phrase hierarchy and the two lexicons.  Approximately
10 elapsed seconds is necessary for \reviewb to search 
the compressed sequence for 100 different symbols.  The
modifiers at a certain location in a stream can be found in
less than 0.1 seconds.

It is more 
difficult to assess the effectiveness of \rephine, 
or phrase browsing in general.  The main feature
of phrase browsing is the 
user-driven navigation between symbols in the phrase hierarchy.  This
interface experience cannot be easily quantified.
If the searching phase of \rephine is assessed using
precision as the evaluation metric, then problems similar to those encountered with 
Boolean queries need to be addressed, since every result returned contains
the symbol being sought, and precision is 1.0.  

On the other hand, the proportion of relevant results (recall) can 
be measured, but in a slightly different manner than in the traditional
definition.  Unlike with a Boolean query or an exhaustive string search, 
recall is not guaranteed to
be 1.0, because of the way in which the pairing process of \repair breaks up
phrases.  For example, suppose part of a message contains the
five symbols:  $\alpha$$\beta$$\delta$$\beta$$\delta$.
Suppose also that the frequency of $\alpha$$\beta$ throughout the message
is higher than the pair $\beta$$\delta$, and that the replacements
made are \tc{1} $\rightarrow$ $\alpha$$\beta$ followed by
\tc{2} $\rightarrow$ $\beta$$\delta$.  With these replacements, the segment
$\alpha$$\beta$$\delta$$\beta$$\delta$
reduces to \tc{1}$\delta$\tc{2}.  As a consequence, if the symbol \tc{2} is selected
during phrase browsing, then only the second occurrence of it
in this segment is found.  The first occurrence
has been split by the replacement with symbol \tc{1}, and is not reported 
to the user.  

Experiments were conducted with the first \mib{20} of \news (\wsja), compressed
with punctuation-aligned \repair, to evaluate the average recall.  Unlike
results from \tabref{tab:prepair-stats} on \pgref{tab:prepair-stats}, 
phase 1 of \remerge has been applied,
removing 16 duplicates in the phrase hierarchy.  For each symbol in the phrase
hierarchy, the frequency of the symbol and all other higher
generation symbols that contained it were compared with the actual
frequency in the word sequence.  The results from these
experiments are presented in \tabref{tab:rephine-recall}.

\tab{ccccc}
{
Symbol      & Number of symbols & \multicolumn{3}{c}{Distribution of recall values} \\ \cline{3-5}
length      & in phrase hierarchy & Median      & Mean   & Standard   \\
            & ($\Ph$) & recall       & recall & deviation  \\
}
{
%%  39,638 changed artificially to 39,637 for 0-length word, total symbols actually 231,546
%%  Upper range of #3 is 2.500
%%  total symbols 16 less than ch4 due to Remerge phase 1
{\D\D}1     &  \D39,637 & 1.000 & 1.000 & 0.000\\
{\D\D}2     &  \D79,693 & 0.750 & 0.706 & 0.295\\
{\D\D}3     &  \D60,825 & 1.000 & 0.794 & 0.260\\
{\D\D}4     &  \D25,245 & 1.000 & 0.862 & 0.225\\
{\D\D}5     &  \D10,932 & 1.000 & 0.917 & 0.184\\
{\D\D}6     & \D\D4,998 & 1.000 & 0.938 & 0.163\\
{\D\D}7     & \D\D2,895 & 1.000 & 0.954 & 0.141\\
{\D\D}8     & \D\D1,686 & 1.000 & 0.954 & 0.144\\
{\D\D}9     & \D\D1,180 & 1.000 & 0.957 & 0.136\\
10+         & \D\D4,454 & 1.000 & 0.955 & 0.136\\[1.0ex]
Overall     &   231,545 & 0.975 & 0.904 & 0.168 \\
}{Recall of the symbols in the \wsja phrase hierarchy.
Median recall values for each symbol length are in the third column,
followed by the mean, and the standard deviation.
}{Recall of symbols in the phrase hierarchy (\wsja)}{tab:rephine-recall}

In \tabref{tab:rephine-recall}, the 231,545 distinct symbols formed from
\wsja were separated into ten symbol groupings based on their expanded
length, measured in word tokens.  Since the zero-length
word token for separating a long non-word token is of no interest,
it does not appear in the table.  

The second column of the table shows
the size of each grouping.  The majority of symbols were 2 or 3 word tokens
in length.  The remaining three columns indicate some gross statistics
for the distribution of recall values for these symbols:  the median,
the mean, and the standard deviation.
Generally, results
were worse for symbols which represented 2 to 4 word tokens.
For the remaining categories, the average recall was at least 0.9
and their corresponding medians meant that at least half of the 
symbols had a recall of 1.000.  This result is further reinforced
by the standard deviation ranging from 0.000 to 0.184.  The average
recall and standard deviations were less satisfactory for symbols lengths of 2
to 4.  When the symbol length is 2, the median recall was only 
0.750.

The results from the table can be interpreted as follows.
Short phrases are more difficult to accurately find than primitives and
long phrases.  For example, in \wsja, the phrase ``prime 
ministr'', can be easily split depending on the words immediately
before and after it.  Note, however, that the most common phrase
that includes each word is exactly the one most likely to have
a high recall.  And, as phrases get longer, since their 
underlying components are more frequent, \repair does not 
separate them.

One anomaly is not shown in the experiments with \rephine.  Even
though precision across the set of symbols in the phrase hierarchy
is always 1.0, not every symbol can be 
searched.  In particular, while \rephine is able to locate
every symbol chosen through phrase browsing,
many combinations of words do not exist in the phrase hierarchy
and cannot be searched for.
In \subfigref{fig:rephine-scr-browse}{c}, the phrase 
``prime minist'' was extended to the left, and one phrase found
was ``canadian prime minister''.  However, the
phrase ``canadian prime'' does not exist in the phrase hierarchy,
and therefore, cannot be searched for by \rephine.  
The problem is the symbol pairing order employed by \repair 
(and \remerge) makes no attempt to place every multi-symbol combination
into the phrase hierarchy.  This point is expanded further later in this chapter.

Phrases for the dictionary are selected by \repair and \remerge 
based solely on symbol frequency, and in this sense, \rephine is
closely related to \sequitur's phrase browser, \phind.
\citet{nwp97:acmdl} identified the phrase boundary problem
with the phrase hierarchies produced by \sequitur; a problem
which also affects the phrase hierarchies of \repair.  The phrase
boundary problem refers to two symbols which expand to identical
strings, but have differing compositions.
For example,
it is possible that ``canadian prime \textbar\ minist'' and
``canadian \textbar\ prime minist'' become two separate symbols
in the phrase hierarchy, creating no problems for compression,
but confusing the browsing process.  
On the other hand, phase 1 of \remerge explicitly addresses 
this problem when phrase hierarchies are merged.  Recall
that phase 1 expands each symbol into a string of primitives
for comparison.  Symbols that represent duplicate strings are
removed.

Another drawback with \rephine is apparent in
the first screen of \subfigref{fig:rephine-scr-browse}{a},
which presents a list of the word tokens 
in the document collection.  Such an exhaustive list is similar
to a Boolean or ranked query system which lists every
word found in the collection, and is analogous
to a book index containing every word found in the book.
The advantage of \rephine for phrase browsing is not the
word lexicon,
but the relationships formed between words in order
to combine into longer phrases.  Nevertheless, such an opening
window is distracting to users.  Subsequent lists also
suffer from the same problem, but to a lesser degree.
The words in the first window and phrases in the remaining
windows of \figref{fig:rephine-scr-browse} could be
filtered so that unhelpful symbols are removed.  

Case folding and stemming by \prepair reduce the size of the
word lexicon and phrase hierarchy, while also improving browsing 
by providing one word which represents several similar ones.
\chapref{chap:prepair} showed the effect stemming and case folding had 
on the the word lexicon of \wsja 
(\tabref{tab:prepair-lexicon-size} on \pgref{tab:prepair-lexicon-size}).
\tabref{tab:rephine-cfstem} updates those results for
\news.  Recall that the rules for stemming
only apply to words in lower case.  In the third row of
the table, if the part of a word 
in upper case is examined by the stemming algorithm, and
no case folding is applied first, then the word would be
left unchanged.  For example, ``Searching'' stems to ``Search'',
while ``SearchinG'' does not.  The third
column of the table shows the number of primitives
in the phrase hierarchy, while the last column indicates
the number of phrases.  Case folding and stemming has an
effect on the size of the word lexicon, confirming the
results for the smaller \wsja file.  Moreover, 
the average lengths of words in the lexicon 
are slightly longer than the values listed in 
\tabref{tab:prepair-lexicon-size}, as indicated in the fourth
column of the table.

\tab{ccccccccc}
{
Case & \multirow{3}*{Stem} & Unique      & Average & \multicolumn{4}{c}{Percentage of primitive extensions} & Unique \\ \cline{5-8}
fold      &      & primitives  & length  & Left & Right & \multirow{2}*{Both} & \multirow{2}*{None} & phrases \\
          &      &             & (chars) & only & only  &      &      &         \\
}
{
No        & No   & 533,205    & 8.00 & 10.2\% & 11.3\% & 18.5\%  & 60.0\% & 8,464,282 \\
Yes       & No   & 448,278    & 8.10 & \D9.6\%& 11.4\% & 18.8\%  & 60.2\% & 8,269,436 \\
No        & Yes  & 401,448    & 7.34 & 10.3\% & 11.7\% & 18.2\%  & 59.8\% & 8,147,217 \\
Yes       & Yes  & 336,739    & 7.46 & \D9.7\%& 11.8\% & 18.1\%  & 60.4\% & 7,940,620 \\
}
{The effect case folding and stemming has on the phrase 
hierarchy created from punctuation-aligned \repair for
\news with method E of \remerge.  The last row corresponds to
the application of both case folding and stemming of word tokens, 
and reflects the arrangement used for the experiments with 
\news in this chapter.}{The effect 
case folding and stemming has on the phrase hierarchy (\news)}
{tab:rephine-cfstem}

In the fifth, sixth, seventh, and eighth columns of the table, 
the set of primitives has been divided into four
groups based on whether or not each one can be extended. 
In all cases, about 60\% of the symbols do not participate as
a component of a higher generation symbol.  However,
simply eliminating these symbols from the opening 
window of \rephine is too extreme.  If a word token 
fails to become part of a phrase, it does not mean that the 
word is not interesting or is infrequent.

\citet{nwp97:acmdl} noticed a similar problem with the
phrase browser \phind.  The
solution adopted was to filter words and phrases based on
word categories.  Words were labelled as {\emph {rare}}, 
{\emph {interesting}}, or {\emph {common}} based on
a set of frequency thresholds.  When the phrase hierarchy
is explored, an expanded phrase is shown only if it
differs from the previous phrase by at least one interesting word.
For example, if ``book'' is an interesting word, but ``the'' 
is a common word, then the expansion of ``book''  
to ``the book'' would not be displayed.  The reason is that adding a common word to 
the original word does not add any value for the user.  
However, ``book'' can be expanded directly to ``the new book'', 
provided the word ``new'' is also an interesting word.
The drawback of this heuristic is that a common word
may still be of interest to users, and a rare word does not
necessarily represent a typographical error.

At the centre of the problem with \sequitur, 
and \repair as well, is that selecting
phrases based on frequency generates too many phrases,
with many differing by only a common word like ``the''.  
\citet{pwcb00:acmdl} proposed the solution of combining
\sequitur with \kea, since \kea was known to select
too few phrases by itself.  Likewise, the combination of
\repair with a mechanism similar to \kea may improve the number of
phrases presented by \rephine.
A more detailed look at the phrases created by \repair
for \rephine is provided next.

\newsubsection{Comparing \rephine with Noun Phrases}{subsec:rephine-nlp}

Another part of \rephine that is worth evaluating is the usefulness of 
the phrases presented to the user.  The quality of the
phrases is not related to precision or recall because 
the primary concern is the type of phrases available for
browsing, and not the result from searching.

\citet{wolff80:ls} produced phrases through a recursive
pairing mechanism called \mka.  In \mka, pairs are chosen based on
the part-of-speech categories assigned to the word tokens.
The quality of the phrases was compared to the phrases
selected by a human linguist.  In a similar fashion, the
phrases made available by \repair can be compared 
with the ones identified by a reference program employing 
natural language processing (NLP) techniques.

As with other IR systems based on NLP techniques, the basis
for these experiments is that a user is more interested in a
complete noun phrase such as ``prime minister'' than 
a broken phrase like ``Canadian prime''.  The aim of the experiments is to
report on the probability that a symbol in the phrase hierarchy
is a noun phrase, and the probability that noun phrases
exist in the phrase hierarchy.
Noun phrases are identified through an NLP
technique called \linkgram \citep{st91:tech} embodied in a
program by the same name, 
available from \myurl{http://www.link.cs.cmu.edu/link/}.  The
latest version of the software is 4.1, dated 
August 2000.

The \linkgram system processes a sentence at a time.  
In the first step, individual words are assigned to part-of-speech 
categories by consulting a static dictionary.
The categories of novel words are guessed through
a technique the authors call ``morpho-guessing''.  Words have links to
both sides for combining with other words in the sentence.
For example, the two words ``the book'' can merge because of
a right-facing ``D$+$'' link on the word ``the''  
and a corresponding left-facing ``D$-$'' link on the noun
``book''.  The ``D'' in both cases symbolise a determiner
such as ``the'' and ``an''.
An edge is drawn between these two words to indicate
that they are compatible with each other.  When a grammatical
sentence is correctly parsed, a valid {\emph {linkage}} is said
to be formed.  Several requirements 
must be satisfied in order to obtain a linkage.  One of these
requirements is that the graph formed must be connected, 
with all of the words in the sentence reachable.

Subsequent work on the system \citep{gls95:tech} permits 
{\emph {null links}} and post-processing of valid linkages with a
phrase parser for identifying constituents such as noun phrases and
verb phrases.  If a sentence cannot be successfully parsed, then 
words in the sentence are removed incrementally until a successful
parsing can be found.  When the sentence is finally parsed, each
removed word introduces a null link into the linkage.  When a linkage
for a sentence has been established, the post-processing phrase
parser combines words to form constituents of one or more words.
On-line documentation for the phrase parser\footnote{Available
at \myurl{http://www.link.cs.cmu.edu/link/ph-explanation.html}.}
reports that the constituents determined by the parser were
75\% correct on the Penn Treebank 
text\footnote{The Penn Treebank project is at 
\myurl{http://www.cis.upenn.edu/~treebank/}.}, 
which is newspaper text already annotated linguistically.

A sample parsing of a headline from \wsja 
is shown in \figref{fig:rephine-linkgram}.
In the figure the headline is shown in part (a).  The \linkgram
system considers it a valid sentence since two linkages are 
identified, as shown in parts (b) and (c).  In 
\subfigref{fig:rephine-linkgram}{a}, the ``G'' link is used to
pair ``South'' with ``Korea''.  A ``G'' link connects proper
nouns together, identified by the upper case letters at the beginning
of each of these words.  Similarly, a ``G'' link is inserted
between ``Current'' and ``Account''.  The second linkage case folds
``South'' so that the first ``G'' link is converted to an ``A''
link.  An ``A'' link connects a adjective with the noun it follows.
The ``Ss'' link pairs a subject with a verb, while ``Ost'' edges
connect a transitive verb to a singular object.  So, \linkgram
incorrectly treats ``'s'' as a verb in both linkages.
Linkages are scored so that ones with less null links or a lower edge
length total are displayed first.
The constituents for the linkage in (b) is
presented in \subfigref{fig:rephine-linkgram}{d}.  At the beginning
of each constituent is its classification.
The abbreviation ``VP'' denotes a verb phrase, 
while ``NP'' is a noun phrase.

\fig{
\begin{tabular}{c}
{\texttt {South Korea's Current Account}} \\[0.5ex]
(a) News headline \\[1.5ex]
\includegraphics*{./linkage-1.eps} \\[0.5ex]
(b) First valid linkage \\[1.5ex]
\includegraphics*{./linkage-2.eps} \\[0.5ex]
(c) Second valid linkage \\[1.5ex]
\parbox{5cm}{\begin{tabbing}
xxx\=xxx\=xxx\=xxx\= \kill
(S (NP {\texttt {South Korea}}) \\
 \>(VP {\texttt {'s}} (NP {\texttt {Current Account}})))
\end{tabbing}
} \\
(d) Constituents derived from the first linkage \\
\end{tabular}
}{An example parsing of a headline from \wsja using the \linkgram
system.  The headline is shown along the top as part (a).  Two
linkages are determined by \linkgram, which are shown in (b) and (c).
The linkage from (b) is post-processed for constituent identification,
yielding the output in (d).  Noun phrases are marked with ``NP''.}{Example
parsing of a headline with \linkgram (\wsja)}
{fig:rephine-linkgram}

As with previous work in IR and NLP, the focus of the 
phrase comparison is with noun phrases.  
The number of unique phrases that \linkgram found in \wsja was
compared with \rephine's phrase hierarchy.  Since only grammatical
sentences can produce linkages, the test data of news
articles were filtered so that sequences of words that are not 
expected to give many noun phrases were omitted.  First, \sgml tags
were removed and multiple consecutive whitespaces collapsed into
a single space.  Sentences were given to \linkgram one at
a time.  Then, the constituent analysis was determined
by the phrase parser.

As with the earlier experiments to measure recall, the \wsja document was
used as test data.  The computing resources of the \linkgram program were
quite high.  It required just over 5 hours
to process the first \mib{1} of \wsja
on a \vipe test machine.  This time included
the above steps, starting from the \sgml tag filtering, and ending
with the identification of constituents.  Note that
the amount of work performed by \linkgram is more than what is
necessary to identify noun phrases.  Other systems, such as the
\keyphind browser of \citet{gpwnf99:dss}, simply categorise words
according to their parts-of-speech, and then form noun phrases based on a set
of fixed rules.  Even so, \citeauthor{gpwnf99:dss}\ required 4 days to process
\gib{1} of text on a Pentium 233 MHz machine running Linux.
In contrast, \linkgram must first successfully find a linkage for a
sentence, and only when this is done, can it isolate the noun phrases.

After \linkgram completes, constituents which are noun phrases,
and are not composed of other phrases, were
recorded.  The noun phrases were case folded
and then stemmed with the Porter stemming algorithm employed
by \prepair.  In order to comply with the 16 character per word
limitation imposed by \prepair, spaces were inserted into long
words.  Finally, a list of all of the distinct noun phrases in
\wsja was prepared.

\figref{fig:rephine-sgmlsample} illustrates how a news article
in \wsja was parsed, with linefeeds altered to improve display.
Text that was stripped from the article is shown shaded.  All \sgml tags
are removed, as well as text within most tags.  Only text
between opening and closing {\tl}HL{\tg} and {\tl}TEXT{\tg} tags
were retained, since the headline and the body of the article has
the best chance of providing complete sentences.  The noun phrases
found by \linkgram for this example are shown in boxes.  The headline used
as an example in 
\figref{fig:rephine-linkgram} was taken from the article shown in
\figref{fig:rephine-sgmlsample}.  

\fig{
\begin{tabular}{l}
\input{./wsja_sample_parse.sgml}
\end{tabular}
}
{An example article from the \wsja document, with noun
phrases indicated with outlined boxes.  Shaded boxes
represent parts of the article that were removed prior to
applying \linkgram.  The article has been formatted with
linefeeds for improved display.}{Example article
with noun phrases identified (\wsja)}{fig:rephine-sgmlsample}

The \linkgram system successfully found 147,294 grammatical
sentences in \wsja.  Within these sentences, 744,721 noun
phrases were identified, of which 216,661 were unique
after stemming and case folding.  These unique
phrases were divided into groups based on their lengths, in words,
and compared with the phrases available to \rephine, which were
also grouped according to length.  The phrases for \rephine are
identical to the ones reported in \tabref{tab:rephine-recall}.
Information about these
two sets of phrases are shown in \tabref{tab:rephine-nlpph-stats}.
The sizes of the sets of phrases are listed by lengths in the
second and third columns.
Note that noun phrases are only found if a valid linkage has been determined.
In the fourth column of
this table, the number of phrases common to the phrase
hierarchy and in the set of noun phrases has been tabulated.
The probability of a symbol in $\Ph$ being a noun phrase is given in fifth
column.  The reverse, the
probability of randomly selecting a noun phrase which is in
the phrase hierarchy, is listed in the last column of the table.  

\tab{cccccc}
{
Phrase      & Number of        & Number of    & Size of                & Probability & Probability \\
length      & phrases in       & unique noun  & $\Ph \cap {\text{NP}}$ & $\Ph$ has & noun phrase \\
            & hierarchy ($\Ph$)& phrases (NP) &                        & noun phrase & in \Ph \\
}
{
{\D\D}1     & \D39,637  &    13,172 &        13,172 & 0.332 &    1.000 \\
{\D\D}2     & \D79,693  &    78,527 &        21,692 & 0.272 &    0.276 \\
{\D\D}3     & \D60,825  &    72,492 &       \D6,350 & 0.104 &    0.088 \\
{\D\D}4     & \D25,245  &    32,419 &       \D1,578 & 0.063 &    0.049 \\
{\D\D}5     & \D10,932  &    12,304 &       \D\C397 & 0.036 &    0.032 \\
{\D\D}6     & \D\D4,998 &    \D4,939 &      \D\C144 & 0.029 &   0.029  \\
{\D\D}7     & \D\D2,895 &    \D1,696 &     \D\C\D32 & 0.011 &   0.019  \\
{\D\D}8     & \D\D1,686 &    \D\C662 &    \D\C\D\D8 & 0.005 &   0.012  \\
{\D\D}9     & \D\D1,180 &    \D\C247 &    \D\C\D\D2 & 0.002 &   0.008  \\
10+         & \D\D4,454 &    \D\C203 &    \D\C\D\D1 & 0.000 &   0.005  \\[1.0ex]
Total       & 231,545   &    216,661 &       43,376 & 0.187 &   0.200 \\
}{Quantifying the quality of symbols in the phrase hierarchy of
\wsja by calculating the intersection of noun phrases, as identified by 
\linkgram, with the phrase hierarchy.  Symbols have been
divided into 10 groups, based on length (in words).  The second and
third columns provide insight into the number of symbols (or 
noun phrases).  The intersection of these two columns are reported
in the fourth column.  The last two columns indicate the proportion of
phrases in $\Ph$ which are noun phrases (fourth column divided
by the second column), and the probability a noun phrase also being in
$\Ph$ (fourth column divided by the third column), 
respectively.}{Comparing the phrase hierarchy against the
noun phrases (\wsja)}{tab:rephine-nlpph-stats}

As \tabref{tab:rephine-nlpph-stats} shows, the majority of phrases in
the phrase hierarchy and noun phrases from \linkgram are
2 or 3 words in length.  The overlap between the two sets
is around 28\% for phrases of length two for both measures, 
a somewhat disappointing result.  Worse, as the phrases
increase in length, there is even less correlation with the phrase 
hierarchy, as the last two columns show.  This difference is partly due to the methods used by \repair
and \linkgram to build phrases.  The \linkgram system operated on sentences
individually, so phrases were guaranteed to be aligned on sentence boundaries.
Also, \sgml markup did not appear in any of the sentences, because of the
preliminary filtering.  On the other
hand, while punctuation-aligned \repair encourages alignment with punctuation,
phrases can still span sentences, since they are not processed individually.
Furthermore, since \repair was applied on the original version of \wsja,
any phrase that contained markup would not have a corresponding phrase
from \linkgram.  The analysis in \tabref{tab:rephine-nlpph-stats}
is based solely on the existence of phrases, and takes no account of the
frequency.  Because \repair finds common phrases, it is likely 
that the scores would
be considerably better if the probabilities were assessed over the 
sequence instead 
of over the phrase hierarchy.

Making phrases available in the phrase hierarchy is only half of \rephine's
purpose since matching contexts must also be located in
the reduced sequence.  Using the same methodology as established for
\tabref{tab:rephine-recall}, the recall of the phrases that exist
in both the phrase hierarchy and the list of noun phrases is presented
in \tabref{tab:rephine-recall-nlp}.  This time, rather than
presenting recall values for all symbols in the phrase
hierarchy, only the ones that were also noun phrases identified
by \linkgram are considered.  The second column of this
table lists the number of noun phrases that were also 
in the phrase hierarchy, copied from the fourth column of 
\tabref{tab:rephine-nlpph-stats}.  In \tabref{tab:rephine-recall-nlp},
as with the earlier experiment, 
the median recall value for each phrase length group is reported.
The mean and standard deviation of
each set of recall values is also shown.

\tab{ccccc}
{
Phrase      & Size of                & \multicolumn{3}{c}{Distribution of recall values} \\ \cline{3-5}
length      & $\Ph \cap {\text{NP}}$ & Median       & Mean   & Standard   \\
            &                        & recall       & recall & deviation  \\
}
{
{\D\D}1     &   13,172 & 1.000 & 1.000 & 0.000\\
{\D\D}2     &   21,692 & 0.800 & 0.720 & 0.298\\
{\D\D}3     &  \C6,350 & 1.000 & 0.787 &  0.271\\

{\D\D}4     &  \D1,578 & 1.000 & 0.845 & 0.242\\

{\D\D}5     &  \D\C397 & 1.000 & 0.831 &  0.264\\
{\D\D}6     &  \D\C144 & 1.000 & 0.849 & 0.250\\

{\D\D}7     & \D\C\D32 & 1.000 & 0.853 & 0.258\\
{\D\D}8     &\D\C\D\D8 & 1.000 & 1.000 &  0.000\\
{\D\D}9     &\D\C\D\D2 & 1.000 & 1.000 & 0.000\\
10+         &\D\C\D\D1 & 1.000 & 1.000 & 0.000\\[1.0ex]
Overall     &   43,376 & 0.980 & 0.889 &  0.158\\
}{Recall effectiveness of the noun phrases in the phrase hierarchy
for \wsja.  The median has been calculated
based on phrase length (in words)
in the third column, with the mean, and the
standard deviation indicated in the last two columns.}
{Recall of noun phrases in the phrase hierarchy (\wsja)}
{tab:rephine-recall-nlp}

The results from \tabref{tab:rephine-recall-nlp} mirror those
of \tabref{tab:rephine-recall}, but note the small size of the
sample for phrases of length 7 or more.  Again, the average 
recall is above 0.7, and more than half
of the symbols had perfect recall.  The standard deviation 
is wider and affects more groupings than in 
\tabref{tab:rephine-recall}.  

\newsection{Tools for Retrieval}{sec:rephine-summary}

The \rephine system allows users to perform two inter-related
tasks.  The first of these is phrase browsing.  Starting
from a list of the distinct words found in the document, the user can
either extend the current symbol by adding symbols to the left or 
to the right, or decompose
the current symbol into its two constituents.  A graph structure
with one symbol from the phrase hierarchy at each node
was established for the purpose of phrase browsing.  Each node
has six pointers for navigation and defining its structure.

Once the user has found a phrase of interest, the graph
structure is consulted again to determine the complete set of phrases
containing the chosen one.  This set of symbols, $\Criteria$,
is the basis for the search in the reduced word sequence to determine 
all of the locations at which any symbol in $\Criteria$ appears, along
with a suitable context for each.  The second job of \rephine
is to interactively display this set of results one by one to the user.
Each result consists of word tokens which have been stemmed and
case folded.  The non-words that follow each word token are
also initially unavailable.  Each presented result can be refined by the 
user by requesting modifiers from the three modifier streams, in any
combination.  If at this stage, another symbol of interest has been
isolated, it can be passed back to the phrase browsing part of 
\rephine.

In this chapter, \rephine has been demonstrated with a document collection 
\mib{1,000} in size on a \vipe.  Before considering larger collections, note that
only three of the seven streams available to \rephine must reside
in memory.  The four remaining streams are compressed in blocks so that
only parts of them are read from disk into memory.  The phrase hierarchy
and the two lexicons occupy 1.8\% of disk space compared to the size of
\news, and all of them must be accessible by \rephine in decoded form.
The complexity and size of the phrase hierarchy ensures that it requires 
the most memory during browsing after it has been fully decoded.

Based on these observations, several
approaches may be selected in order to browse larger collections.
First, \secref{sec:remerge-append} showed how \remerge can be extended
so that documents are appended to a compressed representation while
keeping the phrase hierarchy static.  Experiments demonstrated this
technique with character-based \repair, and an analysis of the changes
necessary for punctuation-aligned \repair was discussed.  In particular,
a method of handling novel words is required.  Second, either the maximum
generation or the maximum symbol length can be limited so that the
phrase hierarchy only contains moderately long phrases.  Third, while all
word tokens must exist in the phrase hierarchy as primitives, additional
heuristics, such as those offered by \kea and \linkgram, 
may be used when building the phrase hierarchy.  Instead of reducing
the number of phrases shown by \rephine, this approach would prevent
certain symbols from being added to the phrase hierarchy.  Finally, any
combination of these three methods can be used in order for larger document
collections to be compressed by \repair and \remerge, and subsequently
browsed.

The \rephine implementation has not reached
the stage where it could sensibly be made the 
subject of a user study.  As with \sequitur, too many symbols 
exist in \repair's phrase hierarchy, and some filtering is
necessary to improve usability.  The removal of symbols
through NLP techniques has been trialled by other IR systems, and
it is likely that \rephine would be more feasible if similar 
techniques could be employed.  An extension of this idea is to 
rank symbols using some metric so that those that are expected to
be the most useful are presented at the top.  Since there are
reasons why users would want frequent and
and infrequent phrases, simply sorting symbols
by frequency would not help.

User studies have been conducted to quantify the usefulness
of phrase browsing retrieval systems \citep{gpwnf99:dss, jp01:jcdl}, 
with some results being in favour of phrase browsing, and some less
conclusive.  \citet{wm01:sigirforum} concluded that
more work is still required in the area of phrase browsing
evaluation with users.  But even if the suggested changes 
were adopted for \rephine, it is expected that
a comparative user study would not favour \rephine.  While providing
users with a phrase browser was one of the aims of this development, the
necessary tension between browsing and compression inevitably means
that \rephine is not as careful as a dedicated phrase
searching mechanism.  One necessary improvement appears to be
achieving higher recall.
It is possible that the solution lies in
closely regulating word-based \repair's phrase selection 
heuristic.  As word-aligned \repair showed, preventing
certain pairs of symbols from joining does not necessarily
degrade compression effectiveness (see 
\figref{fig:prepair-bargraph-all} on 
\pgref{fig:prepair-bargraph-all}).

The benefits to
compression are easier to quantify than the benefits of phrase
browsing.  Since the latter quantity cannot be measured,
it is also difficult to assess when a reasonable compromise
between the two has been achieved.
Instead, the goal of this thesis has been to 
identify areas where compression and retrieval effectiveness and
efficiency are lost, so that one can be adjusted in favour
of another, as necessary.

\rephine is built on top of a compression mechanism, a structure which
limits its ability to browse and retrieve data.  
Throughout this thesis, the amount of compression effectiveness
lost in favour of faster retrieval times has been highlighted as
a means of compromising between the two competing sets of constraints.  In this
chapter, the current drawbacks with \rephine
represent the amount of
retrieval functionality lost in favour of supporting the
lossless compression of a document collection
with the underlying mechanism.
Employing NLP techniques with \rephine
may be the ideal solution for compensating
for \rephine's loss in retrieval effectiveness.  As mentioned
above, at the same time, NLP techniques may also reduce the
number of symbols that reside in memory during browsing.

Regardless of the improvements made to \rephine, phrase
browsing in general is one of many IR techniques.  This chapter
has also provided a glimpse into other querying techniques, 
including ranked queries,
proximity queries, and Boolean queries.  The system a user 
chooses depends on factors such as how familiar the user is
with the document, whether query terms include phrases, and whether
the order and the positions of the words in a phrase matter.
As \citet{gpwnf99:dss} pointed out, no single IR method 
is perfect and the most viable solution 
might be to provide ``several search tools that can 
make up for each others' shortfalls''.  

This chapter concludes the discussion in this thesis
about the balance between compression and phrase browsing.
Before summarising the thesis in \chapref{chap:summary},
the next chapter discusses a separate topic related to the
compression of small \html files.

}
{The sequence of steps performed by \rephine during phrase
browsing.  Dashed boxes indicate actions 
performed by the system, while solid boxes require
input from the user.  Shaded cylinders represent one of
the seven streams accessible by \rephine, with dashed lines
from them representing file access.  The user can exit
the system at any time.}{Sequence of steps
performed by \rephine during phrase browsing}{fig:rephine-alg}

In the first step of the figure, \rephine decodes the phrase
hierarchy, the word lexicon, and the non-word lexicon, and
maintains all three within memory.  Next, a directed
graph (DG) is built using the symbols of the phrase
hierarchy, to permit efficient searching.  Details
about the structure of the graph is given in the next section.
The third step is interactive phrase browsing.  
In browsing the phrase
hierarchy, the user interactively traverses the DG
until a symbol of interest, $\alpha$ say, has been isolated.

In the fourth step, the symbol of interest is used to identify all
other symbols in the DG which contain $\alpha$.
When that search completes, a set of symbols $\Criteria$ has been 
established.  Next, $\Criteria$ is input to the fifth
step, in which the \reviewb decoder is used to locate every
context in which any symbol in $\Criteria$ occurs.  The \reviewb
decoder returns a result set, $\Result$.  Suppose a symbol
$\beta$ contains the symbol $\alpha$ as a component.  In order
to avoid reporting the same location multiple times, when an
occurrence of $\beta$ has been found in the reduced word sequence, 
only $\beta$ is reported.
The size of $\Result$ is denoted as $|\Result|$, and members
are enumerated from $i=1$ to $i=|\Result|$.

The first result's context, $\Result_1$, is
displayed to the user.  At this point, the user may choose
any combination of five actions.  These actions are:  return to
phrase browsing; toggle case folding modifiers, stemming modifiers, 
or non-word insertion for $\Result_1$; or examine
the next or previous result.  It is assumed that the result
set is a circular list, so that the previous result to $\Result_1$
is $\Result_{|\Result|}$.  The three actions related to the modifiers
initiate seek operations to the corresponding positions in 
their respective compressed
streams.  After the modifiers have been retrieved, $\Result_i$
is re-displayed to the user.  The application of 
modifiers can be toggled in any order.  The first time
modifiers are required, access to the streams are performed.
Assuming requested modifiers are buffered in memory, subsequent 
requests do not need to access the file again.
Finally, if the contexts that were obtained were inadequate in
satisfying the user's need, any part of a context can be used as
input into the phrase browsing part of \rephine.

Later in this chapter, the entire phrase browsing process is detailed 
through a brief example which demonstrates the choices available
to the user.  But first, in the next section, the directed graph
required for phrase browsing is explained.

\newsection{Phrase Hierarchy as a Directed Graph}{sec:rephine-datastruct}

Once the phrase hierarchy has been decoded, a directed
graph (DG) is 
constructed.  Every symbol in the phrase hierarchy is assigned
to a separate node in the DG.  Each node contains six
directed edges connecting it to other nodes, as shown in 
\figref{fig:rephine-graph}.
In this figure, the symbols are based on character-based
\repair instead of punctuation-aligned \repair in order to simplify 
the discussion.  The current node is represented by the
symbol ``w{\textbar}o'', with the vertical bar (\textbar)
separating the symbol into its two components.
Two of the six pointers
are {\emph {child pointers}}, and refer to a symbol's structure
These pointers were described in 
\chapref{chap:remerge} as part of phase 1 of \remerge,
but were called components at that time.
The remaining four pointers are collectively called
{\emph {navigational pointers}}.
Navigational pointers are added for the purpose of
browsing and have not been required in the previous discussion.

\fig{
\includegraphics*{./graph.eps}
}{The six pointers used by \rephine to link nodes representing
symbols in the phrase hierarchy into a directed graph.  The symbols in
the example are based on character-based \repair, for illustrative
purposes only.  The node of interest is in the centre, with the
symbol ``w{\textbar}o''.  A vertical bar (\textbar) separates
the expanded symbols into left and right components.
Names of each pair of pointers are provided to the right of the
figure.
}{The six pointers used by a \rephine graph}
{fig:rephine-graph}

The purpose of each pointer is illustrated in the figure.
Pointers appear in pairs and are designated as being 
``left'' or ``right''.  The four navigational
pointers are in two pairs -- the {\emph {sibling pointers}}, 
and the {\emph {parent pointers}}.  

In the example figure, the current symbol is in
the centre, with its six pointers listed in full.  All of the other
nodes in the graph
have six pointers each as well, but the outgoing edges are 
omitted in the figure for simplicity.  The centre symbol 
is ``w{\textbar}o'', with left and right
components (or children) of ``w'' and ``o'', respectively.
The left sibling symbol of ``w{\textbar}o'' is another symbol which 
also has ``w'' as a left child.  A chain is formed from the 
centre node, with all symbols with ``w'' as a 
left child being linked together with their
left sibling pointers.  Within a chain, the nodes are unsorted,
so any node can serve as the front of the list.
Each chain ends with a final
{\emph {empty node}}.  The child pointers of all primitives
also lead to empty nodes.
Similarly, the right sibling pointer from 
``w{\textbar}o'', leads to another symbol with ``o'' as a right child.  

One important operation is to
obtain the entire chain when the current symbol is any symbol
along that chain.
The child and parent pointers help solve this problem.
Parent pointers directly lead to symbols which are a single 
{\emph {extension}} away from the current symbol.  An extension
occurs when a symbol is juxtaposed
to the current symbol on either side, arriving at another
symbol in the phrase hierarchy.  For example, in 
\figref{fig:rephine-graph}, the right parent pointer from
``w{\textbar}o'' allows a right extension to the symbol 
``wo{\textbar}od''.
\figref{fig:rephine-expandgraph} shows \figref{fig:rephine-graph}
at a larger scale to demonstrate how an entire chain of symbols
can be obtained from anywhere in the list.  The ``w{\textbar}o''
is at the head of the list of right siblings, 
and the current symbol is $\alpha$.
Both of these symbols have the same right component.
Every symbol along that chain
has a right child link to the primitive ``o''.  The front of
the chain can be reached by taking the right child pointer from
$\alpha$, followed by the left parent link of ``o'',
as shown by the dashed lines in the figure.
This is possible provided that
graph construction ensures that every node's parent pointer
leads to the beginning of a chain of parents.  If a symbol
does not have a parent, the parent is represented as
an empty node.

\fig{
\input{./expandgraph.pstex_t}
}{The directed graph structure created by \rephine to link symbols
together, at a larger scale than \figref{fig:rephine-graph}.  The
node of interest is ``w{\textbar}o'' in the centre.  The dashed
arrows shows how an entire chain of symbols can be obtained from
an arbitrary symbol $\alpha$ within the chain.  Also shown are two
siblings of the symbol ``wo{\textbar}od''.
}{The directed graph structure of \rephine, at a larger scale}
{fig:rephine-expandgraph}

The graph structure created by \rephine
resembles the data structure employed by \mka \citep{wolff75:bjp}
for the recursive pairing of symbols for discovering sentence
structure.  Each
symbol in the dictionary created by \mka has major and minor links.
A major forwards link is the same as a left child pointer in
\rephine's graph, while a major backwards link is identical
to a right child pointer.  Minor links operate in a similar
fashion to the sibling pointers of \rephine.

Equipped with this graph, five operations are available to
the user.  Assuming the user's currently selected symbol
is ``w{\textbar}o'' of \figref{fig:rephine-graph} and 
\figref{fig:rephine-expandgraph}, the
five operations are as follows.  First, the user can decompose
the symbol into its two components through the child pointers.
Second, if the right parent pointer from ``w{\textbar}o'' 
is used, followed by all left sibling pointers until the empty
node is reached, then all right extensions of ``w{\textbar}o'' can
be retrieved.  In \figref{fig:rephine-expandgraph}, this operation
would obtain the set 
of two symbols, \{``wo{\textbar}od'', ``wo{\textbar}uld''\}.
Likewise, a third operation is to extend ``w{\textbar}o''
to the left.  The union of these last two sets results in a
fourth function, extending ``w{\textbar}o'' in both directions.  

The fifth and final operation is reserved for decoding, instead
of phrase browsing.  Decoding requires identification of
all symbols which contain the current symbol $\alpha$,
either directly or indirectly.  The
located symbols are denoted in 
\figref{fig:rephine-alg} as $\Criteria$.  In order to find all
symbols in the phrase hierarchy graph which contain $\alpha$, 
a breadth-first search
with the navigational pointers is sufficient.  Let $\beta$
represent a symbol which is an ancestor of $\alpha$.  If $\beta$
has only one occurrence of $\alpha$, 
then all of its parents does as well.  But $\alpha$
must appear in either its left or its right component, which
can be determined based on how $\beta$ was reached.  So, only
one of the sibling pointers need to be followed.  If $\beta$
has multiple occurrences of $\alpha$, with at least one in each
component, then during the breadth-first search, $\beta$ is reached
twice.  Care must be taken to ensure that when $\beta$ is reached
the second time, only the other sibling pointer is traversed, and not the
parent pointers.  If this strategy is followed, then 
the nature of the navigational pointers
ensures that all symbols which include ``w{\textbar}o'' are included.

Construction of the DG requires $O(|\Sigma| + |\rho|)$ time,
for a phrase hierarchy of $|\Sigma|$ primitives and $|\rho|$
phrases.  Since the chiastic slide and interpolative coding have been
used to code the phrase hierarchy, the phrase hierarchy is already
in increasing generation order, and within each generation,
in increasing chiastic slide value order.  As no symbol is in
the same generation as either of its two components, constructing
the graph from the first primitive symbol in sequential order
results in graph construction resembling a bottom-up 
approach.
If the phrase hierarchy is stored in memory as an array, 
and if each of the six graph
pointers are represented as array indexes, then locating a
symbol's components requires $O(1)$ time.  
For each phrase, an insertion
is performed into the appropriate parent pointers of 
each of its two components, so that each node is added
at the front of its two chains of siblings in $O(1)$ time.
That is, a single linear pass through the phrase hierarchy
is sufficient to construct the graph.

The memory space required for the phrase hierarchy is linear in
the number of symbols in the phrase hierarchy.
The graph structure adds an additional 6 words of memory 
per symbol.  Also, the phrase hierarchy may
have every symbol expanded into primitives, or symbol
expansion may be deferred until the symbol is required.  
The choice made depends which is more important:
memory space or response time.  Experiments demonstrating
the advantages and disadvantages of phrase
hierarchy expansion were reported in \tabref{tab:repair-despair}
on \pgref{tab:repair-despair}.  In
the implementation of \rephine described in this chapter,
a buffer of recently decoded symbols is maintained and consulted
before decoding a symbol.  This approach was also employed in
\repair's decoder, \despair.

\newsection{Phrase Browsing Example}{sec:rephine-example}

The \rephine system performs two main tasks:  phrase browsing
and decoding symbol contexts for presentation.  Examples of both tasks
are presented in Figures \ref{fig:rephine-scr-browse} and 
\ref{fig:rephine-scr-decode}.  The 
aspects of \rephine related to the graphical user
interface were 
implemented with version 2.0.2 of the \gtk 
library\footnote{Available from \myurl{http://www.gtk.org/}.}.

The data for the example
is \news, processed with punctuation-aligned
\repair exactly as was described in the experiments of
\chapref{chap:remerge}.  The sequence of word tokens in
those experiments were partitioned into blocks of 20,971,520
symbols each, and then merged using method E of \remerge 
(\secref{sec:remerge-expt}).
Modifier and sequence coding were done with indexed \shuff
and \reviewb respectively.

In the previous section, the choices available to the user
for phrase browsing from the current symbol were discussed.
The user also needs to be given a starting point to browse
from.  In \rephine, the user is first presented with the
document's entire lexicon, as shown in
\subfigref{fig:rephine-scr-browse}{a}.  The extensive list
can be daunting, and one solution
is to filter words, discussed later
in the chapter.  In \subfigref{fig:rephine-scr-browse}{a}, the user
has selected the word ``prime''.  
The first symbol is extended to
the right in \subfigref{fig:rephine-scr-browse}{b}, and the
phrase ``prime minist'' has been selected.  While phrases in this 
window are ordered according to their locations in the DG, 
\citet{jp99:dl} and \citet{pwcb00:acmdl} describe systems %% Second cite due to witten00:cpm
where phrases can be ordered according to the number of documents
in which they appear, or their frequency in the collection.
While details about the implementation of these two features are
not available, it is expected that either the additional information
is stored by the system, or calculated at run-time.  Also note
the variations on the phrase ``prime minist'' due to the different
spellings of the second word.  While stemming and case folding
have reduced the size of the phrase hierarchy, it is still 
possible that two seemingly equivalent words stem differently.
The chosen phrase is extended to the left and the phrase 
``canadian prime minist'' has been highlighted in 
\subfigref{fig:rephine-scr-browse}{c}.  As this window shows,
unfortunately, the words ``Canadian'' and ``Canada's'' stem
to different words.  The identified phrase of 
\subfigref{fig:rephine-scr-browse}{c} is extended in both
directions, resulting in \subfigref{fig:rephine-scr-browse}{d}
and its 7 phrases.  Of these 7 phrases, 3 are right extensions
and 4 are left extensions to the phrase ``canadian prime minist''.
The selected phrase is broken down into its two components through
its child pointers, as shown in 
\subfigref{fig:rephine-scr-browse}{e}.  This phrase browsing
example commenced with the word ``prime'' and, in four steps,
arrived at the name of a Canadian prime minister.  In \rephine,
the refinement of a query term is done interactively with the
help of the phrase hierarchy's graph.

%%  A double-page figure
%%  Snapshot created using the command:
%%  import -frame -window "Main browsing session" ~/thesis/t/ch07/scrshots/snapshot1.eps
%%  Then, export to PS with xv (greyscale).
\fig{
    \setlength{\tabcolsep}{3pt}
    \begin{tabular}{ccc}
      \multicolumn{2}{c}{\includegraphics*[scale=0.6]{./scrshots/browse-0.ps}} &
      \includegraphics*[scale=0.6]{./scrshots/browse-1.ps} \\ 
      \multicolumn{2}{c}{(a) Initial window} & (b) Extend right \\ \\
      \includegraphics*[scale=0.6]{./scrshots/browse-2.ps} &
      \includegraphics*[scale=0.6]{./scrshots/browse-3.ps} &
      \includegraphics*[scale=0.6]{./scrshots/browse-4.ps} \\
      (c) Extend left & (d) Extend both ways & (e) Decompose \\
    \end{tabular}
%  }
%  \end{leftfullpage}
}{A sample phrase browsing session of the \news document.}
{Sample phrase browsing session of \rephine (\news)}{fig:rephine-scr-browse}
%\end{figure}

A sample decoding session is illustrated in 
\figref{fig:rephine-scr-decode}.  The selected phrase of
\subfigref{fig:rephine-scr-browse}{e} is the starting point
for the decoding.  The first of 17 occurrences
of the phrase ``pierr elliott trudeau''
is given in \subfigref{fig:rephine-scr-decode}{a}.  The results
are presented to the user in the order in which they occur in
the original document.  In the first window, the phrase exists
in a longer phrase, which has been marked.  A fixed context
of 10 symbols in either direction of the highlighted symbol
has been selected for \rephine.  In the absence of detailed information
about the
non-words, a space character is inserted between each word
token, with linefeeds added as appropriate.  
Even though the words in this window 
have been stemmed and case folded, reasonable guesses can
be made with most of the words.  In 
\subfigref{fig:rephine-scr-decode}{b}, the user has skipped
ahead to the 15th result.  As this section of text may be
of interest, the case folding is first reversed in 
\subfigref{fig:rephine-scr-decode}{c}, and then the stemming
modifiers are applied in 
\subfigref{fig:rephine-scr-decode}{d}.  The corresponding modifiers are
located by the system and only the small number of required
modifiers are transmitted to the \rephine window.  Modifiers can 
be toggled in any order, and allow the user to progressively
change the displayed context.  Finally, in 
\subfigref{fig:rephine-scr-decode}{e},
the non-words are inserted into the text and the window is 
enlarged for improved display.  The text exists in the 
original \news document as shown.  

\fig{
    \setlength{\tabcolsep}{3pt}
    \begin{tabular}{ccc}
      \includegraphics*[scale=0.6]{./scrshots/decode-0.ps} &
      \includegraphics*[scale=0.6]{./scrshots/decode-1.ps} &
      \includegraphics*[scale=0.6]{./scrshots/decode-2.ps} \\ 
      (a) Result 1 of 17 & (b) Result 15 of 17 & (c) Reverse case folding\\ \\
      \includegraphics*[scale=0.6]{./scrshots/decode-3.ps} &
      \multicolumn{2}{c}{\includegraphics*[scale=0.6]{./scrshots/decode-4.ps}} \\
      (d) Reverse stemming & \multicolumn{2}{c}{(e) Add non-words} \\
    \end{tabular}
}{A sample decoding session of the \news document, 
continuing from \subfigref{fig:rephine-scr-browse}{e}.}
{Sample decoding session of \rephine (\news)}{fig:rephine-scr-decode}

At this stage, the user's information need may be satisfied.  
Alternatively, the 
user may decide to select the phrase ``Progressive 
Conservative'' as the new starting point for phrase browsing.
Then, the starting point for the browsing session becomes
the phrase ``progress conserv''.

\rephine applies the modifiers only when requested by the user.
This approach takes advantage of the fact that the modifier
streams are encoded separately.  At least two other ways of
handling modifiers are possible, which have not been implemented
for \rephine.  If the cost of decoding and transmitting the 
modifiers is acceptable, the user can be given the option to 
reverse all modifiers by clicking one button.  This feature
can be added by making a small change to the \rephine 
interface.  Another option is to display initial word
tokens with default stems and case information applied.  
That is, the most frequent stemming and
case folding modifier is recorded for each word in the lexicon
during pre-processing with
\prepair.  The additional disk space required is $2|\Sigma|$ 
integers, uncompressed,
for a compressed document collection of $|\Sigma|$ primitives.
When phrases are first displayed (for example,
in \subfigref{fig:rephine-scr-decode}{a}), these default modifiers
can be applied immediately without any intervention from the user.
While the phrases presented may occasionally be
linguistically incorrect, they should be valid most of
the time.
These default modifiers can be transmitted as needed, or once
at the beginning of the browsing session.  A similar idea was
adopted by \citet{pwcb00:acmdl}.  
The aim of both of these alternatives is to make the interface
more user-friendly by minimising the amount of exposure users
have to case folded, stemmed words.

\newsection{Evaluating \rephine}{sec:rephine-evaluate}

Information retrieval systems are generally evaluated
in terms of efficiency and effectiveness.  Efficiency is
measured as the amount of time required to return a response
to the user.  Effectiveness is an assessment of the extent to which
the set of results satisfies the user's information need.

Suppose an IR system indexes the words in a 
collection of $n$ documents, and that for some set of queries,
the number of documents that are relevant is known.
The effectiveness of the
IR system for the query is determined by the number of
documents the system returns which are relevant.
Two standard methods of quantifying
effectiveness are {\emph {precision}} and {\emph {recall}}.
These and other methods of judging retrieval effectiveness 
are described by various sources, including \citet{korfhage97:book},
\citet{br99:book}, and \citet{wmb99:book}.

Precision and recall are expressed on a scale from 0 to 1.
Precision is the fraction of returned results that are relevant, 
while recall is the proportion of all relevant
documents that are returned.  If a user poses a query for which
the set of relevant documents in the collection is $r$,
and the set of documents returned by the system is $a$, then
precision and recall are defined as follows:
\neweqn{
\textrm{Precision} = \frac{|r \cap a|}{|a|} \\
\textrm{Recall} = \frac{|r \cap a|}{|r|}
}{eqn:rephine-prec-rec}

Ideally, the effectiveness of \rephine should
be evaluated in a similar manner.  However, significant 
differences between \rephine and traditional IR systems prevent 
any straightforward comparison.

Instead, \rephine is evaluated in three steps.  In the
first step, an overview of past work related to phrase browsing
is summarised.  
Second, \rephine is analysed in detail in the context of related
work and traditional query methods.  Some possible improvements
to \rephine are also proposed.
Finally, the phrase hierarchy used by 
\rephine is compared to noun phrases formed
through natural language processing techniques.

\newsubsection{Related Work}{subsec:rephine-related}

Previous work in the area of phrase browsing can be classified based
on the method employed for drawing phrases from a document.
Phrases can be obtained through
the semantic meaning of the words using natural language processing
techniques, or statistically, by applying a mechanism similar to
\repair.

Most work with natural language processing (NLP) techniques
are related to tagging words according to their parts-of-speech (that is,
noun, verb, etc.)\ and forming phrases according to the rules of
grammar.  A significant amount of work has concentrated on 
{\emph {noun phrases}}.
A noun phrase is a linguistic unit.  In it,
a noun is both required and designated as the most semantically
important word and, as a result, is assigned as
the phrase's head.
\figref{fig:rephine-np} shows an example of a noun
phrase, divided into its various parts.

\fig{
\includegraphics*{./nounphrase.eps}
}{Definition of a noun phrase, adapted from 
\citet[pg.~209]{greenbaum96:book}.}{Definition of a noun phrase}
{fig:rephine-np}

While some retrieval systems treat a query as being
made up of individual words, occasionally, a query should be
seen as a set of one or more noun phrases.  If a user posed
the query ``University of Melbourne'', ideally, documents 
about other universities or the city of Melbourne are not
of interest.  In fact, the seemingly unimportant word ``of''
is crucial for this phrase, since ``University in Melbourne'' has
a completely different meaning.  \citet{kirsch98:sigirforum}
provided some insight into the Internet search engine Infoseek 
and noted that most users submitted queries which were noun phrases
and that queries were, on average, 2.2 words in
length.  

Several researchers have made use of noun phrases for their
phrase-based IR systems.
The \clarit system \citep{ez96:acl} 
identified both simplex and complex noun phrases  
for the purpose of improved
precision and recall in information retrieval, without the need
for training data.  The difference between a simplex noun phrase and
a complex noun phrase is that a simplex noun phrase omits all prepositional
phrases (the postmodifier in \figref{fig:rephine-np}).
\citet{av97:sigir} developed the \paraphrase system
which partitioned documents into clusters and summarised each 
cluster with noun phrases for the purpose of browsing.  A 
variant of the Brill tagger \citep{brill94:aaai} labelled
the words according to their parts-of-speech.  \linkit
by \citet{wek01:jcdl} identified noun phrases, which were subsequently
displayed by the \intellindex interface.  A part-of-speech
tagger was used to determine simplex noun phrases, which were then grouped
according to their heads.  As a result, phrases such as 
``the university bookstore'' was associated with the
phrase ``a bookstore''.

Extraction of keyphrases (document keywords in the form of 
phrases) with machine learning techniques was investigated by 
\citet{fpwgn99:ijcai} and embedded within a system dubbed
\kea.  Several interfaces have been built on top of \kea,
including \kniles, \phrasier, and \keyphind.  \kniles and
\phrasier uses the 
phrases identified by \kea to link documents together via
hyperlinks \citep{jones99:interact, jp99:dl}.  Browsing
phrases identified by \kea are supported by the \keyphind
system \citep{gpwnf99:dss}.  \keyphind removes all but twelve
phrases for each document in a collection and constructs three
types of indexes with the phrases:  
a word-to-phrase index, a phrase-to-document
index, and a document-to-phrase index.  A user study was conducted
which compared the usability of \keyphind to a traditional IR system
based on ranked queries.  The results of the study showed that
\keyphind was preferred for certain browsing tasks, such as collection
evaluation.  A more recent user study \citep{jp01:jcdl}
with \kea and keyphrases looked at computer science papers and
compared the phrases selected by \kea with those specified by
the authors of each paper.  The study's conclusion was that the
phrases selected by \kea were almost as good as ones chosen by
the authors.

Phrase selection through statistical methods is more relevant to
\rephine than through semantic methods.  
The \phind system of \citet{nwp97:acmdl} selected 
phrases using the \sequitur compression mechanism, which is similar
to \repair and was discussed in detail in \chapref{chap:repair}.  Like
word-based \repair described here, \sequitur was modified to 
operate on word tokens
in order to ensure word alignment.  Punctuation marks were removed and
words were case folded to lower case.  Word stems were handled by 
the phrase
browser so that if the word ``book'' was chosen, a search for the word
with different endings was performed.  Follow-up work on a browsing
system built on top of \sequitur was described by \citet{nwp99:ijdl}.
In order to browse large document collections, a disk-based browsing
system is proposed.  The phrase hierarchy is placed on disk with
an inverted index.  Smaller auxiliary data structures are
kept in main memory for accessing the phrase hierarchy.

\citet{wza99:adc} describe an index structure called a {\emph {nextword index}}
which provides explicit support for phrase querying.
This multi-level data structure is an index to pairs of words 
that appear in the document collection.  The word lexicon forms the
top-level of the index.  Each word in the lexicon 
has a nextword list that, together, indicates the word pairs that exist
in the collection.  Every word pair has a position
list as the third and final level in the nextword index.  The position
list specifies the documents and the document positions where
the pairs of words occur.  Subsequent extensions to this work include
compaction techniques for improving storage requirements
\citep{bwz01:spire}, and a description of how nextword indexes can be
used for mono-directional phrase browsing \citep{bwz01:acsc}.  
The nextword index for a transformed version of \wsjb 
occupied 56\% of the collection size.  When a query phrase with
more than two words is processed, it must be partitioned into pairs of
words for look-up into the index.  The position lists obtained
using the word pairs need to be combined in order to find locations
of the original, longer phrase.  
\citeauthor{bwz01:acsc} [\citeyear{bwz01:acsc, bwz02:sigir}] 
showed how the word
pairs should be selected in order to optimise querying time, and then
implemented a retrieval system which combined a
nextword index with a word-level IR system.
It was shown that the addition of the nextword index provided
more efficient retrieval when the first word is a common word.

Several methods combine semantics with statistics for phrase 
selection.  \citet{wolff80:ls} tagged words according to 
their parts-of-speech, and replaced frequently
occurring pairs of word classifications with a recursive pairing
mechanism similar to \repair.  \citet{pwcb00:acmdl} improved on
\phind by combining \sequitur with \kea.  A Brill tagger was combined with 
\kea and applied to the documents in order to identify boundaries
between noun phrases.  Next, \sequitur was applied to combine
words into phrases, with constraints enforced to prevent any crossing
over the boundaries.

A separate, yet related area to retrieval with phrases
is that of locality-based indexing
(or proximity indexing \citep[pg.~101]{br99:book}).  Locality-based
indexing creates an index at the granularity of individual word
tokens instead of documents.  When multiple terms are included in
the query, the results are chosen not only because of query
frequency, but also the distance between query terms within 
the documents.  Some past works with locality-based retrieval
include those of \citet{bsms95:trec}, \citet{ccb95:trec},
\citet{ht95:trec}, and \citet{dm99:sigir}.  

\newsubsection{Analysing \rephine}{subsec:rephine-analyse}

The \rephine system of this chapter can be analysed in
terms of its efficiency, effectiveness, and similarity and
differences with other work.
With respect to the efficiency of \rephine, the time
required by decompression has already been presented.
Experiments in \chapref{chap:remerge}
showed that the total decoding time for \news 
was 236.9 seconds, when \shuff was used as the coder
(\tabref{tab:remerge-pa-time} on 
\pgref{tab:remerge-pa-time}).  For the purpose of \rephine,
less time is required since only the phrase hierarchy and
the two lexicons are fully decoded.  Decoding the
two lexicons required 0.8 seconds (\tabref{tab:remerge-pa-otherstats} on
\pgref{tab:remerge-pa-otherstats}).  The total time required for phrase 
hierarchy decoding, including graph construction by \rephine was
12.3 seconds, 
averaged over three trials on a \vipe.  Experiments
in \chapref{chap:review} showed that 
searching for symbols in the reduced word sequence
encoded with \reviewb is fast, but depends on the number of 
symbols searched for (see Figures \ref{fig:review-ws-freq}
and \ref{fig:review-ws-len} on pages
\pageref{fig:review-ws-freq} and \pageref{fig:review-ws-len}, 
respectively).  The values in these graphs
provide a good estimate of the amount of time required
to search a sequence coded with \reviewb.  Finally, locating 
100 positions in a stream of modifiers with indexed Huffman
coding requires no more than 2.2 elapsed seconds
if a block size of 32,768 bytes was chosen, shown in
\tabref{tab:review-modtime} on \pgref{tab:review-modtime}.

In summary, 13.1 CPU seconds is required by \rephine to decode
the phrase hierarchy and the two lexicons.  Approximately
10 elapsed seconds is necessary for \reviewb to search 
the compressed sequence for 100 different symbols.  The
modifiers at a certain location in a stream can be found in
less than 0.1 seconds.

It is more 
difficult to assess the effectiveness of \rephine, 
or phrase browsing in general.  The main feature
of phrase browsing is the 
user-driven navigation between symbols in the phrase hierarchy.  This
interface experience cannot be easily quantified.
If the searching phase of \rephine is assessed using
precision as the evaluation metric, then problems similar to those encountered with 
Boolean queries need to be addressed, since every result returned contains
the symbol being sought, and precision is 1.0.  

On the other hand, the proportion of relevant results (recall) can 
be measured, but in a slightly different manner than in the traditional
definition.  Unlike with a Boolean query or an exhaustive string search, 
recall is not guaranteed to
be 1.0, because of the way in which the pairing process of \repair breaks up
phrases.  For example, suppose part of a message contains the
five symbols:  $\alpha$$\beta$$\delta$$\beta$$\delta$.
Suppose also that the frequency of $\alpha$$\beta$ throughout the message
is higher than the pair $\beta$$\delta$, and that the replacements
made are \tc{1} $\rightarrow$ $\alpha$$\beta$ followed by
\tc{2} $\rightarrow$ $\beta$$\delta$.  With these replacements, the segment
$\alpha$$\beta$$\delta$$\beta$$\delta$
reduces to \tc{1}$\delta$\tc{2}.  As a consequence, if the symbol \tc{2} is selected
during phrase browsing, then only the second occurrence of it
in this segment is found.  The first occurrence
has been split by the replacement with symbol \tc{1}, and is not reported 
to the user.  

Experiments were conducted with the first \mib{20} of \news (\wsja), compressed
with punctuation-aligned \repair, to evaluate the average recall.  Unlike
results from \tabref{tab:prepair-stats} on \pgref{tab:prepair-stats}, 
phase 1 of \remerge has been applied,
removing 16 duplicates in the phrase hierarchy.  For each symbol in the phrase
hierarchy, the frequency of the symbol and all other higher
generation symbols that contained it were compared with the actual
frequency in the word sequence.  The results from these
experiments are presented in \tabref{tab:rephine-recall}.

\tab{ccccc}
{
Symbol      & Number of symbols & \multicolumn{3}{c}{Distribution of recall values} \\ \cline{3-5}
length      & in phrase hierarchy & Median      & Mean   & Standard   \\
            & ($\Ph$) & recall       & recall & deviation  \\
}
{
%%  39,638 changed artificially to 39,637 for 0-length word, total symbols actually 231,546
%%  Upper range of #3 is 2.500
%%  total symbols 16 less than ch4 due to Remerge phase 1
{\D\D}1     &  \D39,637 & 1.000 & 1.000 & 0.000\\
{\D\D}2     &  \D79,693 & 0.750 & 0.706 & 0.295\\
{\D\D}3     &  \D60,825 & 1.000 & 0.794 & 0.260\\
{\D\D}4     &  \D25,245 & 1.000 & 0.862 & 0.225\\
{\D\D}5     &  \D10,932 & 1.000 & 0.917 & 0.184\\
{\D\D}6     & \D\D4,998 & 1.000 & 0.938 & 0.163\\
{\D\D}7     & \D\D2,895 & 1.000 & 0.954 & 0.141\\
{\D\D}8     & \D\D1,686 & 1.000 & 0.954 & 0.144\\
{\D\D}9     & \D\D1,180 & 1.000 & 0.957 & 0.136\\
10+         & \D\D4,454 & 1.000 & 0.955 & 0.136\\[1.0ex]
Overall     &   231,545 & 0.975 & 0.904 & 0.168 \\
}{Recall of the symbols in the \wsja phrase hierarchy.
Median recall values for each symbol length are in the third column,
followed by the mean, and the standard deviation.
}{Recall of symbols in the phrase hierarchy (\wsja)}{tab:rephine-recall}

In \tabref{tab:rephine-recall}, the 231,545 distinct symbols formed from
\wsja were separated into ten symbol groupings based on their expanded
length, measured in word tokens.  Since the zero-length
word token for separating a long non-word token is of no interest,
it does not appear in the table.  

The second column of the table shows
the size of each grouping.  The majority of symbols were 2 or 3 word tokens
in length.  The remaining three columns indicate some gross statistics
for the distribution of recall values for these symbols:  the median,
the mean, and the standard deviation.
Generally, results
were worse for symbols which represented 2 to 4 word tokens.
For the remaining categories, the average recall was at least 0.9
and their corresponding medians meant that at least half of the 
symbols had a recall of 1.000.  This result is further reinforced
by the standard deviation ranging from 0.000 to 0.184.  The average
recall and standard deviations were less satisfactory for symbols lengths of 2
to 4.  When the symbol length is 2, the median recall was only 
0.750.

The results from the table can be interpreted as follows.
Short phrases are more difficult to accurately find than primitives and
long phrases.  For example, in \wsja, the phrase ``prime 
ministr'', can be easily split depending on the words immediately
before and after it.  Note, however, that the most common phrase
that includes each word is exactly the one most likely to have
a high recall.  And, as phrases get longer, since their 
underlying components are more frequent, \repair does not 
separate them.

One anomaly is not shown in the experiments with \rephine.  Even
though precision across the set of symbols in the phrase hierarchy
is always 1.0, not every symbol can be 
searched.  In particular, while \rephine is able to locate
every symbol chosen through phrase browsing,
many combinations of words do not exist in the phrase hierarchy
and cannot be searched for.
In \subfigref{fig:rephine-scr-browse}{c}, the phrase 
``prime minist'' was extended to the left, and one phrase found
was ``canadian prime minister''.  However, the
phrase ``canadian prime'' does not exist in the phrase hierarchy,
and therefore, cannot be searched for by \rephine.  
The problem is the symbol pairing order employed by \repair 
(and \remerge) makes no attempt to place every multi-symbol combination
into the phrase hierarchy.  This point is expanded further later in this chapter.

Phrases for the dictionary are selected by \repair and \remerge 
based solely on symbol frequency, and in this sense, \rephine is
closely related to \sequitur's phrase browser, \phind.
\citet{nwp97:acmdl} identified the phrase boundary problem
with the phrase hierarchies produced by \sequitur; a problem
which also affects the phrase hierarchies of \repair.  The phrase
boundary problem refers to two symbols which expand to identical
strings, but have differing compositions.
For example,
it is possible that ``canadian prime \textbar\ minist'' and
``canadian \textbar\ prime minist'' become two separate symbols
in the phrase hierarchy, creating no problems for compression,
but confusing the browsing process.  
On the other hand, phase 1 of \remerge explicitly addresses 
this problem when phrase hierarchies are merged.  Recall
that phase 1 expands each symbol into a string of primitives
for comparison.  Symbols that represent duplicate strings are
removed.

Another drawback with \rephine is apparent in
the first screen of \subfigref{fig:rephine-scr-browse}{a},
which presents a list of the word tokens 
in the document collection.  Such an exhaustive list is similar
to a Boolean or ranked query system which lists every
word found in the collection, and is analogous
to a book index containing every word found in the book.
The advantage of \rephine for phrase browsing is not the
word lexicon,
but the relationships formed between words in order
to combine into longer phrases.  Nevertheless, such an opening
window is distracting to users.  Subsequent lists also
suffer from the same problem, but to a lesser degree.
The words in the first window and phrases in the remaining
windows of \figref{fig:rephine-scr-browse} could be
filtered so that unhelpful symbols are removed.  

Case folding and stemming by \prepair reduce the size of the
word lexicon and phrase hierarchy, while also improving browsing 
by providing one word which represents several similar ones.
\chapref{chap:prepair} showed the effect stemming and case folding had 
on the the word lexicon of \wsja 
(\tabref{tab:prepair-lexicon-size} on \pgref{tab:prepair-lexicon-size}).
\tabref{tab:rephine-cfstem} updates those results for
\news.  Recall that the rules for stemming
only apply to words in lower case.  In the third row of
the table, if the part of a word 
in upper case is examined by the stemming algorithm, and
no case folding is applied first, then the word would be
left unchanged.  For example, ``Searching'' stems to ``Search'',
while ``SearchinG'' does not.  The third
column of the table shows the number of primitives
in the phrase hierarchy, while the last column indicates
the number of phrases.  Case folding and stemming has an
effect on the size of the word lexicon, confirming the
results for the smaller \wsja file.  Moreover, 
the average lengths of words in the lexicon 
are slightly longer than the values listed in 
\tabref{tab:prepair-lexicon-size}, as indicated in the fourth
column of the table.

\tab{ccccccccc}
{
Case & \multirow{3}*{Stem} & Unique      & Average & \multicolumn{4}{c}{Percentage of primitive extensions} & Unique \\ \cline{5-8}
fold      &      & primitives  & length  & Left & Right & \multirow{2}*{Both} & \multirow{2}*{None} & phrases \\
          &      &             & (chars) & only & only  &      &      &         \\
}
{
No        & No   & 533,205    & 8.00 & 10.2\% & 11.3\% & 18.5\%  & 60.0\% & 8,464,282 \\
Yes       & No   & 448,278    & 8.10 & \D9.6\%& 11.4\% & 18.8\%  & 60.2\% & 8,269,436 \\
No        & Yes  & 401,448    & 7.34 & 10.3\% & 11.7\% & 18.2\%  & 59.8\% & 8,147,217 \\
Yes       & Yes  & 336,739    & 7.46 & \D9.7\%& 11.8\% & 18.1\%  & 60.4\% & 7,940,620 \\
}
{The effect case folding and stemming has on the phrase 
hierarchy created from punctuation-aligned \repair for
\news with method E of \remerge.  The last row corresponds to
the application of both case folding and stemming of word tokens, 
and reflects the arrangement used for the experiments with 
\news in this chapter.}{The effect 
case folding and stemming has on the phrase hierarchy (\news)}
{tab:rephine-cfstem}

In the fifth, sixth, seventh, and eighth columns of the table, 
the set of primitives has been divided into four
groups based on whether or not each one can be extended. 
In all cases, about 60\% of the symbols do not participate as
a component of a higher generation symbol.  However,
simply eliminating these symbols from the opening 
window of \rephine is too extreme.  If a word token 
fails to become part of a phrase, it does not mean that the 
word is not interesting or is infrequent.

\citet{nwp97:acmdl} noticed a similar problem with the
phrase browser \phind.  The
solution adopted was to filter words and phrases based on
word categories.  Words were labelled as {\emph {rare}}, 
{\emph {interesting}}, or {\emph {common}} based on
a set of frequency thresholds.  When the phrase hierarchy
is explored, an expanded phrase is shown only if it
differs from the previous phrase by at least one interesting word.
For example, if ``book'' is an interesting word, but ``the'' 
is a common word, then the expansion of ``book''  
to ``the book'' would not be displayed.  The reason is that adding a common word to 
the original word does not add any value for the user.  
However, ``book'' can be expanded directly to ``the new book'', 
provided the word ``new'' is also an interesting word.
The drawback of this heuristic is that a common word
may still be of interest to users, and a rare word does not
necessarily represent a typographical error.

At the centre of the problem with \sequitur, 
and \repair as well, is that selecting
phrases based on frequency generates too many phrases,
with many differing by only a common word like ``the''.  
\citet{pwcb00:acmdl} proposed the solution of combining
\sequitur with \kea, since \kea was known to select
too few phrases by itself.  Likewise, the combination of
\repair with a mechanism similar to \kea may improve the number of
phrases presented by \rephine.
A more detailed look at the phrases created by \repair
for \rephine is provided next.

\newsubsection{Comparing \rephine with Noun Phrases}{subsec:rephine-nlp}

Another part of \rephine that is worth evaluating is the usefulness of 
the phrases presented to the user.  The quality of the
phrases is not related to precision or recall because 
the primary concern is the type of phrases available for
browsing, and not the result from searching.

\citet{wolff80:ls} produced phrases through a recursive
pairing mechanism called \mka.  In \mka, pairs are chosen based on
the part-of-speech categories assigned to the word tokens.
The quality of the phrases was compared to the phrases
selected by a human linguist.  In a similar fashion, the
phrases made available by \repair can be compared 
with the ones identified by a reference program employing 
natural language processing (NLP) techniques.

As with other IR systems based on NLP techniques, the basis
for these experiments is that a user is more interested in a
complete noun phrase such as ``prime minister'' than 
a broken phrase like ``Canadian prime''.  The aim of the experiments is to
report on the probability that a symbol in the phrase hierarchy
is a noun phrase, and the probability that noun phrases
exist in the phrase hierarchy.
Noun phrases are identified through an NLP
technique called \linkgram \citep{st91:tech} embodied in a
program by the same name, 
available from \myurl{http://www.link.cs.cmu.edu/link/}.  The
latest version of the software is 4.1, dated 
August 2000.

The \linkgram system processes a sentence at a time.  
In the first step, individual words are assigned to part-of-speech 
categories by consulting a static dictionary.
The categories of novel words are guessed through
a technique the authors call ``morpho-guessing''.  Words have links to
both sides for combining with other words in the sentence.
For example, the two words ``the book'' can merge because of
a right-facing ``D$+$'' link on the word ``the''  
and a corresponding left-facing ``D$-$'' link on the noun
``book''.  The ``D'' in both cases symbolise a determiner
such as ``the'' and ``an''.
An edge is drawn between these two words to indicate
that they are compatible with each other.  When a grammatical
sentence is correctly parsed, a valid {\emph {linkage}} is said
to be formed.  Several requirements 
must be satisfied in order to obtain a linkage.  One of these
requirements is that the graph formed must be connected, 
with all of the words in the sentence reachable.

Subsequent work on the system \citep{gls95:tech} permits 
{\emph {null links}} and post-processing of valid linkages with a
phrase parser for identifying constituents such as noun phrases and
verb phrases.  If a sentence cannot be successfully parsed, then 
words in the sentence are removed incrementally until a successful
parsing can be found.  When the sentence is finally parsed, each
removed word introduces a null link into the linkage.  When a linkage
for a sentence has been established, the post-processing phrase
parser combines words to form constituents of one or more words.
On-line documentation for the phrase parser\footnote{Available
at \myurl{http://www.link.cs.cmu.edu/link/ph-explanation.html}.}
reports that the constituents determined by the parser were
75\% correct on the Penn Treebank 
text\footnote{The Penn Treebank project is at 
\myurl{http://www.cis.upenn.edu/~treebank/}.}, 
which is newspaper text already annotated linguistically.

A sample parsing of a headline from \wsja 
is shown in \figref{fig:rephine-linkgram}.
In the figure the headline is shown in part (a).  The \linkgram
system considers it a valid sentence since two linkages are 
identified, as shown in parts (b) and (c).  In 
\subfigref{fig:rephine-linkgram}{a}, the ``G'' link is used to
pair ``South'' with ``Korea''.  A ``G'' link connects proper
nouns together, identified by the upper case letters at the beginning
of each of these words.  Similarly, a ``G'' link is inserted
between ``Current'' and ``Account''.  The second linkage case folds
``South'' so that the first ``G'' link is converted to an ``A''
link.  An ``A'' link connects a adjective with the noun it follows.
The ``Ss'' link pairs a subject with a verb, while ``Ost'' edges
connect a transitive verb to a singular object.  So, \linkgram
incorrectly treats ``'s'' as a verb in both linkages.
Linkages are scored so that ones with less null links or a lower edge
length total are displayed first.
The constituents for the linkage in (b) is
presented in \subfigref{fig:rephine-linkgram}{d}.  At the beginning
of each constituent is its classification.
The abbreviation ``VP'' denotes a verb phrase, 
while ``NP'' is a noun phrase.

\fig{
\begin{tabular}{c}
{\texttt {South Korea's Current Account}} \\[0.5ex]
(a) News headline \\[1.5ex]
\includegraphics*{./linkage-1.eps} \\[0.5ex]
(b) First valid linkage \\[1.5ex]
\includegraphics*{./linkage-2.eps} \\[0.5ex]
(c) Second valid linkage \\[1.5ex]
\parbox{5cm}{\begin{tabbing}
xxx\=xxx\=xxx\=xxx\= \kill
(S (NP {\texttt {South Korea}}) \\
 \>(VP {\texttt {'s}} (NP {\texttt {Current Account}})))
\end{tabbing}
} \\
(d) Constituents derived from the first linkage \\
\end{tabular}
}{An example parsing of a headline from \wsja using the \linkgram
system.  The headline is shown along the top as part (a).  Two
linkages are determined by \linkgram, which are shown in (b) and (c).
The linkage from (b) is post-processed for constituent identification,
yielding the output in (d).  Noun phrases are marked with ``NP''.}{Example
parsing of a headline with \linkgram (\wsja)}
{fig:rephine-linkgram}

As with previous work in IR and NLP, the focus of the 
phrase comparison is with noun phrases.  
The number of unique phrases that \linkgram found in \wsja was
compared with \rephine's phrase hierarchy.  Since only grammatical
sentences can produce linkages, the test data of news
articles were filtered so that sequences of words that are not 
expected to give many noun phrases were omitted.  First, \sgml tags
were removed and multiple consecutive whitespaces collapsed into
a single space.  Sentences were given to \linkgram one at
a time.  Then, the constituent analysis was determined
by the phrase parser.

As with the earlier experiments to measure recall, the \wsja document was
used as test data.  The computing resources of the \linkgram program were
quite high.  It required just over 5 hours
to process the first \mib{1} of \wsja
on a \vipe test machine.  This time included
the above steps, starting from the \sgml tag filtering, and ending
with the identification of constituents.  Note that
the amount of work performed by \linkgram is more than what is
necessary to identify noun phrases.  Other systems, such as the
\keyphind browser of \citet{gpwnf99:dss}, simply categorise words
according to their parts-of-speech, and then form noun phrases based on a set
of fixed rules.  Even so, \citeauthor{gpwnf99:dss}\ required 4 days to process
\gib{1} of text on a Pentium 233 MHz machine running Linux.
In contrast, \linkgram must first successfully find a linkage for a
sentence, and only when this is done, can it isolate the noun phrases.

After \linkgram completes, constituents which are noun phrases,
and are not composed of other phrases, were
recorded.  The noun phrases were case folded
and then stemmed with the Porter stemming algorithm employed
by \prepair.  In order to comply with the 16 character per word
limitation imposed by \prepair, spaces were inserted into long
words.  Finally, a list of all of the distinct noun phrases in
\wsja was prepared.

\figref{fig:rephine-sgmlsample} illustrates how a news article
in \wsja was parsed, with linefeeds altered to improve display.
Text that was stripped from the article is shown shaded.  All \sgml tags
are removed, as well as text within most tags.  Only text
between opening and closing {\tl}HL{\tg} and {\tl}TEXT{\tg} tags
were retained, since the headline and the body of the article has
the best chance of providing complete sentences.  The noun phrases
found by \linkgram for this example are shown in boxes.  The headline used
as an example in 
\figref{fig:rephine-linkgram} was taken from the article shown in
\figref{fig:rephine-sgmlsample}.  

\fig{
\begin{tabular}{l}
\input{./wsja_sample_parse.sgml}
\end{tabular}
}
{An example article from the \wsja document, with noun
phrases indicated with outlined boxes.  Shaded boxes
represent parts of the article that were removed prior to
applying \linkgram.  The article has been formatted with
linefeeds for improved display.}{Example article
with noun phrases identified (\wsja)}{fig:rephine-sgmlsample}

The \linkgram system successfully found 147,294 grammatical
sentences in \wsja.  Within these sentences, 744,721 noun
phrases were identified, of which 216,661 were unique
after stemming and case folding.  These unique
phrases were divided into groups based on their lengths, in words,
and compared with the phrases available to \rephine, which were
also grouped according to length.  The phrases for \rephine are
identical to the ones reported in \tabref{tab:rephine-recall}.
Information about these
two sets of phrases are shown in \tabref{tab:rephine-nlpph-stats}.
The sizes of the sets of phrases are listed by lengths in the
second and third columns.
Note that noun phrases are only found if a valid linkage has been determined.
In the fourth column of
this table, the number of phrases common to the phrase
hierarchy and in the set of noun phrases has been tabulated.
The probability of a symbol in $\Ph$ being a noun phrase is given in fifth
column.  The reverse, the
probability of randomly selecting a noun phrase which is in
the phrase hierarchy, is listed in the last column of the table.  

\tab{cccccc}
{
Phrase      & Number of        & Number of    & Size of                & Probability & Probability \\
length      & phrases in       & unique noun  & $\Ph \cap {\text{NP}}$ & $\Ph$ has & noun phrase \\
            & hierarchy ($\Ph$)& phrases (NP) &                        & noun phrase & in \Ph \\
}
{
{\D\D}1     & \D39,637  &    13,172 &        13,172 & 0.332 &    1.000 \\
{\D\D}2     & \D79,693  &    78,527 &        21,692 & 0.272 &    0.276 \\
{\D\D}3     & \D60,825  &    72,492 &       \D6,350 & 0.104 &    0.088 \\
{\D\D}4     & \D25,245  &    32,419 &       \D1,578 & 0.063 &    0.049 \\
{\D\D}5     & \D10,932  &    12,304 &       \D\C397 & 0.036 &    0.032 \\
{\D\D}6     & \D\D4,998 &    \D4,939 &      \D\C144 & 0.029 &   0.029  \\
{\D\D}7     & \D\D2,895 &    \D1,696 &     \D\C\D32 & 0.011 &   0.019  \\
{\D\D}8     & \D\D1,686 &    \D\C662 &    \D\C\D\D8 & 0.005 &   0.012  \\
{\D\D}9     & \D\D1,180 &    \D\C247 &    \D\C\D\D2 & 0.002 &   0.008  \\
10+         & \D\D4,454 &    \D\C203 &    \D\C\D\D1 & 0.000 &   0.005  \\[1.0ex]
Total       & 231,545   &    216,661 &       43,376 & 0.187 &   0.200 \\
}{Quantifying the quality of symbols in the phrase hierarchy of
\wsja by calculating the intersection of noun phrases, as identified by 
\linkgram, with the phrase hierarchy.  Symbols have been
divided into 10 groups, based on length (in words).  The second and
third columns provide insight into the number of symbols (or 
noun phrases).  The intersection of these two columns are reported
in the fourth column.  The last two columns indicate the proportion of
phrases in $\Ph$ which are noun phrases (fourth column divided
by the second column), and the probability a noun phrase also being in
$\Ph$ (fourth column divided by the third column), 
respectively.}{Comparing the phrase hierarchy against the
noun phrases (\wsja)}{tab:rephine-nlpph-stats}

As \tabref{tab:rephine-nlpph-stats} shows, the majority of phrases in
the phrase hierarchy and noun phrases from \linkgram are
2 or 3 words in length.  The overlap between the two sets
is around 28\% for phrases of length two for both measures, 
a somewhat disappointing result.  Worse, as the phrases
increase in length, there is even less correlation with the phrase 
hierarchy, as the last two columns show.  This difference is partly due to the methods used by \repair
and \linkgram to build phrases.  The \linkgram system operated on sentences
individually, so phrases were guaranteed to be aligned on sentence boundaries.
Also, \sgml markup did not appear in any of the sentences, because of the
preliminary filtering.  On the other
hand, while punctuation-aligned \repair encourages alignment with punctuation,
phrases can still span sentences, since they are not processed individually.
Furthermore, since \repair was applied on the original version of \wsja,
any phrase that contained markup would not have a corresponding phrase
from \linkgram.  The analysis in \tabref{tab:rephine-nlpph-stats}
is based solely on the existence of phrases, and takes no account of the
frequency.  Because \repair finds common phrases, it is likely 
that the scores would
be considerably better if the probabilities were assessed over the 
sequence instead 
of over the phrase hierarchy.

Making phrases available in the phrase hierarchy is only half of \rephine's
purpose since matching contexts must also be located in
the reduced sequence.  Using the same methodology as established for
\tabref{tab:rephine-recall}, the recall of the phrases that exist
in both the phrase hierarchy and the list of noun phrases is presented
in \tabref{tab:rephine-recall-nlp}.  This time, rather than
presenting recall values for all symbols in the phrase
hierarchy, only the ones that were also noun phrases identified
by \linkgram are considered.  The second column of this
table lists the number of noun phrases that were also 
in the phrase hierarchy, copied from the fourth column of 
\tabref{tab:rephine-nlpph-stats}.  In \tabref{tab:rephine-recall-nlp},
as with the earlier experiment, 
the median recall value for each phrase length group is reported.
The mean and standard deviation of
each set of recall values is also shown.

\tab{ccccc}
{
Phrase      & Size of                & \multicolumn{3}{c}{Distribution of recall values} \\ \cline{3-5}
length      & $\Ph \cap {\text{NP}}$ & Median       & Mean   & Standard   \\
            &                        & recall       & recall & deviation  \\
}
{
{\D\D}1     &   13,172 & 1.000 & 1.000 & 0.000\\
{\D\D}2     &   21,692 & 0.800 & 0.720 & 0.298\\
{\D\D}3     &  \C6,350 & 1.000 & 0.787 &  0.271\\

{\D\D}4     &  \D1,578 & 1.000 & 0.845 & 0.242\\

{\D\D}5     &  \D\C397 & 1.000 & 0.831 &  0.264\\
{\D\D}6     &  \D\C144 & 1.000 & 0.849 & 0.250\\

{\D\D}7     & \D\C\D32 & 1.000 & 0.853 & 0.258\\
{\D\D}8     &\D\C\D\D8 & 1.000 & 1.000 &  0.000\\
{\D\D}9     &\D\C\D\D2 & 1.000 & 1.000 & 0.000\\
10+         &\D\C\D\D1 & 1.000 & 1.000 & 0.000\\[1.0ex]
Overall     &   43,376 & 0.980 & 0.889 &  0.158\\
}{Recall effectiveness of the noun phrases in the phrase hierarchy
for \wsja.  The median has been calculated
based on phrase length (in words)
in the third column, with the mean, and the
standard deviation indicated in the last two columns.}
{Recall of noun phrases in the phrase hierarchy (\wsja)}
{tab:rephine-recall-nlp}

The results from \tabref{tab:rephine-recall-nlp} mirror those
of \tabref{tab:rephine-recall}, but note the small size of the
sample for phrases of length 7 or more.  Again, the average 
recall is above 0.7, and more than half
of the symbols had perfect recall.  The standard deviation 
is wider and affects more groupings than in 
\tabref{tab:rephine-recall}.  

\newsection{Tools for Retrieval}{sec:rephine-summary}

The \rephine system allows users to perform two inter-related
tasks.  The first of these is phrase browsing.  Starting
from a list of the distinct words found in the document, the user can
either extend the current symbol by adding symbols to the left or 
to the right, or decompose
the current symbol into its two constituents.  A graph structure
with one symbol from the phrase hierarchy at each node
was established for the purpose of phrase browsing.  Each node
has six pointers for navigation and defining its structure.

Once the user has found a phrase of interest, the graph
structure is consulted again to determine the complete set of phrases
containing the chosen one.  This set of symbols, $\Criteria$,
is the basis for the search in the reduced word sequence to determine 
all of the locations at which any symbol in $\Criteria$ appears, along
with a suitable context for each.  The second job of \rephine
is to interactively display this set of results one by one to the user.
Each result consists of word tokens which have been stemmed and
case folded.  The non-words that follow each word token are
also initially unavailable.  Each presented result can be refined by the 
user by requesting modifiers from the three modifier streams, in any
combination.  If at this stage, another symbol of interest has been
isolated, it can be passed back to the phrase browsing part of 
\rephine.

In this chapter, \rephine has been demonstrated with a document collection 
\mib{1,000} in size on a \vipe.  Before considering larger collections, note that
only three of the seven streams available to \rephine must reside
in memory.  The four remaining streams are compressed in blocks so that
only parts of them are read from disk into memory.  The phrase hierarchy
and the two lexicons occupy 1.8\% of disk space compared to the size of
\news, and all of them must be accessible by \rephine in decoded form.
The complexity and size of the phrase hierarchy ensures that it requires 
the most memory during browsing after it has been fully decoded.

Based on these observations, several
approaches may be selected in order to browse larger collections.
First, \secref{sec:remerge-append} showed how \remerge can be extended
so that documents are appended to a compressed representation while
keeping the phrase hierarchy static.  Experiments demonstrated this
technique with character-based \repair, and an analysis of the changes
necessary for punctuation-aligned \repair was discussed.  In particular,
a method of handling novel words is required.  Second, either the maximum
generation or the maximum symbol length can be limited so that the
phrase hierarchy only contains moderately long phrases.  Third, while all
word tokens must exist in the phrase hierarchy as primitives, additional
heuristics, such as those offered by \kea and \linkgram, 
may be used when building the phrase hierarchy.  Instead of reducing
the number of phrases shown by \rephine, this approach would prevent
certain symbols from being added to the phrase hierarchy.  Finally, any
combination of these three methods can be used in order for larger document
collections to be compressed by \repair and \remerge, and subsequently
browsed.

The \rephine implementation has not reached
the stage where it could sensibly be made the 
subject of a user study.  As with \sequitur, too many symbols 
exist in \repair's phrase hierarchy, and some filtering is
necessary to improve usability.  The removal of symbols
through NLP techniques has been trialled by other IR systems, and
it is likely that \rephine would be more feasible if similar 
techniques could be employed.  An extension of this idea is to 
rank symbols using some metric so that those that are expected to
be the most useful are presented at the top.  Since there are
reasons why users would want frequent and
and infrequent phrases, simply sorting symbols
by frequency would not help.

User studies have been conducted to quantify the usefulness
of phrase browsing retrieval systems \citep{gpwnf99:dss, jp01:jcdl}, 
with some results being in favour of phrase browsing, and some less
conclusive.  \citet{wm01:sigirforum} concluded that
more work is still required in the area of phrase browsing
evaluation with users.  But even if the suggested changes 
were adopted for \rephine, it is expected that
a comparative user study would not favour \rephine.  While providing
users with a phrase browser was one of the aims of this development, the
necessary tension between browsing and compression inevitably means
that \rephine is not as careful as a dedicated phrase
searching mechanism.  One necessary improvement appears to be
achieving higher recall.
It is possible that the solution lies in
closely regulating word-based \repair's phrase selection 
heuristic.  As word-aligned \repair showed, preventing
certain pairs of symbols from joining does not necessarily
degrade compression effectiveness (see 
\figref{fig:prepair-bargraph-all} on 
\pgref{fig:prepair-bargraph-all}).

The benefits to
compression are easier to quantify than the benefits of phrase
browsing.  Since the latter quantity cannot be measured,
it is also difficult to assess when a reasonable compromise
between the two has been achieved.
Instead, the goal of this thesis has been to 
identify areas where compression and retrieval effectiveness and
efficiency are lost, so that one can be adjusted in favour
of another, as necessary.

\rephine is built on top of a compression mechanism, a structure which
limits its ability to browse and retrieve data.  
Throughout this thesis, the amount of compression effectiveness
lost in favour of faster retrieval times has been highlighted as
a means of compromising between the two competing sets of constraints.  In this
chapter, the current drawbacks with \rephine
represent the amount of
retrieval functionality lost in favour of supporting the
lossless compression of a document collection
with the underlying mechanism.
Employing NLP techniques with \rephine
may be the ideal solution for compensating
for \rephine's loss in retrieval effectiveness.  As mentioned
above, at the same time, NLP techniques may also reduce the
number of symbols that reside in memory during browsing.

Regardless of the improvements made to \rephine, phrase
browsing in general is one of many IR techniques.  This chapter
has also provided a glimpse into other querying techniques, 
including ranked queries,
proximity queries, and Boolean queries.  The system a user 
chooses depends on factors such as how familiar the user is
with the document, whether query terms include phrases, and whether
the order and the positions of the words in a phrase matter.
As \citet{gpwnf99:dss} pointed out, no single IR method 
is perfect and the most viable solution 
might be to provide ``several search tools that can 
make up for each others' shortfalls''.  

This chapter concludes the discussion in this thesis
about the balance between compression and phrase browsing.
Before summarising the thesis in \chapref{chap:summary},
the next chapter discusses a separate topic related to the
compression of small \html files.

