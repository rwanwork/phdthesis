\newchapter{Text Compression}{chap:tc}

Compression is the fundamental area of text management
which recognises and removes redundancies, and represents 
the remaining content in a more
economical way so that the underlying information is retained.
The compressed document is then transmitted 
or stored for later use.  The receiver of the compressed document
must reverse the transformation in order to obtain the original.
The reduction in document size usually comes at the expense
of another factor, such as processing time for the conversion.

This chapter provides some background information on 
compression, and is structured as follows.  \secref{sec:tc-fundamental}
presents fundamental theory which forms the foundation of all
compression methods.  \secref{sec:tc-compsys} 
gives an overview of compression systems, and how they are
divided into its three parts:
model, probability estimator, and coder.  \secref{sec:tc-model}
describes models and probability estimators in the context of
three classes of
algorithms prevalent in text compression:  dictionary-based, statistical,
and block sorting.  \secref{sec:tc-coder} gives insight into
static and entropy coding techniques.  The focus of this 
chapter shifts to practice in \secref{sec:tc-practice},
when some complete compression systems are described.
Details of the framework for the experiments performed
throughout this thesis are covered in \secref{sec:tc-expt}.
Finally, this chapter concludes in \secref{sec:tc-summary}
with a discussion of the
effect compression has on searching.

\newsection{Fundamentals in Compression}{sec:tc-fundamental}

Some definitions and background about
the area of information theory as the foundation for compression
are required as a first step.  We suppose that a
{\emph {message}} is created by some
{\emph {source}}.  Whether a message is created for transmission
or storage for archival purposes, its recipient is the 
{\emph {receiver}}.  The {\emph {channel}} is the medium between
the source and the receiver, and can be a communications line or
a disk drive.  Between the source and the channel is the process
which performs compression:  the {\emph {compressor}}.  Its counterpart
is the {\emph {decompressor}}, employed by the receiver to 
reconstruct the message.

The smallest unit in the message is a
{\emph {symbol}}.  The set of all possible symbols available
to the source for building messages is the {\emph {alphabet}},
denoted as $\Sigma$.
The symbols in the alphabet depend on the source and the type
of messages being transmitted.  For example, if the source
only produces telephone numbers, then an alphabet
which includes the 10 digits and the hyphen is sufficient.  
The size of this
alphabet (denoted as $|\Sigma|$) 
is 11.  In this thesis, all messages are encoded using the \eascii
character set \citep{latin1:std}, an extension to the \ascii
character set \citep{ascii:std}.  The \ascii alphabet
consists of 128 symbols, including the 26 letters of the English
alphabet in both cases, the 10 digits, and a range of
punctuation marks and non-printing characters.  
The \eascii alphabet adds an additional 128 symbols to
\ascii, to permit the encoding of text in most western European
languages.  The symbols in a message encoded with \eascii are also
called {\emph {characters}}.

Each symbol in a message conveys some information to the 
receiver.  The amount of 
information conveyed depends on the probability of the
symbol occurring in the message.
Since a highly probable symbol offers little
surprise to the receiver, the information
content is small.  On the other hand, an unlikely symbol 
has a high information content.
If $\alpha$ represents a symbol in the alphabet, 
then the relationship between its information content, 
$I(\alpha)$, and its probability, 
$P(\alpha)$, was defined by \citet{shannon48:bell} as:

\neweqn{I(\alpha) = -\log_2 P(\alpha)\,.}{eqn:tc-info}

In \eqnref{eqn:tc-info}, any base can be selected for
the logarithm.  However, since compression is usually performed
with computers based on the binary counting system,
a base of two is a convenient choice, and the information 
content is measured in units of bits.  For example, in a message
representing a sequence of throws of a six-sided dice,
the probability of any value appearing is $1/6$.  The
information content of each of the 6 events is
$-\log_2 (1/6) = 2.58$ bits.

If the information content of all of the symbols in some 
alphabet $\Sigma$
are averaged, then the {\emph {zero-order entropy}} of the 
source is obtained 
\citep{shannon48:bell}:

%% If the frequency of $\alpha$ and
%% the length of the message is taken into account, then the number
%% of bits per symbol (bps) for each instance of $\alpha$ in the 
%% message is given by:

%% \neweqn{\text{bps}(\alpha) = P(\alpha) \cdot I(\alpha) = - P(\alpha) \cdot \log_2 P(\alpha)}{eqn:tc-info2}

\neweqn{H = \sum_{\alpha \in \Sigma} P(\alpha) \cdot I(\alpha)
= -\sum_{\alpha \in \Sigma} P(\alpha) \cdot \log_2 P(\alpha)\,.}
{eqn:tc-entropy}

If the probabilities are derived from a specified message, rather than the
source, then the quantity described in \eqnref{eqn:tc-entropy}
is the per-symbol zero-order {\emph {self-information}} of the message.  
\citeauthor{shannon48:bell} demonstrated that the entropy
is the lowest limit (in bits per symbol) with which a message 
from the source can, on average, be coded.

The definitions for information content and entropy depend on
two factors.  First, they assume that the probability of the
symbols in the alphabet can be obtained.  While other definitions
of entropy exist, this definition assumes a Markov source
that generates the symbols in a message independently and 
identically distributed.  That is, the value of a symbol
does not affect the values of neighbouring symbols.  Of course,
this is not true for the letters that appear in an 
English message, for example.  Despite these requirements, the 
definition of entropy has served as a guide for work in
compression for many decades.  A more thorough coverage of
information theory can be found in books such as 
\citet{ct91:book}.

\newsection{Compression Systems}{sec:tc-compsys}

Many types of compression systems exist in order to satisfy
a range of needs.  A system is {\emph {lossless}}
if the decoder is able to recreate the original message
exactly.  A system that does not achieve this goal is
{\emph {lossy}}.  Lossy compression is commonly
found in the audio and video compression fields, since
changes to the message that go unnoticed by the human senses
are considered acceptable.  Some work has been conducted in
the area of lossy compression for text.  \citet{ngbpll97:sigcomm}
considered the lossy compression of \html documents by transforming
them to improve bandwidth, provided they rendered correctly
within
a web browser.  \citet{kpt86:spe} applied a syntax-directed
compression scheme to Pascal source code.  As a higher priority 
was given to the program source code, other free-form text such
as comments and extraneous whitespaces were modified as necessary.
Despite these counterexamples, text compression is usually
taken to imply lossless compression, because a missing character or
word from a sentence can greatly alter the meaning of the 
sentence.

Whether a compression system is tailored for audio, video, or
text, it can usually be divided
into three tasks:  modelling, probability estimating, and
coding \citep{rl81:ieeeit}.  
The model receives the message from the source and 
accumulates knowledge about it.  The definition
of entropy earlier was based on independently occurring
symbols, which is not true for most messages.  
If the probability a symbol occurs is based on
neighbouring symbols, it is during the first phase of the 
compression system that this relationship is captured.
Next, the probability
estimator assigns probabilities to symbols in the message to form 
the model, usually
based on the observed frequencies derived from the part of the 
message processed.  
Finally, the coder maps the output of the probability
estimator to a set of codewords based
on the {\emph {channel alphabet}}, which in this thesis
is assumed to be binary.

As shown in \figref{fig:tc-modelcoder}, these three components 
are situated before (and after) the
channel, forming the compressor (and decompressor).  
The decompressor reverses the 
operations performed by the compressor to obtain the original 
message.  In a compression system, these three parts may each
exist in isolation, or be tightly bounded together, with no
apparent division between them.

\fig{\includegraphics*{./modelcoder.eps}}{The separation
of a compression system into a model, probability 
estimator, and coder.}{Separation of a compression system}
{fig:tc-modelcoder}

Each component of the compression system operates {\emph {statically}}, 
{\emph {semi-statically}}, or {\emph {adaptively}}.  A static
algorithm makes use of fixed, pre-determined statistics to process
the input.  These statistics are presumed to be available at both ends
of the
channel, so that explicit transmission is unnecessary.  The
drawback to such a regime is that it is only suitable when
the static probabilities match those found in the input.  When
there is a mismatch, compression effectiveness degrades.

On the other hand, an adaptive algorithm learns about the
input throughout the compression process, and typically 
commences with bland statistics.  As compression progresses,
these statistics are updated, 
in an attempt to match those possessed by the message.  
Like static algorithms, adaptive ones have the advantage of 
requiring only a single pass.  Furthermore, adaptive mechanisms
are suitable for any type of message.

Semi-static algorithms encode the message in two
passes.  The first pass collects information about the message.
That knowledge is then applied in the second pass.
The primary difference between a static algorithm
and a semi-static equivalent is that the information that exists
on both sides of the channel in the static approach must be
explicitly transmitted by the semi-static one.  Thereafter,
the process is the same.

Closely related to these three algorithm classifications are
the categories {\emph {on-line}} and {\emph {off-line}}.  An
on-line algorithm processes a symbol, and then immediately commits
it to the channel as a stream of bits before reading in the 
next symbol.  An off-line
mechanism has the luxury of examining a significant block of the
input before sending anything to the decoder, at the cost of
buffering symbols in memory.

In the next two sections, models and coders are discussed in
greater detail through sample algorithms.  Coverage of
probability estimators is absorbed into the 
description of models, because of their close relationship.

\newsection{Modelling}{sec:tc-model}

The model analyses and learns about
the message, and in the process, identifies the redundancies
that allow the message to be represented in an 
alternate, yet cost-effective manner.
Three different models are considered in this chapter:  dictionary-based,
statistical, and block sorting.

A short passage has been 
chosen as the example for most of this thesis.  The 
``Woodchuck'' tongue twister from
\citet[pg.~30]{oo59:book} is shown in \figref{fig:tc-woodchuck}.
All words in the message have been case folded, and all punctuation
has been removed.  In addition, all whitespace characters have
been collapsed into a single space, as indicated in the figure
as ``\tvs''.  While linefeeds are used in the figure to
improve display, they are not considered to be part of the message.

\fig{
\begin{tabular}{l}
{\texttt {how{\tvs}much{\tvs}wood{\tvs}could{\tvs}a{\tvs}woodchuck{\tvs}chuck{\tvs}}} \\
{\texttt {if{\tvs}a{\tvs}woodchuck{\tvs}could{\tvs}chuck{\tvs}wood{\tvs}}} \\
{\texttt {as{\tvs}much{\tvs}wood{\tvs}as{\tvs}a{\tvs}woodchuck{\tvs}would{\tvs}chuck{\tvs}}} \\
{\texttt {if{\tvs}a{\tvs}woodchuck{\tvs}could{\tvs}chuck{\tvs}wood}} \\
\end{tabular}
}{The ``Woodchuck'' message.  A space is indicated by a ``\tvs''
character.  Linefeeds have been inserted to improve readability.}
{The ``Woodchuck'' message}{fig:tc-woodchuck}

\newsubsection{Dictionary-Based Models}{subsec:tc-dict}

Dictionary-based models (or {\emph {macro encoding}} mechanisms
\citep{ss82:jacm}) seek
to represent the input message in a more cost effective way by building a
dictionary of phrases.  Sections of the message are substituted by
references to the dictionary's entries.  Compression is achieved if 
the cost of expressing the dictionary is less than the total number
of symbols it replaces.

A static dictionary-based model would hard-code the phrases in the
compressor and the decompressor.  The dictionary can also be built
semi-statically or adaptively so that each compressed message has
an associated dictionary.  In this case, either the 
dictionary must
be explicitly sent to the decompressor, or enough information must
be transmitted for the decompressor to reconstruct the dictionary.
The overall effectiveness of a dictionary-based model depends on the 
frequency with which dictionary phrases occur, and their lengths.

This section focuses on two dictionary-based models, both of which
operate adaptively and on-line.  The first is the \lza algorithm
by \citet{zl77:ieeeit}, which creates a dictionary by maintaining
a window of recently seen symbols.  Each symbol in the window
is considered to be the beginning of a phrase of arbitrary length.  
Sequences (or strings) of 
symbols in the message are replaced by phrase references.

\algo{\begin{minipage}{\textwidth}
\DontPrintSemicolon

\SetFuncSty{textnormal}
\SetArgSty{textnormal}
\SetDataSty{textnormal}

\SetKwInOut{Input}{input}
\SetKwInOut{Output}{output}

\SetKwFunction{appendString}{appendString}
\SetKwFunction{findLongestMatch}{findLongestMatch}

\SetKwData{poffset}{\var{prev\_offset}}
\SetKwData{offset}{\var{offset}}

\Input{input message $\mathcal{M}$, window size $N$}
\Output{encoded message $\mathcal{M}'$}

\BlankLine

\lnl{algline:tc:lza:1}
\poffset $\leftarrow$ 0; \offset $\leftarrow$ 0 \;
\lnl{algline:tc:lza:2}
$i \leftarrow 0$; $\omega \leftarrow \lambda$ \;

\lnl{algline:tc:lza:3}
\For{$i < |\mathcal{M}|$}{
\lnl{algline:tc:lza:4}
  \appendString{$\omega$, $\mathcal{M}[i]$} \;
\lnl{algline:tc:lza:5}
  $\offset \leftarrow$ \findLongestMatch{$\omega$, $\mathcal{M}[i-N]$ ... $\mathcal{M}[i-1]$} \;
\lnl{algline:tc:lza:6}
  \If{$\offset = 0$}{
\lnl{algline:tc:lza:7}
    output $\langle{\poffset}, |\omega|-1, \mathcal{M}[i]\rangle$ to $\mathcal{M}'$ \;
\lnl{algline:tc:lza:8}
    $\poffset \leftarrow 0$ \;
\lnl{algline:tc:lza:9}
    $\omega \leftarrow \lambda$ \;
  }
\lnl{algline:tc:lza:10}
  \Else{
\lnl{algline:tc:lza:11}
    $\poffset \leftarrow \offset$ \;
  }
\lnl{algline:tc:lza:12}
  $i \leftarrow i + 1$ \;
}

\BlankLine

\lnl{algline:tc:lza:13}
\Return ($\mathcal{M}'$) \;
\end{minipage}
}{Algorithm for \lza.}
{\lza}{alg:tc-lza}

\algref{alg:tc-lza} presents the algorithm for \lza.  
The input and output messages are represented as $\mathcal{M}$
and $\mathcal{M}'$, respectively.  The size of the window is
denoted as $N$.  As each symbol is read, 
it is appended to
the string $\omega$, which is initialised to the empty string
$\lambda$ at the beginning of the algorithm on line 
\ref{algline:tc:lza:2}.  On line \ref{algline:tc:lza:5}, a 
search for $\omega$
is performed on the dictionary of phrases for the set of strings
that have $\omega$ as a prefix.  A match is indicated by a
distance offset backwards, counting from the end of the window.
When $\omega$ cannot be found, an offset of $0$ is returned from
the search, and the string $\omega$ is encoded as a 3-tuple of
the form ${\langle}d,l,s{\rangle}$.  The location in the window
of the last successful match is specified by $d$, while the length
of this match (one less than the length of $\omega$) is specified
by $l$.  The last
symbol in $\omega$ that caused the mismatch is transmitted directly 
as $s$.
As a special case, when the string $\omega$ only contains one symbol,
$\alpha$, the novel symbol is replaced with the tuple:  
${\langle}0,0,\alpha{\rangle}$.  

The size
of the window is restricted to some value $N$, and as more of the message
is processed, symbols at the beginning of the window are removed 
so that the window is kept recent.  The eradication of 
old symbols and the addition of
new ones give the illusion of a window sliding over the message.
So, mechanisms based on \lza are sometimes referred to as ``sliding window''
compression schemes.  For efficiency reasons, $N$
is usually chosen as a power of 2.

\tabref{tab:tc-lz77} illustrates the \lza algorithm with an example.  The
sample text is formed from part of the ``Woodchuck'' message, 
``woodchuck{\tvs}chuck{\tvs}if''.  Each line in the figure indicates
one 3-tuple.
The dictionary window is indicated by a shaded box, while $\omega$ is
placed in a framed box.  The window is $N=8$ symbols in length.
In row (a) of the table, the window is initially empty, and
the tuple ${\langle}0,0,$w${\rangle}$ is transmitted, to get the first
character through.  Later, 
in row (c), the second occurrence of the letter ``o''
creates an $\omega$ two characters in size.  At the second to last step in
the table, $\omega$ is 7 characters in length and the longest
match is 6 characters from the end of the window.  At this stage, the
first two characters in the message, ``wo'', have departed from the
window.

\tab{ccc}
{ & Tuple & Message \\}
{
(a) & ${\langle}0,0,$w${\rangle}$      & {\texttt {\nogreyboxtt{w}oodchuck{\tvs}chuck{\tvs}if}} \\ [0.5ex]
(b) & ${\langle}0,0,$o${\rangle}$      & \greyboxtt{w}{\texttt {\nogreyboxtt{o}odchuck{\tvs}chuck{\tvs}if}} \\ [0.5ex]
(c) & ${\langle}1,1,$d${\rangle}$      & \greyboxtt{wo}{\texttt {\nogreyboxtt{od}chuck{\tvs}chuck{\tvs}if}} \\ [0.5ex]
(d) & ${\langle}0,0,$c${\rangle}$      & \greyboxtt{wood}{\texttt {\nogreyboxtt{c}huck{\tvs}chuck{\tvs}if}} \\ [0.5ex]
(e) & ${\langle}0,0,$h${\rangle}$      & \greyboxtt{woodc}{\texttt {\nogreyboxtt{h}uck{\tvs}chuck{\tvs}if}} \\ [0.5ex]
(f) & ${\langle}0,0,$u${\rangle}$      & \greyboxtt{woodch}{\texttt {\nogreyboxtt{u}ck{\tvs}chuck{\tvs}if}} \\ [0.5ex]
(g) & ${\langle}3,1,$k${\rangle}$      & \greyboxtt{woodchu}{\texttt {\nogreyboxtt{ck}{\tvs}chuck{\tvs}if}} \\ [0.5ex]
(h) & ${\langle}0,0,${\tvs}${\rangle}$ & w\greyboxtt{oodchuck}{\texttt {\nogreyboxtt{{\tvs}}chuck{\tvs}if}} \\ [0.5ex]
(i) & ${\langle}6,6,$i${\rangle}$ & {\texttt {wo}}\greyboxtt{odchuck{\tvs}}{\texttt {\nogreyboxtt{chuck{\tvs}i}f}} \\ [0.5ex]
(j) & ${\langle}0,0,$f${\rangle}$ & {\texttt {woodchuck}}\greyboxtt{{\tvs}chuck{\tvs}i}{\texttt {\nogreyboxtt{f}}} \\ [0.5ex]
}{Dictionary-based compression using the \lza 
algorithm.  Each row illustrates the string 
$\omega$ being encoded and its associated 3-tuple.  
On the right, the shaded region is the current window, which 
grows until a fixed size and then ``slides'' across the 
text.  The unshaded region is the section of text being 
processed ($\omega$).}{Dictionary-based compression using \lza}{tab:tc-lz77}

%% \fig{\begin{tabular}{llp{7cm}}
%% \multirow{3}*{(a)} &
%% \multirow{3}*{{\textless}0,0,l{\textgreater}} & 
%% \multirow{3}*[0.5pt]{\subfigure[]{\label{fig:tc-lz77a}\input{./lz77-a.pstex_t}}} \\
%% & \\
%% & \\
%% \multirow{3}*{(b)} &
%% \multirow{3}*{{\textless}6,2,a{\textgreater}} & 
%% \multirow{3}*[5pt]{\subfigure{\label{fig:tc-lz77b}\input{./lz77-b.pstex_t}}} \\
%% & \\
%% & \\
%% \multirow{3}*{(c)} &
%% \multirow{3}*{{\textless}13,5,c{\textgreater}} & 
%% \multirow{3}*[5pt]{\subfigure{\label{fig:tc-lz77c}\input{./lz77-c.pstex_t}}} \\
%% & \\
%% & \\
%% \end{tabular}}{Dictionary-based compression using the \lza 
%% algorithm.  The shaded region is the current window, which 
%% grows until a fixed size and then ``slides'' across the 
%% text.}{Dictionary-based compression using \lza.}{fig:tc-lz77}

In terms of memory, the \lza algorithm requires at least space
proportional to $N$ in order to hold the sliding window.  However,
since searching for $\omega$ in the dictionary can be the most
costly operation in the algorithm, additional data structures are used
to improve efficiency.  Some basic data structures include 
sorted lists and binary search trees.  Other possible mechanisms
include tries (also called multi-way tries \citep{sedgewick97:book}) and
hash tables.  A trie is a tree where each node has an outdegree equal to
$|\Sigma|$.  When searching for a string $\omega$, reading each
character results in one transition from the current node to one of its
$|\Sigma|$ child nodes.  A hash table can be implemented so that the
first few characters of each phrase are used to calculate a position in
the table.  The data structure chosen must also take
into account the fact that symbols leave the window as well as join
it.  So, phrases are continually added and removed from the data structure.

In comparison, the decoder is
simpler than the compressor.  While a sliding window must be 
maintained, string searching is not required.  Each tuple in
the compressed message indicates explicitly where in the window
to copy phrases from, and the number of symbols to copy.  Hence, 
decoding is extremely quick.
In fact, the fast decoding time is a trait shared by most 
dictionary-based algorithms.  Reconstructing the message 
by directly copying strings from the dictionary is attractive from
a pragmatic point of view.

Some variations on the \lza algorithm include \lzr and \lzss.  The
\lzr scheme \citep{rpe81:jacm} removes the size of the sliding window so that all
of the message seen so far is used as phrases.  \citet{bell86:ieeec} designed
\lzss to remove the restriction on constant tuple sizes by mixing the
output stream with characters and back pointers.  A bit flag was used
to indicate whether a literal character or a back pointer exists in
the stream.  This work was taken further by \citet{fenwick93:dcc}, who
investigated other lengths of bit flags.

The second dictionary-based algorithm discussed in this chapter
is the \lzb algorithm of \citet{zl78:ieeeit}.  Like \lza,
\lzb is adaptive, operates on-line, and replaces sections of
the message with pointers to the dictionary.  However, rather
than using a sliding window of symbols as a phrase book,
an explicit dictionary is built.  \algref{alg:tc-lzb} lists the
algorithm for \lzb.  As before, symbols are read one at a time and 
appended to the string
$\omega$.  If $\omega$ exists in the dictionary $D$, then the next
input symbol is considered.  But if $\omega$ cannot be found, then it
is replaced with a 2-tuple, ${\langle}p,s{\rangle}$.  The last successful
match for a phrase (that is, $\omega$ without its last character) is
specified by $p$ as a position in the dictionary.  The last
symbol in $\omega$ is encoded explicitly as a literal, $s$.  Whenever
a new phrase has been found, it is added to the dictionary on
line \ref{algline:tc:lzb:8} of the algorithm.
Dictionary entries are indexed from 1 so that
a new symbol, $\alpha$, is encoded as ${\langle}0,\alpha{\rangle}$.
The dictionary is initially empty.

\algo{\begin{minipage}{\textwidth}
\DontPrintSemicolon

\SetFuncSty{textnormal}
\SetArgSty{textnormal}
\SetDataSty{textnormal}

\SetKwInOut{Input}{input}
\SetKwInOut{Output}{output}

\SetKwFunction{appendString}{appendString}
\SetKwFunction{searchDictionary}{searchDictionary}

\SetKwData{ppos}{\var{prev\_position}}
\SetKwData{pos}{\var{position}}

\Input{input message $\mathcal{M}$}
\Output{encoded message $\mathcal{M}'$}

\BlankLine

\lnl{algline:tc:lzb:1}
\ppos $\leftarrow$ 0; \pos $\leftarrow$ 0 \;
\lnl{algline:tc:lzb:2}
$i \leftarrow 0$; $\omega \leftarrow \lambda$ \;

\lnl{algline:tc:lzb:3}
\For{$i < |\mathcal{M}|$}{
\lnl{algline:tc:lzb:4}
  \appendString{$\omega$, $\mathcal{M}[i]$} \;
\lnl{algline:tc:lzb:5}
  $\pos \leftarrow$ \searchDictionary{$\omega$, $D$} \;
\lnl{algline:tc:lzb:6}
  \If{$\pos = 0$}{
\lnl{algline:tc:lzb:7}
    output $\langle{\ppos}, \mathcal{M}[i]\rangle$ to $\mathcal{M}'$ \;
\lnl{algline:tc:lzb:8}
    add $\omega$ to $D$ \;
\lnl{algline:tc:lzb:9}
    $\ppos \leftarrow 0$ \;
\lnl{algline:tc:lzb:10}
    $\omega \leftarrow \lambda$ \;
  }
\lnl{algline:tc:lzb:11}
  \Else{
\lnl{algline:tc:lzb:12}
    $\ppos \leftarrow \pos$ \;
  }
\lnl{algline:tc:lzb:13}
  $i \leftarrow i + 1$ \;
}

\BlankLine

\lnl{algline:tc:lzb:14}
\Return ($\mathcal{M}'$) \;
\end{minipage}
}{Algorithm for \lzb.}
{\lzb}{alg:tc-lzb}

A sample application of the \lzb algorithm on the same 
message as \tabref{tab:tc-lz77} is shown in 
\tabref{tab:tc-lz78}.  Each row in the table shows a section
of text being encoded as a 2-tuple (indicated by 
the column marked ``Tuple'').  The
position in the dictionary of each new string is shown
in the next column, followed by the progress of the compression.
The shaded region represents
text that has already been seen, while the framed box draws
attention to the string $\omega$.  Due to the short example,
the message is encoded only as strings of length 1 or 2.
Since the dictionary gradually grows, the repetition of the
string ``chuck{\tvs}'' is not captured within the text shown, 
and more pointers are required than the equivalent \lza parsing.
Also, in row (k), note that the letter ``k'' is encoded as a
literal, even though it is the second time it has been seen.  The
first time it was encountered, it was part of a string of length
2.

\tab{ccc@{\,}c@{\,}lc}
{ 
& \multirow{2}*{Tuple} & \multicolumn{3}{c}{Current} & \multirow{2}*{Message} \\ 
&                      & \multicolumn{3}{c}{dictionary} & \\
}
{
(a) & ${\langle}0,$w${\rangle}$ &   1  & $\rightarrow$ & {\texttt {w}} & {\texttt {\nogreyboxtt{w}oodchuck{\tvs}chuck{\tvs}if}} \\ [0.5ex]
(b) & ${\langle}0,$o${\rangle}$ &   2  & $\rightarrow$ & {\texttt {o}} & \greyboxtt{w}{\texttt {\nogreyboxtt{o}odchuck{\tvs}chuck{\tvs}if}} \\ [0.5ex]
(c) & ${\langle}2,$d${\rangle}$ &   3  & $\rightarrow$ & {\texttt {od}} & \greyboxtt{wo}{\texttt {\nogreyboxtt{od}chuck{\tvs}chuck{\tvs}if}} \\ [0.5ex]
(d) & ${\langle}0,$c${\rangle}$ &   4  & $\rightarrow$ & {\texttt {c}} & \greyboxtt{wood}{\texttt {\nogreyboxtt{c}huck{\tvs}chuck{\tvs}if}} \\ [0.5ex]
(e) & ${\langle}0,$h${\rangle}$ &   5  & $\rightarrow$ & {\texttt {h}} & \greyboxtt{woodc}{\texttt {\nogreyboxtt{h}uck{\tvs}chuck{\tvs}if}} \\ [0.5ex]
(f) & ${\langle}0,$u${\rangle}$ &   6  & $\rightarrow$ & {\texttt {u}} & \greyboxtt{woodch}{\texttt {\nogreyboxtt{u}ck{\tvs}chuck{\tvs}if}} \\ [0.5ex]
(g) & ${\langle}4,$k${\rangle}$ &   7  & $\rightarrow$ & {\texttt {ck}} & \greyboxtt{woodchu}{\texttt {\nogreyboxtt{ck}{\tvs}chuck{\tvs}if}} \\ [0.5ex]
(h) & ${\langle}0,${\tvs}${\rangle}$ &   8  & $\rightarrow$ & {\texttt {{\tvs}}} & \greyboxtt{woodchuck}{\texttt {\nogreyboxtt{\tvs}chuck{\tvs}if}} \\ [0.5ex]
(i) & ${\langle}4,$h${\rangle}$ &   9  & $\rightarrow$ & {\texttt {ch}} & \greyboxtt{woodchuck{\tvs}}{\texttt {\nogreyboxtt{ch}uck{\tvs}if}} \\ [0.5ex]
(j) & ${\langle}6,$c${\rangle}$ &  10  & $\rightarrow$ & {\texttt {uc}} & \greyboxtt{woodchuck{\tvs}ch}{\texttt {\nogreyboxtt{uc}k{\tvs}if}} \\ [0.5ex]
(k) & ${\langle}0,$k${\rangle}$ &  11  & $\rightarrow$ & {\texttt {k}} & \greyboxtt{woodchuck{\tvs}chuc}{\texttt {\nogreyboxtt{k}{\tvs}if}} \\ [0.5ex]
(l) & ${\langle}8,$i${\rangle}$ &  12  & $\rightarrow$ & {\texttt {{\tvs}i}} & \greyboxtt{woodchuck{\tvs}chuck}{\texttt {\nogreyboxtt{{\tvs}i}f}} \\
(m) & ${\langle}0,$f${\rangle}$ &  13  & $\rightarrow$ & {\texttt {f}} & \greyboxtt{woodchuck{\tvs}chuck{\tvs}i}{\texttt {\nogreyboxtt{f}}} \\ [0.5ex]
}
{Dictionary-based compression using the \lzb
algorithm.  Each row illustrates the string 
$\omega$ being encoded, with the generated 2-tuple 
in the first column.  The position in the dictionary
and the equivalent string of the new tuple is shown in
the second column.
At the end of each row is the progress of the algorithm.
The shaded region indicates text that has already been
processed, while the unshaded region is the section of
text being processed ($\omega$).  Dictionary strings grow 
one character longer at each step.}
{Dictionary-based compression using \lzb}{tab:tc-lz78}

In comparison to the \lza algorithm, \lzb searches for phrases
in an explicit dictionary, rather than a sliding window of
phrases.  The relevance of the dictionary to the text being compressed is
no longer limited by a window size.
Even so, in practice, a limitation on memory usage 
is imposed on \lzb so that the amount of space is bounded.  Also,
while phrases are explicitly added into $D$, note that the addition
of a phrase does not give assurance that it will be used.  

The same data structures described for \lza can also be employed by
\lzb for maintaining phrases.
But since phrases are never deleted from the dictionary, one 
complication inherent in \lza is eliminated by \lzb.  The \lzb algorithm,
though, requires the dictionary to be maintained by both the decompressor
and the compressor.  As a result, decoding with \lzb is usually slower
than \lza.  

In \lzb implementations, several approaches exist when dealing 
with a full dictionary.  First, the dictionary can
be deleted, and re-constructed from scratch.  Second, a least recently
used policy can be employed, so that the oldest phrases are removed to
make space for new ones \citep{tischer87:acsc}.  Alternatively,
if every symbol in the
alphabet is used to initialise the dictionary (such as the case with 
\lzw by \citet{welch84:computer}), then it can be kept static for 
the remainder of the message.  Because of this, the output from
the \lzw algorithm only contains pointers to phrases.  Combinations of
these three policies are possible; for example \lzw rebuilds
the dictionary if compression degrades significantly.

\newsubsection{Statistical Models}{subsec:tc-stat}

Instead of building a dictionary, statistical models directly
estimate the probability for each symbol in the message.  As an
example, suppose symbols are words and the following sentence 
fragment has to be
completed with a word, ``I purchased a book from the ...''. 
Many plausible
guesses exist, but most of them would be sources of
books, such as ``bookstore'' and ``Internet''.  
This conclusion can be drawn from the information provided
by the preceding words.

Generalising the problem to symbols, rather than words,
statistical models combine with probability estimation modules
in order to gather statistics and provide estimates
for each symbol as it is encountered. Restricting the context to 
the preceding symbols permits messages to be processed in a single
pass.   
An {\emph{order-$K$}} context uses the previous 
$K$ symbols to predict the current symbol.  By definition, an 
order-0 context does not use any preceding symbols, while an order of minus-one
assigns the same probability to every symbol.  The \eascii
standard can be thought of as a very simple compression system
based on a fixed minus-one order context.  An order-1 statistical
model predicts
each symbol in the context of one immediately preceding symbol.
Intuitively, the higher
the order, the more accurately the model is able to predict
symbols.  However, if the size of the alphabet is $|\Sigma|$, 
then an increase in the context by one symbol
results in an increase in memory usage by a factor of as much
as $|\Sigma|$, and in
general, for an order-$K$ context, the amount of
memory required is potentially proportional to $|\Sigma|^{(K + 1)}$.

The Prediction by Partial Matching (\ppm) algorithm 
\citep{cw84:ieeec} is one example of a statistical model.  
In order to conserve memory, instead of creating
all possible contexts initially, \ppm records contexts
as they are encountered
in the message.  If a symbol cannot be predicted in the current
context, then the algorithm reduces the context by 
one symbol and tries again, until a context can be found.

Implementations of the \ppm algorithm are typically 
parameterised in terms of the maximum order $K$. 
The encoding of a symbol $\alpha$ is first attempted using
the previous $k = K$ symbols.  If $\alpha$ has not occurred in
this context, then a special {\emph {escape symbol}}
is transmitted to indicate that a context
of length $k - 1$ should be attempted.  This is repeated
until $\alpha$ can be estimated based on a previously seen context, 
or when $k = -1$ and the symbol is encoded
as a literal.

Several variations on the \ppm algorithm exist, differing primarily
in the probability assigned to the escape symbol.  The probability
given to the escape symbol is related to the {\emph 
{zero frequency problem}} \citep{wb91:ieeeit}.  The zero frequency
problem refers to the dilemma of assigning a probability to an
event that has not occurred.  Since there is a chance it could occur
later, novel events that have not yet occurred must still be assigned 
non-zero probabilities.

With respect to \ppm, the probability of the escape symbol is
an instance of the zero frequency problem since the escape symbol
is emitted when a symbol has not occurred in the current context.
Several methods related to this problem have been explored.
Method A of \ppm increases the count of each context by 1 
for the escape probability.  Method B subtracts 1 from the 
counts of every context so that an event is considered to have
occurred only if has happened twice.  The remaining counts
are allocated to the escape symbol.  Methods A and B were described
by \citet{cw84:ieeec}.  Method C \citep{moffat90:ieeec} assigns 
a probability
of $d / (n + d)$, where $d$ is the number of distinct symbols seen
in the current context, and $n$ is the total number of symbols 
encountered in that context.  Method D \citep{howard93:phd}, 
estimates the probability of the escape symbol as $d / (2n)$.  
Implementations of \ppm
which make use of the latter two methods generally achieve better
compression effectiveness than variants using methods A and B \citep{wmb99:book}.  
\citet{ct97:cj} looked at \ppm with an unbounded context.

An example of the \ppm algorithm is shown in \tabref{tab:tc-ppm}.
The sample message is again ``\texttt{woodchuck{\tvs}chuck{\tvs}if}''.
A row in the table represents the processing of a character in
the message.  The message is in the last column, with the shaded region 
representing text that has already been
seen.  The framed box indicates the character under
consideration.  In this example, the maximum order is fixed at $K = 2$.  
So, an attempt is made to first predict each character using the two
immediately preceding characters.
The columns labelled ``Order'' and ``Context'' indicate the order
and context that is finally used to predict the current character.
Novel characters are always encoded 
using a $-1$ order model.
In the example, four of the letters in the second occurrence of 
``\texttt{chuck}'' can be predicted using a second order model.

\tab{cccc}
{ & Order & Context & Message \\ }
{
(a) & -1 & - & \nogreyboxtt{w}{\texttt{oodchuck{\tvs}chuck{\tvs}if}} \\
(b) & -1 & - & \greyboxtt{w}{\nogreyboxtt{o}}{\texttt{odchuck{\tvs}chuck{\tvs}if}} \\
(c) &  0 &   & \greyboxtt{wo}{\nogreyboxtt{o}}{\texttt{dchuck{\tvs}chuck{\tvs}if}} \\
(d) & -1 & - & \greyboxtt{woo}{\nogreyboxtt{d}}{\texttt{chuck{\tvs}chuck{\tvs}if}} \\
(e) & -1 & - & \greyboxtt{wood}{\nogreyboxtt{c}}{\texttt{huck{\tvs}chuck{\tvs}if}} \\
(f) & -1 & - & \greyboxtt{woodc}{\nogreyboxtt{h}}{\texttt{uck{\tvs}chuck{\tvs}if}} \\
(g) & -1 & - & \greyboxtt{woodch}{\nogreyboxtt{u}}{\texttt{ck{\tvs}chuck{\tvs}if}} \\
(h) &  0 &   & \greyboxtt{woodchu}{\nogreyboxtt{c}}{\texttt{k{\tvs}chuck{\tvs}if}} \\
(i) & -1 & - & \greyboxtt{woodchuc}{\nogreyboxtt{k}}{\texttt{{\tvs}chuck{\tvs}if}} \\
(j) & -1 & - & \greyboxtt{woodchuck}{\nogreyboxtt{\tvs}}{\texttt{chuck{\tvs}if}} \\
(k) &  0 &   & \greyboxtt{woodchuck{\tvs}}{\nogreyboxtt{c}}{\texttt{huck{\tvs}if}} \\
(l) &  1 & \texttt{c} & \greyboxtt{woodchuck{\tvs}c}{\nogreyboxtt{h}}{\texttt{uck{\tvs}if}} \\
(m) &  2 & \texttt{ch} & \greyboxtt{woodchuck{\tvs}ch}{\nogreyboxtt{u}}{\texttt{ck{\tvs}if}} \\
(n) &  2 & \texttt{hu} & \greyboxtt{woodchuck{\tvs}chu}{\nogreyboxtt{c}}{\texttt{k{\tvs}if}} \\
(o) &  2 & \texttt{uc} & \greyboxtt{woodchuck{\tvs}chuc}{\nogreyboxtt{k}}{\texttt{{\tvs}if}} \\
(p) &  2 & \texttt{ck} & \greyboxtt{woodchuck{\tvs}chuck}{\nogreyboxtt{\tvs}}{\texttt{if}} \\
(q) & -1 & - & \greyboxtt{woodchuck{\tvs}chuck{\tvs}}{\nogreyboxtt{i}}{\texttt{f}} \\
(r) & -1 & - & \greyboxtt{woodchuck{\tvs}chuck{\tvs}i}{\nogreyboxtt{f}} \\
}
{Statistical modelling using \ppm.  Each row indicates the 
order and context used to predict the current character.
The progress of the algorithm is presented at the end of 
each row, with the current character placed in a box.  The
shaded region indicates the characters that have been processed.}
{Statistical modelling using \ppm}{tab:tc-ppm}

The amount of memory required by the \ppm algorithm is
higher than dictionary-based algorithms, due to the number of
contexts that have to be maintained.  Typical implementations 
use several megabytes of memory for data structures.  \citet{bcw90:book}
showed how an implementation of \ppm would use a trie of
context nodes, with each node representing a unique context
of symbols.  Nodes are interlinked with 
{\emph {vine pointers}} so that when a shorter context is needed, 
it can be reached in $O(1)$ time.
Even so, \ppm tends to execute slower for both encoding and 
decoding in comparison to \lza and \lzb because of the overhead of
maintaining the context tree, and the fact that \ppm operates on a
per symbol basis.  It is widely accepted that the combination of \ppm 
with an arithmetic coder (to be discussed
below), is one of the best general-purpose compression 
systems\footnote{The Canterbury Corpus web site 
at {\myurl {http://corpus.canterbury.ac.nz/}} compares a wide range
of compression systems.}. 
\citet{bcw90:book} summarise work by others (including
\citet{shannon48:bell}, \citet{rl81:ieeeit}, and \citet{bw87:tech}),
and show that an equivalent 
statistical model exists for any dictionary-based algorithm which
selects phrases in a greedy fashion.  This relationship was further
discussed by \citet{bw94:jacm}.
So, a greedy dictionary-based model cannot outperform a 
statistical one.

\newsubsection{Block Sorting}{subsec:tc-bsort}

The last modelling technique considered in this chapter is the block
sorting mechanism devised by \citet{bw94:tech}.  Also called the 
Burrows-Wheeler transform (or BWT),
the algorithm differs from dictionary-based and statistical ones 
in that it applies a transformation to the message to improve
compression.
A subsequent move-to-front mechanism is invoked 
in order to underscore
the locality created by the transformation, and eventually, a 
zero-order coder is applied.

The block sorting transform is best explained through an example.
\figref{fig:tc-bs} demonstrates the algorithm with the string,
``woodchuck''\footnote{See other sources, such as \citet[pg.~66]{wmb99:book}, 
for an example of block sorting using a longer message with more
redundancy.}.
The first step in block sorting is to partition the message into 
blocks so that each one can be processed independently in an off-line 
manner.  For each block of $n$ symbols, a set of $n$ strings is created,
starting from each symbol in the block.  When the end of the 
string is reached, the string wraps-around to the beginning.
The $n$ strings for the example string are 
presented in \subfigref{fig:tc-bs}{a}.  The block of strings is
then sorted by comparing from the second-to-last symbol until
the beginning of the string, in a reverse lexicographic order.  
This is shown in \subfigref{fig:tc-bs}{b}.
Sorting into context order gives a good indication about 
what characters appear after a particular conditioning sequence.  
For example, the first two strings of \subfigref{fig:tc-bs}{b} show that
in this string, the character ``c'' only forms a context for
``h'' and ``k''.
The sequence of last characters of the block of sorted strings
is shown with a shaded column in the figure.  This string of characters
is encoded for transmission.  The 
encoder notifies the decoder as to the location of the symbol 
at the beginning of the original message.  In this example, it is 
the ``w'' in the fifth position of the string, as indicated by the asterisk (*).



\fig{
\begin{tabular}{ll@{}l@{}ll}
\LCC
& & \llightgray & & \\
{\texttt {woodchuck}} & {\texttt {uckwoodc}} & {\texttt {h}} & & {\texttt {c h}} \\
{\texttt {oodchuckw}} & {\texttt {woodchuc}} & {\texttt {k}} & & {\texttt {c k}} \\
{\texttt {odchuckwo}} & {\texttt {huckwood}} & {\texttt {c}} & & {\texttt {d c}} \\
{\texttt {dchuckwoo}} & {\texttt {ckwoodch}} & {\texttt {u}} & & {\texttt {h u}} \\
{\texttt {chuckwood}} & {\texttt {oodchuck}} & {\texttt {w}} &*& {\texttt {k w}}* \\
{\texttt {huckwoodc}} & {\texttt {dchuckwo}} & {\texttt {o}} & & {\texttt {o o}} \\
{\texttt {uckwoodch}} & {\texttt {chuckwoo}} & {\texttt {d}} & & {\texttt {o d}} \\
{\texttt {ckwoodchu}} & {\texttt {kwoodchu}} & {\texttt {c}} & & {\texttt {u c}} \\
{\texttt {kwoodchuc}} & {\texttt {odchuckw}} & {\texttt {o}} & & {\texttt {w o}} \\
\ECC
\multicolumn{1}{c}{(a)} & \multicolumn{3}{c}{(b)} & \multicolumn{1}{c}{(c)} \\
\end{tabular}}{Example of the block sorting algorithm with the
word ``woodchuck''.  The $n$ strings (for a block of $n$ 
characters) are obtained by permuting the original word as shown 
in (a).  Then, the strings are sorted in (b), with the first letter
of the word indicated by the asterisk (``*'').  
The characters indicated in the grey
column are transformed using MTF before being transmitted to the
receiver.  Decoding is shown in (c).}
{Block sorting example}{fig:tc-bs}

The decoder receives the permuted string, and sorts it.  In
\subfigref{fig:tc-bs}{c}, the sorted string on the left is
placed next to the received unsorted string, shown to its right.  
The sorted string of received characters
is identical to the column of second-to-last characters in 
\subfigref{fig:tc-bs}{b}.  That is, each character in the sorted string
forms a one-symbol context for the character to its right.
Moreover, in both strings, the relative order of characters
which have the same value is preserved.  The second ``o'' in the
sorted string (seventh character) is the same 
character as the second ``o'' in the
unsorted string (last character).
One symbol is decoded by examining a character from the 
received string, then the same character in the sorted version,
and finally, the character it forms a context with, to its right,
as illustrated in \subfigref{fig:tc-bs}{c}.  In this example, the first character in
the unsorted string is the ``w'' in position 5.  This character
is the same as the ``w'' at the end of the sorted string, which forms 
a context for the second ``o'' in the unsorted string.

The difference between the transmitted string and the original message
is that the permuted string has localised character occurrences.
In the first two strings of \subfigref{fig:tc-bs}{b}, the
characters ``h'' and ``k'' appear in the context of ``c''.  Even if
the message was longer, only the few character that appear after a
``c'' would cluster in this region.  

To exploit the localised clustering, a move-to-front (MTF) 
transformation is then
applied.  The transformation translates each symbol
in the string to a number that indicates how many distinct symbols
have been observed since this one was last seen.
A symbol that
was recently seen gets translated into a small value.
In a cluster of symbols where there is little variation, sequences
of small numbers result.  
In the end, the permuted string becomes a list of
numbers with a skewed distribution of mainly small values, and it is that
sequence that is passed to a coder.  If the seven distinct characters
in ``woodchuck'' are initially ordered alphabetically, then the MTF values
for the permuted string is $\{3, 4, 3, 6, 7, 7, 7, 5, 3 \}$.  If the
original message was longer, then clusters of symbols in the permuted
sequence would generate runs of small MTF numbers.  As an alternative,
\citet{wm01:dcc} compressed the output from block sorting directly
using other schemes, without
applying the MTF transform.

Generally, systems which implement block sorting with 
MTF yield compression ratios close to \ppm due to their exploitation
of symbol contexts \citep{ct97:cj}.  As the most time consuming operation
in BWT is string sorting, others have considered ways of improving
this phase of the algorithm \citep{sadakane98:dcc, seward00:dcc, seward01:dcc}.
Several other authors have explored refinements to improve compression
effectiveness \citep{deorowicz00:spe, fenwick96:acsc, fenwick96:cj}.

\newsection{Coders}{sec:tc-coder}

The model and probability estimator identifies the redundant
information in the message and produces a
{\emph {reduced message}}.  This reduced message
must then be coded to achieve the desired
transmission across the channel.
Based on the description of models presented earlier in this 
chapter, the reduced message may contain components of
a 3-tuple from a \lza implementation, a stream of symbols indicating which
\ppm contexts were chosen, or MTF values for a message permuted with
block sorting.

The coder obtains the reduced message, the set of distinct symbols
in the reduced message ($D$), and the probabilities with which these
symbols occur ($P$).  From these three pieces of information, the
coder creates a set of {\emph {codewords}} ($C$) which map each symbol
in the reduced message to a string of letters in the channel 
alphabet.  Compression has been achieved if the set of codewords
(collectively called {\emph {codes}}) chosen produce an output
message shorter than the original message.  Furthermore, if 
the codes are selected so that no codeword is a prefix of any other
codeword, then the codes are {\emph {prefix-free}}.  Prefix-free codes
ensure that the compressed message is uniquely decodable, without the 
need to buffer codewords.

Two classes of coders are presented next.  Methods in the first class 
assign codewords statically, based solely on a symbol's value
within the alphabet.
The second class of coders are entropy coders,
which use the supplied symbol probabilities to produce a 
compressed message whose length is intended to be close to the
self-information of the reduced message.

\newsubsection{Static Coders}{subsec:tc-static}

Static coders map each symbol in the reduced message to a
particular codeword, without acknowledging the
probabilities with which the symbols occur.  The simplest scheme
is a unary code, which encodes the symbol $x$ as a codeword
made up of $x - 1$ one-bits, and terminated by a zero bit.
Unary codes are biased toward smaller values
of $x$, since the codeword for $x$ is exactly $x$ bits in length.

Another straightforward coding mechanism is binary, which
allocates a codeword of length $\lceil \log_2 N \rceil$ bits
for each $x$, for some upper limit $N$.  Unlike unary coding, 
binary coding assumes a 
uniform distribution of the symbols and a definite range.  
Some examples of binary
coding are the \ascii and \eascii character sets, which encode
each symbol in 7 and 8 bits, respectively.  
The second and third columns of
\tabref{tab:tc-staticcodes} list the unary and binary codes
for the values 1 to 9, respectively.
The binary codes in this table are based on an
upper limit of $N = 8$.  Because of this, no binary code is possible
for $x = 9$.

\tab{clcll}
{ $x$ & \multicolumn{1}{c}{unary} & binary & \multicolumn{1}{c}{gamma} & \multicolumn{1}{c}{delta} \\}
{
1     & 0         & 000 & \greybox{0}      & \greybox{0} \\
2     & 10        & 001 & \greybox{10}0    & \greybox{100}0 \\
3     & 110       & 010 & \greybox{10}1    & \greybox{100}1 \\
4     & 1110      & 011 & \greybox{110}00  & \greybox{101}00 \\
5     & 11110     & 100 & \greybox{110}01  & \greybox{101}01 \\
6     & 111110    & 101 & \greybox{110}10  & \greybox{101}10 \\
7     & 1111110   & 110 & \greybox{110}11  & \greybox{101}11 \\
8     & 11111110  & 111 & \greybox{1100}00 & \greybox{11000}000 \\
9     & 111111110 & --- & \greybox{1100}01 & \greybox{11000}001 \\
}{Some sample static codes for the values 1 to 9.  The prefix of the
gamma and delta codes are shaded in grey.}{Sample static codes}
{tab:tc-staticcodes}

The fourth column of the table shows the codewords produced by
gamma coding \citep{elias75:ieeeit}.  Gamma codes offer a compromise
between the skewed distribution assumed by unary coding, and the
uniform distribution of binary coding.  A gamma code has two parts,
a prefix and a suffix.
The prefix, shown shaded in the fourth column, uses a unary code to specify
the binary length of the symbol.  The suffix part
specifies the exact symbol in binary within that order of magnitude.
For the symbol $x$, the prefix specifies 
$1 + \lfloor \log_2 x \rfloor$ in $1 + \lfloor \log_2 x \rfloor$ 
bits, while the suffix indicates $x - 2^{\lfloor \log_2 x \rfloor}$
in $\lfloor \log_2 x \rfloor$ bits.  The total length of the gamma
code for $x$ is thus $1 + 2\lfloor \log_2 x \rfloor$ bits.  If the prefix
is coded using a gamma code, and the suffix in binary, then
the delta codes result, also described by \citeauthor{elias75:ieeeit}.
The last column of \tabref{tab:tc-staticcodes} 
presents some delta codes, with the prefix again highlighted.  The length
of a delta code for a symbol $x$ is 
$1 + 2\lfloor \log_2 \log_2 2x \rfloor + \lfloor \log_2 x \rfloor$ bits.

There are two benefits of using \citeauthor{elias75:ieeeit}
codes instead of unary and binary codes.  In comparison to
unary codes, \citeauthor{elias75:ieeeit} codes are 
shorter for all but a very small number of values of $x$.  
And while the length of binary codes are even
shorter, binary codes are assigned based on a known range,
which might not be a reasonable restriction.
On the other hand, \citeauthor{elias75:ieeeit} codes and 
unary codes can represent arbitrarily large values.  All four 
codes are prefix-free.

\newsubsection{Huffman Coding}{subsec:tc-huffman}

Entropy coders assign codes derived from symbol probabilities that
are calculated directly 
from symbol frequencies in the reduced message, or through other means.
Either way, they are assumed to
be true for the symbols being processed.  Entropy coders that
decide on codewords based on the observed frequency of symbols in the reduced
message can do so using either semi-static or adaptive mechanisms.  Two
entropy coding regimes are considered in this section:  semi-static
Huffman coding, and adaptive arithmetic coding.

Huffman coding is a famous entropy coding technique 
\citep{huffman52:ire}.  This algorithm assigns codewords 
so that frequent symbols 
obtain shorter codewords than rare symbols.  The Huffman algorithm
calculates codewords incrementally, by first
placing each symbol of the alphabet in a {\emph {package}}, and
then by repeatedly merging the two packages with the lowest combined
probabilities,
until only one package remains.  Whenever two packages
are combined, the probabilities are summed.  Every symbol has an
associated codeword length which is incremented by one every time
the symbol participates in a merge. 
When the algorithm concludes, the set of 
lengths are used to assign codewords to each symbol.

\figref{fig:tc-huffman} explains the Huffman algorithm through
an example.  Let the set of packages be $D$, the associated probabilities
of each symbol in the alphabet be $P$, and their corresponding codeword
lengths $L$.  In the example, the alphabet consists of the four symbols
$D = \{\{\textrm{w}\}, \{\textrm{x}\}, \{\textrm{y}\}, \{\textrm{z}\}\}$,
with respective probabilities $P = \{0.1, 0.2, 0.2, 0.5\}$.  The codeword
lengths are all initialised to 0 so that, $L = \{0, 0, 0, 0\}$.  
First,  the two packages with the 
lowest probabilities are merged in \subfigref{fig:tc-huffman}{b}.  
Since the packages $\{\textrm{x}\}$ and $\{\textrm{y}\}$
have the same probabilities, either one can be combined with
$\{\textrm{w}\}$.  The selected packages, shown shaded, are merged and
their codeword lengths are incremented by 1.
Once the package $\{\textrm{z}\}$, is processed in \subfigref{fig:tc-huffman}{d},
the final set of codeword lengths is $L = \{3, 3, 2, 1\}$.  

\fig{
\begin{tabular}{cc}
  \begin{tabular}{ccc}
  \hline
  $D$ & $P$ & $L$ \\
  \hline
  w & 0.1 & 0 \\
  x & 0.2 & 0 \\
  y & 0.2 & 0 \\
  z & 0.5 & 0 \\
  \hline
  \end{tabular}
  &
  \begin{tabular}{ccc}
  \hline
  $D$ & $P$ & $L$ \\
  \hline
\LCC
  \lightgray & & \\
  w & \multirow{2}*{0.3} & 0 \\
  x &                    & 1 \\
\ECC
  y & 0.2                & 0 \\
  z & 0.5                & 0 \\
  \hline
  \end{tabular} \\ [0.5ex]
  (a) Initial packages. & (b) Combine w with x. \\ [1.0ex]

  \begin{tabular}{ccc}
  \hline
  $D$ & $P$ & $L$ \\
  \hline
\LCC
  \lightgray & & \\
  w & \multirow{3}*{0.5} & 2  \\
  x &                    & 2  \\
  y &                    & 1 \\
\ECC
  z & 0.5                & 0 \\
  \hline
  \end{tabular}
  &
  \begin{tabular}{ccc}
  \hline
  $D$ & $P$ & $L$ \\
  \hline
\LCC
  \lightgray & & \\
  w & \multirow{4}*{1.0} & 3   \\
  x &                    & 3   \\
  y &                    & 2 \\
  z &                    & 1 \\
\ECC
  \hline
  \end{tabular} \\ [0.5ex]
  (c) Combine wx with y. & (d) Combine wxy with z. \\
\end{tabular}}{The merging of packages to build a set of
Huffman codes.  Lightly shaded packages are merged and
their codeword lengths incremented by 1.}
{Huffman code construction}{fig:tc-huffman}

Next, a set $C$ of prefix-free codewords is assigned to each 
symbol based on $L$ to complete Huffman's algorithm.  
One set of codewords that satisfy this
condition is $C = \{000, 001, 01, 1\}$.  Note that the assignment of
codewords is not unique.  If there are $n$ packages,
then there are $n - 1$ merge operations.  Each merge operation is
equivalent to appending a 0 bit to one package and a 1 bit to the
other, once $L$ has been calculated.  As the algorithm does not
state the bit a package receives, there are $2^{n - 1}$
sets of Huffman codes.  Moreover, as shown in the example, when
multiple packages share the same probability, ties are broken
arbitrarily.  

An alternative method to the Huffman code construction described is the
bottom-up method which employs a binary tree 
\citep{johnsonbaugh93:book, salomon00:book, clrs01:book}.
\figref{fig:tc-huffman-pic} shows a tree for the codewords $C$ that
were assigned using the lengths $L$ from 
the example of \figref{fig:tc-huffman}.  Symbols are placed at
the leaves of the tree.  In this graph representation,
symbol lengths translate to depths of symbols in the tree, so
that each edge signifies a bit in the codeword.  
A bit stream is decoded by traversing the tree
starting from the root, so that a bit permits one edge transition.
When a leaf has been reached, the symbol at
that leaf is output and the process begins again from the root.
While visualising the decoding of Huffman codes with a tree 
is helpful, \citet{mt97:ieeec} 
presents a more efficient method for decoding which replaces
tree construction by both the encoder and the decoder with a
set of arrays.  The primary requirement is that the codes must
be {\emph {canonical}}.  A set of Huffman codes are
canonical if they are ordered lexically when sorted by length.

According to the codewords of \figref{fig:tc-huffman},
the reduced message ``yzzw'' would be coded in 7 bits as 
0111000.  Unfortunately, this bit stream alone is not 
enough for the decoder to recreate the original message.
A {\emph {prelude}} is required which indicates the alphabet
of the reduced message ($D$).  
Details about the construction and 
transmission of the prelude for Huffman coding is described
by \citet{tm00:ieeec}.

\fig{
\includegraphics*{./huffman.eps}
}{Visualising Huffman codes as a binary tree.}{Huffman codes as 
a binary tree}{fig:tc-huffman-pic}

Other variations to the Huffman algorithm include length-limited
coding \citep{lh90:jacm, lm02:dcc} and adaptive Huffman coding
\citep{vitter87:jacm, knuth85:jalg}.  In length-limited coding, codewords
are assigned so that none of them are longer than $L_m$ bits, for some
upper limit $L_m$.  Adaptive
Huffman coding differs from the semi-static paradigm described above
in that the probabilities of the symbol are modified while the message is
processed.  Moreover, like the \ppm algorithm described earlier, the
zero-frequency problem re-emerges since a probability must be assigned
to novel symbols.

\newsubsection{Arithmetic Coding}{subsec:tc-arith}

All of the coding techniques described so far have one trait
in common.  Regardless of the value of a symbol, 
or its probability in the message, the symbol must be coded in at 
least one bit.  Arithmetic coding by \citet{rissanen76:ibm} 
and \citet{wnc87:cacm}, and later revised 
by \citet{mnw98:tis}, eliminates this restriction by coding
sequences of symbols at a time.  

\figref{fig:tc-arith} explains the algorithm through the sample
message ``yzzw'', using the symbol probabilities used earlier.
At first, the interval [0, 1) is divided according to the 
cumulative probabilities of $D$.  As each symbol in the message
is encountered, the section corresponding to the symbol is 
further divided.  In the example, after ``y'' has been processed,
the interval [0.3, 0.5) is divided into four parts based on
the symbol probabilities.  Once the last symbol is read,
the interval [0.45, 0.455) has been isolated.  In binary, with
the decimal removed, this interval is [0111001, 0111010001).  The
message can be encoded by specifying any number within this interval.
One sequence that meets this requirement is 011101 (or, 0.453125).

\fig{
\includegraphics*{./arith.eps}
}{The division of intervals when arithmetic coding is applied
to the message ``yzzw'', for the four symbol alphabet,
$\{\textrm{w}, \textrm{x}, \textrm{y}, \textrm{z}\}$.
The intervals are broken down further with
each symbol encountered.}{Arithmetic coding example}
{fig:tc-arith}

As with Huffman coding, the prelude must be sent to the decoder.
Furthermore, in order for the decoder to mimic the encoder, the
probabilities of the symbols need to be transmitted as well.  Then,
the decoder operates by dividing and selecting intervals in the
same order as the encoder.  Several other simplifications
have been made.  For example, an actual implementation would
use integers instead of real numbers so as to avoid any rounding
error.  Also, intervals must be scaled when they become too narrow.
\citet{mt02:book} provides a detailed discussion about the
implementation issues associated with arithmetic coding.

Generally, arithmetic coding
performs better than other coding techniques for highly skewed
probabilities -- especially when the self-information of any
of the symbols is less than one bit.  In fact, the compression
effectiveness attained by arithmetic coding is better able 
to approach the self-information of the message than any other
coding technique.  However, in comparison to other coding methods, 
arithmetic coding is relatively slow for both encoding and decoding,
since an interval has to be divided for each symbol in the message.

\newsection{Compression in Practice}{sec:tc-practice}

Three compression systems are explained next, 
in the context of the algorithm descriptions presented
above.  These systems, two of which are readily available on the
Internet, are used throughout this thesis as baseline comparisons.
The three implementations are \gzip, \ppmd, and \bzip.

The dictionary-based system \gzip \citep{gzip:program} employs the \deflate
algorithm \citep{deflate:rfc}, which combines a
variant of the \lza algorithm with Huffman coding.  The \deflate
algorithm partitions a message into blocks, and processes a block at
a time for efficiency reasons.  The \deflate algorithm 
uses 1-tuples for literal symbols and 2-tuples
for references into the dictionary, and neither type of tuple specifies
the next symbol.  The sliding window size is fixed at
32 KiB\footnote{In this thesis, 
\kib{1} = $2^{10}$ bytes = 1,024 bytes; 
\mib{1} = $2^{20}$ bytes = 1,048,576 bytes; 
\gib{1} = $2^{30}$ bytes = 1,073,741,824 bytes.
\citep{iec60027:std}.}.  
The \deflate algorithm assumes that the message is
encoded in bytes, so that literal symbols fall within the range of
0 to 255.  Literal symbols and phrase lengths are encoded using one
set of Huffman codes, while distances are encoded with another set.
For each block, \deflate chooses between static or semi-static Huffman
codes.  If semi-static Huffman codes are selected, then the prelude is
situated at the beginning of the compressed block.
A publicly available library called \zlib \citep{zlib:program} is available,
which allows \deflate to be compiled into any program.  
The default compression 
effectiveness of \gzip can be improved by adding the 
{\texttt {-9}} option.  The {\texttt {-9}} option forces \gzip to 
search more exhaustively for a longer match.  All experiments with
\gzip in this thesis use version 1.3.2 with the {\texttt {-9}}
option.

The \ppmd system by Alistair Moffat (University of Melbourne)
couples a \ppm mechanism based
on escape method D with an arithmetic coder.  
Two parameters can be set by the user to alter the program's behaviour.
The first parameter specifies the maximum context, while the second
one sets a limit on the amount of memory that can be used.  Context nodes are created
by \ppmd until the memory limit is reached.  If this happens while
a message is being processed, then all context nodes are deleted, and
compression continues with a fresh trie.  Experiments in this investigation
make use of a seventh order model with a memory limit of \mib{255}, both
chosen to favour compression effectiveness.

The \bzip program \citep{bzip:program} combines block sorting with a
Huffman coder.  The block sorting mechanism divides the message into
blocks ranging from 100,000 to 900,000 bytes in size, depending 
on the value of a command line argument.  In this thesis, the 
program parameter {\texttt {-9}} is always given so that 
a block size of 900,000 bytes is used.  The documentation that
accompanies the program states that the memory required by \bzip for
encoding with this block size is around \mib{7.4}, while decoding
requires about \mib{3.6}.  A compression library similar to \zlib
called \libbzip is available from the \bzip web site.  The
version of \bzip used in this thesis is 1.0.2.  Details of the implementation
of \bzip are given by \citet{seward00:dcc} and \citet{seward01:dcc}.

Two additional programs are purely entropy coders which presume a zero-order
model, and hence that all modelling processes have already been allowed for.  
Input to these coders is a sequence of 
integers.  The \shuff
system \citep{mt97:ieeec, tm00:ieeec}
is a semi-static Huffman coder that compresses an integer sequence
in blocks in two passes.  The end of a block is determined either
by an explicit size from the user, or when a special end-of-block
symbol has been found in the input stream.
The second coding mechanism is an arithmetic
coder called \uint, developed as part of a package of software to
demonstrate the attributes of arithmetic coding by \citet{mnw98:tis}.

With the exception of the \ppmd system,\footnote{One publicly available
\ppm system which employs escape method D is at 
\myurl{http://www.cs.waikato.ac.nz/~singlis/}.} the remaining four 
programs are publicly
available for download, with source code, from web sites mentioned
in their respective bibliography entries at the back of this thesis.

\newsection{Experimental Framework}{sec:tc-expt}

Experiments are performed in this thesis in order to validate
the compression and retrieval ideas presented.  
All tests are carried out on a \vipe running Debian Linux.
Programs implemented for this thesis are written in the
C programming language and
compiled with version 2.95.4 of the GNU {\texttt {gcc}} compiler with the
{\texttt {-O3}} optimisation flag.
Compression systems have been evaluated in terms of efficiency
and effectiveness.  Unless
otherwise stated, compression times are expressed in CPU seconds,
averaged over three trials.  CPU seconds reflects the amount of 
time the computer spent on the program, but excluding the time spent 
accessing files, for example.  Due to the small variations
in execution times, an average calculated from three runs was
sufficient for most of the experiments.

Compression ratios throughout the investigation are 
primarily expressed in one of two different ways.  
If the original message is composed of 32-bit integers, 
then the ratios are expressed in bits per symbol (bps).  However,
for most of this thesis, the message is encoded in \eascii and
the units employed are bits per character (bpc).

The test data that was chosen for this 
investigation included the Large Canterbury 
Corpus\footnote{URL:  {\myurl {http://corpus.canterbury.ac.nz/}}} 
(\largecc) and data
from the Text REtrieval 
Conference\footnote{URL:  {\myurl {http://trec.nist.gov/} }}
(\trec).  The Large Canterbury Corpus includes three files:  
{\texttt {E.coli}}, {\texttt {world192.txt}}, and {\texttt {bible.txt}}.  
The {\texttt {E.coli}} file 
consists of the complete genome of the E.coli bacterium and only contains
the four characters a, c, g, and t, with no whitespace.  The file
{\texttt {world192.txt}} is the CIA world fact book, 
while {\texttt {bible.txt}} is the King James version of the bible.  
The sizes of these three files are listed in \tabref{tab:tc-bpc}.

Three data files were drawn from the news articles of disks 1 and 2
of \trec's 
\tipster collection \citep{har95:ipm}.  The largest data
file is \news, which is composed of news articles from the Wall
Street Journal (\wsj) from 1987 to 1992, as well as articles 
from the Associated Press (\ap)
from 1988 and 1989.  All of the news articles have been marked
up with the Standard Generalized Markup Language (\sgml)
\citep{sgml:std}.  Due to the wide range in years covered by
the articles, and the fact that the \sgml tags differ between
\wsj and \ap, there are some discontinuities of style caused by the
concatenation.  However, the homogeneous data collection of news articles
permits a data file of
about \mib{1,000} in size to be processed.  Small excerpts from
\wsj and \ap are shown in \figref{fig:tc-sgmlsample}.  The 
remaining two data files from \trec are \wsjb and \wsja.  The 
\wsjb file includes all of the Wall Street Journal articles from
\trec, which is \mib{508} in total.  A smaller data file, \wsja,
consists of just the first \mib{20} of \wsjb.

\fig{
\begin{tabular}{c}
\input{./wsja_sample.sgml} \\ [0.5ex]
(a) Excerpt from \wsj (\wsja and \wsjb). \\ [2ex]
\input{./ap_sample.sgml} \\ [0.5ex]
(b) Excerpt from \ap. \\
\end{tabular}}{Two excerpts from the \news data set.}
{Excerpts from the \news data set}{fig:tc-sgmlsample}

The three compression systems, \gzip, \bzip, and \ppmd, were
applied to these 6 data files.  The results are presented in
\tabref{tab:tc-bpc} and \tabref{tab:tc-time}.  Since \shuff and \uint
are coding mechanisms and not general-purpose compression systems,
experiments with them are detailed later in \chapref{chap:repair}
with different data files.

\tabref{tab:tc-bpc}
shows the exact file sizes of the 6 files and the compression 
effectiveness attained with the compression programs.  Both \gzip and
\bzip were executed with the {\texttt {-9}} option, while \ppmd
used a seventh order model and \mib{255} of memory in order to 
favour compression effectiveness.  For all 6 files, the statistical model
of \ppmd consistently performed the best, followed by \bzip, 
and then \gzip.  This order is expected, given their respective
models and coders, as well as the amount of memory available for
them, and the consequential use of very large blocks by \ppmd.

\tab{lcccc}
{
Filename & File size & \gzip & \bzip & \ppmd \\
}
{
{\texttt {E.coli}}        & \C\D\D4,638,690    & 2.240 & 2.158 & 2.034 \\
{\texttt {world192.txt}}  & \C\D\D2,473,400    & 2.333 & 1.584 & 1.454 \\
{\texttt {bible.txt}}     & \C\D\D4,047,392    & 2.326 & 1.671 & 1.564 \\
\wsja         & \C\D20,971,520     & 2.908 & 2.078 & 1.656 \\
\wsjb         & \C533,196,049      & 2.941 & 2.105 & 1.637 \\
\news         & 1,048,491,001      & 2.963 & 2.145 & 1.674 \\
}{Compression ratios for the Large Canterbury Corpus, \wsja, 
\wsjb, and \news using \gzip, \bzip, and \ppm, expressed in 
bits per character.  File sizes are
indicated in bytes.}{Compression ratios using \gzip, 
\bzip, and \ppm (all data)}{tab:tc-bpc}

\tabref{tab:tc-time} examines the compression experiments with respect
to time in CPU seconds, averaged over three trials on the test machine discussed
earlier.  In almost every scenario, \gzip required the least 
amount of time for 
both encoding and decoding, followed by \bzip, and then \ppmd.  The
notable exception is the compression of {\texttt {E.coli}}, where \gzip
required the most time for encoding.  This phenomenon is due to the small
set of characters in the file, which made looking for the longest match
more time consuming.  

The tables demonstrate a common trade-off in compression between 
efficiency and effectiveness.  Generally, the best compression
effectiveness comes at the cost of execution time.  These results
also indicate why these three programs were selected for this chapter.  
Besides demonstrating the three types of models mentioned earlier, 
the \gzip system was chosen for its speed, while \ppmd was chosen for
its compression effectiveness.  The \bzip system demonstrates the
compromise in time and space between the other two systems.  Throughout
this thesis, these two tables are frequently referred to as benchmarks
for these compression systems and the test data.

Finally, note that \gzip is particularly fast for decoding,
a trait that is common with most dictionary-based algorithms, and one
that forms the foundation for much of this thesis, as explained in 
the next section of this chapter.

\tab{lccccccc}
{
\multirow{2}*{Filename} & \multicolumn{3}{c}{Encoding (s)} & & \multicolumn{3}{c}{Decoding (s)} \\ \cline{2-4}\cline{6-8}
& \gzip & \bzip & \ppmd & & \gzip & \bzip & \ppmd \\
}
{
{\texttt {E.coli}}        & \D14.0  & \D\D4.4 & \C\D\D3.6 &  & \D0.2   & \D\D1.5   & \C\D\D3.6  \\
{\texttt {world192.txt}}  & \D\D0.6 & \D\D2.1 & \C\D\D3.3 &  & \D0.1   & \D\D0.7   & \C\D\D3.4  \\
{\texttt {bible.txt}}     & \D\D1.8 & \D\D3.4 & \C\D\D4.4 &  & \D0.1   & \D\D1.1   & \C\D\D4.5  \\
\wsja         & \D\D6.4 & \D17.9  & \C\D31.0  & & \D0.8   & \D\D6.2   & \C\D31.5 \\
\wsjb         & 170.0   & 447.3   & \C800.9 & & 20.4  & 155.8 & \C821.5\\
\news         & 323.3   & 871.5   & 1,600.8 & & 40.3  & 304.3 & 1,641.0\\
}{Compression times for the Large Canterbury Corpus, \wsja, \wsjb, and 
\news using \gzip, \bzip, and \ppm.  Times are reported in CPU seconds
on a \vipe, averaged over three trials.}
{Compression times using \gzip, \bzip, and \ppm (all data)}{tab:tc-time}

\newsection{Effect of Compression on Browsing}{sec:tc-summary}

The central theme in this thesis is the exploration of the 
compromises required in order to accommodate both 
compression and browsing.
The compression systems surveyed 
in this chapter can reduce a message to less than a fifth of its
original size.  However, several aspects of their underlying
algorithms preclude any form of retrieval, let alone browsing,
without performing a complete decompression.

The support for searching a compressed message is important,
since time and space could be saved while still being
able to obtain information from the compressed representation.
The solution offered by the \zgrep script 
\citep{zgrep:program}, is to decompress a reduced message 
created by \gzip temporarily, and then 
search this copy with a standard pattern matcher.
However, while such a solution is convenient for the user,
the solution is somewhat unsettling given that the compressed
message is fully decoded.  The source of the problem is
the compression algorithms themselves.

The main drawback to the dictionary-based and statistical models
presented is their adaptive, on-line nature.  
In order to retrieve from an arbitrary point in a message 
processed with either
\lza, \lzb, or \ppm, everything
before it must be decoded.  In the case of block sorting, only
the block where the point of interest is located must be decoded.
So, extending this solution to the other algorithms, creating more
blocks would improve retrieval feasibility.  Generalising the
solution, {\emph {synchronisation points}} could be inserted 
throughout the
compressed message so that retrieval from a location in the
compressed stream only requires a full decode from the last 
synchronisation point.  However, since synchronisation points 
indicate a location in the compressed message where the model
is re-built, as the frequency of these points increase, compression
can be expected to suffer.

Furthermore, while compression systems are measured in terms of
effectiveness and efficiency, retrieval systems are usually
interactive.  That is, users are more concerned with response
times and expect a short waiting period for results to appear.
The importance of access time provides another reason
why a full decode of the compressed document is discouraged.

In addition to these two points, coding algorithms present a
third problem, which is byte-alignment.  Even though the
channel alphabet is based on the binary system, computers
access data in bytes consisting of 8 bits each.  The low-level
operations on bits which are intrinsic to static and entropy
coding algorithms also affect both retrieval and decoding 
times.

These three problems are considered in the remainder
of this thesis, beginning with \chapref{chap:repair}.  At the
heart of this investigation is the \repair algorithm of 
\citet{lm00:procieee}, which is described next.  The
algorithm is dictionary-based and has many similarities with
\lza and \lzb.  However, \repair possesses other qualities that
make it ideal for browsing.  The chapters that follow 
\chapref{chap:repair} build on the algorithm and address the 
three problems posed here.

