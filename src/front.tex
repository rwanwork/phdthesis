%%  Title page, abstract, declaration, preface, and acknowledgments in 
%%  the order required by the School of Graduate Studies

%%  Definitions

%%  title definition
\newcommand{\prevdegree}{BSc (University of British Columbia, 1997)}
\newcommand{\degreelocation}{Department of Computer Science 
and Software Engineering \\The University of Melbourne}
\pretitle{\begin{center}\Huge\textbf}
\posttitle{\par\end{center}\vspace{1cm}}
\preauthor{\begin{center}
           \huge \lineskip 0.5em}
\postauthor{\large \vspace{1cm} \par \degreelocation \end{center} \vspace*{\fill}}
\predate{\vspace{1cm} \begin{center}\large}
%\postdate{\\Produced on acid-free paper \end{center}}
\postdate{\end{center}}

\newcommand{\disclaimer}[1]{%
  \gdef\Disclaim{#1}}
\newcommand{\Disclaim}{}
\renewcommand{\maketitlehookc}{%
  \par\noindent \Disclaim}

\disclaimer{\centering Submitted in partial fulfilment of the requirements of the degree of\\ Doctor of Philosophy (with coursework component)}


%%  2003/06/05 title
\title{Browsing and Searching\\Compressed Documents}
\author{Raymond Wan}
\date{December 2003}  %%  Date of resubmission was December 18, 2003
\maketitle
\thispagestyle{empty}
\clearpage

\pagestyle{rwanfootonly}
%%  Abstract
\vspace*{\fill}
\begin{abstract}
Compression and information retrieval are two areas of
document management that exist separately due to the conflicting
methods of achieving their goals.  This research examines a 
mechanism which provides lossless compression and phrase-based 
browsing and searching of large document collections.
%% In this research a mechanism is described for combining
%% compression and phrase-based searching on large text documents.
The framework for the investigation is an
existing off-line dictionary-based compression algorithm.

An analysis of the algorithm, supported by previous work and
experiments, highlights two factors that are important for
retrieval:  efficient decoding, and a separate dictionary
stream.  However, three areas of improvement are necessary, prior to 
the inclusion of the algorithm into a browsing system.

First, in order to accommodate retrieval, the algorithm must
produce a dictionary built up on words, rather than 
characters.  A pre-processing stage is introduced which
separates the message into words and non-words, along with
word modifiers.

Second, the memory requirements of the algorithm prevent the
processing of large documents.  Earlier work has proposed
a solution which separates the message into individual
blocks prior to compression.  Here, a post-processing stage is 
proposed which combines the blocks in a series of phases.
Experiments show the trade-offs between the number of phases
performed and the improvements in compression levels.

Organisations which provide access to an information retrieval 
system to users are sometimes concerned with the amount of
disk space available.  But users have a different point
of view and may place response time at a higher priority.
Indeed, faster computers and network connections translate
into plummeting patience levels of users.
The last improvement to the compression algorithm is
two new coding schemes which 
replace the entropy coder that was used in previous work.  
While deploying them sacrifices compression effectiveness, these
two mechanisms offer improved efficiency, as shown through
experiments.

With the enhancements to the compression algorithm in hand,
a technique to efficiently support phrase browsing is presented.
Phrase contexts can be searched and progressively refined through 
the word modifiers.
Because of the three changes to the algorithm, phrases are more
visually appealing, larger documents can be processed, and response
times are improved.

%% \textsc command does not work?
%%% A brief diversion into the compression of HTML
%%% files completes the investigation.

%% Finally, experiments with the compression of HTML files
%% are conducted.  A simulation with a medium-sized web server
%% shows how compression of these small files can be improved
%% through the hyperlinks that connect them.

\vspace*{\fill}
\clearpage
\end{abstract}

%%  Declaration
\renewcommand{\abstractname}{Declaration}
\vspace*{\fill}
\begin{abstract}
\noindent This is to certify that
\begin{enumerate}
\item the thesis comprises only my original work towards the PhD except 
where indicates in the Preface,
\item due acknowledgement has been made in the text to all other material 
used, and
\item the thesis is less than 100,000 words in length, exclusive of 
tables, maps, bibliographies, and appendices.
\end{enumerate}
\vspace*{2cm}

\noindent Raymond Wan \\
\noindent \prevdegree
\vspace*{\fill}
\clearpage
\end{abstract}

%%  Preface
\renewcommand{\abstractname}{Preface}
\vspace*{\fill}
\begin{abstract}

\begin{center}
{\emph {Publications arising from this thesis.}}
\end{center}

\noindent The general concept of phrase browsing documents compressed
with character-based \repair was introduced as a poster presentation at
the 24th Annual International {ACM} {SIGIR} 
Conference on Research and Development in Information 
Retrieval \citep{wm01:sigir}, and then 
subsequently expanded for the 2001 Symposium on String Processing 
and Information Retrieval \citep{mw01:spire}. \\

\noindent Merging compressed blocks, as covered in \chapref{chap:remerge},
was presented at the 2002 Symposium for Combinatorial
Pattern Matching \citep{wm02:cpm}. \\

\noindent The results from the experiments of \chapref{chap:web} 
on compression with \html hyperlinks
was presented at the 2001 Australasian Database Conference
\citep{wm01:adc}.

\vspace*{\fill}
\clearpage
\end{abstract}

%%  Acknowledgments
\renewcommand{\abstractname}{Acknowledgements}
%\vspace*{\fill}
\begin{abstract}

As expected, completing work of this magnitude is due to 
the support of many people.  As I didn't know anyone
in Australia before I arrived here for
study, there are a lot of relationships I am
thankful for.

The continued support over the years from
my supervisor, Alistair Moffat, has made much of this
thesis possible.  I began this degree in the hope of 
learning, and fortunately, Alistair has made sure of that
from the first day until the last.

The past and present members of Alistair's 
group have contributed to an excellent support network, 
and it has been a pleasure to
be a part of it.  Tim Bell and Owen de Kretser helped me from
the very beginning as I struggled with the transition back into
university life.  Tim's continued maintenance of the ``vi''
machines, and calmness whenever I brought them to a grinding 
halt, is also very much appreciated.  Encouragement from Anh Ngoc Vo 
over the years has certainly kept me going from day-to-day.
Mike Ciavarella, Mike Liddell, Yugo Isal, Lily Sun, and Tony Wirth 
have further contributed to my personal life and my studies.

I'm also fortunate to have worked with (or next to) many 
other people in the department.  Hongjian Fan, Kevin Glynn, 
Bernard Pope, Robert Shelton, Qun Sun, Xiuzhen (Jenny) Zhang, 
are only some of the people who
have shared an office with me at some stage -- four
offices in all, spread across two buildings!  Inga 
Sitzmann and Lei Zheng also helped complete my network of
postgraduates within the department, while working with Linda
Stern on several occasions as a tutor has always been both 
enjoyable and enriching.  

Glen Gibb and Leslie Young have been great friends 
during my stay in Melbourne, and through them, I have met
many others.
The love and support given to me from Ju Hyung (Juliet) Lee has 
only grown since we first met.  During that time, she's been
in the unfortunate position of bearing the brunt of the stress,
and I thank her for that.

\newpage
I'm also thankful to many other friends who live overseas,
and which I've had little opportunity to meet while living here.
Wilson Lee has always been a close friend, and his advice 
is something which I've depended on over the years.
Joyce Cunningham and Yoshiko Yoshimura provided the 
motivation to consider further study, 
while Ed Knorr has always been a 
mentor to me.  Ed's
suggestion to do a research degree instead of a coursework one
because ``I won't regret it'' was what started everything
here.  Now that I'm at the end, I can say that he was
right after all.

And last, but most of all, to my parents,
for their support and caring.
Their patience while their son studied on the other side
of the world is much appreciated, as is everything else they've
done for me.

\vspace*{\fill}
\clearpage
\end{abstract}


