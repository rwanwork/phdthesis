\newchapter{Selecting Phrases for Browsing}{chap:prepair}

The previous chapter showed how \repair reduces a message by
repeatedly selecting active phrases, replacing their corresponding
active pairs with new symbols, and adding these new symbols
to the phrase hierarchy.  Active phrases are chosen in order of
decreasing number of active pairs, with ties broken by choosing
the active phrase created first.

\repair's approach to phrase construction uses primitives as the
smallest building block in the phrase hierarchy.
In the previous chapter, primitives were defined as
the \eascii representations used in the message.  However, 
people do not view messages as characters, but as clusters of
alphabetic characters which form words with 
whitespace characters in between.
A user that browses phrases drawn
from a \repair phrase hierarchy should be presented with 
meaningful words rather than arbitrary strings of characters that 
might not align with word boundaries.  In order to accomplish
this, \repair needs to treat whitespace and non-whitespace
characters differently.

This chapter considers the problem of how \repair can build a
phrase hierarchy made up of useful phrases for
browsing.  The policy used by \repair to select phrases based on
decreasing frequency alone is insufficient for this purpose.
Three alternative methods are considered which complement
\repair's phrase selection heuristic.  

The remainder of this chapter is structured as follows.  
In \secref{sec:prepair-intro}, an overview of information
retrieval and compression systems which operate on words rather
than characters is provided, along with the definition of ``word''
that is used in this chapter.
Then, the three methods of extending
\repair are examined.  In \secref{sec:prepair-align},
the first method adds rules to \repair in order to align words 
to the word boundaries implied by the non-words that 
surround them.  The second method introduces
a word-based pre-processing phase which explicitly separates a
document into two streams:  one for words and the other for non-words.
In \secref{sec:prepair-puncbased}, a third method combines the first two
methods by using the word-based pre-processing step with a
new set of rules for punctuation marks.
\secref{sec:prepair-expt} compares these three methods, and shows the
impact of the changes through experiments.
Finally, the chapter 
concludes with a summary of the three methods described.

\newsection{Forming Words from Characters}{sec:prepair-intro}

\repair, as described in the last chapter, views documents at the
character level.  However, people read and understand documents as a
sequence of words.  In a retrieval system, the query terms 
that a user provides and the results that the system returns
should be in the form of words.  Just as index-based retrieval systems 
build indexes using words, so should a phrase
browsing system using \repair.

Information retrieval systems generally parse their input 
into three different classes of symbols:  word symbols, inter-word
symbols, and special processing symbols 
\citep{kowalski97:book}.  Documents are treated as an
alternating sequence of word symbols and inter-word symbols (or
non-word symbols).  Some retrieval systems also make use of special 
processing symbols for separating documents
\citep{wmb99:book} or for indicating meta-data, such as
\sgml tags.

Symbols are encoded individually by a compression system
which assume a memoryless source.  Because the characters
that form words in a document do not appear independently,
word-based mechanisms have also been employed for compression.
A word-based compression system requires
choices to be made with respect
to how the document is parsed.  When a document is parsed as 
words, similar to a retrieval system, suitable definitions of 
words, non-words, and special processing symbols (if any) are 
required.  These different symbol types may exist in the 
reduced message as different streams, or as a single, unified
stream.  Finally, some methods of parsing a message into words
replaces the words with references to a lexicon (or dictionary) 
that also needs to be encoded and transmitted.

\citet{bstw86:cacm} devised a word-based parsing scheme which
separated a message into two alternating streams:  one of 
alphanumeric symbols and another of non-alphanumeric symbols.
A move-to-front scheme (MTF) followed by Huffman coding was
used to encode word references.  \citet{moffat89:spe}
undertook experiments on the \citeauthor{bstw86:cacm}\ scheme, 
and also extended character-based \ppm to word-based \ppm
and found improvement in compression effectiveness with a
first-order model, but little additional improvement with a 
second-order model.  An alternating sequence of words and non-words
was also used, with a 20 character limit on each.  \citet{hc92:dcc} 
investigated word-based compression for \lzb, where the 
lexicon was transmitted implicitly by using an escape 
mechanism to indicate the appearance of a new word.  
They also investigated first-order context modelling
by conditioning a word based on the parts-of-speech
(noun, verb, article, adjective, or other) of the
preceding word.  The parts-of-speech of words were
determined through dictionary look-up.  Words that
did not appear in the dictionary were assigned to
the category that would achieve the best compression.
\citet{im01:dcc} devised a compression
scheme that combined block sorting with
a word-based model.  Words and non-words were limited to
12 characters each and a single lexicon was produced
by adapting the {\em {spaceless words}} approach of
\citet{mnzb00:tis}.  The spaceless words approach
combines the word and non-word lexicons into a single
lexicon.  During decoding, a space character is
added after every word.  If a word is followed by an
explicitly coded non-word, then no space is added and
the non-word is decoded in its place instead.  
\citet{im01:dcc} reported a compression ratio of 1.69 bpc
for \wsja when combining word-based block sorting
with a suitable coder.

\citet{nw97:cj}, parsed semi-structured documents 
into words so that 
\sequitur could be applied to references to words in
the word lexicon.  In semi-structured data, words may
be conditioned on other words that are not immediately
adjacent.  As a trivial example, a closing tag in an 
\sgml message appears because of an opening tag earlier
in the message.  \sequitur used heuristics based on the
amount of gain that could be obtained from a distant
word prediction.  Experiments showed that \sequitur attained
better compression effectiveness than other systems, including
\gzip and a \ppm implementation using escape method C, for a
sample semi-structured document.  Compression
effectiveness further improved when the heuristics were 
employed.

The changes to \repair described in this chapter reflect
previous work in word-based compression.  However, some
different choices were made because the goal of this
investigation is to balance phrase browsing with compression.
Before discussing word-based compression for \repair,
some definitions are required.  A word character is any
alphabetic or numeric character in the \eascii character
set.  In order to accommodate the \sgml test 
data used in this thesis, the three characters ``\tl'', ``\tg'', 
and ``/'' are also considered to be word characters.  Any other 
symbol, including whitespace and punctuation marks, are 
non-word characters.  No special processing symbols are
defined at this time. 
A sequence of contiguous word (or non-word) 
characters is a {\emph {token}}.  A document is viewed as
an alternating sequence of word tokens and non-word tokens.

The original version of 
\repair described by \citet{lm00:procieee} and in
\chapref{chap:repair} is called character-based \repair.  This
distinction is required in order to differentiate it from
three additional word-aware versions of \repair that are 
described in the next three sections.

\newsection{Word-aligned \repair}{sec:prepair-align}

The first method of processing a document as words is called
{\emph {word-aligned \repair.}}  Word-aligned \repair augments
rules to the part of the \repair algorithm that decides
which pairs of symbols are permitted as active pairs.

\figref{fig:prepair-short-phrases} provides the motivation
behind this variation of \repair.  The figure expands the
phrases generated by character-based \repair in
\tabref{tab:repair-short} on \pgref{tab:repair-short} into 
all primitives.  The phrases in \figref{fig:prepair-short-phrases}
are listed in the order in which they are added to the phrase
hierarchy.  Three of these eight phrases contain the space character,
a non-word symbol.  While browsing phrases with non-word 
symbols is visually unappealing, a more noticeable problem is
its effect on the choice of phrases made by \repair.  Since the
phrases ``{\tvs}w'' and ``d{\tvs}'' in \figref{fig:prepair-short-phrases}
occur with a higher frequency in
the message than ``wo'' and ``od'', the likelihood of the word ``wood''
from being formed as a phrase is reduced.  The
problem is more obvious with longer documents and is caused 
by all non-word symbols, not just the space character.

\fig{\begin{tabular}{l}
\tc{1} $\rightarrow$ ch\\%
\tc{2} $\rightarrow$ d{\tvs}\\%
\tc{3} $\rightarrow$ {\tvs}w\\%
\tc{4} $\rightarrow$ ck\\%
\tc{5} $\rightarrow$ oo\\%
\tc{6} $\rightarrow$ chu\\%
\tc{7} $\rightarrow$ {\tvs}woo\\%
\tc{8} $\rightarrow$ chuck\\%

\end{tabular}}
{Expanding the phrases from the example of 
\tabref{tab:repair-short} into all primitives.}
{Expanding the phrases of \tabref{tab:repair-short}}
{fig:prepair-short-phrases}


Word-aligned \repair solves this problem by introducing the four
rules shown in \tabref{tab:prepair-wa-rules}.  As each phrase
is added to the phrase hierarchy, it is labelled as a word 
symbol (W), a non-word symbol (N), or a mixture (M).  The 
primitives (characters) are also labelled as 
``W'' or ``N'' during initialisation.  When
two adjacent symbols are being considered as potential active
pairs, the word-aligned rules are consulted.  The second
and third columns represent the classification of the 
two symbols under consideration.  The first and fourth columns 
represent the surrounding context in which that pair of symbols occurs
in the sequence. An ``X'' indicates that the classification is
unimportant.  If the pair of symbols satisfies any one of the four
rules, then it can become an active pair.  
The fifth column of the table indicates the category of the 
corresponding symbol after it has been added to the phrase 
hierarchy.  For example, when an active phrase representing 
active pairs composed of two non-word symbols is added to the
phrase hierarchy, it is also classified as a non-word symbol.
The last column provides examples which satisfy each
rule, with boxes representing the two symbols that are
being considered as an active pair.

\tab{cccccc}{Before-left & Left & Right & After-right & Result & Example \\}
{ \\ [-2.5ex]
X        & N & N & X       & N & woodchuck\greyboxline{{\tvs}}\greyboxline{{\tvs}}chuck{\tvs}{\tvs}wood \\ [0.5ex]
 X        & W & W & X       & W & woodchuck{\tvs}{\tvs}\greyboxline{chu}\greyboxline{ck}{\tvs}{\tvs}wood \\ [0.5ex]
 X        & M & M & X       & M & \greyboxline{woodchuck{\tvs}{\tvs}}\greyboxline{chuck{\tvs}{\tvs}}wood \\ [0.5ex]
{ N {\textbar} M } & W & N & {W {\textbar} M} & M & woodchuck{\tvs}{\tvs}\greyboxline{chuck}\greyboxline{{\tvs}{\tvs}}wood \\ [0.5ex]
}
{The rules used by word-aligned \repair for deciding when a pair
of symbols can become active pairs.  If a pair of adjacent symbols
satisfy any of the four rules, then it becomes an active pair.}
{Word-aligned \repair rules}{tab:prepair-wa-rules}

The alignment of words is caused by the fourth rule.
The fourth rule ensures that a word symbol can only be
combined with a non-word symbol, forming a mixed symbol,
if both of them are 
{\emph {complete}}.  A complete word is a sequence of one
or more word symbols, such that on both sides of it, 
there are only non-words.  A similar definition applies to 
non-words.  As there is only one such rule, words can only
be left components of mixed symbols, and non-words can only
be right components.  The first three rules simply allow 
longer phrases of the same type to be formed.

These four rules are applied at two locations
in the \repair algorithm of \algref{alg:repair-alg}.  They
are used when the message is initially scanned 
(step \ref{algline:repair-alg:scan}), and after an active 
pair replacement is made and the frequency counts of the
newly introduced symbol with its adjacent neighbours is incremented 
(step \ref{algline:repair-alg:incrcount}).  This
ensures that the active pair counts for active phrases only include
active pairs that satisfy one of these rules.  During recursive
pairing, even though the sequence changes from the time
when an active pair is identified to when 
it is replaced, the classification of active pairs remains
correct.  The first three rules classifies pairs of symbols
regardless of their contexts.  More importantly, changes in the 
sequence do not affect pairs classified using the fourth rule.  
Suppose a pair of symbols is classified as
a mixed pair when the symbol preceding the left component
is a non-word symbol.  Later, when this pair is replaced,
the preceding non-word symbol may have become part of a mixed
symbol.  But since all mixed phrases start with a word
symbol and end with a non-word symbol, the fourth rule still 
holds for this active pair.

\tabref{tab:woodchuck-short-wa} shows word-aligned \repair applied
to the same ``Woodchuck'' message, with the third column showing
the expanded phrases.  The expansion of phrase \tc{6} is 
the word ``wood'', as desired.  The pair ``d{\tvs}'' does not
exist as a phrase because, in this message, ``d'' does not appear
by itself as a complete word.  As another example, 
\figref{fig:woodchuck-complete-wa-phrases} shows the expanded phrases
generated by word-aligned \repair when it is applied to the
complete ``Woodchuck'' message (see \pgref{fig:tc-woodchuck}).
The word-alignment rules ensure that word symbols and non-word
symbols continue to grow until they are surrounded by symbols of
a different type.  Then, mixed symbols are created by having
complete word symbols combine with complete non-word symbols on
the right.  In the figure, every symbol represents phrases
that are either part of an English word, a complete English
word, or an alternating sequence of complete words followed by 
complete non-words.  In this example, the longest non-word symbol 
is only one symbol long (the space character, {\tvs}), 
so complete non-word phrases that are longer than one primitive 
do not exist in the phrase hierarchy.

\tab{lll}
{\multicolumn{1}{c}{Sequence} & \multicolumn{1}{c}{Phrase hierarchy} & \multicolumn{1}{c}{Expanded phrases} \\}
{how{\tvs}much{\tvs}wood{\tvs}could{\tvs}a{\tvs}woodchuck{\tvs}chuck & & \\
how{\tvs}mu\tc{1}{\tvs}wood{\tvs}could{\tvs}a{\tvs}wood\tc{1}uck{\tvs}\tc{1}uck & \tc{1} $\rightarrow$ c h & $\rightarrow$ ch \\
how{\tvs}mu\tc{1}{\tvs}wo\tc{2}{\tvs}could{\tvs}a{\tvs}wo\tc{2}\tc{1}uck{\tvs}\tc{1}uck & \tc{2} $\rightarrow$ o d & $\rightarrow$ od \\
how{\tvs}mu\tc{1}{\tvs}wo\tc{2}{\tvs}could{\tvs}a{\tvs}wo\tc{2}\tc{1}u\tc{3}{\tvs}\tc{1}u\tc{3} & \tc{3} $\rightarrow$ c k & $\rightarrow$ ck \\
how{\tvs}mu\tc{1}{\tvs}\tc{4}\tc{2}{\tvs}could{\tvs}a{\tvs}\tc{4}\tc{2}\tc{1}u\tc{3}{\tvs}\tc{1}u\tc{3} & \tc{4} $\rightarrow$ w o & $\rightarrow$ wo \\
how{\tvs}mu\tc{1}{\tvs}\tc{4}\tc{2}{\tvs}could{\tvs}a{\tvs}\tc{4}\tc{2}\tc{5}\tc{3}{\tvs}\tc{5}\tc{3} & \tc{5} $\rightarrow$ \tc{1} u & $\rightarrow$ chu \\
how{\tvs}mu\tc{1}{\tvs}\tc{6}{\tvs}could{\tvs}a{\tvs}\tc{6}\tc{5}\tc{3}{\tvs}\tc{5}\tc{3} & \tc{6} $\rightarrow$ \tc{4} \tc{2} & $\rightarrow$ wood \\
how{\tvs}mu\tc{1}{\tvs}\tc{6}{\tvs}could{\tvs}a{\tvs}\tc{6}\tc{7}{\tvs}\tc{7} & \tc{7} $\rightarrow$ \tc{5} \tc{3} & $\rightarrow$ chuck \\
}
{Applying word-aligned \repair to the same brief ``Woodchuck'' 
message used in \tabref{tab:repair-short} on 
\pgref{tab:repair-short}.}
{Sample application of word-aligned \repair}
{tab:woodchuck-short-wa}

\fig{\begin{tabular}{llll}
(W) & \tc{1} $\rightarrow$ ch                      & (M) & \tc{16} $\rightarrow$ wood{\tvs} \\
(W) & \tc{2} $\rightarrow$ wo                      & (W) & \tc{17} $\rightarrow$ could \\
(W) & \tc{3} $\rightarrow$ od                      & (M) & \tc{18} $\rightarrow$ could{\tvs} \\
(W) & \tc{4} $\rightarrow$ ck                      & (W) & \tc{19} $\rightarrow$ if \\
(W) & \tc{5} $\rightarrow$ chu                     & (W) & \tc{20} $\rightarrow$ as \\
(W) & \tc{6} $\rightarrow$ wood                    & (W) & \tc{21} $\rightarrow$ mu \\
(W) & \tc{7} $\rightarrow$ chuck                   & (M) & \tc{22} $\rightarrow$ a{\tvs}woodchuck{\tvs}could{\tvs} \\
(M) & \tc{8} $\rightarrow$ a{\tvs}                 & (M) & \tc{23} $\rightarrow$ if{\tvs} \\
(W) & \tc{9} $\rightarrow$ ld                      & (M) & \tc{24} $\rightarrow$ as{\tvs} \\
(W) & \tc{10} $\rightarrow$ woodchuck              & (W) & \tc{25} $\rightarrow$ much \\
(M) & \tc{11} $\rightarrow$ chuck{\tvs}            & (M) & \tc{26} $\rightarrow$ a{\tvs}woodchuck{\tvs}could{\tvs}chuck{\tvs} \\
(W) & \tc{12} $\rightarrow$ uld                    & (M) & \tc{27} $\rightarrow$ chuck{\tvs}if{\tvs} \\
(M) & \tc{13} $\rightarrow$ woodchuck{\tvs}        & (M) & \tc{28} $\rightarrow$ wood{\tvs}as{\tvs} \\
(M) & \tc{14} $\rightarrow$ a{\tvs}woodchuck{\tvs} & (M) & \tc{29} $\rightarrow$ much{\tvs} \\
(W) & \tc{15} $\rightarrow$ co                     & (M) & \tc{30} $\rightarrow$ chuck{\tvs}if{\tvs}a{\tvs}woodchuck{\tvs}could{\tvs}chuck{\tvs} \\

\end{tabular}}
{Expanded phrases generated by word-aligned \repair on the
complete ``Woodchuck'' message, in the order in which
they were added to the phrase hierarchy.  In front of 
each phrase, its word-aligned category is shown.
No phrases in this example are non-words (N).}
{Word-aligned \repair phrases}
{fig:woodchuck-complete-wa-phrases}

In a phrase hierarchy with $|\Sigma|$ primitives and $|\rho|$
phrases, word-aligned \repair requires an extra
$(|\Sigma| + |\rho|)$ words of memory compared to 
character-based \repair.  Every primitive and phrase in the 
phrase hierarchy is assigned a category.  Since the categories 
can be represented in 2 bits, an implementation which conserves
memory is possible which only uses an extra $2(|\Sigma| + |\rho|)$ 
bits.  
Categories could also be assigned to each symbol in
the sequence.  However, for a message of length $n$,
this approach requires $n$ words of
memory, which is typically larger than $(|\Sigma| + |\rho|)$.
Furthermore, augmenting categories to the sequence is
unnecessary because the category of a symbol in
the sequence is the same as its category in the phrase.
The phrase ``wood'' is a word symbol (W), regardless of the context
in which it appears.  When an occurrence of ``wood'' 
is considered as a component of an active pair, then its
adjacent neighbours in the sequence are checked.

There are two advantages to word-aligned \repair.  First, the 
alignment performed by word-aligned \repair operates
transparent to the decompressor.  
The augmented rules only affect the compressor.
Second, word-aligned \repair allows compound words 
like ``woodchuck'' to be split up into its constituent
words, provided they exist elsewhere in the message by 
themselves.

Experimental results with word-aligned \repair are provided
at the final section of this thesis.  However, problems with 
word-aligned \repair exist, and are addressed in the next section
by word-based \repair.

\newsection{Word-based \repair}{sec:prepair-wordbased}

Word-aligned \repair creates phrases that are more useful than
those of character-based \repair,
by aligning phrases to word boundaries.  However, several problems
remain in the phrase hierarchy which require a different solution.
First, two words used in the complete ``Woodchuck'' message cannot be 
found in the phrase hierarchy of \figref{fig:woodchuck-complete-wa-phrases}:
``how'' and ``would''.  Each word exists in the
message only once and while ``how'' can be located by performing a
direct search on the reduced sequence using primitives, ``would'' cannot 
be found since it has been broken into two phrases:  \tc{2} and 
\tc{12}.  Second, punctuation was removed in the complete ``Woodchuck''
message, but if it was included, then the phrase ``wood{\tvs}''
is treated differently from ``wood?''.  To a user, both phrases have
the same meaning, except the second one marks the end of a 
sentence.  Furthermore, case information of letters was removed from 
the message, but word-aligned \repair would have treated ``Wood{\tvs}''
as being a different phrase from ``wood{\tvs}''.

That is, while word-aligned \repair is an improvement over character-based
\repair, some problems remain.  
The essence of the problems is that word-aligned \repair does 
not force a message 
into words, but merely applies a set of rules only when a 
decision is required.  With respect to the first problem 
above, since ``how'' and ``would'' each appears once in the 
message, no decision was required by \repair, resulting in 
neither word appearing in the phrase hierarchy.
Instead, a word-based scheme is required, similar to the ones used by 
other compression systems described earlier.  Word-based 
compression requires an additional processing stage 
where a message is divided into word tokens and 
non-word tokens.  The processing can occur as the
message is compressed, or it may complete before compression
begins.  

The pre-processing used by word-based \repair needs to satisfy
two criteria in order to strike a balance between the requirements
of retrieval and compression.  First, the phrases
produced by \repair after the pre-processing should improve
compared to ones created by the character-based and word-aligned
\repair variants.  For example, phrases should be aligned
on word boundaries.
Second, like other compression algorithms, compression
must be lossless.  \figref{fig:prepair-wordbased} shows a structure
that satisfies these two conditions.

\fig{\includegraphics*{./wordbased.eps}}
{The pre-processing stage used by word-based \repair to separate
a message into non-words (top path) and case folded, stemmed 
words (bottom path).  Punctuation-aligned \repair, described
in \secref{sec:prepair-puncbased}, also follows a similar
scheme, except for the addition of punctuation flags for
\repair, shown as dashed lines.}{The pre-precessing stage 
used by word-based \repair}{fig:prepair-wordbased}

In \figref{fig:prepair-wordbased}, processes are shown as
rectangles, while the flow of data are indicated as lines.
Each process is further elaborated later in this section.
Initially, the message is separated into two streams of tokens:
one for words and the other for non-words.  The word tokens are then
case folded and stemmed. Lexicons for words and non-words
are maintained separately and updated as the message is processed.  The word
references are then compressed by \repair.  The punctuation flags,
shown as dashed lines, are explained in \secref{sec:prepair-puncbased}.
All of these steps are performed by the 
pre-processor, dubbed \prepair, to emphasise its use before
\repair.

\newsubsection{Separating the Message}{subsec:prepair-separate}

The message is parsed as an alternating sequence of word
and non-word tokens.  A token is at most 16 characters in
length.  For longer words (and non-words), a special
zero-length token exists in both types of tokens.  For
example, if a word of length 21 characters is encountered,
then it is split so that two words of lengths 16 and 5
are encoded, with a zero-length non-word in between.  
This parsing scheme is similar to the one described by 
\citet{moffat89:spe}.

Furthermore, to accommodate the \sgml test files used by this
thesis, the definition of words is extended.  The ``\tl''
character specifies the start of a new word token while 
the ``\tg'' character specifies the end of one, regardless of the
context in which they occur.  This definition 
ensures that \sgml tags are tokens, regardless of whether or not
there is whitespace surrounding them.

\newsubsection{Case folding}{subsec:prepair-casefold}

After a word token has been identified, it is case folded to 
lower case.   
Case folding is a technique commonly used by retrieval systems for 
constructing and accessing an index.
It permits a user to enter query terms 
without considering whether the case of the words 
being given to the system matches the case of the indexed terms.
However, in retrieval systems, the index is separate
from the message that it indexes.  So, the case folding can be a 
lossy transformation because it need not be reversed.  
In contrast, the case folding done by
\prepair needs to be reversible since \repair is used for
constructing phrases for both browsing and compression.  Therefore,
a modifier is
produced for each word token when case folded.  When the 
modifier is applied to the case folded token, the original 
word is produced.

Since any character in a word might be in upper case and so require folding,
a modifier of at least 16 bits is required, with each bit corresponding 
to a character position.
A bit value of ``1'' means that the corresponding source character 
is in upper case.  The bit flags are in reversed order with respect
to the characters in the word token.  That is, a word token with
only the first character in upper case results in a case folding
modifier with the least significant bit turned on.  This ensures
that all word tokens with only the initial character in upper case
have the same case folding modifier, regardless of word length.
Furthermore, an extra bit is used to indicate the special case 
when every character is in upper case, so that the case folding
modifier is 17 bits in length.
While encoding a token, every character is checked.
When decoding, as soon as no other bits in
the modifier are set to ``1'', the rest of the word is not
processed.

Reversible case folding is also used by the Length
Index Preserving Transform (\lipt) of \citet{am01:itcc}.  
\lipt is a compression pre-processor that uses a static
dictionary of lower case words.  As most words that 
contain upper case characters either have only the first 
character in upper case or all of the characters in upper case, 
special 
symbols are inserted into the message to indicate this.  
For other combination, individual characters are flagged if they 
are in upper case.  \prepair operates similar to \lipt except that
the case folding modifiers of \prepair exist as a stream
separate from the words.
Also, \prepair assigns a bit flag to every character that is
processed while \lipt adds a character modifier only if an upper case
character is found.  

\newsubsection{Stemming}{subsec:prepair-stem}

After the word token is case folded, it is stemmed.  
Stemming, or conflation \citep{frakes92:book}
is a technique used 
in information retrieval to remove word suffixes to
produce its root form.  For example, ``searches'',
``searching'', and ``search'' all stem to the word ``search''.
This allows users to query a retrieval system
without worrying about entering the correct ending for a
word.  Moreover, stemming also reduces the number of 
entries in the lexicon.

Several authors have proposed stemming algorithms 
\citep{lovins68:mtcl, dawson74:ballc, porter80:program}.  Of
these, the Porter stemming algorithm is used 
in this chapter, based upon the recommendation of  
\citet{kowalski97:book}. 
 
All stemming algorithms conflate a word by applying steps which
successively remove endings.  A step is applied if the word
matches a set of conditions.  Some
conditions include the length of the word, the characters
in the suffix, or the characters before the suffix.
The algorithms do not consider the semantic meaning of the
word, and each word in a document is stemmed independently of
adjacent words.  This is acceptable for most information 
retrieval systems since the stemmed word is used internally
to search through an index.  However, these results would
appear unexpected to a user.  \tabref{tab:prepair-stemwrong}
shows a few examples from applying the Porter stemming
algorithm.  In most cases, the root word obtained is not an
English word.  Of particular interest, the word ``skies'' is transformed 
to the unrelated word ``ski'', instead of ``sky'', through 
the removal of the ending ``es''.  Nevertheless, the benefits
of stemming outweigh the costs, and provided all related words
and no unrelated words stem to the same string most of the 
time, no problems are
caused.  As words are combined to form longer words 
during browsing, it is expected that neighbouring words 
make up for any important contexts lost during stemming.  

\tab{lll}
{\multicolumn{1}{c}{Original word} & \multicolumn{1}{c}{Root word obtained} & \multicolumn{1}{c}{Expected root word} \\}
{
character{\D}     & charact{\D}{\D}     & character \\
frequency{\D}     & frequenc{\D}     & frequent{\D} \\
retrieval{\D}     & retriev{\D}{\D}     & retrieve{\D} \\
searchable      & searchabl       & search{\D}{\D} \\
skies{\D}{\D}{\D}{\D}{\D} & ski{\D}{\D}{\D}{\D}{\D}{\D} & sky{\D}{\D}{\D}{\D}{\D} \\
pies{\D}{\D}{\D}{\D}{\D}{\D} & pi{\D}{\D}{\D}{\D}{\D}{\D}{\D} & pie{\D}{\D}{\D}{\D}{\D}\\
}{Some unexpected results from the Porter stemming algorithm.  
Words are listed in the first column.  The root form obtained by
the algorithm is shown in the second column, while the third column 
displays the root form a person most likely would provide.}
{Unexpected results from the Porter stemming algorithm}
{tab:prepair-stemwrong}

The stemming algorithm used by \repair contains minor changes to
the original stemming algorithm by Porter.  Details 
of the changes are provided in 
\appref{chap:porter}.  The American spelling assumed by the implementation
is retained in order to accommodate the spelling used by the test
data.

As with case folding, compression implies a  
requirement to be able to losslessly reverse any stemming performed.
A stream of stemming modifiers is generated, with each
word token having a corresponding modifier.
\appref{chap:porter} lists the steps to the algorithm and 
shows that a modifier
of length 23 bits is sufficient to indicate how to unstem a
word.  Each step (and sub-step) is 
assigned bit positions in the modifier, independent of other
steps.  While it might have been possible to reduce the number
of bits used by the modifier by combining steps, separating the
steps keeps the variant used by \repair as close to
the original as possible.

Since stemming reduces the length of a word, an alternative design
for \prepair would apply the 16 character
limit on word tokens after stemming.  However, this approach
was not taken for two reasons.  First, \prepair would be less
efficient because a buffer of characters which varies in length
would have to be maintained.  
Second, the one-to-one correspondence between
word tokens and stemming modifiers could no longer be guaranteed.
Reversing stemming not only increases the length of
a stem, but changes characters within it as well.  So, if
a long word is split across two word tokens, it may be incorrect
to assign the stemming modifier to only the second word token.
The consequence of this decision is that words longer than 16
characters would not be stemmed, even if their stemmed form
is less than the limit.

There are two reasons for incorporating case folding and stemming
into word-based \repair.  During phrase browsing, these transformations
group words that are derived from the same root word 
to improve usability, as shown later in \chapref{chap:rephine}
when the phrase browser is presented.  Also,
both case folding and stemming help reduce the number of entries in
the word lexicon.  
The effect these two transformations have on the number of distinct
word tokens is shown in \tabref{tab:prepair-lexicon-size} for the
test file \wsja.  A detailed discussion of the experiments with 
\wsja, including compression times and
compression ratios, is provided later, in \secref{sec:prepair-expt}.
Some of those results are presented here to complete
the discussion on case folding and stemming.
Case folding and stemming do not have to be used together.
However, because the implementation of 
the Porter stemming algorithm used by word-based \repair
only stems lower case words; words with endings in upper case 
are untouched.  This occurred often in \wsja
since certain sections of every news article was 
entirely in upper case.

\tab{lcc}
{                       & Unique & Average length \\
                        & tokens & in lexicon     \\
                        &        & (chars)        \\}
{
word lexicon & & \\
{\D\D\D}no transformations      & 74,686  & 7.31 \\
{\D\D\D}case folding only       & 60,748  & 7.38 \\
{\D\D\D}stemming only           & 50,412  & 6.26 \\
{\D\D\D}case folding and stemming & 39,638  & 6.27 \\
non-word lexicon                & \D1,399 & 7.68 \\}
{Statistics from \prepair which indicate the effects
of case folding and stemming on the word lexicon.  
Transformations to word tokens do not affect the non-word
lexicon, which is also shown.  The test data used was
\wsja.}{Relationship between lexicon size and 
word token transformations (\wsja)}
{tab:prepair-lexicon-size}

Lexicon statistics when the application 
of case folding and stemming is varied are shown in 
\tabref{tab:prepair-lexicon-size}.  In all cases, 
\prepair identified 3,356,915 word tokens.  Both case folding
and stemming decrease the number of unique word tokens, as
shown in the first column.  The average length of a word token 
in the message is 4.92 characters.
Note that this value is smaller than the average token lengths
in the lexicons, as shown in the second column of
\tabref{tab:prepair-lexicon-size}.  As one would expect, in an
English document, short words occur more frequently than long
ones.  The small averages imply that extending the limit
of 16 characters per token would achieve little benefit for
document collections such as \wsja.

Case folding and stemming transformations have no
effect on the non-word lexicon, whose size is shown in the last
row of \tabref{tab:prepair-lexicon-size}. 
The number of entries in the non-word lexicon is much less 
than the word lexicon.  The average length of a 
non-word token in the message is 1.34 characters, much less than
the 7.68 characters in the non-word lexicon.

While modifiers for case folding and stemming can be stored
together, they have been separated and placed into their own
individual streams.  This separation ensures a one-to-one
correspondence between a word token and its associated 
modifiers.  This approach highlights an important point regarding
how modifiers are used.  While users usually search for words,
it would be difficult to envision a user searching for a 
particular case folding modifier.  Instead, when a user has
found a word token of interest, the position $k$ of the word token
in the message needs to be determined.  Then, to reverse
case folding and stemming, the system must retrieve the modifiers
at position $k$ in the respective streams.  Experiments with
several compression systems were applied later in this chapter
in order to process the two streams in the context of 
this requirement.
But first, the next section provides more details about the 
two lexicons.

\newsubsection{Encoding the Lexicons}{subsec:prepair-lexicon}

After a word token is case folded and stemmed, it is located 
in the word lexicon and replaced with an ordinal number
indicating its position in the lexicon.  If the word token is
novel, then it
is replaced with the number representing the next 
available position, and the lexicon is updated.  
The pointer back to the lexicon is appended to the output word sequence.
After \prepair completes, its lexicon is written as a separate file 
while the stream of word sequence numbers
is then passed to \repair, to produce 
the {\emph {reduced word sequence}}.  Finally, the reduced word
sequence is compressed, to yield the 
{\emph {compressed word sequence}}.  

In contrast to the word sequence, the stream of 
non-word tokens are processed without any stemming, case folding, 
or application of \repair.  Also, the non-word sequence is
called the {\emph {non-word modifier}} stream.  That is, the
symbols in this stream are considered modifiers which
transform word tokens by appending non-word symbols to them.

The \prepair system keeps each lexicon in memory with a
splay tree \citep{st85:jacm}.  A splay tree
is a binary search tree that is rotated after every node access or 
modification to ensure that the most recently used node is at 
the root of the tree.  The number of entries in the lexicon is 
equal to the number of unique case folded, stemmed words in the
document.  \citet{hzw02:tis} describes an alternative data
structure for maintaining lexicons called a {\emph {burst trie}}.

Three methods of compressing the lexicons exist.
The first method involves the application of a transformation 
called {\emph {front coding}} \citep{ghlr75:ncc}.  
Front coding is a technique often used to encode a sorted list
of strings.  Many adjacent strings in a sorted list have 
similar prefixes.  Front coding exploits this fact by encoding
every word in the lexicon as a triple made up of two numbers and
a substring.  The first number indicates the number of characters
the prefix of the current word matches to the prefix of the 
previous word (0 if nothing matches).  The second number is the 
remaining length of the current word, without the length of the 
matching prefix.  The substring of the remainder of
the word is the last part of the triple.
Recall that in word-based \repair, word and non-word tokens are 
limited to 16 characters in length.  Therefore, 8 bits are 
sufficient for indicating the two numeric triples of each word.

The second method of lexicon encoding uses one of the 
compression systems described in \chapref{chap:tc}. 
The lexicon is written in sorted order, with each token
preceded by one byte, indicating its length.  In the experiments
described shortly, this is called literal encoding.  Then
one of the following three
compression systems are applied:  \gzip, \bzip, and \ppmd.  
Similar to the experiments in the
previous chapter, \gzip and \bzip were executed using 
the {\texttt {-9}} option in order to favour compression
effectiveness.  Likewise, \ppmd used \mib{255} of
memory and a seventh order context.  The third scheme 
combines the first two by
applying one of these compression systems to the output of
the front coder.
\citet{nw97:cj} also used \ppm after front coding for
compressing \sequitur's word lexicon.

\tabref{tab:prepair-wordlexicon-itself} and 
\tabref{tab:prepair-wordlexicon-wsja} show the
compression effectiveness of the word lexicons with respect to itself
and with respect to the size of \wsja, respectively.  Case folding
and stemming have been varied along with the three compression
strategies.  The two views of the results are necessary because
these transformations affect the size of the lexicons.  As a benchmark,
the tables also provide compression ratios when literal coding is
used without any additional compression system.
Compression ratios are reported in bits per
character (bpc) with the best result in each table highlighted.

\tabnoline{c}{}
{
  \begin{tabular}{lccccc}
  \\ \cline{2-3} \cline{5-6}
  & Literal coding & Front coding & & Literal coding & Front coding \\ \cline{2-3} \cline{5-6}

none & 8.000 & 3.518 & & 8.000 & 3.370\\
\gzip & 3.411 & 2.054 & & 3.289 & 1.928 \\
\bzip & 3.770 & \greybox{1.771} & & 3.714 & \greybox{1.677} \\
\ppmd & 3.442 & 1.864 & & 3.416 & 1.751 \\ \cline{2-3} \cline{5-6}
  & \multicolumn{2}{c}{(a) No case folding or stemming} & & 
    \multicolumn{2}{c}{(b) Case folding only}  \\

  \\
  \\ \cline{2-3} \cline{5-6}
  & Literal coding & Front coding & & Literal coding & Front coding \\ \cline{2-3} \cline{5-6}

none & 8.000 & 3.958 & & 8.000 & 3.801 \\
\gzip & 3.857 & 2.512 & & 3.701 & 2.341 \\
\bzip & 4.233 & \greybox{2.202} & & 4.130 & \greybox{2.085} \\
\ppmd & 3.926 & 2.363 & & 3.905 & 2.226 \\ \cline{2-3} \cline{5-6}
  & \multicolumn{2}{c}{(c) Stemming only} & &
    \multicolumn{2}{c}{(d) Case folding and stemming}  \\
  \end{tabular}
}
{Comparing the use of front coding 
with several compression systems on  
the word lexicon of \wsja while varying the use of
case folding and stemming.  Results are in bits per character with
respect to the size of the lexicon.}
{Compression results for word lexicons with respect to themselves (\wsja)}
{tab:prepair-wordlexicon-itself}

\tabnoline{c}{}
{
  \begin{tabular}{lccccc}
  \\ \cline{2-3} \cline{5-6}
  & Literal coding & Front coding & & Literal coding & Front coding \\ \cline{2-3} \cline{5-6}
  none & 0.237 & 0.104 & & 0.194 & 0.082 \\
  \gzip & 0.101 & 0.061 & & 0.080 & 0.047 \\
  \bzip & 0.112 & \greybox{0.052} & & 0.090 & \greybox{0.041} \\
  \ppmd & 0.102 & 0.055 & & 0.083 & 0.042 \\ \cline{2-3} \cline{5-6}
  & \multicolumn{2}{c}{(a) No case folding or stemming} & & 
    \multicolumn{2}{c}{(b) Case folding only}  \\

  \\
  \\ \cline{2-3} \cline{5-6}
  & Literal coding & Front coding & & Literal coding & Front coding \\ \cline{2-3} \cline{5-6}
none & 0.140 & 0.069 & & 0.110 & 0.052 \\
\gzip & 0.067 & 0.044 & & 0.051 & 0.032 \\
\bzip & 0.074 & \greybox{0.038} & & 0.057 & \greybox{0.029} \\
\ppmd & 0.068 & 0.041 & & 0.054 & 0.031 \\ \cline{2-3} \cline{5-6}
  & \multicolumn{2}{c}{(c) Stemming only} & &
    \multicolumn{2}{c}{(d) Case folding and stemming}  \\
  \end{tabular}
}
{Comparing the use of front coding 
with several compression systems on  
the word lexicon of \wsja while varying the use of
case folding and stemming.  Results are in bits per character
with respect to the size of \wsja.}
{Compression results for word lexicons with respect to file size (\wsja)}
{tab:prepair-wordlexicon-wsja}

When literal coding is not coupled with any compression 
mechanism, the size of the word lexicons are shown in the 
tables in the rows marked ``none'', under the first columns.  
Case folding and stemming both decrease the amount of
space occupied.  In all four combinations, 
\gzip achieves the best compression when no front coding is
applied.
Of course, a smaller word lexicon means that
some compression effectiveness must be paid for by additional
modifier streams, as described later in this chapter.
Once front coding is used, compression with any of the three
systems surveyed is better than applying the same program by itself.
In particular, in all cases, front coding
with \bzip offers the best compression effectiveness.

\tabref{tab:prepair-nonwordlexicon} shows the effect of applying
these compression mechanisms on the non-word lexicon.  On the
left, the compression results are reported with respect to the
size of the lexicon; on the right, they are presented with respect
to the size of \wsja.  According to the table on the right, the non-word
lexicon is only 0.005 bpc when output as literal text, while the
word lexicon is at least 0.110 bpc when case folding and stemming are
both applied.  Again, front coding with \bzip provided good compression
effectiveness at 0.001 bpc.  Other regimes are also able to attain similar
compression ratios for the non-word lexicon.  These tables do not 
mention the time required for compressing the lexicons.  Later
in this chapter, detailed experiments with \wsja show that both 
lexicons are small in size compared to the other streams.  Processing 
the lexicons contributes little to the overall compression and 
decompression times.

\tabnoline{lccccc}{}
{
\\ \cline{2-3} \cline{5-6}
  & Literal coding & Front coding & & Literal coding & Front coding \\ \cline{2-3} \cline{5-6}
none & 8.000 & 2.925 & & 0.005 & 0.002 \\
\gzip & 2.632 & 1.464 & & 0.002 & 0.001 \\
\bzip & 2.588 & 1.446 & & 0.001 & 0.001 \\
\ppmd & 2.385 & \greybox{1.373} & & 0.001 & \greybox{0.001} \\ \cline{2-3} \cline{5-6}
  & \multicolumn{2}{c}{(a) Relative to the lexicon} & &
    \multicolumn{2}{c}{(b) Relative to \wsja}  \\
}
{Comparing the difference between the use of front coding 
with several compression systems on 
the non-word lexicon of \wsja.  Results are in bits per character.}
{Compression results for the non-word lexicon (\wsja)}
{tab:prepair-nonwordlexicon}

\newsection{Punctuation-aligned \repair}{sec:prepair-puncbased}

The pre-processing step of \prepair solves many of the problems
that existed with character-based \repair and word-aligned 
\repair.  A word in the message is no longer broken into more
than one phrase.  Moreover, all words, even infrequent ones,
are in the word lexicon and are therefore primitives in the
phrase hierarchy.

One minor problem remains, which is related to the treatment of
punctuation marks.
In English, punctuation marks such as commas and full stops
separate phrases or sentences from each other.  Word-based \repair
ignores these punctuation marks and allows phrases to
span over them.  A minor change
to word-based \repair, called punctuation-aligned \repair, 
encourages phrases to end with punctuation marks.

Punctuation-aligned \repair identifies a set of punctuation marks
as special processing symbols.  The set of six punctuation marks
are:  the
exclamation mark (!), the full stop (.), the question mark
(?), the comma (,), the colon (:), and the semicolon (;).
The first three punctuation marks set off a sentence from the
surrounding text, while the remaining three separate an
independent clause, a phrase, or a subordinate clause from the rest
of the sentence.  During parsing with \prepair, these punctuation 
marks are identified without understanding the context in which they
occur.  Because of this, \prepair treats a full stop in an abbreviation
as if a sentence had just ended.
All of these punctuation
marks are categorised as {\emph {separating punctuation marks}},
in contrast to {\emph {enclosing punctuation marks}} 
\citep{greenbaum96:book}.  Enclosing
punctuation marks, such as parentheses, highlight a section of text
usually through a pair of delimiters.

%%  all 6 punctuation marks:
%%  bible.txt:  119721/122115 = 98.0%
%%  wsj20:  481423/1072725 = 44.9%
%%  wsj20:  481423/788325 = 61.1% (without <, >, and slashes (1/2 of the > brackets))
While other punctuation marks could have been identified, adding too
many would have degraded the performance of \prepair.  Moreover, 
\citet{greenbaum96:book} presented results from a study 
of American texts which consisted of writing totalling 72,000 words.
The documents were all samples of ``journalism, learned  writing, 
and fiction''.  The results showed that the six punctuation
marks used by punctuation-aligned \repair accounted for 95.9\% of
all punctuation marks.  Full stops and commas alone contributed to
91.8\% of the punctuation marks found.  In comparison, these six
punctuation marks make up 98.0\% and 61.9\% of the punctuation marks
in {\texttt {bible.txt}} (Large Canterbury Corpus, described in 
\chapref{chap:tc}) and \wsja.  The percentage of punctuation marks 
for \wsja is lower, even though this value has already excluded \sgml
tags.  A few lines from \wsja were provided in 
\subfigref{fig:tc-sgmlsample}{a} on \pgref{fig:tc-sgmlsample}.  As the
figure shows, other punctuation marks such as hyphens and slashes
appear, which do not segment sequences of words into phrases.

In the same way that word-aligned \repair modifies 
character-based \repair, punctuation-aligned \repair augments
a set of rules to word-based \repair which dictate what
active pairs can be chosen.  These rules are shown in
\tabref{tab:prepair-pb-rules}.  Every symbol in the
word sequence is assigned a category.  Either it is a 
word that is not followed by any of the 6 punctuation
marks (W), or it is (P).  The table gives the four ways
these two symbol types, shown under the first two columns
of the table, can combine.  If the combination is
permitted, and subsequently replaced, the new symbol
obtains a category as shown in the third column.  
The fourth
column of the table gives some examples in which the
symbol combinations may occur using variations on
``Melbourne University'', an unofficial name for The
University of Melbourne.

\tab{cccl}{Left & Right & Result & \multicolumn{1}{c}{Example} \\}
{ \\[-2.5ex]
W & W & W & \greyboxline{Melbourne }\greyboxline{University }\\ [0.5ex]
P & P & P & \greyboxline{M.}\greyboxline{U.}\\ [0.5ex]
W & P & P & \greyboxline{Melbourne }\greyboxline{University.  }\\ [0.5ex]
P & W & Not allowed & In \greyboxline{Melbourne, }\greyboxline{universities }\\ [0.5ex]
}
{The rules used by punctuation-aligned \repair for 
deciding when a pair of symbols can become active 
pairs.  An active pair is allowed if any of the first 
three rules apply; an active pair is disallowed if the
last rule applies.}
{Punctuation-aligned \repair rules}{tab:prepair-pb-rules}

Whether or not a word is followed by a punctuation mark,
it can become an active pair with symbols of the same 
category.  If it is replaced, the new symbol inherits
the same category as its components.  The rule in the
second row permits abbreviations and words in a series,
separated by commas, to form.  
However, symbols of two different types can only combine if
the right component is a symbol followed by a punctuation
mark.  The reversed combination (fourth row in the table)
is disallowed.  Note that these rules do
not pair symbols based on the categories of the surrounding
symbols.  While word-aligned
\repair forced complete words and non-words to be formed by
examining the contexts in which symbols occur, 
punctuation-aligned \repair is less restrictive.  
For example, in the second rule of \tabref{tab:prepair-pb-rules}, 
two symbols 
with P categories can combine regardless of the classification of the
symbol to the left of the left component.
Punctuation-aligned \repair merely encourages word tokens 
to combine to the left if they are followed by punctuation marks.
This design choice recognises the fact that phrases can form
without punctuation marks, so confining phrases to exist only 
between punctuation marks would be inappropriate.

There are three other differences between word-aligned
and punctuation-aligned \repair.  First, a mixture (M)
category is unnecessary since anything that combines with
a word token terminated by a punctuation mark is a
(P) token.  Second, the symbol
categories are stored with the sequence and not with the
phrase hierarchy.  Unlike word-aligned \repair, the
category of a word token depends entirely on the
presence of a punctuation mark after it.  Therefore,
maintaining the categories in the phrase hierarchy
is not possible.  As a result, for a message of 
$n$ symbols, $n$ extra words of memory are required
to implement punctuation-aligned \repair.  Since only one
bit is ever used, this approach is wasteful, and a more
careful implementation is possible which operates on 
the bit level.  Then, only $n$ extra bits of memory would
be needed.
Finally, the information about punctuation marks
has been separated from the word tokens by \prepair.
In order to implement punctuation-aligned \repair, 
punctuation flags (shown as dashed lines in
\figref{fig:prepair-wordbased} on 
\pgref{fig:prepair-wordbased}) need to be passed
to \repair along with the word sequence.  For each
word token, the presence or absence of a punctuation 
mark following it is indicated by a flag.  These flags are
only used by \repair, and the output structure of 
punctuation-aligned \repair
remains identical to that of word-based \repair.

\figref{fig:prepair-biblephrases} shows the
difference between the four versions of \repair with respect to
the type of phrases created.  The document used is 
{\texttt {bible.txt}}, chosen because the \sgml markup in the 
\news file, and the files that are derived from it, would clutter
the example.  After being compressed, the first 20 symbols
have been expanded into primitives for each version
of \repair.  Since character-based and word-aligned \repair 
do not remove non-word characters, the space and linefeed characters
are represented as ``\tvs'' and ``\tb'', respectively, in the figure.
Case folding and stemming were used for word-based 
and punctuation-aligned \repair, with a space added between 
each word to facilitate display.

\fig{
\begin{tabular}{l}
\greyboxtt{In{\tvs}the{\tvs}}
\greyboxtt{beginning{\tvs}}
\greyboxtt{God{\tvs}}
\greyboxtt{created{\tvs}the{\tvs}}
\greyboxtt{heaven{\tvs}and{\tvs}the{\tvs}earth} \\
\greyboxtt{.{\tvs}And{\tvs}the{\tvs}}
\greyboxtt{earth{\tvs}was{\tvs}}
\greyboxtt{without{\tvs}form,{\tvs}and{\tvs}void}
\greyboxtt{;{\tvs}and{\tvs}darkness{\tvs}was{\tvs}} \\
\greyboxtt{upon{\tvs}the{\tvs}face{\tvs}of{\tvs}the{\tvs}}
\greyboxtt{deep}
\greyboxtt{.{\tvs}And{\tvs}the{\tvs}}
\greyboxtt{Spirit{\tvs}of{\tvs}God{\tvs}}
\greyboxtt{moved{\tvs}} \\
\greyboxtt{upon{\tvs}the{\tvs}face{\tvs}of{\tvs}the{\tvs}water}
\greyboxtt{s.{\tvs}{\tb}And{\tvs}God{\tvs}}
\greyboxtt{said,{\tvs}Let{\tvs}there{\tvs}be{\tvs}} \\
\greyboxtt{light} 
\greyboxtt{:{\tvs}and{\tvs}there{\tvs}was{\tvs}}
\greyboxtt{light}
 \\ [0.5ex]
\multicolumn{1}{c}{(a) Character-based \repair} \\ [1.0ex]
\greyboxtt{In{\tvs}the{\tvs}beginning{\tvs}}
\greyboxtt{God{\tvs}}
\greyboxtt{created{\tvs}the{\tvs}}
\greyboxtt{heaven{\tvs}and{\tvs}the{\tvs}}
\greyboxtt{earth.{\tvs}} \\
\greyboxtt{And{\tvs}the{\tvs}earth{\tvs}}
\greyboxtt{was{\tvs}without{\tvs}}
\greyboxtt{form,{\tvs}and{\tvs}void;{\tvs}}
\greyboxtt{and{\tvs}darkness{\tvs}was{\tvs}} \\
\greyboxtt{upon{\tvs}the{\tvs}face{\tvs}of{\tvs}the{\tvs}}
\greyboxtt{deep}
\greyboxtt{.{\tvs}}
\greyboxtt{And{\tvs}the{\tvs}Spirit{\tvs}of{\tvs}God{\tvs}}
\greyboxtt{moved{\tvs}} \\
\greyboxtt{upon{\tvs}the{\tvs}face{\tvs}of{\tvs}the{\tvs}waters.{\tvs}{\tb}}
\greyboxtt{And{\tvs}God{\tvs}said,{\tvs}Let{\tvs}}
\greyboxtt{there{\tvs}be{\tvs}} \\
\greyboxtt{light:{\tvs}} 
\greyboxtt{and{\tvs}there{\tvs}was{\tvs}}
\greyboxtt{light.{\tvs}{\tb}}
 \\ [0.5ex]
\multicolumn{1}{c}{(b) Word-aligned \repair} \\ [1.0ex]
\greyboxtt{in the begin }
\greyboxtt{god creat }
\greyboxtt{the heaven and the earth } \\
\greyboxtt{and the earth was } 
\greyboxtt{without form and void }
\greyboxtt{and dark }
\greyboxtt{was } \\
\greyboxtt{upon the face } 
\greyboxtt{of the deep } 
\greyboxtt{and the spirit of god }
\greyboxtt{move } \\
\greyboxtt{upon the face of the water } 
\greyboxtt{and god said let there be }
\greyboxtt{light } \\
\greyboxtt{and there was } 
\greyboxtt{light }
\greyboxtt{and god } 
\greyboxtt{saw the }
\greyboxtt{light }
\greyboxtt{that it was }
 \\ [0.5ex]
\multicolumn{1}{c}{(c) Word-based \repair with case folding and stemming} \\ [1.0ex]
\greyboxtt{in the begin }
\greyboxtt{god }
\greyboxtt{creat }
\greyboxtt{the heaven and the earth } \\
\greyboxtt{and the earth was } 
\greyboxtt{without form }
\greyboxtt{and void }
\greyboxtt{and dark was } \\
\greyboxtt{upon the face } 
\greyboxtt{of the deep } 
\greyboxtt{and the spirit of god }
\greyboxtt{move } \\
\greyboxtt{upon the face of the water } 
\greyboxtt{and god said }
\greyboxtt{let there be light } \\
\greyboxtt{and there was } 
\greyboxtt{light }
\greyboxtt{and god } 
\greyboxtt{saw the }
\greyboxtt{light }
 \\ [0.5ex]
\multicolumn{1}{c}{(d) Punctuation-aligned \repair with case folding and stemming} \\ [1.0ex]
\end{tabular}
}{Expanding the first 20 symbols of {\texttt {bible.txt}} into primitives,
after being compressed with character-based \repair,
word-aligned \repair, word-based \repair, and
punctuation-aligned \repair.  For the purposes of display, 
{\tb} and {\tvs} are used to indicate linefeeds and spaces, respectively,
in (a) and (b).  Non-words are not shown in (c) and (d).}
{Comparing phrases from the four versions of \repair 
({\texttt {bible.txt}})}{fig:prepair-biblephrases}

Since character-based \repair forms phrases using only frequency,
word boundaries are not adhered to, as shown in 
\subfigref{fig:prepair-biblephrases}{a}.  For example, in the fourth
line, the word ``waters'' spans the first and the second phrases of
that line.  Also, some phrases commence with non-words.  Word-aligned
\repair handles both of these problems by using its rules
to guide the recursive pairing.  While a phrase may contain part of a
word, no word spans across phrases, as shown in 
\subfigref{fig:prepair-biblephrases}{b}.  Also, all phrases start 
with a complete word and end with a complete non-word.
Alternatively, they can also contain all word characters, or all 
non-word characters.  A phrase categorised as a mixture cannot begin 
with a non-word character.

The phrases from word-based \repair in 
\subfigref{fig:prepair-biblephrases}{c} have been case folded and 
stemmed.  Since the smallest atomic units are word tokens, word-based
\repair ensures that no word spans across one or more phrases.
Moreover, non-word characters do not appear in 
the expanded phrases anymore.  While case folding and stemming have 
transformed some of the words, they still resemble the original words,
when compared with \subfigref{fig:prepair-biblephrases}{a}.  Word-based
\repair ignores the location of punctuation marks.  Punctuation-aligned
\repair rectifies this problem by encouraging word tokens followed by a
punctuation mark to only pair with a symbol to its left.
For example, comparing word-based \repair with punctuation-aligned 
\repair, in the second line of \subfigref{fig:prepair-biblephrases}{d},
``without form, and void'' is no longer accepted as a phrase.  Also,
while word-based \repair created the phrases ``and god said let there be''
and ``light'' in the fourth line, punctuation-aligned \repair separated
these seven words according to the locations of the comma and the colon.

\newsection{Experiments}{sec:prepair-expt}

The compression efficiency and effectiveness of 
the three methods of 
word-aware document parsing were compared with 
each other and with character-based 
\repair through experiments.
Partial results from these experiments have already been
shown in \secref{sec:prepair-wordbased}.  In all of
these experiments, \wsja was used as test data
and the test machine was a \vipe.

Before the four versions of \repair are compared, a detailed look
at the various streams of word-based \repair are examined.  
\secref{sec:prepair-wordbased} showed compression results for
the word and non-word lexicons of \wsja.  For both of these 
streams, front coding followed by \bzip offered the best 
compression effectiveness.  Statistics about the remaining
four streams produced by \prepair are listed in 
\tabref{tab:prepair-fileinfo}.  

\tab{lc@{}c@{}cccc}
{ & \multicolumn{3}{c}{\multirow{2}*{Symbol range}} & Distinct & Number of  & Self- \\
  & & &                              & values   & symbols    & information \\}
{
case folding modifier  & 0 & {\D}to{\D}& \C\D65,536   & \D\D\C72  & 3,356,915 & {\D}1.227 \\
stemming modifier      & 0 & {\D}to{\D}& 7,356,416    & \D\D\C346 & 3,356,915 & {\D}2.292 \\
non-word modifier      & 0 & {\D}to{\D}& \C\D\D1,398  & \D\D1,399 & 3,356,915 & {\D}1.750 \\
word sequence          & 0 & {\D}to{\D}& \C\D39,637   & \D39,638  & 3,356,915 & 10.272    \\
reduced word sequence  & 0 & {\D}to{\D}& \C235,929    & 218,822   & 1,582,065 & 15.829 \\
}{Statistics about some of the streams from \prepair,
and the reduced word
sequence from \repair on the \wsja file.  
The column headed self-information
is the zero-order self-information for the frequency distribution of
the symbols in that file, measured for each file in bits per
symbol relative to the size of that file.}{Statistics for
the modifiers and the reduced word sequence (\wsja)}
{tab:prepair-fileinfo}

The word sequence is compressed with \repair, by treating the
word sequence symbols as primitives.  A hierarchy of phrases
is constructed for browsing, as well as a reduced word sequence.
In \chapref{chap:repair}, experiments showed that the phrase
hierarchy can be effectively stored by transforming it
with chiastic slide, and then encoding it with interpolative coding.
Statistics about the reduced word sequence of \wsja are also given
in \tabref{tab:prepair-fileinfo}.

The three streams of modifiers are stored as 32-bit integers.
The number of distinct symbols in any of these streams is less than
the word sequence.  Also the self-information of these streams is
lower than the word sequence.

The modifiers are also compressed, but
since a phrase hierarchy is not required for browsing, compression
systems other than \repair can also be applied.  Results obtained
by applying five compression systems to \wsja are presented in 
\tabref{tab:prepair-modbpc}.  The compression mechanisms employed
include the zero-order Huffman coder \shuff, \gzip, \bzip, 
\ppmd, and
\repair.  The \shuff program processed the modifiers as a single block,
while \gzip and \bzip were executed with the {\texttt {-9}}
option in order to favour compression effectiveness.  The \ppmd system
operated with a seventh order model and a limit of \mib{255} of memory.
The three modifiers were also processed with \repair as a single block,
with \shuff used to encode the sequence, also as a single block.
While \gzip, \bzip, and \ppmd compress the modifiers as individual 
bytes, both \repair and \shuff treat each 32-bit integer as a symbol.

\tab{lccccc}
{
   & \multirow{2}*{\shuff} & \multirow{2}*{\gzip} & \multirow{2}*{\bzip} & \multirow{2}*{\ppmd} & \repair \\
   &        &       &       &       & with \shuff \\
}
{
case folding modifiers & 0.232 & 0.200 & 0.142 & 0.229 & {\greybox {0.125}} \\
stemming modifiers     & {\greybox {0.388}} & 0.583 & 0.457 & 0.520 & 0.394 \\
non-word modifiers     & 0.320 & 0.367 & 0.261 & 0.362 & {\greybox {0.224}} \\
}{Compression effectiveness of the modifiers from 
word-based \repair for \wsja.}
{Compression ratios for word-based \repair modifiers (\wsja)}
{tab:prepair-modbpc}

The compression results from \tabref{tab:prepair-modbpc} show the compression
results in bits per character, relative to the size of \wsja.  The best
compression ratio for each modifier is highlighted.  While the case folding
and non-word modifiers are most effectively processed by \repair (with \shuff),
\shuff alone provided the best compression ratio for the stemming modifiers.
Moreover, \bzip also gave better compression effectiveness than \shuff 
for the same modifiers as \repair.  Both \repair and \bzip are better
at capturing the repetition in the modifier streams.

The time required to process these streams with the five systems are
reported in \tabref{tab:prepair-modtime}, with the fastest encoding and
decoding time for each stream highlighted.  The \shuff system operated
the fastest for both encoding and decoding, partly due to the simplistic
zero-order model that it is based on.  The \bzip program was particularly
slow for encoding the non-word modifier, but otherwise yielded consistent
encoding and decoding times.  Decoding with the dictionary-based systems
(\gzip and \repair) are fast, and in one case, gave the same decoding time
as \shuff.

\tab{lccccc}
{
   & \multirow{2}*{\shuff} & \multirow{2}*{\gzip} & \multirow{2}*{\bzip} & \multirow{2}*{\ppmd} & \repair \\
   &        &       &       &       & with \shuff \\
}
{
Encoding time & & & & & \\
{\D}{\D}case folding modifiers       & {\greybox {0.4}}  & 20.4 & \D4.8  & 4.0  & \D9.5  \\
{\D}{\D}stemming modifiers           & {\greybox {0.4}}  & 51.0 & \D4.6  & 5.0  & 13.3 \\
{\D}{\D}non-word modifiers           & {\greybox {0.4}}  & 17.1 & 70.6   & 5.3  & 10.4 \\
Decoding time & & & & & \\
{\D}{\D}case folding modifiers       & {\greybox {0.1}}  &\D0.2 & \D1.3  & 4.6  & {\greybox {0.1}}  \\
{\D}{\D}stemming modifiers           & {\greybox {0.2}}  &\D0.3 & \D1.9  & 5.5  & 0.3  \\
{\D}{\D}non-word modifiers           & {\greybox {0.1}}  &\D0.3 & \D3.4  & 5.9  & 0.2  \\
}{Compression times for the modifiers produced by word-based \repair on
\wsja, reported in CPU seconds and averaged over three runs each.}
{Compression times for word-based \repair modifiers (\wsja)}
{tab:prepair-modtime}

These results show that a combination of \shuff alone and \repair coupled 
with \shuff seem to offer the best compression effectiveness for the
modifiers.  In addition, both systems offer fast decoding times, an 
important factor to consider when designing a retrieval system.  
While these two
systems satisfy the needs of compression, they are inappropriate
for retrieval for other reasons.  These reasons are related to how
these streams are used in the browsing system, a problem which is
covered later beginning in \chapref{chap:review}.
The method used for encoding the reduced sequence and the modifiers
is revisited at that time.

\tabref{tab:prepair-stats} presents some statistics when
\repair is applied to either the original \wsja file of 20,971,520
characters (character-based and word-aligned \repair) 
or to the word sequence of 3,356,915 symbols (word-based 
and punctuation-aligned \repair) produced from \prepair.  The
results for character-based \repair are taken from
\tabref{tab:repair-stats} on \pgref{tab:repair-stats}.  
Word-aligned \repair produces
a larger phrase hierarchy than character-based \repair,
but with less generations and a longer reduced sequence.  Many
phrases in the word-aligned phrase hierarchy include a 
phrase ``$\alpha$'' in one generation, and then
the phrase ``$\alpha${\tvs}'' in the next generation. 
This effect is also evident in the phrase hierarchy 
of \figref{fig:woodchuck-complete-wa-phrases} on
\pgref{fig:woodchuck-complete-wa-phrases}.

%%  Note:  \Seq length includes the terminating 0 symbol.
\tab{lccccc}
{\multicolumn{1}{c}{\multirow{2}*{Method}} & \multirow{2}*{$|\Sigma|$} & \multirow{2}*{$|\rho|$}  & \multirow{2}*{$|\Seq|$}     & \multirow{2}*{Generations} & Longest phrase \\
                         &            &          & &             & (in primitives) \\}
{ 
character-based        & \D\D\C90 & 310,486 & 1,904,118 & 21 & 1,257 \\
word-aligned           & \D\D\C90 & 343,520 & 1,912,566 & 19 & 1,269 \\
word-based             & 39,638   & 196,291 & 1,582,065 & 15 & \C215 \\
punctuation-aligned      & 39,638   & 191,924 & 1,634,652 & 13 & \C215 \\
}{Statistics from experiments with the four variants
of \repair with \wsja.  The first two methods are applied
to \wsja directly, where the characters are primitives.
The other two methods are used on the word sequence 
produced by \prepair.}
{Statistics for the four variants
of \repair (\wsja)}{tab:prepair-stats}

As \tabref{tab:prepair-stats} shows, word-based and punctuation-aligned
\repair are similar, but both produce a shorter reduced sequence
and a smaller phrase hierarchy than character-based 
and word-aligned \repair.  Also, the maximum phrase generation
between the two is 15.  The identification of word tokens with \prepair
allow contexts of characters to be found during pre-processing,
before \repair is used.  

The last column of the table indicates the longest phrase of
each method, measured in primitives.  The
longest phrase for word-based \repair
is 215 symbols, which expands to 1,325 characters in length.
This is a measure of stemmed, case folded
words with no non-words.  Without these two transformations, 
the same phrase also has more primitives than the longest
phrase for character-based \repair:  1,421 characters.

The compression effectiveness of the four versions of \repair is listed
in \tabref{tab:prepair-bpc}.  The test file is \wsja and compression 
ratios are given in bits per character relative to the original file.
The results for character-based \repair are from 
\tabref{tab:repair-bpc} on \pgref{tab:repair-bpc}.  
Compression levels for four variants of 
word-based \repair are presented, which each differ by whether or not
case folding and stemming are applied.  Punctuation-aligned \repair
imposes rules on \repair which prevents certain phrases from forming.
However, the three modifier streams are unchanged.  In the table,
only results for punctuation-aligned \repair with both case folding
and stemming are shown.  All lexicons are encoded with front coding
followed by \bzip.  The reduced word sequence and the three streams of
modifiers are encoded using \shuff.  

\sidetab{lcccccccc}
{
& Word & Phrase & Reduced & Case folding & Stemming & Non-word & Non-word & \\
& lexicon & hierarchy & (word) & modifiers & modifiers & lexicon & modifiers & Total \\
&         &           & sequence &         &           &         &           & \\
}
{
character-based             &  -    & 0.223 & 1.551 &   -   &   -   &  -    &   -   & 1.774  \\
word-aligned                &  -    & 0.201 & 1.533 &   -   &   -   &  -    &   -   & 1.734  \\
word-based                  &       &       &       &       &       &       &       &        \\
{\D}{\D}no transformations  & 0.052 & 0.154 & 1.320 &   -   &   -   & 0.001 & 0.320 & {\greybox {1.708}}  \\
{\D}{\D}case folding only   & 0.041 & 0.152 & 1.285 & 0.232 &   -   & 0.001 & 0.320 & 2.031  \\
{\D}{\D}stemming only       & 0.038 & 0.145 & 1.264 &   -   & 0.380 & 0.001 & 0.320 & 2.148  \\
{\D}{\D}both                & 0.029 & 0.143 & 1.230 & 0.232 & 0.388 & 0.001 & 0.320 & 2.343  \\
punctuation-aligned         &       &       &       &       &       &       &       &        \\
{\D}{\D}both                & 0.029 & 0.138 & 1.254 & 0.232 & 0.388 & 0.001 & 0.320 & 2.362  \\
}{Comparison of the compression ratios achieved by the four 
versions of \repair for \wsja, in bits per character.}{Compression
ratios of the four versions of \repair (\wsja)}{tab:prepair-bpc}

The best compression achieved is by word-based \repair,
when no word transformations are applied.  Compression levels
degrade as case folding and stemming are included.  Note that 
the rules for stemming assume that the word tokens have been
case folded.  Hence, the stemming modifiers occupy 0.380 bpc or
0.388 bpc, depending on whether or not the word tokens have been
case folded.
When both transformations are used, a compression effectiveness
of 2.343 bpc is attained.  Even if the best selection of compression
mechanisms is chosen for the modifiers (see the highlighted
numbers in \tabref{tab:prepair-modbpc}), the compression ratios
drops by 0.203 bpc to 2.140 bpc.  So, despite the fact that
word-based \repair and punctuation-aligned \repair produce
a smaller phrase hierarchy, character-based \repair and
word-aligned \repair achieve better compression effectiveness.

One reason for the poorer compression levels of word-based and
punctuation-aligned \repair
is that separating the message into several independent streams
disturbs the contexts in the message.  One possible direction
for future work is to encode the modifier streams more carefully.
Symbols in these streams do not appear
independently of neighbouring symbols, and are even conditioned based on
symbols from another stream.  For example, if the first letter of a word
is in upper case, the word is probably the beginning of a sentence.
One would expect the few words that follow to be in lower case.  Moreover, 
one of the non-word characters that immediately precede it is probably
a sentence ending punctuation mark.  And, since the stems of words
correspond to their parts-of-speech (for example, ``-ing'' is 
associated with verbs), and these categories restrict the order of
words in a sentence, it is expected that some correlation exists 
between stemming modifiers.  This observation means that the
encoding of the symbol in position $k$ of the case folding modifiers
can be conditioned by preceding symbols as well as symbols in the
other modifier streams, or the word sequence.

\tabref{tab:prepair-bpc} also shows that word-aligned \repair 
achieves slightly better compression than character-based 
\repair, even though the former adds more phrases to the 
phrase hierarchy.  This result shows that character-based \repair's 
use of frequency alone as a heuristic does not always choose
the best phrases.  A heuristic which forms complete words before 
combining with complete non-words to the right improves compression
effectiveness over character-based \repair.

The compression effectiveness of the four versions of \repair
are presented in \figref{fig:prepair-bargraph-all} as a graph,
with the compression results for \wsja with \gzip, \bzip, and
\ppmd (see \tabref{tab:tc-bpc} on \pgref{tab:tc-bpc}) indicated
as horizontal lines.  As a further guide, but not shown in the
figure, \citet{im01:dcc} used a word-based block sorting
mechanism and a Huffman coder on \wsja and reported 
compression effectiveness of 1.69 bits per character.
Word-based and punctuation-aligned \repair both performed
case folding and stemming.  All four versions of \repair have
been broken down by stream.  Word-based and punctuation-aligned
\repair both achieve slightly worse compression than \bzip, but better than
\gzip.  However, both of these versions of \repair require more
memory than either \bzip or \gzip.  Even when neither case folding nor
stemming are used with word-based \repair, the compression 
effectiveness of 1.708 is still slightly worse than the 1.656 bpc
offered by \ppmd.

\fig{\includegraphics*{./bargraph-all.eps}}
{Comparing the compression ratios of the four variants
of \repair for \wsja, broken down by stream.  Also shown are
compression results from \chapref{chap:tc} using
\gzip, \bzip, and \ppmd.}{Compression ratios for the four
variants of \repair (\wsja)}{fig:prepair-bargraph-all}

Other than compression effectiveness, the second factor to 
consider when evaluating compression systems is
execution time.  All times reported below
have been averaged over 3 trials.  \tabref{tab:prepair-chartime}
displays the encoding and decoding times of character-based
\repair (from \tabref{tab:repair-time} on \pgref{tab:repair-time})
and word-aligned \repair.  The rules embedded within
word-aligned \repair reduces the number of active phrases 
and active pairs that character-based \repair keeps.
As a result, word-aligned \repair creates a longer
reduced sequence, but encodes the message in less time.  However,
no difference exists for the other program execution times shown
in \tabref{tab:prepair-chartime}.

\tab{lcc}
{ 
  & Character-based & Word-aligned \\
  & \repair         & \repair \\
}
{
Encoding time    & & \\
{\D}{\D}\repair  & 88.5  & 61.2 \\
{\D}{\D}\shuff   & \D1.6 & \D1.6 \\
{\D}{\D}Total    & 90.1  & 62.8 \\
Decoding time    & & \\
{\D}{\D}\shuff   & \D2.5 & \D2.5 \\
{\D}{\D}\despair & \D0.4 & \D0.4 \\
{\D}{\D}Total    & \D2.9 & \D2.9 \\
}{Compression times for character-based \repair and word-aligned \repair
for \wsja.  Times have been averaged over three trials and are given in
CPU seconds.}{Compression times for character-based and 
word-aligned \repair (\wsja)}{tab:prepair-chartime}

The time required to compress \wsja by word-based \repair,
with case folding and stemming, is
shown in \tabref{tab:prepair-wordtime}.  If both case folding
and stemming are not used, the parsing performed by \prepair
reduces to 7.8 seconds.  Since the only difference between
word-based \repair and punctuation-aligned \repair is the 
way phrases are selected, the only times affected are those
related to \repair.  Punctuation-aligned \repair takes
16.3 and 1.1 seconds to encode with \repair and \shuff and
0.9 and 0.3 seconds to decode with \despair and \shuff.
Similar to word-aligned \repair, the augmented rules to 
punctuation-aligned reduces the amount of time to encode
the word sequence.

\tab{llcc}{
\multicolumn{1}{c}{\multirow{2}*{Stream}} & \multicolumn{1}{c}{Compression} & Encoding & Decoding \\
                      & \multicolumn{1}{c}{system}      & time     & time     \\}
{
\prepair               &                   & 14.2   & \D2.8  \\
word lexicon & front coding and \bzip      &  \D0.1   & \D0.1    \\
word sequence          &                   & &       \\
                       & \repair or \despair & 19.6   & \D0.9  \\
                       & \shuff            & \D1.1  & \D0.2  \\
case folding modifiers & \shuff            & \D0.3  & \D0.1  \\
stemming modifiers     & \shuff            & \D0.3  & \D0.1  \\
non-word lexicon & front coding and \bzip  & \D0.1  & \D0.1 \\ %%  Actually, both are < 0.1
non-word modifiers     & \shuff            & \D0.3  & \D0.1  \\
\hline
Total                  &                   & 36.0   & \D4.4  \\
}{Compression and decompression times for word-based
\repair in CPU seconds.  Case folding and stemming have been performed
and front coding followed by \bzip is used for encoding
the lexicons.  Times have been averaged over three trials.}
{Compression and decompression times for \prepair (\wsja)}
{tab:prepair-wordtime}

Finally, \figref{fig:prepair-wsjphrases} shows the difference in
phrases between word-aligned \repair and word-based \repair by
expanding the first 20 symbols of \wsja into primitives.  As a 
reference point, the same section of \wsja was presented in
\figref{fig:repair-wsjhead} on \pgref{fig:repair-wsjhead}.
Punctuation-aligned \repair is not shown because, for this section
of text, only a small difference with word-based \repair exists.
An extensive comparison of phrases between the four versions of 
\repair was discussed earlier with {\texttt {bible.txt}} 
(\figref{fig:prepair-biblephrases}).  \figref{fig:prepair-wsjphrases}
presents the phrases for \wsja for completeness, since most of this
thesis uses \wsja, or files similar to \wsja, as test data.

\fig{
\begin{tabular}{l}
\greyboxtt{{\tl}DOC{\tg}{\tb}{\tl}DOCNO{\tg}{\tvs}}
\greyboxtt{WSJ8703}
\greyboxtt{24}
\greyboxtt{-}
\greyboxtt{0001{\tvs}{\tl}/DOCNO{\tg}{\tb}{\tl}HL{\tg}{\tvs}} \\
\greyboxtt{John{\tvs}Blair{\tvs}} 
\greyboxtt{Is{\tvs}Near{\tvs}}
\greyboxtt{Accord{\tb}To{\tvs}Sell{\tvs}}
\greyboxtt{Unit,{\tvs}} \\
\greyboxtt{Sources{\tvs}Say{\tl}/HL{\tg}{\tb}} 
\greyboxtt{{\tl}DD{\tg}{\tvs}} 
\greyboxtt{03/}
\greyboxtt{24}
\greyboxtt{/87{\tl}/DD{\tg}}
\greyboxtt{{\tb}} \\
\greyboxtt{{\tl}SO{\tg}{\tvs}WALL{\tvs}STREET{\tvs}JOURNAL{\tvs}(J){\tl}/SO{\tg}{\tb}{\tl}IN{\tg}{\tvs}} 
\greyboxtt{REL{\tb}} \\
\greyboxtt{TENDER{\tvs}OFFERS,{\tvs}MERGERS,{\tvs}ACQUISITIONS{\tvs}(TNM){\tb}}{\crarrow} \\
\greyboxtt{MARKETING,{\tvs}ADVERTISING{\tvs}(MKT){\tb}} \\
\greyboxtt{TELECOMMUNICATIONS,{\tvs}BROADCASTING,{\tvs}TELEPHONE,{\tvs}}{\crarrow} \\
\greyboxtt{TELEGRAPH{\tvs}(TEL){\tvs}{\tl}/IN{\tg}{\tb}{\tl}DATELINE{\tg}{\tvs}NEW{\tvs}YORK{\tvs}}{\crarrow} \\
\greyboxtt{{\tl}/DATELINE{\tg}{\tb}{\tl}TEXT{\tg}{\tb}{\tvs}{\tvs}{\tvs}} 
\greyboxtt{John{\tvs}Blair{\tvs}\&amp;{\tvs}Co.{\tvs}}
 \\ [0.5ex]
\multicolumn{1}{c}{(a) Word-aligned \repair} \\ [1.0ex]
\greyboxtt{{\tl}doc{\tg} {\tl}docno{\tg} } 
\greyboxtt{wsj870324 } 
\greyboxtt{0001 {\tl}/docno{\tg} {\tl}hl{\tg} } 
\greyboxtt{john blair } \\
\greyboxtt{is near } 
\greyboxtt{accord to sell } 
\greyboxtt{unit sourc say } 
\greyboxtt{{\tl}/hl{\tg} {\tl}dd{\tg} } 
\greyboxtt{03/24/87 } \\
\greyboxtt{{\tl}/dd{\tg} {\tl}so{\tg} wall street journal j {\tl}/so{\tg} {\tl}in{\tg} }
\greyboxtt{rel } \\
\greyboxtt{tender offer merger acquisit tnm market advertis mkt } \\
\greyboxtt{telecommunicatio ns broadcast telephon telegraph tel}{\crarrow}\\ 
\greyboxtt{{\tl}/in{\tg} {\tl}dateline{\tg} new york {\tl}/dateline{\tg} {\tl}text{\tg} } \\
\greyboxtt{john blair amp co } 
\greyboxtt{is close } 
\greyboxtt{to an agreement } 
\greyboxtt{to sell its } \\
\greyboxtt{tv station } 
\greyboxtt{advertis } 
\greyboxtt{represent }
 \\ [0.5ex]
\multicolumn{1}{c}{(b) Word-based \repair with case folding and stemming} \\ [1.0ex]
\end{tabular}
}{Expanding the first 20 symbols of \wsja into primitives,
parsed with word-aligned \repair and word-based \repair.
For the purposes of display, 
{\tb} is used to indicate a linefeed and long phrases 
have been broken at the points indicated by {\crarrow}.}
{Expanding the beginning of the sequence with
word-aligned and word-based \repair (\wsja)}{fig:prepair-wsjphrases}

The same comparisons that were made for 
\figref{fig:prepair-biblephrases} also apply to 
\figref{fig:prepair-wsjphrases}.  However, a couple of 
points of interest exists in 
\figref{fig:prepair-wsjphrases}.  Note that \sgml
tags are treated as word tokens, and all of them have been
case folded.  Also, the word
``telecommunications'' on line 5 of 
\subfigref{fig:prepair-wsjphrases}{b} is too
long and has been broken in two.  However, after breaking
it, no stemming rules could be applied, so this word
appears unchanged.  If no maximum word length was imposed,
the word ``telecommunications'' would have stemmed to
``telecommun''.  

\newsection{Phrases for Browsing}{sec:prepair-summary}

Three additional heuristics for selecting phrases
have been proposed in this chapter which extend the
capabilities of character-based \repair.
Experiments have been performed
which measured the difference in compression
effectiveness achieved and compression times.
However, the initial motivation
for embarking on this investigation has been to create
phrases that are suitable for phrase browsing.

From \figref{fig:prepair-biblephrases} and 
\figref{fig:prepair-wsjphrases}, the most visually
appealing phrases are those of word-based \repair
and punctuation-aligned \repair.  Case folding and
stemming reduce the size of the lexicon and allow
users to browse phrases without considering the
case and stems used in the document.  The separation
of non-words from the message permits browsing without
the clutter caused by them.  While not appearing 
in the phrase hierarchy, non-word tokens which contain
punctuation marks are used by 
punctuation-aligned \repair to guide the choice of 
phrases made.

There are six streams created by the message 
pre-processor, \prepair.  After \repair compresses the
word sequence, a phrase hierarchy and a reduced word
sequence are created.  In \tabref{tab:prepair-streams},
these seven streams (in boldface), their purpose,
and the method employed in this chapter for compressing
them are shown.  Further
details about how these seven streams are used by a phrase
browsing system are given
in \chapref{chap:review} and \chapref{chap:rephine}.  In
\chapref{chap:review}, the four instances of ``\shuff''
are replaced by compression systems that are more suited
to the intended purposes of the reduced word sequence and the 
three modifier streams.  The encoding methods for the lexicons
and the phrase hierarchy remains unchanged for the remainder
of this thesis.

\newlength{\mytablength}
\setlength{\mytablength}{\textwidth}
\addtolength{\mytablength}{-0.1cm}

\tabnoline{@{}l}{}
{
\begin{tabularx}{\mytablength}{|l|X|X|X|}
\hline
Output from  & \mc{1}{\mr{{\textbf {word lexicon}}}} & \mc{2}{\mr{word sequence}} \\
\prepair     &                             & \mc{2}{} \\ \cline{1-4}

\mr{Purpose} & \mc{2}{\mr{phrase browsing}} & \mc{1}{\mr{searching for symbols}}  \\
             & \mc{2}{} & \\ \cline{1-4}
\mr{Model}   &                 & \mc{2}{\mr{\repair}}      \\
             & \mc{1}{front coding}    & \mc{2}{}             \\ \cline{1-1}\cline{3-4}
\multirow{3}*{Coder}   & \mc{1}{and \bzip} & \mc{1}{{\textbf {phrase hierarchy}}} & \mc{1}{{\textbf {reduced word sequence}}} \\
             & & \mc{1}{chiastic slide and} & \mc{1}{\mr{\shuff}} \\ 
             & & \mc{1}{interpolative coding} &        \\ \cline{1-4}
\end{tabularx} \\
\multicolumn{1}{c}{(a) Words} \\ [1.5ex]

\begin{tabularx}{\mytablength}{|l|X|X|X|X|}
\hline
Output from  & \mc{1}{{\textbf {non-word}}} & \mc{1}{{\textbf {non-word}}} & \mc{1}{{\textbf {case folding}}} & \mc{1}{{\textbf {stemming}}} \\
\prepair     & \mc{1}{{\textbf {lexicon}}}  & \mc{1}{{\textbf {modifiers}}} & \mc{1}{{\textbf {modifiers}}} & \mc{1}{{\textbf {modifiers}}}  \\ \cline{1-5}
\mr{Purpose} & \multicolumn{4}{c|}{\mr{skipping to symbols}} \\
             & \multicolumn{4}{c|}{} \\ \cline{1-5}
\mr{Model}   &                 & \mc{1}{\multirow{4}*{\shuff}}& \mc{1}{\multirow{4}*{\shuff}}& \mc{1}{\multirow{4}*{\shuff}}\\
             & \mc{1}{front coding}    &  & & \\ \cline{1-1}
\mr{Coder}   & \mc{1}{and \bzip}       & &  & \\
             &                 &          &          &        \\ \cline{1-5}
\end{tabularx} \\
\multicolumn{1}{c}{(b) Non-words and modifiers} \\ [1.0ex]
}{The seven streams produced by \prepair and \repair.  The purpose
of each stream, and the compression methods employed in this chapter
for these streams are also shown.  Entries marked ``\shuff'' 
are replaced in \chapref{chap:review} with systems better
suited to those specific tasks.}{The seven streams by \prepair and 
\repair}{tab:prepair-streams}

Phrase browsing can be
performed with only the word lexicon and the 
phrase hierarchy to determine the words used in
the message and the relationships between them.
Together, these two represent just 
0.167 bits per character (punctuation-aligned 
\repair, from \tabref{tab:prepair-bpc}), or just
over 2\% of the original document size, yet provide a very
good guide to the context of the document. 
The lexicons are relatively small, and can be compressed
using methods other than front coding with \bzip.

The word lexicon and word phrase hierarchy are used
for browsing.  But when the context in which the phrases
occur are
required, then the compressed word sequence (1.254 bpc) 
can be used.  The 3 streams so far are only an 
approximation of the document;
words in the phrases need to have the case folding and
stemming reversed.  Also, the non-words within 
phrases have to be re-inserted.  If punctuation-aligned
\repair is used for compression, then the original
document can be obtained by decoding all of the streams
and applying a post-processor to combine them.  For
retrieval, though, only a portion of the modifiers 
and the non-word sequence may be required.  If a 
one-to-one correspondence
between a word and its case folding, stemming, 
and non-word modifiers is ensured, then
a phrase can be reproduced during retrieval
by skipping to precise locations in each of these streams.
In this chapter, the Huffman coding used to encode the modifiers
prevent this operation from occurring 
without decoding the entire document.  However,
this issue and the problem of efficiently searching for
phrase contexts in the compressed word sequence are considered
later, in \chapref{chap:review}.  Also, more details about
how a user browses phrases and searches for the contexts
in which the phrases occur is examined in 
\chapref{chap:rephine}.

But first, in the next chapter, a different problem 
is considered.
So far, \repair has been used to compress a \mib{20}
document.  Ideally, larger documents should be compressed
and browsed and some ways of achieving this is discussed
next.

