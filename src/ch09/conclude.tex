\newchapter{Summary}{chap:summary}

This thesis details the
compromises required in order to satisfy the two opposing
tasks of compression and phrase browsing.
While a compression system reduces the size of a document by
removing any redundancies,
a retrieval system generally augments redundant information to the
document in order to facilitate efficient access.  These 
two document management methods usually exist in 
isolation; but this thesis considers the costs and benefits
of combining them.  Throughout, the compromise
between retrieval and compression has translated into a 
balance between disk space and retrieval time.

\chapref{chap:tc} began with an overview of the foundations
of text compression, and continued by listing three
complete compression systems (\gzip, \bzip,
and \ppmd) and two coding mechanisms (arithmetic coding and
Huffman coding) as examples.
In \chapref{chap:repair}, a fourth compression program called 
\repair \citep{lm00:procieee} was introduced.  Character-based
\repair, as it was later called, reduces the length of a document
(or message) by recursively replacing the most frequently 
occurring pair of symbols with a new symbol until no pair of 
adjacent symbols exists twice or more in the message.
\repair produces a phrase hierarchy which is encoded
using chiastic slide and interpolative coding, and a sequence
whose coding is varied.

The \repair system forms the basis of the 
retrieval system in this thesis because of two other properties.  
As a dictionary-based compression mechanism, the decoding 
time is fast, an important concern for interactive retrieval
systems.  Also, the phrase hierarchy is
both compact and rich with information.  After encoding, it 
occupies only 0.223 bpc compared to the original data, 
but contains information about the characters in
the document and the relationships they form to make partial
or complete words and phrases.  A retrieval system based on
browsing these symbols has the
advantage of not requiring any of the auxiliary data structures 
to be permanently stored.
\chapref{chap:repair} concluded by noting three drawbacks to 
character-based \repair, which are addressed by the enhancements that
form the main body of this thesis.

\chapref{chap:prepair} added a pre-processing step to \repair
called \prepair.  The \prepair system separates a message
into two alternating streams of word tokens and non-word tokens.
Each of these streams is accompanied by a lexicon.  Word tokens
are also case folded and stemmed in the interest of reducing
the size of the word lexicons, resulting in a total of seven
streams.  
Pairing rules are augmented to \repair to prevent phrases
from straddling punctuation marks.
Combining all of these additions results in a system called
punctuation-aligned \repair.  The phrase hierarchy,
coupled with the word lexicon, now contains the words that 
occur in the document and the relationships they form to make
phrases.

In order to tackle the problem of memory usage,
\chapref{chap:remerge} first described how \citet{lm00:procieee}
segments a large message into blocks.
Then, \remerge was described as a method of combining compressed
blocks in phases.  An extension to \remerge 
which allows text to be appended to a previously compressed
document is also presented.

\chapref{chap:review} looks at alternative encoding mechanisms 
for the reduced word sequence and the three
modifier streams (case folding, stemming, and non-words).
The reduced word sequence is encoded with
a byte-aligned coder called \review, while the modifiers 
are encoded with an indexed variant of \shuff.  The decision
to switch to these methods from \shuff was based on improved
access times rather than compression effectiveness.

Attention shifted from compression to phrase 
browsing in \chapref{chap:rephine}.  The \rephine system 
enables the symbols
that make up the phrase hierarchy to be browsed, and the 
contexts in which they occur to be retrieved.  Symbols
can be extended, or broken into their components.
Once a symbol of interest has been isolated, the reduced word
sequence is searched for the symbol in question,
and the contexts are
displayed one-by-one.  Further refinement through the 
inclusion of case folding, stemming, and intervening non-word
characters is also possible.

\chapref{chap:web} takes a detour from the primary theme of the
thesis by looking at the issue of web page delivery and compression.

\newsection{Putting it all together}{sec:summary-all}

The systems mentioned from \chapref{chap:repair} to
\chapref{chap:rephine} work in concert to provide document storage 
and browsing.  Together, they are nicknamed
\restore, and are summarised in \figref{fig:conclude-restore}.
In the figure, boxes indicate the major components that comprise
\restore, while lines between them indicate the flow of data.
On the right side of the figure, the
matching chapter of each system is shown.

\fig{\includegraphics*{./restore.eps}}
{A sketch of the overall \restore system, with relevant
chapters indicated.}{Sketch of the \restore system}
{fig:conclude-restore}

As the pieces of \restore came together, the loss in
compression effectiveness in order to satisfy browsing has
been noted throughout the thesis.  Overall, the compression 
effectiveness of the \news document has
worsened from 1.674 bpc with \ppmd to 2.523 bpc with 
punctuation-aligned \repair.  Similarly, by
building a retrieval system on top of a compression mechanism,
drawbacks become apparent.  For example, the browser
is unable to accurately judge the usefulness of symbols in 
the phrase hierarchy, and resorts to displaying all of them 
instead.  This is the equilibrium point between compression
and browsing for \restore.  

Opportunities exist for future work that may end
up bringing compression and browsing closer together.  
For example, as
\figref{fig:prepair-bargraph-all} on \pgref{fig:prepair-bargraph-all}
shows, the modifiers add almost 1.0 bpc to the compressed message
when encoded independently with \shuff.  Compression levels would probably
improve if these three streams were compressed together, so that 
the context for any one stream is formed from the two remaining
ones.  While compression levels closer to character-based \repair
might then be attained, the effect on browsing would need to 
be re-assessed.

\secref{sec:rephine-summary} gave few directions for possible
improvements to phrase browsing.  One suggestion is
filtering primitives and phrases from \rephine so that only
useful ones are shown.  Alternatively, a ranking scheme can
be adopted which shows interesting symbols at the top of a
\rephine window.  Another important direction for \restore is
to modify the appending phase of \remerge (\secref{sec:remerge-append}
on \pgref{sec:remerge-append}) to allow novel words to be
handled.  Then, documents larger than \mib{1,000} can be processed.

\newsection{Towards the Future}{sec:summary-future}

Compression and information retrieval are both well-studied
areas in the field of document management.  However, the 
successful coupling of these two areas offers new opportunities
for future work.  Effective storage and improved accessibility 
ensure that collected data can accumulate while still remaining
useful.

But is there room in information retrieval for phrase browsing?
This question can be divided into two parts.  First, is there a need
for phrase browsing?  Also, are there challenges in developing
phrase browsing?  

A closer look at how people approach new books will help in addressing 
the first question.  Indexes and 
table of contents help people expedite their search for information
between book covers.  However,
the words and phrases that lie on these pages also allow the 
reader to browse the book, essentially
for free.  No extra amount of work needs to be expended 
by the author or the publisher to facilitate browsing.  
If electronic documents are extensions of books, then it is 
expected that the
tools already available for the latter will be useful for the former.

So, the problem for phrase browsing is to adapt concepts
from printed matter to electronic media.  Solutions employed
by index-based systems cannot be replicated for phrase
browsing.  While metrics such as precision and recall
have been established for querying, the effectiveness of phrase browsing
is more difficult to measure.  Precision and recall 
indicate how well a system has performed and how it
compares to other systems.
While phrase browsing can be assessed through 
user studies, precise work in this area is notoriously difficult.
Also, the techniques and data structures
for index-based systems differ from what is needed
in a phrase browsing system.

In an ideal situation, 
an information retrieval system should provide a selection
of tools which can appeal to a variety of users.  Index-based
querying and phrase browsing are two possible tools, and just as
this thesis has looked at combining compression with phrase browsing,
another area of work may lie in combining multiple information
retrieval techniques into a single, succinct system.

Not only are tools merging to form more complex systems, but 
the format of data is changing as well.
While this thesis has presented techniques for browsing 
compressed document collections in \sgml markup, the documents
have been treated like flat text.  As more documents are published
electronically with \sgml or \xml markup, compressing and
browsing them using the tags is becoming another important 
direction for the future.
The structural information in these document formats might
improve compression levels and allow users to selectively
browse parts of the documents.


