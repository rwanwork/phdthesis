\newchapter{Coding Considerations}{chap:review}

\chapref{chap:tc} explained how compression systems 
can be thought of as three separate components:
modelling, probability estimation, and coding
(see \figref{fig:tc-modelcoder}, 
\pgref{fig:tc-modelcoder}).  In the chapter that followed,
the \repair algorithm was demonstrated as a method of
reducing the length of a message through the recursive replacement of 
character digrams.  Later, pre-processing and post-processing 
stages were introduced which, when working with \repair, 
provided a combination of tools for parsing a document into words, 
reducing the length of the sequence of word tokens, and finally merging
blocks so that larger messages can be handled.  The compression
aspects from \chapref{chap:repair} to \chapref{chap:remerge}
have placed emphasis on the modelling and probability estimation 
components of a compression system.

A relatively small amount of attention was given
to the third component:  coding.  The
goal of this chapter is to rectify that omission by 
considering several methods of coding, and concluding with a
selection of coders which follows the theme of this thesis.  At the end
of this chapter, a coding scheme which provides a suitable 
balance between compression and browsing for punctuation-aligned
\repair is described.

The chapter is structured as follows.  
\secref{sec:review-pastchaps} summarises the thesis so far with
respect to coding.  \secref{sec:review-huffblocks} and 
\ref{sec:review-review} propose two
coding mechanisms.  The first builds on the Huffman coder
that has already been used throughout this thesis, while the 
second relaxes the 
requirements of entropy coding altogether in order to achieve
more efficient phrase browsing.    
The effectiveness and efficiency of these two
coding mechanisms are described
in \secref{sec:review-expt}.  \secref{sec:review-summary}
concludes this chapter with an overview of the advantages
and disadvantages of the coding methods that have been considered.

\newsection{Coding for Compression and Phrase Browsing}{sec:review-pastchaps}

The coding techniques employed in previous chapters
have been biased towards compression in that they favour compression
effectiveness over other more pragmatic factors. This chapter introduces
alternatives which attempt to compromise between 
compression and the special needs of phrase browsing.  Before examining these
alternatives, a brief review of the 
previous chapters with respect to coding is appropriate.

The coding component of a compression system encodes the output
of the probability estimator into a bit stream for transmission.  
Improved compression effectiveness
is achieved when the number of bits required is reduced.
The least number of bits possible for losslessly encoding
the message is defined by the self-information (or self-entropy)
of the message, as
covered in \chapref{chap:tc}.  An approach which
attempts to achieve a compression ratio close to the 
self-information is an entropy coder.

While there are several factors to consider when deciding on
a coder for a compression system, the use of an entropy coder
implies that some importance has been given to compression 
effectiveness.  The need to permit phrase browsing in the
compressed message shifts the importance to decoding time 
and searchability. In any interactive system, a timely response
to the user is essential.  Sacrificing a controlled amount of compression 
effectiveness might be justified, if a shorter response time can 
be assured.

In the case of punctuation-aligned (or word-based) \repair,
seven streams are produced by \prepair, \repair, and \remerge.
These streams serve different purposes, but they can
be divided into two broad categories based on their coding 
requirements for phrase browsing.  
There are the data
streams that need to be decoded completely in order to be
useful, and there are those that need to be only partially
decoded if only a fraction of the original message is to be presented.

\newsubsection{The Phrase Hierarchy and Lexicons}{subsec:review-phlex}

The first category of streams includes the word lexicon, the
phrase hierarchy, and the non-word lexicon.  The two 
most important streams for phrase browsing are the word 
lexicon and the phrase hierarchy.  The word lexicon lists 
every unique word token found in the document, while the 
phrase hierarchy contains information about the relationships 
formed between the word tokens as they form longer phrases.  Separate from
these two streams is the non-word lexicon, which is not
required for phrase browsing.  However, the commonality that 
exists between all three of these streams is that they are encoded
so that compression effectiveness is favoured.  
As a result, the compression mechanisms applied
require that they must be completely decoded in order
to be useful.  

The two lexicons are encoded with front-coding, and can then be processed by
a standard compression tool such as \bzip.  The two-dimensional phrase 
hierarchy is mapped to a single dimension via the chiastic
slide before being encoded with interpolative coding.
These two methods of encoding are biased towards compression
effectiveness. But the strings involved have to be fully decoded to support
phrase browsing, so this approach is acceptable. 
When these compression methods are applied, these 
three streams were small in comparison to the remaining four.  
For example, the \mib{1,000} of \news was processed with punctuation-aligned
\repair and method E of \remerge in \secref{sec:remerge-expt}, 
these three streams represent 6.4\% of the total
compressed message, or 0.144 bits per character
with respect to the original message 
(see \tabref{tab:remerge-pa-otherstats}
on \pgref{tab:remerge-pa-otherstats} and 
\tabref{tab:remerge-pa-bpc}
on \pgref{tab:remerge-pa-bpc}).

\newsubsection{The Modifiers and Reduced Sequence}{subsec:review-modseq}

The other four streams possess a much simpler structure than
the phrase hierarchy or the lexicons.  Each stream is composed 
of symbols which can be arbitrarily decoded without examining
neighbouring symbols.  A symbol is represented as a number,
which refers
to an entry in the non-word lexicon, a set of instructions 
to reverse stemming or case folding, or a symbol in the phrase 
hierarchy.

Strictly speaking, in each stream, a number (or reference) 
does not
appear independently of its neighbours.  References in the
reduced word sequence do not appear at random with equal probability
because after \repair (and \remerge), no pair of adjacent symbols 
appear twice or more in the entire sequence.  
One could also argue that the appearance
of certain combinations of case folding modifiers precludes the same or
some other type
immediately after. For example, if the first letter of a word token is in 
upper case, it is probably signalling the beginning of a new
sentence, and one would expect the previous non-word to include a period, 
and the next few words to be entirely in lower
case.  When the modifier streams were first discussed
in \chapref{chap:prepair}, results from experiments with 
higher-order models were shown
(\tabref{tab:prepair-modbpc} on \pgref{tab:prepair-modbpc}).  
However, in the interest of obtaining
a modifier from any position in the stream without decoding
everything before it, a zero-order model was
advocated in \chapref{chap:prepair}.  Because of this, coding was handled
by the Huffman coder, \shuff.

Two entropy coding schemes for a sequence of references were
discussed in \chapref{chap:tc}.  They were
arithmetic coding and Huffman coding, with available 
implementations being \uint and \shuff, respectively.
\chapref{chap:tc} and \chapref{chap:repair} demonstrated 
that \uint achieved better overall compression 
effectiveness, but required more time for both encoding
and decoding.  \citet{mz92:dcc} argued that arithmetic coding is 
unsuitable for an information retrieval system due to the 
decoding time involved, which translates to waiting time for
the user.  Instead, Huffman coding was employed and was shown
to yield a small loss in compression effectiveness, while 
providing an estimated improvement of 40 times over an
implementation of arithmetic coding by \citet{wbn91:dcc}.
Following the work by \citeauthor{mz92:dcc}, 
\uint was abandoned as a potential coding candidate in a 
phrase browsing system with \repair.  The systems detailed
since
\chapref{chap:repair} have only considered \shuff as the
coder for \repair's modifier streams and reduced 
sequence.

Unfortunately, applying the Huffman coder
\shuff to these four streams does not meet
the needs of phrase browsing.  To illustrate this, a 
description of how the streams need to be accessed by a phrase
browser is necessary.  The phrase hierarchy and the word
lexicon allow the user to manoeuvre around the primitives and phrases,
with the exact steps performed being the 
topic of \chapref{chap:rephine}.
Phrase browsing ceases when the user has decided on a symbol
of interest, $\alpha$.  The user will then examine the
contexts in which $\alpha$ occurs.  Within the reduced
word sequence, $\alpha$ may appear directly or indirectly,
as part of a symbol in a higher generation.  That is, the browsing 
system must translate the symbol $\alpha$ into a set of 
symbols $\Criteria = (\alpha_1, \alpha_2, ..., \alpha_k)$
where $\alpha = \alpha_1$, but $\alpha_2$ to $\alpha_k$
are higher-generation symbols which contain $\alpha$.
Precise details about the search through the phrase hierarchy
is provided in the next chapter, but can be
summarised as a recursive search for all symbols which
contain $\alpha$.  The important point for this chapter
is that a search for every symbol in $\Criteria$ is
done next on the reduced word sequence.

The locations of the symbols of $\Criteria$ are 
determined and recorded in the set $\Result$, along with
some suitable context for each.  One plausible context may be the
10 symbols before and after each location.  
Results in $\Result$ are interactively shown to the user by 
using the phrase hierarchy and the word lexicon to translate
the reduced word sequence references to words.

If the displayed result does not satisfy the user's information need,
further clarification may be required. The initial crude result displays
stemmed, case folded word tokens with non-word tokens removed.
To improve visual appeal, the initial display might
add a space character between each word token.
Ultimately, further refinement of the displayed result 
through the decoding of the three modifier streams is 
required, and if the $i$th symbol in $\Result$ ($\Result_i$) 
is being displayed to the user,
the corresponding modifiers for the symbol and
its context need to be located and applied.

There is an important distinction between the purposes of
the reduced word sequence and the three modifier streams.
A set of symbols are {\emph {searched}} for in the reduced 
word sequence, based on specified symbol values.
The modifier streams are then synchronised to
the same locations.  The symbols in the modifier streams are
located by their positions, and not their values.
Also, locating modifiers for 
$|\Result|$ contexts is unnecessary, since not
every result in $\Result$ may require clarification.  So,
one approach to the problem of modifier look-up is to obtain one 
position and its context at a time, as required.

With these points in mind, the next two sections look
at ways of coding the four streams to enable the necessary
searching and synchronisation operations.

\newsection{Huffman Coding in Blocks}{sec:review-huffblocks}

Motivated by the discussion of compression effectiveness in
\chapref{chap:prepair}, the application of several compression
systems on
the three modifier streams was considered and the costs in
time and benefits in space were assessed.  
While compression
effectiveness was good, there are two drawbacks to the
approaches with respect to retrieval.  First, because
the user is waiting for the results, decoding should be
fast.  While previous results with dictionary-based systems
such as \gzip and \repair have shown that decoding can be
done efficiently, the time required cannot be ignored.  
Second, each time the modifier stream is accessed,
only one position is sought.
Once that symbol has been found, only it and its 
neighbouring symbols are required.  Decoding the entire
stream from the beginning until the symbol of interest 
seems wasteful.

If time is crucial, then a flat binary code can be employed,
as shown in \tabref{tab:tc-staticcodes} on 
\pgref{tab:tc-staticcodes}.  That is,
each symbol in each of the modifier streams is encoded
as an $L$-bit integer.  The parameter $L$ is chosen so
that no symbol in the stream is larger than $2^L$.  Two
passes over the stream are required for encoding, with
the first pass used to determine $L$, and the second pass
for encoding each symbol.  Since every symbol is fixed at
$L$ bits in width, any symbol can be retrieved provided $L$
is specified to the decoder.
Retrieval is fast and is only limited by the
physical characteristics of the storage medium.

In this section, a third alternative is presented which
compromises between these extremes.
Compression is achieved by encoding each
stream directly with a zero-order Huffman coder (\shuff).  
Unlike in earlier chapters,
each stream is partitioned into blocks 
before being processed.  In \chapref{chap:remerge}, 
a message was divided into blocks by \repair because of
limitations in memory.  Now, separating the streams of 
phrase numbers into blocks of $B$ symbols 
serves a second purpose -- to allow 
any arbitrary
block to be decoded by creating a partial index $I$, as shown in
\figref{fig:review-skipblock}.  Each entry in the index
points to the beginning of the corresponding 
block of Huffman coded symbol numbers.  

\fig{
\input{./skipblock-block.pstex_t}}
{The Huffman coding approach with an index which leads
to the beginning of each block. Each block is self-contained, and
includes the same number of coded symbols.}{Indexed Huffman coding}
{fig:review-skipblock}

The algorithm for this approach, called {\emph {indexed \textsc{shuff}}},
is shown in \algref{alg:review-huffblocks}.  Each block of $B$
symbols is encoded with \shuff in isolation and appended to the
output stream, $\Seq'$.  The size of each compressed block is 
appended to the index $I$.  

\algo{\begin{minipage}{\textwidth}
\dontprintsemicolon

\SetFuncSty{textnormal}
\SetArgSty{textnormal}
\SetDataSty{textnormal}

\SetKwInOut{Input}{input}
\SetKwInOut{Output}{output}

\SetKwFunction{encodeShuff}{encodeShuff}

\SetKwData{block}{\var{current\_block}}

\SetKwData{false}{{\textsc {FALSE}}}
\SetKwData{true}{{\textsc {TRUE}}}

\Input{$\Seq$, block size $B$}
\Output{$\Seq'$, index $I$}

\BlankLine

\lnl{algline:review:huffblocks:1}
$i \leftarrow 0$; $\Seq' \leftarrow \{\}$; $I \leftarrow \{\}$\;
\lnl{algline:review:huffblocks:2}
\For{$i < |\Seq|$}{
\lnl{algline:review:huffblocks:3}
  $\block \leftarrow$ \encodeShuff{$\Seq[i...i+B-1]$} \;
\lnl{algline:review:huffblocks:4}
  append $|\block|$ to $I$ \;
\lnl{algline:review:huffblocks:5}
  append $\block$ to $\Seq'$ \;
\lnl{algline:review:huffblocks:6}
  $i \leftarrow i + B$ \;
}

\lnl{algline:review:huffblocks:7}
\Return ($\Seq'$, $I$) \;
\end{minipage}}{Algorithm for Indexed \shuff.}
{Indexed \shuff}{alg:review-huffblocks}

The number of
blocks created for a stream of $n$ symbols is $\lceil n/B \rceil$.
When a symbol is being sought, the fixed block size
ensures that a position $p$ in the stream must be in 
block $\lceil p/B \rceil$.  If line 
\ref{algline:review:huffblocks:4} of the algorithm 
is changed so that $I$ contains
cumulative compressed block sizes, then only two index
entries need to be obtained in order to find the beginning and ending
of the corresponding block.  Otherwise, every index entry up
until the two being sought need to be decoded.
Within the compressed stream
($\Seq'$), position $p$ plus some context to the left and right
is retrieved.  If the size of the window of symbols
is smaller than $B$, at most two adjacent blocks
need to be accessed from disk, in addition to the entries
in the index.

It is important to note that the \shuff coder has the ability to
divide an input stream into
blocks, according to options provided by the user.  However,
the starting locations of each block are invisible to 
the decoding program, preventing any block from being
processed without decoding all previous blocks.

Since each block of $B$ symbols are compressed individually,
indexed \shuff allows the input stream to be read in blocks
of $B$ symbols at a time.  Indexed \shuff is expected to take
longer than compressing the entire stream with \shuff as a single
block because of
the number of disk accesses necessary.
Experiments later in this chapter survey the effect various
choices for $B$ have on time and space.

\newsection{Byte-Aligned Coding in Blocks}{sec:review-review}

The reduced word sequence needs to support a search operation and not
just the seek mechanisms described in the previous section. 
For efficiency reasons, it is preferred that the sequence be searched
while in compressed form.  Decoding the sequence before
searching requires extra time.  Moreover, it may be more
efficient to search through a compressed message because it
is smaller than its corresponding uncompressed form.

Motivated by these observations, 
\citet{ab92:dcc} described the problem of searching a message
in compressed form and labelled it as the {\emph {compressed 
matching problem}}.  Since then, 
searching in a compressed message has received a significant
amount of attention, with the searching mechanism generally being
directly linked to the compression algorithm employed.  Some recent
work in this area include searching in
data encoded with Huffman 
coding \citep{ks01:dcc, tmksfsa02:spire} and searching
in messages which have been compressed with \lzb 
\citep{nktsa01:dcc} and \sequitur \citep{mhmsta01:dcc}.

The problem considered by \citeauthor{ks01:dcc} is that of 
searching
in a Huffman coded text.  As Huffman codewords are variable length,
a match may be a false one if it is not aligned on a 
codeword boundary.  The solution proposed by \citeauthor{ks01:dcc}
extends earlier work which showed how Huffman
codes tend to resynchronise after errors \citep{kw00:dcc}.  If a
possible match is found at position $i$, then their algorithm
jumps back a constant number of bits and decodes all of 
the bits up until position $i - 1$ with a Huffman tree.  
Probability analysis combined with experiments demonstrate
that the quality of the search results improves as the
size of the section of bits decoded increase.  However, their
method is unsuitable for the reduced sequence 
because several patterns may need to be searched 
simultaneously.  The number of symbols in $\Criteria$ can be
large, depending on the symbol selected by the user, and
the number of other symbols that rely on it.  Ideally,
a search method should make a single pass through the reduced
sequence, regardless of the size of $\Criteria$.

\citet{tmksfsa02:spire} described a method of searching 
compressed documents without prior modification.  They also 
demonstrated their technique on multi-byte character texts 
as well as semi-structured texts, and showed that any 
prefix-free code, including Huffman codes, could be searched.
The basis of their idea was to combine synchronisation with
string searching by constructing a pattern matching machine,
which is then run on the entire file.

\citet{manber97:tis}
and \citet{mnzb00:tis} looked at byte-aligned codewords to
improve efficiency and to avoid the problem of aligning with
codeword boundaries.
Decoding byte-aligned codewords is fast, since
repeated bit operations like bit shifting and bit masking are
eliminated.  \citeauthor{manber97:tis} devised a mechanism for 
compressing and searching \ascii text which resembles the
pairing of \repair.  Since \ascii is limited to only 128 symbols, frequent
pairs of symbols were replaced non-recursively until the 
compressed message had 256
unique symbols.  By converting the search pattern in a similar
fashion, the pattern is searched for directly in the compressed
message using a string-matching technique such as \citet{bm77:cacm}.
Alternatively, \citeauthor{mnzb00:tis}\ implemented Huffman coding
with a radix-256 code.  That is, the
nodes in the generated Huffman code tree had an out-degree of
256, instead of 2 for a binary code.  Building upon this idea,
a Tagged Huffman code was used for compression and searching.  
Tagged Huffman coding represented words as groups of 7 bit 
codewords (radix-128), with an eighth bit reserved for
indicating the first byte of a codeword, to ensure alignment 
while searching.  In their experiments,
Tagged Huffman coding offered compression effectiveness which was
around 5 bits per symbol worse than binary Huffman coding.  
However, overall search times required about half of the 
time of a direct search on the uncompressed message.  More 
recently, \citet{bfge03:spire} extended this work by generalising
Tagged Huffman codes to $({S},{C})$-Dense Codes.  $({S},{C})$-Dense 
Codes consist of continuers and stoppers.
Stoppers are bytes which indicate the last byte of a codeword.
Other bytes are continuers.  Tagged Huffman coding is a special
case when there are 128 stoppers and 128 continuers.  
\citeauthor{bfge03:spire}\ varied the size of these two sets of
codes to determine the combination which yielded the best 
compression ratio for various document collections.

The entropy coding requirements for the reduced word sequence
are further eased compared to this previous work, in favour
of retrieval.  A coder called \review has
been developed which operates with a corresponding codeword length
limit $L$.  Blocks are created so that each block has at most
$2^L$ distinct values.  Byte-alignment is achieved by restricting
$L$ to be either 8 or 16.  These two variants are denoted as
\reviewa and \reviewb, respectively.

A sketch of a \review block and part of the
following block is shown in \figref{fig:review-review}.  Each
block consists of a prelude and codewords, similar to a \shuff
block.  The prelude specifies the $2^L$ different symbol 
values that exist in the block.  At the beginning of each
block, four values are also required, shown shaded in the 
figure.

\fig{
\input{./review-block.pstex_t}
}
{One \review block and part of the next one.  A \review block
is composed of four integers in the header (shown shaded), a 
prelude, and a section of codewords.  The arrow at the top of 
the figure indicates the potential to skip the codewords
of a block if no match is known to
exist in that block.}{Sketch of \review blocks}
{fig:review-review}

The four values in the header provide information about the
entire block to improve the search time of \review.  In
addition to the byte-alignment of codewords, the informational
headers allow a block to be bypassed when none of the symbols
being sought ($\Criteria$) are known to exist inside.  After
phrase browsing, a user may have honed in on a phrase which
is highly descriptive, and exists in only certain parts of the
document collection.  Areas in the collection which do not
contain the phrase should be skipped.  This function is provided
by the last three of the shaded values of \figref{fig:review-review}.
In order to determine whether or not a symbol in $\Criteria$
exists in a block, the prelude must be decoded.  If no symbols
in $\Criteria$ are found, then the block of codewords may be
skipped, as shown by the solid, curved arrow in the figure.  There are
$B_k$ symbols in block $k$, which translates to $B_k(L/8)$ bytes.
The number of distinct values in the block ($D_k$) is required
to aid in decoding the prelude, and is $2^L$ for every block 
except for the last one.  The number of original word tokens covered by
a block is specified in the header as $W_k$, and is necessary for
synchronisation with the modifier streams.  
After \repair, the number of word tokens represented by a block 
is equal to or greater than the number of symbols it contains.
When a block is skipped, not only does the number of phrase symbols 
that have been passed need to be known ($B_k$), but the number 
of original word tokens as well. Then, within a block, as 
symbols are read, a count of the number
of word tokens skipped and read since the beginning of the stream
is maintained, so that each word token's corresponding modifier
can be determined accurately.  In order to support this requirement,
both encoding and decoding of 
a reduced word sequence requires the lengths (in number of word tokens) 
of each symbol in the phrase hierarchy.

In later experiments, \review is applied to the modifier streams
as well, to assist in its understanding.  As with indexed \shuff,
every block, except for the one which contains the position being
sought, needs to be skipped.  The first value in the header
indicates the size of the prelude in bytes to allow both 
the prelude and the codewords
of a block to be skipped, as shown by the dashed line of
\figref{fig:review-review}.  The block of interest can be located
by summing $B_k$ for every block.  In order to bypass
the prelude and codewords of the current block $k$, the number of 
bytes is calculated as $P_k + B_k(L/8)$.

The algorithm for \review is presented in \algref{alg:review-review}.
\review processes a stream ($\Seq$) by scanning it sequentially
while keeping track of the number of distinct symbol values 
seen so far ($D_k$).  The distinct symbol values are maintained
in an array \var{symbol\_map}, whose size is proportional to
the number of distinct symbols in the entire sequence (the
size of the alphabet, $|\Sigma|$).
Once $2^L$ different values have been 
seen (line \ref{algline:review:review:9}), a block is created.  
The distinct symbols that appear in the block
are mapped to $L$-bit integers on
line \ref{algline:review:review:14}, and recorded in the
prelude.  The prelude is buffered so that $P_k$
can be placed first within a block.  The codewords
are produced by mapping the symbol values to
$L$-bit integers, by consulting the \var{symbol\_map} array
on line \ref{algline:review:review:22}.
The symbols in the block are represented as fixed $L$-bit
codewords, and retain $L$-bit alignment throughout the block.

\algo{\begin{minipage}{\textwidth}
\DontPrintSemicolon

\SetFuncSty{textnormal}
\SetArgSty{textnormal}
\SetDataSty{textnormal}

\SetKwInOut{Input}{input}
\SetKwInOut{Output}{output}

\SetKwFunction{writeSymbolToPreludeBuffer}{writeSymbolToPreludeBuffer}
\SetKwFunction{writeHeader}{writeHeader}
\SetKwFunction{writePreludeBuffer}{writePreludeBuffer}

\SetKwData{symmap}{\var{symbol\_map}}
\SetKwData{code}{\var{code}}
\SetKwData{blockstart}{\var{block\_start}}

\SetKwData{false}{{\textsc {FALSE}}}
\SetKwData{true}{{\textsc {TRUE}}}

\Input{$\Seq$, length of symbols in $\Ph$, alignment $L$ in bits}
\Output{$\Seq'$}

\BlankLine

\lnl{algline:review:review:1}
$\Seq' \leftarrow \{\}$;
$P_k \leftarrow 0$;
$B_k \leftarrow 0$;
$D_k \leftarrow 0$;
$W_k \leftarrow 0$;
$\blockstart \leftarrow 0$ \;

\lnl{algline:review:review:2}
initialise \symmap \;
\lnl{algline:review:review:3}
\For{$i < |\Seq|$}{
\lnl{algline:review:review:4}
  $W_k \leftarrow W_k + |\alpha_i|$ \;
\lnl{algline:review:review:5}
  $B_k \leftarrow B_k + 1$ \;
\lnl{algline:review:review:6}
  \If{$\symmap[\alpha_i] = 0$}{
\lnl{algline:review:review:7}
    $\symmap[\alpha_i] \leftarrow 1$ \;
\lnl{algline:review:review:8}
    $D_k \leftarrow D_k + 1$ \;
  }
\lnl{algline:review:review:9}
  \If{$D_k = 2^L$}{
\lnl{algline:review:review:10}
    $\code \leftarrow 0 $; $j \leftarrow 0$ \;
\lnl{algline:review:review:11}
    \While{$\code < D_k$}{
\lnl{algline:review:review:12}
      \If{$\symmap[j] = 1$}{
\lnl{algline:review:review:13}
        write $\alpha_j$ to prelude buffer \;
\lnl{algline:review:review:14}
        $\symmap[j] \leftarrow \code$ \;
\lnl{algline:review:review:15}
        $\code \leftarrow \code + 1$ \;
      }
\lnl{algline:review:review:16}
      $j \leftarrow j + 1$ \;
    }
\lnl{algline:review:review:17}
    $P_k \leftarrow$ size of prelude buffer \;
\lnl{algline:review:review:18}
    write $P_k$, $B_k$, $D_k$, $W_k$ to $\Seq'$ \;
\lnl{algline:review:review:19}
    write prelude buffer to $\Seq'$ \;
\lnl{algline:review:review:20}
    $j \leftarrow \blockstart$ \;
\lnl{algline:review:review:21}
    \For{$j < i$}{
\lnl{algline:review:review:22}
      write $\symmap[\alpha_j]$ to $\Seq'$\;
    }
\lnl{algline:review:review:23}
    $P_k \leftarrow 0$;
    $B_k \leftarrow 0$;
    $D_k \leftarrow 0$;
    $W_k \leftarrow 0$;
    $\blockstart \leftarrow i$ \;
\lnl{algline:review:review:24}
    re-initialise $\symmap$ \;
\lnl{algline:review:review:25}
    $k \leftarrow k + 1$ \;
  }
}

\lnl{algline:review:review:26}
\Return ($\Seq'$) \;
\end{minipage}
}{Algorithm for \review.}
{\review}{alg:review-review}

While \var{symbol\_map} is scanned, each symbol is recorded
in the prelude on line
\ref{algline:review:review:13}.  Symbols are encoding
as differences from the preceding symbol 
using nibble-aligned codes (4-bit codes).  The mapping
of differences to nibble-aligned codes is specified in
\tabref{tab:review-prelude}.  In the second column of
the table, the codes employed are specified using
hexadecimal.

\tab{ll}
{
Difference & Nibble-aligned \\
& \multicolumn{1}{c}{codes} \\
}
{
{\D\D}1 -- 8      & {\D\D\D}0 -- 7 \\
{\D\D}9 -- 135    & {\D\D\D}80 -- FE \\
{\D\D}{$\ge$}136 & {\D\D\D}FF \\
}{Mapping differences to nibble-aligned codes.  The
codes are specified in hexadecimal.}
{Mapping differences to nibble-aligned codes for \review}
{tab:review-prelude}

The main data structures employed by \review are the
sequence and the \var{symbol\_map} array, requiring
a total of $|\Seq| + |\Sigma|$ words of memory.
Three main operations are performed by \review on these
two structures.  When a block is being built, two 
passes are performed over the reduced sequence.  The
first pass inspects the sequence in order to identify
$2^L$ distinct values.  The second pass maps each
symbol to an $L$-bit codeword.  Also, in order to assign
symbols to codewords, a single pass over the \var{symbol\_map}
structure is necessary for each of the $K$ blocks created.  
Since \var{symbol\_map}
has $|\Sigma|$ entries, $K$ linear passes may be costly.
If an index of size $2^L$ is created so that only the symbols
in \var{symbol\_map} need to be looked at, then the linear
search can be eliminated.  For each block, this index needs
to be sorted at a constant cost of $2^L \log_2 2^L$.  
So, the \review algorithm runs in $O(2|\Seq| + K(2^L \log_2 2^L) + K2^L)$.

When a set $\Criteria$ is searched for in the compressed
sequence, the prelude of every block must be decoded.
If at least one symbol in $\Criteria$ exists in the current
block $k$, then the corresponding codewords are also 
decoded.  Otherwise, the set of codewords is skipped
using the value $B_k$ in the header.
As is demonstrated shortly, since the prelude is about 
36.1\% of the compressed file, a scan for a rare phrase
involves processing at least one-third of the sequence.
If the limited amount of decoding involved in processing 
every prelude cannot be accepted, then one
solution is to create a higher-level index, similar to the 
\glimpse system \citep{mw94:usenix}.  The \glimpse retrieval 
system partitions a file system into
blocks.  The index indicates, for every term in the
file system, the blocks in which it occurs.
Once the blocks of interest have been determined, 
each block is sequentially searched.  

Several potential enhancements to \review are possible.  Disk
access can be reduced by physically separating the preludes
and the codewords into different files.  Then, disk access
is concentrated on the prelude file.  Another modification
would build a top-level index
on blocks so that the preludes of blocks which do not contain 
the symbol of interest are avoided entirely.  Such an index
would be kept in memory.  Extending this last idea to
an inverted index on symbols is also possible.
None of these options were explored in the experiments 
reported in the next section since the search times 
obtained were acceptable.

\newsection{Experiments}{sec:review-expt}

To investigate the compression
effectiveness of the modifier streams and the reduced word
sequence, experiments were conducted using Huffman coding,
built on the \shuff
implementation, \review, and an even simpler binary code.
The efficiency for both skipping to a precise location and
searching are also examined.  Test files were
taken from the punctuation-aligned \repair 
experiments with method E of \remerge on the \news test data,
as described in \chapref{chap:remerge}.  The compression results 
from that chapter
were presented in \tabref{tab:remerge-pa-otherstats} 
(\pgref{tab:remerge-pa-otherstats}) for 
the modifiers and \tabref{tab:remerge-pa-bpc} 
(\pgref{tab:remerge-pa-bpc}) for the 
reduced word sequence.  In those tables,
the modifiers and the reduced word
sequence were coded with \shuff as a single block.  
The two tables in \chapref{chap:remerge} also indicate that the combined
compression ratio of the word lexicon, non-word lexicon, and the
phrase hierarchy was 0.143 bits per character.  The remaining 4
streams occupied 2.090 bits per character.

Additional information pertaining to the four streams 
is listed in \tabref{tab:review-fileinfo}.  The
last column in the table lists the self-entropy of each file.
Note that because these values are relative to the streams
themselves and not to the size of \news, they are not
comparable to the compression ratios achieved in 
\tabref{tab:remerge-pa-otherstats} and
\tabref{tab:remerge-pa-bpc}.  

\tab{lc@{}c@{}cccc}
{ & \multicolumn{3}{c}{Symbol range} & Distinct & Number of  & Self- \\
  & & &                              & values   & symbols    & entropy     \\}
{
case folding modifier  & 0 & {\D}to{\D}& \C\D65,536  & \C\D\D\C634 & 168,721,097 & {\D}1.303 \\
stemming modifier      & 0 & {\D}to{\D}& 7,864,320   & \C\D\D\C541 & 168,721,097 & {\D}2.204 \\
non-word modifier      & 0 & {\D}to{\D}& \C\D11,332  & \C\D11,333  & 168,721,097 & {\D}2.054 \\
reduced word sequence  & 0 & {\D}to{\D}& 8,277,359   & 6,232,453   & {\D}58,685,281 & 18.659 \\
}{Additional information about the modifier streams and
the reduced word sequence from method E of \remerge with
punctuation-aligned \repair on \news.  The column headed self-information
is the zero-order self-information for the frequency distribution of
the symbols in that file, measured for each file in bits per 
symbol relative to the size of that file.}{Statistics
about the modifier streams and reduced word sequence 
(\news)}{tab:review-fileinfo}

The other three columns of the table provide the
range of symbols, the number of distinct values, and the total
number of symbols of each 
stream.  Note that the case folding and stemming mechanism 
employed by \prepair permit $2^{16}$ possible case folding 
values and $2^{23}$ possible stemming values.  In practice it appears
that even a large collection only has a small number of distinct
values used.  In contrast, \prepair assigns non-word
modifier values in sequential order, starting from 1, with 0
reserved for the zero-length non-word symbol.  For the non-word
modifier stream, every symbol within the range appears.  
Since there is a one-to-one correspondence between
a word token and its modifiers, the three modifier 
streams have the same number of values in them.

Compared to the other three streams, the reduced word sequence
is shorter, because \repair and \remerge have been applied.
No pair of adjacent symbols occurs twice 
in this sequence, unless
one of the restrictions imposed by punctuation-aligned \repair has
been violated.  As a result, the self-entropy of the stream is
significantly higher than the others.  Accordingly, the number
of distinct values and the symbol range are also high.  The
symbol range is simply the number of symbols in the phrase
hierarchy (see \tabref{tab:remerge-pa-stats} on 
\pgref{tab:remerge-pa-stats}), with a special end
of block marker.  After recursive pairing, some symbols
in the phrase hierarchy do not appear directly in the reduced word
sequence, but instead have been absorbed by higher generation
symbols.  That is why the number of distinct symbols is less than
the number of entries in the phrase hierarchy.

There are two sets of experiments that are performed in this section.
The first set considers the three modifier streams,
while the second set examines the reduced word sequence in
isolation.  In
both scenarios, the compression ratios are reported, as are the
times for compression and full decompression.  The times
for seek or search operations are also reported, whichever is applicable.
When necessary, the compression ratios are reported in bits
per character (bpc) relative to the size of \news, and in bits
per symbol (bps) relative to the number of symbols in the file 
itself.  The number of bits
per character is comparable with all compression ratios reported
in previous chapters, while the number of bits per symbol is
important when examining the column labelled ``self-entropy''
in \tabref{tab:review-fileinfo}.

Program execution times in this chapter are
reported as both CPU time and elapsed time.  CPU time is a measure of
the amount of computation performed by the program.  All
execution times reported in previous chapters have been CPU costs.  
But because this chapter also considers 
the amount of time a user must wait for a response, 
efficiency is also 
reported as elapsed time (also called ``real time'').
Elapsed time includes 
both user time and the time taken by the system for
the program.  Since all programs are
executed with no data in memory beforehand, the time needed
to access the disk is included in the elapsed time.
All times, both CPU and elapsed, are averaged over three trials
unless otherwise indicated.

\tabref{tab:review-modbpc} and \tabref{tab:review-modtime} 
summarise the experiments with the modifiers.  Each of the
three modifier streams was compressed with indexed \shuff
with block sizes increasing by a factor of 8 from 
from \kib{4} up to \kib{16,384}.  While not
the intended purpose of \review, \reviewl was tested with
$L = 8$ and $L = 16$ as a preview for the second set of 
experiments later.  A flat binary coder (\flatbin) 
was also applied so that each symbol was encoded as a fixed-width
binary code whose length was based on the maximum symbol value
in the stream.  The \flatbin system treated each stream as a
single block.

\tab{rccc}{
\multicolumn{1}{c}{Block size} & \multicolumn{2}{c}{Compression ratio} & Number \\ \cline{2-3}
\multicolumn{1}{c}{($\times$ 1024 symbols)} & bpc & bps & of blocks \\}
{
\multicolumn{4}{l}{case folding modifiers} \\
4       & 0.259    & \D1.608    & 41,192\\
32      & 0.243    & \D1.508    & \D5,149\\
256     & 0.240    & \D1.494    & \D\C644\\
2,048   & 0.240    & \greybox{\D1.492}    & \D\C\D81\\
16,384  & 0.240    & \D1.493    & \D\C\D11\\
{\D\D}\flatbin       & 2.736    & 17.000     & \D\D\D\D1 \\
{\D\D}\reviewa          & 1.288    & \D8.004    & \D\C161 \\
{\D\D}\reviewb          & 2.575    & 16.004     & \D\C161 \\
\hline
\multicolumn{4}{l}{stemming modifiers} \\
4       & 0.429    & \D2.667    & 41,192\\
32      & 0.388    & \D2.414    & \D5,149\\
256     & 0.379    & \D2.358    & \D\C644\\
2,048   & 0.378    & \D2.347    & \D\C\D81\\
16,384  & 0.377    & \greybox{\D2.346}    & \D\C\D11 \\
{\D\D}\flatbin       & 3.701    & 23.000     & \D\C\D\D1 \\
{\D\D}\reviewa          & 1.487    & \D9.240    & \D\C479 \\
{\D\D}\reviewb          & 2.642    & 16.417     & \D\C161 \\
\hline
\multicolumn{4}{l}{non-word modifiers} \\
4       & 0.370    & \D2.298    & 41,192\\
32      & 0.346    & \D2.148    & \D5,149\\
256     & 0.342    & \D2.123    & \D\C644\\
2,048   & 0.341    & \greybox{\D2.119}    & \D\C\D81\\
16,384  & 0.342    & \D2.126    & \D\C\D11\\
{\D\D}\flatbin       & 2.253    & 14.000     & \D\C\D\D1 \\
{\D\D}\reviewa          & 1.291    & \D8.020    & \D1,690\\
{\D\D}\reviewb          & 2.575    & 16.004     & \D\C161\\
}{Compression effectiveness of indexed \shuff, \review, and
\flatbin for the modifier streams.
The second column lists the compression ratios of
each method relative to the size of \news in bits per character
relative to the original file.  In the third column, compression
levels in bits per symbol relative to the number of symbols in that 
stream are reported, with the best values for each stream 
highlighted.  The number of blocks
produced are shown in the last column.}{Compression effectiveness
for the modifier streams (\news)}
{tab:review-modbpc}

Generally, as the block size for indexed
\shuff increased, compression effectiveness improved, as shown
in \tabref{tab:review-modbpc}.  However, the
decrease is small, especially between the larger block sizes.  
Also, the difference in compression ratio
between the smallest block size and the largest block
size shown is no more than 0.052 bits per character.  
Similar conclusions can be drawn from the column titled ``bps''
when compared with the self-entropy of each file 
(\tabref{tab:review-fileinfo}).

The implementation of \review first creates an
initial buffer of 1,048,576 symbols, preventing any block from being
larger than the buffer.  
The low variability in case folding modifier values 
means that after \review has seen just over one million 
symbols, less than $2^{8}$ distinct symbols have 
been found.  As the number of distinct values increases in first
the stemming modifiers, and then the non-word modifiers, more blocks
are made by \reviewa.  However, as \reviewb still only creates
161 blocks in both cases, no buffer has $2^{16}$ or
more distinct 
values.  Compression effectiveness of both types of \review are
unacceptable, though quite expected.  \reviewl
can never achieve compression effectiveness that is better than
$L$ bits per symbol.  

The ideal method for encoding the modifier streams depends
on efficiency as well as effectiveness, as shown in 
\tabref{tab:review-modtime}.
The second and third columns of the table list the encoding and decoding 
times of each of the methods, averaged over three trials, and
represented as CPU time. 
In order to demonstrate the random access abilities, 
100 random positions were selected
from the possible 168,721,097.  During the design of the
experiments, it was noted that access times varied
depending on the position in the stream that is being accessed.
However, if the same 100 positions were used by all of the experiments,
then the access times between systems can be compared.  Moreover,
since the access times were short, \tabref{tab:review-modtime}
presents the total time for accessing all 100 positions in the
same order.  These times are reported as elapsed time and CPU time
in the fourth and fifth columns of the table.  All times have 
been averaged over three trials.  In the entries labelled ``---'', less
than 0.1 seconds was noted, a time which cannot be reliably measured.

\tab{rccccc}
{
 & \multicolumn{2}{c}{Compression} & & \multicolumn{2}{c}{Random access} \\ \cline{2-3} \cline{5-6}
\multicolumn{1}{c}{Block size} & Encoding & Decoding                         & & Elapsed & CPU      \\
\multicolumn{1}{c}{($\times$ 1024 symbols)}            & time     & time     & & time    & time     \\
            & (CPU sec)&(CPU sec) & & (s)& (s) \\
}
{
\multicolumn{6}{l}{case folding modifiers} \\
4                     & 270.6    & 444.5    & & \D\D2.3    &  \D\D1.6    \\
32                    & \D47.1     & \D64.8   & & \D\D2.0    &  \D\D1.3    \\
256                   & \D24.4     & \D18.7   & & \D\D5.2    & \D\D3.0     \\
2,048                 & \D20.7     & \D12.7   & & \D26.3   & \D14.7    \\
16,384                & \D19.7     & \D11.2   & & 194.8  & 110.1   \\
{\D\D}\flatbin        & \D15.3    & \D28.9   & & \greybox{\C\D---}      & \greybox{\D\D0.1}     \\
{\D\D}\reviewa        & \greybox{\D\D8.3}      & \greybox{\D\D3.3}    & & \D\D1.5    & \greybox{\D\D0.1}     \\
{\D\D}\reviewb        & \D\D8.7   & \D\D3.5    & & \D\D2.3    & \greybox{\D\D0.1}     \\
\hline
\multicolumn{6}{l}{stemming modifiers} \\
4                    & 916.5    & 445.1  & & \D\D2.1    & \D\D1.6     \\
32                   & 141.1    & \D66.0   & & \D\D2.2    & \D\D1.5     \\
256                  & \D36.8     & \D19.1   & & \D\D5.1    & \D\D3.0     \\
2,048                & \D23.1    & \D12.4   & & \D26.7   & \D15.3    \\
16,384               & \D21.0     & \D11.3   & & 200.2  & 113.6   \\
{\D\D}\flatbin       & \D17.0     & \D37.6   & & \greybox{\C\D---}      & \greybox{\D\D0.1}     \\
{\D\D}\reviewa       & \D\D9.0    & \D\D4.1    & & \D\D2.0    & \D\D0.4     \\
{\D\D}\reviewb       & \greybox{\D\D8.7}  & \greybox{\D\D4.0}    & & \D\D3.1    & \D\D0.3     \\
\hline
\multicolumn{6}{l}{non-word modifiers} \\
4                    & 261.5    & 443.5  & & \D\D2.1    & \D\D1.5     \\
32                   & \D45.1     & \D65.0   & & \D\D2.2    & \D\D1.3     \\
256                  & \D24.8     & \D18.8   & & \D\D5.1    & \D\D2.9     \\
2,048                & \D20.9     & \D13.1   & & \D26.5   & \D15.2    \\
16,384               & \D19.7     & \D11.3   & & 198.3  & 113.2   \\
{\D\D}\flatbin       & \D14.7     & \D25.2   & & \greybox{\C\D---}      & \greybox{\D\D0.1}     \\
{\D\D}\reviewa       & \greybox{\D\D8.4}      & \greybox{\D\D2.7}    & & \D\D1.9    & \D\D0.3     \\
{\D\D}\reviewb       & \D\D8.7      & \D\D3.7    & & \D\D2.6    & \greybox{\D\D0.1}     \\
}{Compression efficiency of indexed \shuff, \review, and \flatbin for 
the modifier streams of \news.  All times are measured in seconds, with the
best values shaded.  Compression
and decompression times are CPU times averaged over three trials.
Results from randomly accessing 100 positions in each stream have
also been repeated three times each.  These times are shown as
elapsed time and user time in the fourth and fifth columns, 
respectively.  Times indicated by ``---'' are less than 0.1 seconds.}
{Compression efficiency for the modifier streams (\news)}
{tab:review-modtime}

As less blocks are created,
compression and decompression efficiency improves.  
\review is faster than both indexed
\shuff and \flatbin.  
The encoding times of \flatbin are equivalent
to the largest block size tested for indexed \shuff, 
while decoding is slightly worse than indexed \shuff with a 
\kib{256} block size.
The time required for random access is fastest for the
\flatbin coder, followed by the two \review systems, and then
indexed \shuff in increasing block size.  

%% vipe:~/vipe-c/expt/6/search > wc -c grep.out
%% 901252607 grep.out
%% vipe:~/vipe-c/expt/6/search > bc
%% bc 1.06
%% 901252607/1048576
%% 859.5014
The second set of experiments looked at encoding the 
reduced word sequence.  Three methods of searching within
the word sequence were compared.  

The first method applied
version 2.4.2 of GNU \grep \citep{fsf-grep:program}
on a simplified version of \news.  The \grep program searches
for a set of one or more strings and returns the lines which
match.  Including \grep in the experiments allows comparisons
between searching for
a string and searching for a reference to a string in the
phrase hierarchy.  The \news document must be simplified
in order to minimise the number of differences between these
two tasks.  The modified \news document contains every 
symbol in the reduced word sequence expanded into word 
tokens.  No modifiers are applied, and
a single space is inserted between each word token.  In order
to allow \grep to return lines which match, linefeeds were
inserted after each \sgml closing tag.  As long as
no line that is searched contains a \sgml closing tag, this
change ensures that no search string is missed because it
straddles multiple lines.  One minor problem with this arrangement
occurs when a search string appears twice or more in the same line.  
In this
case, \grep simply returns one copy of the line, and does not
continue searching the remainder.  Hence, these 
modifications to the \news document are not perfect, but 
do provide a reasonable estimate for direct searching. 
The modified version of \news is \mib{860} in size.

While \grep is used to search for a particular
string, the string may exist in several symbols in the phrase
hierarchy.  The second and third methods are applied to the
reduced word sequence so that a set of symbols can be searched
for simultaneously.  The second method assumes the sequence
is compressed with \flatbin or \shuff (without an index) using
various block sizes, with the word sequence fully decompressed 
for each search.  The third method compresses the
sequence using \reviewb with searching performed using the 
partial-decoder approach described earlier in 
\secref{sec:review-review}.  Note that \reviewa cannot be sensibly 
applied to the word sequence because of the large number of
distinct symbols.

\tabref{tab:review-ws-bpctime} presents compression and 
decompression results with \shuff, \reviewb, and \flatbin.  Larger
block sizes were selected for \shuff than for indexed \shuff
in order to favour compression effectiveness and decompression
time.  As \tabref{tab:review-modtime} showed, the overhead of
processing many small blocks with indexed \shuff had an unfavourable effect
on the decompression time.  Block sizes for \shuff ranged from
1,048,576 symbols per block up to the entire
stream as a single block ($\infty$).

\tab{lccccc}
{
\multirow{2}*{Method} & \multicolumn{2}{c}{Compression ratio} & Number & \multicolumn{2}{c}{Compression time} \\ \cline{2-3} \cline{5-6}
                  & bpc      & bps     &of blocks& Encoding (s) & Decoding (s)\\
}
{
\shuff block size &          &          &        &         &          \\
{\D\D\D}1,048,576   & 1.183    & 21.144   & \D56   & 104.5   & 25.1     \\
{\D\D\D}2,097,152   & 1.166    & 20.827   & \D28   & \D87.3  & 22.9     \\
{\D\D\D}4,194,304   & 1.151    & 20.570   & \D14   & \D75.9  & 21.1     \\
{\D\D\D}8,388,608   & 1.139    & 20.355   & \D\D7  & \D64.9  & 18.7     \\
{\D\D}16,777,216  & 1.132    & 20.219   & \D\D4  & \D56.3  & 17.0     \\
{\D\D}20,971,520  & 1.131    & 20.199   & \D\D3  & \D53.6  & 16.2     \\
{\D\D}33,554,432  & \greybox{1.127}    & \greybox{20.138}   & \D\D2  & \greybox{\D48.3}  & 15.1     \\
{\D\D}$\infty$    & 1.131    & 20.213   & \D\D1  & \D52.6  & 13.9     \\
\hline
\flatbin          & 1.287    & 23.000   & \D\D1  & \D\D5.8 & 13.1 \\
\reviewb          & 1.402    & 25.044   & 672    & \D64.9   & \greybox{\D7.9}      \\
}{Compression statistics for the reduced word sequence of 
\news, using \shuff, \flatbin, and \reviewb.  All times are reported in
CPU seconds and averaged over three trials, with the best values shaded.}
{Compression statistics for 
the reduced word sequence (\news)}{tab:review-ws-bpctime}

The compression ratios are listed in the second and third columns
of the table in bits per character relative to the source text 
and bits per symbol relative to the file of phrase numbers, respectively.
As the \shuff block size increases, compression effectiveness improves.
The exception occurs with
the last block size.  Enlarging the block size to $\infty$ reduces the total
prelude size, but the coded message increases in size.  The
reason for this unexpected result is that the reduced word sequence,
and the \news file from which it was derived, is not
uniform.  Recall that \news is composed of several
years' worth of news articles from two different sources, with the
Wall Street Journal articles at the beginning and the Associated Press
articles at the end.  The change in document style and the change in
article years was the price paid in order to create a sufficiently
large document for experiments in this thesis.  The \flatbin coder 
attains a compression level which is about 3 bits per symbol worse than \shuff.
\reviewb achieves compression which is 5 bits per
symbol worse than \shuff; not a surprising result given that \reviewb is not an
entropy coder and a prelude must be transmitted for each of the
672 blocks.  The size of the encoded reduced word sequence with
\reviewb was \mib{175.2}, of which the values in the
headers ($W_k$, $D_k$, $B_k$, and $P_k$) and the preludes made 
up \mib{63.3}.  The large preludes ensure that \reviewb yields
compression effectiveness which is also worse than \flatbin.

Following the column of block count are the encoding and decoding
times, in seconds of CPU time, averaged over three trials.  Generally, 
encoding time with \shuff decreases as the block size increases,
except for the ``$\infty$'' block size.
Decoding time also decreases for large block 
sizes as fewer preludes need to be decoded.
The \flatbin coder is the fastest system for encoding
and is faster than \shuff for decoding, irrespective of block size.
\reviewb requires as much time to encode as \shuff with 
a block size of 8 million symbols.  But \review has the lowest
decoding time out of all the systems tested.

More important than
decoding time is searching time, which was the topic for this
set of experiments.  To quantify the searching speeds, each of
\grep, decompressed 
searching, and \reviewb were used to search for 
symbols selected at
random from the phrase hierarchy.  Decompressed searching
includes full decoding with \shuff, 
followed by a direct search on the reduced word sequence.  
Since an infinite block size was employed by \shuff, all the times
for decompressed search include 16.0 seconds of elapsed time and
13.9 seconds of CPU time.  

The symbols searched for were selected based on certain
criteria.  Only symbols which directly appeared
in the reduced word sequence and were not components of any
higher generation symbol in the phrase hierarchy were 
chosen.  These restrictions ensure that the comparison 
was as fair as possible.  Of the 8,277,359 symbols in the phrase
hierarchy, 4,376,175 symbols were deemed valid.

Experiments were conducted from two different perspectives,
based on the fact that not all of the valid symbols are 
equal.  The first experiment highlights the fact that symbols
occur with different frequencies.  The four million valid
symbols were divided into 3 equal groups based on frequency
in the reduced word sequence:  high, medium, and low.
The second set of experiments was based on symbol length,
in word tokens (or primitives).  The valid symbols
were divided into three groups based on their lengths when
expanded, as shown in 
\tabref{tab:review-ws-dist}.  Also shown in this table is
the number of invalid symbols (second column).  Valid 
symbols were divided based on their lengths and then merged
to create three roughly equal symbol length groupings,
as shown in the third and then the fourth columns of the
table.  In the first group, valid symbols of length 1 or 2
primitives are included.  The second group contains valid
symbols of lengths 3 or 4 primitives.  All remaining valid
symbols are in the third group.

\tab{cccc}
{
Symbol  & \multicolumn{2}{c}{Distribution} & Size of \\ \cline{2-3}
lengths & Invalid    & Valid        & group \\
(words) & candidates & candidates   & \\
}
{
{\D\D}1 & \C133,389 & \C203,349 & \multirow{2}*{\C915,908}\\
{\D\D}2 & \C615,548 & \C712,559 & \\
\hline
{\D\D}3 & \C804,927 & 1,149,698 & \multirow{2}*{2,040,548}\\
{\D\D}4 & \C623,040 & \C890,850 & \\
\hline
{\D\D}5 & \C415,727 & \C507,765 & \multirow{6}*{1,419,719}\\
{\D\D}6 & \C275,483 & \C281,307 & \\
{\D\D}7 & \C192,309 & \C156,154 & \\
{\D\D}8 & \C141,607 & \C101,003 & \\
{\D\D}9 & \C106,609 & \C\D62,752 & \\
10+     & \C592,544 & \C310,738 & \\
\hline
Total   & 3,901,184 & 4,376,175   & \\
}{Distribution of invalid and valid symbol for the
experiments with the reduced word sequence.  Phrase hierarchy
symbols are separated based on length (in primitives), with
the valid ones merged to form three symbol length
groupings.}{Distribution of symbols for reduced word
sequence experiments (\news)}{tab:review-ws-dist}

\figref{fig:review-ws-freq} and \tabref{tab:review-ws-freq-reviewb}
present the results from the first set of experiments, based on symbol frequency
in the reduced word sequence.  For each grouping, 1, 10, or 100
symbols were searched for, selected at random.  The same symbols were
searched three times and the time required was averaged, 
producing the elapsed and CPU times shown in the figure.  

\fig{
\begin{tabular}{cc}
\includegraphics*{./time-freq-high.eps} & 
\includegraphics*{./time-freq-medium.eps} \\
(a) High frequency group & (b) Medium frequency group \\
\multicolumn{2}{l}{\includegraphics*{./time-freq-low.eps}} \\ [1ex]
(c) Low frequency group & \\
\end{tabular}
}{Search times with \grep, decompressed search, and \reviewb for the
reduced word sequence based on frequency groupings.  The elapsed
times and CPU times are reported in seconds
and averaged over three trials with the same test symbols.}{Search
times for the reduced word sequence for frequency groupings (\news)}
{fig:review-ws-freq}

As more symbols are searched simultaneously, \grep requires 
more time.  However, the frequency of the symbols has a relatively
small effect on the elapsed and CPU times.  There is a noticeable
difference between elapsed and CPU times for \grep, due to the
number of disk accesses.  Meanwhile, 
decompressed searching requires the same amount of elapsed
and CPU times, regardless of the number of symbols sought
or the frequency of the symbols.  Of the three systems shown in
the figure, \reviewb achieves the best search times.  
As \reviewb searched for
more symbols at the same time, more time is required.
The effect of symbol frequency on search time begins to be noticeable
when as many as 100 symbols are searched for.  This result reflects
the data displayed in the last two columns of 
\tabref{tab:review-ws-freq-reviewb}.  The
third column indicates the number of blocks that were fully
decoded.  When searching with \reviewb, every prelude must
be decoded, but when the symbols being sought are not in the 
block, the rest of the block can be skipped.  As a consequence
of this property, the number of bytes read can never be less
than the combined size of all of the preludes, which is \mib{63.3}.  And,
as the fourth column shows, as more blocks need to be fully
decoded, more bytes need to be read.  So, when either more
symbols or more frequent symbols are searched for,
the number of blocks that are fully decoded and the number
of bytes read increases accordingly.

%% \tab{lcccccc}
%% {
%% \multicolumn{1}{c}{Frequency} & Symbols  & \multicolumn{2}{c}{\grep} & & \multicolumn{2}{c}{Decompressed search} \\ \cline{3-4} \cline{6-7}
%% \multicolumn{1}{c}{group}     & searched & elapsed & CPU & & elapsed & CPU \\
%% }
%% {
%% High   & \D\D1             &  35.1 & \D2.7 & & 21.7 &  18.2 \\
%% High   & \D10              &  34.9 &  11.1 & & 21.7 &  18.2 \\
%% High   & 100               &  58.3 &  38.3 & & 21.7 &  18.2 \\ [1.0ex]
%% Medium & \D\D1             &  34.9 & \D2.9 & & 21.7 &  18.2 \\
%% Medium & \D10              &  34.9 & \D9.9 & & 21.7 &  18.2 \\
%% Medium & 100               &  56.2 &  42.1 & & 21.7 &  18.2 \\ [1.0ex]
%% Low    & \D\D1             &  34.9 & \D3.0 & & 21.0 &  17.5 \\
%% Low    & \D10              &  34.9 &  12.1 & & 21.7 &  18.2 \\
%% Low    & 100               &  57.1 &  43.0 & & 21.7 &  18.2 \\
%% }{Search times with \grep and uncompressed searching 
%% for the reduced word sequence based on frequency groupings.
%% Times are in seconds and have been averaged over 3 trials with
%% the same test symbols.}{Search times with \grep and uncompressed
%% searching for the reduced word sequence based on frequency groupings.}
%% {tab:review-ws-freq-grep}

\tab{lccc}
{
\multicolumn{1}{c}{Frequency} & Symbols  & Number  & Number \\
\multicolumn{1}{c}{group}     & searched & of blocks  & of bytes   \\
          &          & decoded & read    \\
}
{
High   & \D\D1  & \D\D4  & \D\mib{63.9} \\
High   & \D10   & \D58   & \D\mib{73.0} \\
High   & 100    & 423    & \mib{134.0} \\ [1ex]
Medium & \D\D1  & \D\D2  & \D\mib{63.6} \\
Medium & \D10   & \D18   & \D\mib{66.3} \\
Medium & 100    & 161    & \D\mib{90.0} \\ [1ex]
Low    & \D\D1  & \D\D1  & \D\mib{63.4} \\
Low    & \D10   & \D14   & \D\mib{65.6} \\
Low    & 100    & 144    & \D\mib{87.4} \\
}{Search efficiency with \reviewb for the reduced word sequence 
based on frequency groupings.  
The number of blocks fully
decoded and the number of bytes read in order to complete 
the searching are shown in the third and fourth columns.}
{Search efficiency of \reviewb for frequency groupings (\news)}
{tab:review-ws-freq-reviewb}

The results from the second set of experiments with 
the reduced word sequence are contained in 
\figref{fig:review-ws-len} and
\tabref{tab:review-ws-len-reviewb}.  If 
\subfigref{fig:review-ws-freq}{c} is compared with 
\subfigref{fig:review-ws-len}{c}, it would appear that \grep
requires less time as symbols which expand to more words are
sought.  However, symbol length does not have a noticeable
effect on decompressed searching or \reviewb, according to
these graphs.

\fig{
\begin{tabular}{cc}
\includegraphics*{./time-len-short.eps} & 
\includegraphics*{./time-len-medium.eps} \\
(a) Symbols 1 -- 2 words in length & (b) Symbols 3 -- 4 words in length \\
\multicolumn{2}{l}{\includegraphics*{./time-len-long.eps}} \\ [1ex]
(c) Symbols 5+ words in length & \\
\end{tabular}
}{Search times with \grep, decompressed search, and \reviewb for the
reduced word sequence based on symbol length groupings.  The elapsed
times and CPU times are reported in seconds
and averaged over three trials with the same test symbols.}{Search
times for the reduced word sequence for symbol length groupings (\news)}
{fig:review-ws-len}

\tabref{tab:review-ws-len-reviewb} shows how less blocks are
requested as symbol length increases.  But this trend may be caused
by the fact that symbols with more words are expected to occur less
frequently.

%% \tab{lcccccc}
%% {
%% \multicolumn{1}{c}{Symbol} & Symbols  & \multicolumn{2}{c}{\grep} & & \multicolumn{2}{c}{Decompressed search} \\ \cline{3-4} \cline{6-7}
%% \multicolumn{1}{c}{length} & searched & Elapsed & CPU & & Elapsed & CPU \\
%% \multicolumn{1}{c}{group}  &          & time& time    & & time& time    \\
%% }
%% {
%% 1 - 2 & \D\D1  &  34.9 & \D3.3 & & 20.1 &  16.7 \\
%% 1 - 2 & \D10   &  34.9 &  10.7 & & 21.0 &  17.5 \\
%% 1 - 2 & 100    &  51.6 &  37.4 & & 21.0 &  17.5 \\ [1ex]
%% 3 - 4 & \D\D1  &  34.9 & \D3.1 & & 21.0 &  17.5 \\
%% 3 - 4 & \D10   &  34.9 & \D9.1 & & 21.7 &  18.2 \\
%% 3 - 4 & 100    &  51.4 &  37.2 & & 21.7 &  18.2 \\ [1ex]
%% 5+    & \D\D1  &  34.9 & \D2.9 & & 21.7 &  18.2 \\
%% 5+    & \D10   &  34.9 & \D7.5 & & 21.7 &  18.2 \\
%% 5+    & 100    &  42.0 &  26.4 & & 21.7 &  18.2 \\
%% }{Search times with \grep and uncompressed searching 
%% for the reduced word sequence based on symbol length groupings.
%% Times are in seconds and have been averaged over 3 trials with
%% the same test symbols.}{Search times with \grep and uncompressed
%% searching for the reduced word sequence based on symbol length groupings.}
%% {tab:review-ws-len-grep}

\tab{lccc}
{
\multicolumn{1}{c}{Symbol} & Symbols  & Number     & Number \\
\multicolumn{1}{c}{length} & searched & of blocks  & of bytes   \\
\multicolumn{1}{c}{group}  &          & decoded    & read    \\
}
{
1 -- 2  & \D\D1  & \D\D2  & \D\mib{63.6} \\
1 -- 2  & \D10   & \D41   & \D\mib{70.2} \\
1 -- 2  & 100    & 298    & \mib{113.0} \\ [1ex]
3 -- 4  & \D\D1  & \D\D2  & \D\mib{63.6} \\
3 -- 4  & \D10   & \D30   & \D\mib{68.3} \\
3 -- 4  & 100    & 261    & \mib{107.0} \\ [1ex]
5+     & \D\D1  & \D\D2  & \D\mib{63.6} \\
5+     & \D10   & \D32   & \D\mib{68.6} \\
5+     & 100    & 181    & \D\mib{93.7} \\
}{Search efficiency with \reviewb for the reduced word sequence 
based on symbol length groupings.  
The number of blocks fully
decoded and the number of bytes read in order to complete 
the searching are shown in the third and fourth columns.}
{Search efficiency of \reviewb for symbol
length groupings (\news)}{tab:review-ws-len-reviewb}

In conclusion, this section has looked at the coding requirements
of the three modifier streams and the reduced word sequence.  Indexed
\shuff produced the best compression effectiveness when sufficiently
large block sizes were chosen.  While the compression and random access
times offered
by \review and \flatbin for the modifiers were good, indexed \shuff
with small block sizes gave comparable results for skipping to arbitrary
locations within the streams.

Three compression methods were examined for the reduced word sequence.
As with the modifier streams, \shuff provided the best compression
effectiveness.  While the \flatbin coder also gave better compression levels
than \reviewb, \reviewb was found to be more efficient than \shuff and
\flatbin for decompression and searching.
In these experiments, a modified version of \news was
used by \grep, while decompressed searching assumed \shuff with an
infinite block size.  The times for decompressed searching and \reviewb 
were not affected by symbol frequency or symbol length, but both conditions
had a noticeable effect on the number of blocks fully decoded and the
number of bytes read from disk by \reviewb.  Furthermore, as more
symbols are searched for simultaneously, both \reviewb and \grep required 
more time.  

\newsection{Selecting Coders}{sec:review-summary}

This chapter has examined coders for the three modifier streams
and the reduced word sequence stream.  The decoding 
requirements of these four streams for phrase browsing are 
quite different from those that apply to the phrase hierarchy
and the two lexicons. Within these four streams, different decoding
specifications also emerge.  The modifier
streams require random access, whereas the reduced word
sequence is searched.

%% (168721097*3+58685281)/1048491001*8
%% 4.3096
The naive approach to coding these four streams is to simply
encode each reference as a 32-bit
integer.  Retrieval time is fast for both random access and searching,
and the difficulty is allocating enough storage.  Referring to
\tabref{tab:review-fileinfo}, such an approach would require
4.310 bits per character for only these streams.  If this level
of compression is acceptable, then such a coding scheme can
be considered.

At the other extreme, a compression system such as \bzip can be
applied to the modifier streams, as explored in
\chapref{chap:prepair}.  If such an approach was employed, then
the streams must be fully decoded before being used, 
yielding unfavourable response times.

As a balance between these extremes, indexed Huffman coding is a
reasonable choice for the modifier
streams.  As the streams
are required less frequently, if blocks are created with an index
for each block, then at most two blocks are decoded at any 
one time for seeking.  The ideal random access times occur
when a block size of 32,768 symbols is selected.  
If this block size is applied, the combined compression ratio of
the three modifier streams is 0.977 bits per character.

The reduced word sequence must be searchable and
\reviewb combines byte-alignment for the codewords
with the ability to skip blocks when it has been determined that
no symbol of interest occurs within the block.  The disadvantages
of this
approach are that all of the preludes must be decoded as part of every search,
and that compression effectiveness is not as good as if an entropy coder
like \shuff had been used.  One 
improvement to \reviewb, which was not explored, would be to create a
second-level index so that each token in the file is associated directly
with a list of the blocks in which it occurs.  

\reviewb achieves better search times than \grep and decompressed searching.
But the \grep system still performs remarkably well given 
that the search criteria were strings instead of sets of symbols.
Compression of the reduced
sequence with \reviewb requires
1.402 bits per character.

So, through the techniques described in this chapter and the encoding
methods decided earlier for the phrase hierarchy and lexicons, the
compression effectiveness for all seven streams is 2.523 bits per 
character.  \tabref{tab:review-streams} updates \tabref{tab:prepair-streams}
on \pgref{tab:prepair-streams} with the coding mechanisms examined in
this chapter.  The next chapter takes these files and shows how they 
can be used to construct a phrase browsing tool.

\setlength{\mytablength}{\textwidth}
\addtolength{\mytablength}{-0.1cm}

\tabnoline{@{}l}{}
{
\begin{tabularx}{\mytablength}{|l|X|X|X|}
\hline
Output from  & \mc{1}{\mr{{\textbf {word lexicon}}}} & \mc{2}{\mr{word sequence}} \\
\prepair     &                             & \mc{2}{} \\ \cline{1-4}

\mr{Purpose} & \mc{2}{\mr{phrase browsing}} & \mc{1}{\mr{searching for symbols}}  \\
             & \mc{2}{} & \\ \cline{1-4}
\mr{Model}   &                 & \mc{2}{\mr{\repair}}      \\
             & \mc{1}{front coding}    & \mc{2}{}             \\ \cline{1-1}\cline{3-4}
\multirow{3}*{Coder}   & \mc{1}{and \bzip} & \mc{1}{{\textbf {phrase hierarchy}}} & \mc{1}{{\textbf {reduced word sequence}}} \\
             & & \mc{1}{chiastic slide and} & \mc{1}{\mr{\reviewb}} \\ 
             & & \mc{1}{interpolative coding} &        \\ \cline{1-4}
\end{tabularx} \\
\multicolumn{1}{c}{(a) Words} \\ [1.5ex]

\begin{tabularx}{\mytablength}{|l|X|X|X|X|}
\hline
Output from  & \mc{1}{{\textbf {non-word}}} & \mc{1}{{\textbf {non-word}}} & \mc{1}{{\textbf {case folding}}} & \mc{1}{{\textbf {stemming}}} \\
\prepair     & \mc{1}{{\textbf {lexicon}}}  & \mc{1}{{\textbf {modifiers}}} & \mc{1}{{\textbf {modifiers}}} & \mc{1}{{\textbf {modifiers}}}  \\ \cline{1-5}
\mr{Purpose} & \multicolumn{4}{c|}{\mr{skipping to symbols}} \\
             & \multicolumn{4}{c|}{} \\ \cline{1-5}
\mr{Model}   &                 & \mc{1}{\multirow{4}*{indexed \shuff}}& \mc{1}{\multirow{4}*{indexed \shuff}}& \mc{1}{\multirow{4}*{indexed \shuff}}\\
             & \mc{1}{front coding}    &  & & \\ \cline{1-1}
\mr{Coder}   & \mc{1}{and \bzip}       & &  & \\
             &                 &          &          &        \\ \cline{1-5}
\end{tabularx} \\
\multicolumn{1}{c}{(b) Non-words and modifiers} \\ [1.0ex]
}{The seven streams produced by \prepair and \repair.  The purpose
of each stream, and the compression methods employed are taken from
\tabref{tab:prepair-streams} on \pgref{tab:prepair-streams}, updated
with the coding mechanisms in this chapter.}{The seven streams
by \prepair and \repair, updated}{tab:review-streams}


