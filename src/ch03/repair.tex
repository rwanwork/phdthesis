\newchapter{Compression by Recursive Pairing}{chap:repair}

%% * Did not mention:  \repair needs to keep track with a hash table
%%   the pairs that have already been replaced, to prevent duplicates
%%   in the phrase hierarchy.

\repair is a dictionary-based compression algorithm developed by
\citet{lm00:procieee}.  \repair operates off-line and commences the
compression process only after the entire message has been seen.
This is in contrast to dictionary-based algorithms such as \lza 
and \lzb, described in \chapref{chap:tc}, which build 
dictionaries on-line.  

Off-line compression algorithms have gained 
attention recently for two reasons.  First, the amount of memory
available in computers has steadily increased in recent years.
Earlier on-line
algorithms were able to operate with just enough memory to
perform the compression, but not enough to also hold the
entire message.  With more memory to use, off-line algorithms
which examine an entire message before starting
compression offer an alternative to on-line algorithms.
Second, decompression occurs more frequently than compression 
in retrieval systems.  In these circumstances,
an underlying off-line compression mechanism that operates slower
than its on-line counterparts but ensures fast decompression may
be preferred.  
As with on-line dictionary-based algorithms, off-line ones build
a dictionary of phrases.  Occurrences of these phrases in the
message are replaced by references to them.  However, off-line
dictionary-based algorithms are faster because the model does not
have to be updated throughout decompression.

\repair transmits the dictionary explicitly as a separate stream
to the receiver instead of incrementally like \lza.  This 
results in two outputs:  the dictionary, and a sequence of 
references to entries in the dictionary.  These two parts are coded 
independently and transmitted to the decompressor, \despair, 
as two distinct streams in order to
reconstruct the message.  As is discussed later in this chapter, 
it is \repair's separation of the message into two separate
streams that makes it ideal for phrase browsing.

The remainder of the chapter is structured as follows.  
\secref{sec:repair-model} provides an overview of the \repair 
algorithm.  While recent computers have more memory available 
for off-line algorithms to use, memory is still limited and 
\secref{sec:repair-analysis} gives a detailed analysis of 
the data structures used by \repair.  These data structures 
provide a balance between memory space and
execution time.  \secref{sec:repair-coding} describes how the
two outputs of \repair are represented as coded sequences.  \secref{sec:repair-expt}
gives experimental results for the implementation of \repair
used, and \secref{sec:repair-related} describes other algorithms
similar to \repair.  Finally, \secref{sec:repair-retrieve}
discusses the features of \repair that make it attractive for 
fast searching and retrieval, and the issues that prevent it 
from being used for that purpose.

\newsection{The \repair Model}{sec:repair-model}

\repair reduces the length of a message by replacing repeated
pairs of symbols with a new symbol which does not appear in the
message.  
Initially, the most frequently occurring pair of characters
is located and all occurrences of it are replaced with a new symbol.
Each replacement reduces the length of the message by one.
The new symbol is added to a dictionary, together with a 
record of how it should be expanded.  This process is 
repeated, and the next most frequently occurring pair of symbols 
(now including any new symbols introduced by 
\repair) is replaced.  The process terminates when no 
pair of adjacent symbols occurs more than once in the message.  
The source of the name ``\repair''
arises from the recursive pairing nature of the algorithm.

In order to describe \repair in greater detail, some more precise 
definitions and notations are necessary.  The alphabet ($\Sigma$)
of the original message are also called {\emph {primitives}}.  
\repair
creates and adds {\emph {phrases}} to the dictionary.  The set
of all phrases is denoted as $\rho$.  

Primitives and phrases are collectively called {\emph {symbols}}, 
and are denoted using, for example, $\alpha$, $\beta$, $\delta$, and
$\gamma$.  
If it is important to indicate that a symbol $\alpha$ is a 
phrase introduced by \repair and not a primitive, then it 
is circled, \tc{$\alpha$}.  If the
order in which the phrase was added to the dictionary is important,
then the phrase is instead indicated by a circled number, \tc{1}.

Every symbol $\alpha$ has a generation
$g(\alpha)$ which indicates how deep 
it is in the hierarchy of paired symbols.  By definition, 
the generation of any primitive is zero.  The generation of a 
phrase $\alpha$ is one more than the higher generation of its 
two components.

The outputs of \repair are a dictionary, and a sequence of 
references to entries in the dictionary.  
The dictionary contains all of the primitives and
phrases composed out of them by the pairing process.
Since every phrase consists of two symbols from
a lower generation (either shorter phrases or primitives), the
dictionary resembles a hierarchy of phrases.  To underscore this
fact, the dictionary generated by \repair is called the 
{\emph {phrase hierarchy}}, denoted as $\Ph$.
The list of pointers to symbols in the phrase hierarchy is called the 
{\emph {reduced sequence}}, and is denoted as $\Seq$.


A sample application of \repair is shown in 
\tabref{tab:repair-short}.  The sample message used is 
part of the ``Woodchuck'' message of \figref{fig:tc-woodchuck}.
The initial message is shown in the first entry of the first column.
Each subsequent row of the table indicates the identification 
of the most frequently occurring pair of symbols in the previous 
row, and the replacement of all occurrences of it with a new 
symbol.  The new phrase added to the phrase hierarchy as a 
result of that set of replacements is shown in the second column.  
For each phrase, the frequency with which it occurs in the 
message is shown in the third column, and the generation of that 
phrase is shown in the fourth column.  As expected, the 
sequence of replacements performed by \repair is in decreasing 
frequency.  In this example, the generation numbers of the phrases 
are in increasing order, but this is not always the case.

\tab{llcc}
{
\multicolumn{1}{c}{Sequence} & \multicolumn{1}{c}{Phrase hierarchy} & \emph{f} & $g(\alpha)$ \\
}
{
how{\tvs}much{\tvs}wood{\tvs}could{\tvs}a{\tvs}woodchuck{\tvs}chuck & & & \\
how{\tvs}mu\tc{1}{\tvs}wood{\tvs}could{\tvs}a{\tvs}wood\tc{1}uck{\tvs}\tc{1}uck & \tc{1} $\rightarrow$ c h & 3 & 1\\
how{\tvs}mu\tc{1}{\tvs}woo\tc{2}coul\tc{2}a{\tvs}wood\tc{1}uck{\tvs}\tc{1}uck & \tc{2} $\rightarrow$ d \tvs & 2 & 1\\
how{\tvs}mu\tc{1}\tc{3}oo\tc{2}coul\tc{2}a\tc{3}ood\tc{1}uck{\tvs}\tc{1}uck & \tc{3} $\rightarrow$ \tvs w & 2 & 1\\
how{\tvs}mu\tc{1}\tc{3}oo\tc{2}coul\tc{2}a\tc{3}ood\tc{1}u\tc{4}{\tvs}\tc{1}u\tc{4} & \tc{4} $\rightarrow$ c k & 2 & 1\\
how{\tvs}mu\tc{1}\tc{3}\tc{5}\tc{2}coul\tc{2}a\tc{3}\tc{5}d\tc{1}u\tc{4}{\tvs}\tc{1}u\tc{4} & \tc{5} $\rightarrow$ o o & 2 & 1\\
how{\tvs}mu\tc{1}\tc{3}\tc{5}\tc{2}coul\tc{2}a\tc{3}\tc{5}d\tc{6}\tc{4}{\tvs}\tc{6}\tc{4} & \tc{6} $\rightarrow$ \tc{1} u & 2 & 2\\
how{\tvs}mu\tc{1}\tc{7}\tc{2}coul\tc{2}a\tc{7}d\tc{6}\tc{4}{\tvs}\tc{6}\tc{4} & \tc{7} $\rightarrow$ \tc{3} \tc{5} & 2 & 2\\
how{\tvs}mu\tc{1}\tc{7}\tc{2}coul\tc{2}a\tc{7}d\tc{8}{\tvs}\tc{8} & \tc{8} $\rightarrow$ \tc{6} \tc{4} & 2 & 3\\

}
{A sample application of \repair on part of the ``Woodchuck'' message.  
Each line represents the replacement of all occurrences of
a pair of symbols.  The effect on the sequence and the phrase
hierarchy is shown, along with the frequency of the symbol pair,
and the generation number of the replacing phrase.}
{Sample application of \repair}{tab:repair-short}

Several phrases from the same generation may occur with
equal frequencies.  The order in which they are chosen depend
on the data structures used by \repair.  Discussion of the 
method for breaking ties appears in \secref{subsec:repair-time}.

After the last replacement in the table, no pair of 
symbols occurs more than once.  \repair then terminates, and the 
final sequence (last entry in the first column) is 
encoded.  The phrase hierarchy is also a necessary part of the 
coded message, which includes all of the second column as well 
as a description of the set of primitives.

\newsubsection{The Decompressor}{subsec:repair-despair}

The \repair decompressor (nicknamed, for good reason, \despair) 
is simpler than the
compressor.  After decoding the stream representing the phrase 
hierarchy, the entire structure is built within memory so that
each phrase requires two words of memory to indicate its two
components.  

A symbol in the reduced sequence is decoded by expanding the
phrase hierarchy until the string of primitives that it 
represents are found.  A recursive traversal for each symbol 
in the reduced message can be slow, depending on the size of
the phrase hierarchy and the generation of each symbol.
A more economical approach
is to decode every phrase in the phrase hierarchy first, 
so that a decoding table could be made.
Then, for each symbol in the sequence, each decoded symbol can be 
copied directly from the decoding table to the output stream.  
This method is significantly 
faster, and is similar to the \lzb algorithm discussed in the 
previous chapter.
A third approach, which strikes a balance between the other two, 
maintains a buffer of recently decoded symbols for later use.  
When a symbol is decoded, the buffer is consulted first before
it is recursively expanded.

\tabref{tab:repair-despair} presents some experimental results
which compare the differences in
time and memory usage of these three methods.  The test data
is \wsja, which was compressed in experiments described later
in \secref{sec:repair-expt}.  The compressed
message has 90 primitives, 310,486 phrases, and 1,904,118
references.  The longest phrase was composed of 1,257 
primitives, and there were 21 generations in the phrase
hierarchy.  The third method listed in 
\tabref{tab:repair-despair} employs a \kib{256} buffer
of recently expanded symbols.
Decompression time was measured in CPU seconds, averaged 
over 3 trials.  Memory is stated in \mibonly, and represents
the maximum memory consumed during execution.

\tab{lcc}
{
\multicolumn{1}{c}{Method} & Time (s) & Memory (\mibonly) \\
}
{
Recursive expansion       & 4.3 & 11.5 \\   %%  favouring memory
Fully expanded symbols    & 2.2 & 30.2 \\   %%  favouring time
Buffer of recent symbols  & 2.3 & 11.8 \\   %%  compromise
}
{The trade-offs between decoding time and maximum memory usage 
of \despair using the three approaches on \wsja.  The third method
uses a buffer of \kib{256}.}
{Trade-offs between decoding time and
memory usage for \despair (\wsja)}{tab:repair-despair}

Expanding the phrase
hierarchy before decoding the sequence was only slightly faster
than employing a buffer of recently expanded symbols.  The 
difference in time is expected to be larger if the reduced
sequence had more symbols.  That is, the cost of fully expanding
symbols is recovered when more symbols are decoded.
Recursively expanding the phrase hierarchy with each symbol
requires more time, but less memory.  The difference in memory
between the first and the third methods is equal to the size of
the buffer.  Also, note that the amount of extra memory required
by the second method is equal to the size of the document
due to the recursive nature of the phrase hierarchy.

The results show that the creation of a modest buffer offers
a compromise between execution time and memory usage.
This approach has been adopted in the implementation of \despair
for later experiments.

\newsection{Analysis of \repair}{sec:repair-analysis}

As with any compression system, implementation choices in \repair
allow a range of trade-offs between memory space and processing
time.  The three different approaches to the implementation of
\despair is one example of the choices available.  A similar
balance is required by \repair between memory space and 
execution time.  This section discusses data structures 
described by \citet{lm00:procieee}, and the trade-offs 
that they allow.

\newsubsection{Data Structures}{subsec:repair-datastructs}

The three main data structures used by \repair are shown in
\figref{fig:repair-datastruct}, with part of the text of 
\tabref{tab:repair-short} used as an example.  The purpose 
of the data structures is to
locate {\emph {active pairs}} and {\emph{active phrases}}
efficiently.  An active pair is a pair of symbols 
in the sequence that is under consideration for replacement.
Likewise, an active phrase is a potential 
replacement for all
active pairs which have the same left and right components.
For example, if the pair {$\alpha$$\beta$} appears 10 times
in the sequence, each of the occurrences is an
active pair, and the 10 active pairs are associated with a
single active phrase.  Active phrases are required when
considering a set of replacements, while active pairs are 
required for a particular replacement.

\fig{\input{./repair-datastruct.pstex_t}}
{The main data structures used by \repair.  The active pair
pointers for one occurrence of the pair ``ch'' are shown.  
Since the list of
active pairs is a circular list, the ``next'' pointer of the
last ``ch'' leads to the first ``ch''.  The shaded areas
indicate pointers used for the two sets of double-linked
lists, one pair to link items in the hash table and one 
pair to link items in the priority queue structure.}
{Main \repair data structures}{fig:repair-datastruct}

The first data structure is an array of {\emph {sequence nodes}},
shown along the top of \figref{fig:repair-datastruct}.  Initially,
each sequence node holds a primitive from the message so that
for a message of $n$ primitives, $n$ sequence nodes are 
required.  During compression, a pair of adjacent symbols is replaced
with a single symbol.  Each sequence node contains 
a symbol and two pointers.  The pointers serve two purposes.  
Active pairs are formed by taking pairs of adjacent sequence nodes 
so that every node (except for the first and last ones) 
can be a left component and a right component of an active pair.
The first purpose of the sequence node pointers is to 
form a circular double-linked list of active pairs 
so that each node points to the left components of the previous 
and next equivalent active pairs.  In 
\figref{fig:repair-datastruct}, the first ``ch'' active pair 
shown is the second of three that appear in the 
``Woodchuck'' message.  When the
active phrase ``ch'' is chosen, each occurrence of ``ch'' 
can be replaced in $O(1)$ time.  As active pairs are replaced,
gaps appear in the sequence.  The second purpose of the node
pointers is to allow each gap to be traversed in $O(1)$
time, as illustrated in \figref{fig:repair-seqnode}.  
Four snapshots of the sub-sequence of 
\figref{fig:repair-datastruct} are shown at various points
during the recursive pairing process.
As gaps appear in the sequence array, the pointers at
either end of each gap lead to the other end of the gap.  The nodes
within a gap are unused, and are coloured grey.

\fig{\begin{tabular}{l}
\input{./repair-seqnode-a.pstex_t} \\
\multicolumn{1}{c}{(a)} \\[1ex]
\input{./repair-seqnode-b.pstex_t} \\
\multicolumn{1}{c}{(b)} \\[1ex]
\input{./repair-seqnode-c.pstex_t} \\
\multicolumn{1}{c}{(c)} \\[1ex]
\input{./repair-seqnode-d.pstex_t} \\
\multicolumn{1}{c}{(d)} \\
\end{tabular}}{The re-using of sequence node pointers to 
jump over gaps created during the \repair process.  For
simplicity, pointers indicating active 
pairs are not shown.  The parts of a sequence node which
are no longer used have been shaded.}{Re-using \repair sequence nodes}
{fig:repair-seqnode}

At the centre of \figref{fig:repair-datastruct} are active phrase
nodes.  An active phrase node contains a pointer to its active
pairs, as well as a count of the number of active pairs.  Each
active phrase needs to be locatable either by its
active pair count or its components.  To accomplish this, each 
node is placed on two circular double-linked lists.  One
list permits fast look-up of symbol pairs through a hash table,
with collisions resolved by chaining.  The other list is placed in
a priority queue data structure implemented as an array.  Each slot 
in the priority queue corresponds to active phrases of a given number
of active pairs.  The number of slots in the priority queue 
is equal to the number of active pairs associated with the active
phrase with the most active pairs.
The active phrases in a list at a given priority
queue slot are placed so that the oldest active phrase node is at
the head of the list.  Active phrase nodes are attached to the 
hash table and priority queue using circular double-linked lists, 
so as to allow $O(1)$ time addition and removal from their respective 
lists, once they have been located.

A fourth data structure, not shown in the figure, records 
the active phrases that have been 
added to the phrase hierarchy.  Each phrase in the 
phrase hierarchy contains pointers to its two component phrases
and its generation number.

\newsubsection{Space Analysis}{subsec:repair-space}

The amount of space used by the four main data structures of
\repair is linear in the length of the message.  To see this, 
suppose that a message of $n$ primitives is to be processed.  The 
greatest number of active phrases throughout the processing of 
the message is $p$.  The size of the hash table is fixed at $h$ slots.
The most active pairs an active phrase can have is $\lfloor n/2 \rfloor$,
which is equal to the maximum number of slots for the priority queue.  
The most active pairs occur when every primitive in the message is
the same.  In this case, there
are $n - 1$ pairs of primitives, but only half of them are made active
because \repair prevents the existence of overlapping pairs.
For example, when there are three consecutive symbols with 
the same value, only one active pair exists.  Overlapping pairs are 
elaborated further below.

The amount of memory used by the main data structures
and the active phrase nodes in words is summarised in
\tabref{tab:repair-memory}.  Since an active phrase node is
larger than a phrase in the phrase hierarchy, the space
occupied by an active phrase node can be re-used by its corresponding
phrase in the phrase hierarchy.  So, the memory required for
the phrase hierarchy is not shown in \tabref{tab:repair-memory}.
Initially, as many as $n - 1$ active phrases may be created, each
with one active pair.  However, as none of them would be 
candidates for replacement, \repair would then terminate.  
Indeed, in order to minimise memory usage, \citet{lm00:procieee} 
showed how only useful active phrases that have at least two
active pairs each would be created by performing an additional
pass through the message.

\tab{lc}
{Data structure & Space used \\}
{sequence         & $3n$ \\
active phrases   & $6p$ \\
hash table       & $h$ \\
priority queue   & $\lfloor n/2 \rfloor$ \\}
{The amount of memory used by the most important data structures
in \repair, in words.  The initial message has $n$ symbols.
The most active phrases in memory during compression is $p$.  
The hash table has $h$ slots.}
{Memory usage of \repair data structures}{tab:repair-memory}


While the exact relationship between $p$ and $n$ cannot 
be determined theoretically, empirically, it can be shown 
that $n$ is the dominant term for average text.  To investigate
the relationship, \repair was
used to process \wsja.  (Details of these experiments appear
in \secref{sec:repair-expt}.)  For \wsja, 
$|\Sigma| = 90$, $|\rho| = 310\mbox{,}486$, 
and $n = 20\mbox{,}971\mbox{,}520$.  
\figref{fig:repair-activeph} shows the number of active phrases
in memory after a certain number of phrases have been added to
the phrase hierarchy by \repair for \wsja.  The number of 
phrases that have been added to the phrase hierarchy 
(horizontal axis) was sampled at regular intervals.  The total 
number of active phrases peaks at 482,370 after
13,137 phrases have been created in the hierarchy.
Afterwards, the number of active phrases decreases until the last 
replacement, which creates symbol number 310,486, deletes 
the last active phrase.  The size of the priority queue is 
initialised to the most frequent symbol, which for \wsja, 
occurred 466,082 times.

\fig{\includegraphics*{./wsja-activeph.eps}}
{The relationship between the number of
active phrases and the number of phrases added to the
phrase hierarchy for \wsja.}{Number of active
phrases during \repair execution (\wsja)}
{fig:repair-activeph}

Using the above values for 
$n$ and $h$, if $p = 482\mbox{,}370$, then $n$ is easily the
most important factor with respect to memory usage.  
The sequence nodes occupy the most space out of all 
the data structures.  Therefore, \repair requires $3.5n + O(p)$ words 
of memory.

The memory space analysis presented is slightly different from the
one given by \citet{lm00:procieee} due to minor changes to the data 
structures used.  For example, \citeauthor{lm00:procieee} limited 
the size of the priority queue to $\sqrt{n}$ words of memory so 
that the last list in the priority queue contains active phrase nodes
which have $\sqrt{n}$ or more active pairs.  This final list is
again ordered so that the least recently created active phrase 
node is at the head of the list.  However, to locate the next
active phrase on the list with the most active pairs would require
a linear search through the list.  Instead, a larger priority queue 
was chosen to avoid this linear search.
\citeauthor{lm00:procieee} also
showed a method of periodically compacting the sequence nodes if
memory becomes a serious constraint.  By reclaiming the space
occupied by the gaps, no new active phrase nodes need to be
created after the replacement process has begun.  Each compaction 
operation requires an additional pass over the sequence node
array to locate the gaps and an update of the pointers of each
sequence node.

A shorter priority queue and sequence node compaction 
reduces the amount of memory used by \repair by
using additional execution time.  Neither of these approaches were
incorporated into the implementation of \repair for this 
thesis because a higher priority was given to execution time.
Furthermore, if longer messages are processed which require 
more memory than available, then an 
alternative method is employed, which is described in 
\chapref{chap:remerge}.

\newsubsection{Time Analysis}{subsec:repair-time}

The data structures described in the last section
allow a message to be processed efficiently.  With the data
structures described, the \repair algorithm is shown in
\algref{alg:repair-alg}.  

\algo{
\begin{minipage}{13cm}
\begin{raggedright}
\begin{enumerate}
\item Scan the message and create up to $n - 1$ active pairs.  Create 
      active phrases as necessary and add them to the hash table. 
      \label{algline:repair-alg:scan}
\item Add every hashed active phrase with at least two active pairs
      to the appropriate position in the priority queue.  Active
      phrases that have less than the minimum number of active pairs
      are deleted.
\item Recursively pair symbols as follows until no pair of adjacent
      symbols occur twice or more in the sequence.
  \begin{enumerate}
  \item Retrieve the first active phrase node from the list 
        located at the last priority queue slot that is in use. 
        This node corresponds to the active phrase, 
        $\alpha$$\beta$, which currently has the most active pairs.  
        Let the replacement be \tc{$\gamma$}.
    \begin{enumerate}
    \item Locate the next active pair for $\alpha$$\beta$ and its 
          context, $\delta$$\alpha$$\beta$$\delta$.
    \item  Decrement the active phrase frequency counts of 
           $\delta$$\alpha$ and $\beta$$\delta$ and move them one 
           priority queue list to the left.  Remove any active 
           phrase with a frequency count of 0.
    \item Replace $\alpha$$\beta$ with \tc{$\gamma$}. 
    \item Increment the active phrase frequency counts of 
          $\delta$\tc{$\gamma$} and \tc{$\gamma$}$\delta$ and 
          insert them to the end of the appropriate priority 
          queue lists.  Create new active phrases if necessary.
          \label{algline:repair-alg:incrcount}
    \end{enumerate}
  \item Add the phrase \tc{$\gamma$} $\rightarrow \alpha$$\beta$ 
        to the phrase hierarchy. 
  \item Remove the active phrase $\alpha$$\beta$ from the hash 
        table and priority queue.
  \item Remove all active phrases with counts less than 2. 
  \end{enumerate}
\item Transmit the phrase hierarchy and the reduced sequence using
      an appropriate representation.
\end{enumerate}
\end{raggedright}
\end{minipage}
}{The \repair algorithm.}{\repair}{alg:repair-alg}


An analysis of the time complexity of the algorithm 
is as follows.  In the first step, an initial pass over the
sequence is performed which considers every symbol as a left
and a right component with its two immediately adjacent 
symbols.  Active phrases are created if necessary and inserted
into the hash table.  Then, in step 2, a pass is performed
over the hash table to insert active phrases into the priority
queue lists.

The dominant step of the algorithm is the recursive pairing loop 
that begins at step 3.  Each replacement reduces the length of 
the sequence by one symbol.  During a single replacement, the
most costly steps involve adding, removing, and updating active 
phrases.  By using double-linked lists in the hash table and 
priority queue, all of these operations 
execute in $O(1)$ expected time.  Therefore, \repair runs in time
linear to the length of the message.

Furthermore, the number of active 
phrases created by replacing the $k$ active pairs of 
${\alpha}{\beta}$ can never exceed $k$.  That is, after the
priority queue has been initialised in step 2, it does not 
enlarge in size.  Also, once an active phrase has been placed
on the priority queue list at slot $k$, it is guaranteed to
be used by \repair for pairing or moved to a lower slot in the
priority queue.  That is, \repair is guaranteed to terminate.

The algorithm also stipulates how active phrases with equal 
frequency counts are handled.  A ``least recently used'' 
policy is enforced by adding active phrases to the end of the 
priority queue lists.  This ensures that active phrases 
are processed first in decreasing frequency, and then 
in the order in which they were encounterd by
\repair.  An exception to this rule are generation 1 active 
phrases.  Generation 1 active phrases are created in step 1 but
are added to the priority queue in step 2 from the first 
slot in the hash table to the last.  

\newsubsection{Problems with \repair}{subsec:repair-problems}

There are two problems with the \repair algorithm which 
were not discussed by \citet{lm00:procieee}.  They
arise in certain isolated cases.  They appear because
of \repair's handling of overlapping pairs.  
Overlapping pairs occur when a sequence of length 3 or more 
of a single symbol appear in the message.  During
the initial scanning, the second of two overlapping pairs is
not considered as an active pair.  For example, in the
string ``$\text{a}_1\text{a}_2\text{a}_3$'' (where the
subscripts are used to distinguish between occurrences of 
``a''), ``$\text{a}_1\text{a}_2$'' is the only active pair.
A false count would result if these three primitives were 
considered as two active pairs.

However, suppose a section of a sequence appears like
this:  $\delta\alpha{\beta_1}\underline{{\beta_2}{\beta_3}}{\delta}$.  
Every pair of symbols is an active pair except for the underlined
pair because of the three consecutive $\beta$ symbols.  If
the replacement \tc{$\gamma$} $\rightarrow {\alpha}{\beta}$ is made
next, then the sequence becomes 
$\delta$\tc{$\gamma$}$\underline{{\beta_2}{\beta_3}}{\delta}$.
According to the algorithm, after the replacement, two new
active pairs are considered:  $\delta$\tc{$\gamma$} and \tc{$\gamma$}{$\beta_2$}.
But with ${\beta_1}$ replaced, ${\beta_2}{\beta_3}$ should be
considered an active pair.  While the algorithm looks for new
active pairs by examining the immediately preceding and following
symbols to a replacement, when overlapping pairs are concerned,
this example shows that a look-ahead of two symbols from where
the replacement was made is required to ensure that an active
pair is not missed.

Unfortunately, longer sequences of consecutive symbols may require 
more look-ahead.  Suppose there are five consecutive symbols as in 
the following section of a sequence:
$\delta\alpha{\beta_1}\underline{{\beta_2}{\beta_3}}\overline{{\beta_4}{\beta_5}}{\delta}$.
Two pairs of symbols are inactive, as shown.
As before, suppose ${\alpha}{\beta}$ is replaced, resulting in:
$\delta$\tc{$\gamma$}$\underline{{\beta_2}{\beta_3}}\overline{{\beta_4}{\beta_5}}{\delta}$.
The consecutive sequence of four $\beta$'s now only contains one active pair.
After the replacement, the sequence should have been examined up to
four symbols after \tc{$\gamma$} so that the active pair ${\beta_3}$${\beta_4}$ 
gets replaced by the two active pairs ${\beta_2}$${\beta_3}$ and 
${\beta_4}$${\beta_5}$.  The amount of look-ahead required, 
though, depends on the number of consecutive symbols.  In 
the implementation of \repair used, a look-ahead of just two 
symbols is performed, and when more consecutive symbols appear 
in a message, active pairs are missed.  However, a missed active
pair causes a problem only if it would have been replaced.
Suppose in the above example, ${\beta_5}$${\delta}$ is replaced by
\tc{$\alpha$}.  The sequence then becomes
$\delta$\tc{$\gamma$}$\underline{{\beta_2}{\beta_3}}{\beta_4}$\tc{$\alpha$}.
Three $\beta$'s remain with only one active pair among them, which
is correct.  Even though the active pair ${\beta_4}$${\beta_5}$ was
missed, it was unimportant because of a replacement involving 
${\beta_5}$.

The problem of consecutive symbols causing missed active pairs that
should have been replaced generally occurs for only whitespace
in average text.
In the \mib{20} of \wsja mentioned earlier, only one active pair was
missed which could have been replaced.  The problem may be more
serious for pathological cases and binary data, such as
graphic files.  The word-based parsing for \repair 
described in the next chapter make this situation even
more rare, though still possible.

\label{memo:repair:phdups}The second problem with the 
\repair algorithm is the appearance
of duplicate phrases caused by consecutive symbols.  After the
replacement of ${\alpha}{\alpha}$ with \tc{$\gamma_1$}, three
consecutive symbols may appear multiple times in the sequence,
requiring another replacement.  However, based on what pairs of
consecutive symbols are active and which are inactive, two
phrases may result.  The first replacement might be 
\tc{$\gamma_2$} $\rightarrow {\alpha}{\gamma_1}$ and the second
might be \tc{$\gamma_3$} $\rightarrow {\gamma_1}{\alpha}$.
These two phrases are identical if they are expanded
to primitives.  This problem is also not handled by the
implementation of \repair used in this thesis, but 
a resolution is discussed in \chapref{chap:remerge}.

\newsection{Coding for \repair}{sec:repair-coding}

The phrase hierarchy and the sequence produced by the \repair 
model need to be coded separately before transmission.  The 
coding requirements of each are different because of how they are 
used.  The phrase hierarchy needs to be fully 
decoded due to the inter-dependency between symbols across
generations.  On the other hand, it is possible that only
small sections of the sequence are
required during retrieval.  Any part of the sequence can be
displayed as long as the phrase hierarchy is already available.  
This decoding requirement for the sequence is set aside for the 
moment and is considered in \chapref{chap:review}.

\newsubsection{The Phrase Hierarchy}{subsec:repair-ph}

\citet{lm00:procieee} discuss a variety of methods for encoding 
the phrase hierarchy.  The best method, chiastic slide followed 
by interpolative coding, is used in the implementation 
of \repair considered in this thesis.

Interpolative coding \citep{ms00:ir} encodes a sorted list
of non-uniformly distributed integers by first encoding the 
median, and then recursively representing the 
two half lists of values which are less than or
greater than the median.  Since
the number of possible values is restricted by the endpoints of 
the intervals, the medians within each interval can be encoded within the range established by its endpoints instead of the entire list.  This allows 
shorter codes to be used when large number of values cluster 
together.

The two components of each phrase in the phrase hierarchy form
a two-dimensional space of approximately 
$(|\Sigma| + |\rho|)^2$ in size.  To apply
interpolative coding each phrase is mapped to a single number using its
left ($l$) and right ($r$) components.  The method used
is called the chiastic slide, denoted as $\chi(l,r)$.

Chiastic slide values are calculated incrementally starting from
the first generation of phrases.  The way that
generations are defined ensures that each phrase has
at least one of its components in the immediately preceding
generation, as otherwise it would have been placed in an earlier
generation.  To calculate the chiastic slide values for
phrases in generation $i$, the two-dimensional space 
where at least one component is in generation $i - 1$ is 
enumerated.  The set of primitives is also interpolative 
coded, but does not need to be mapped using the chiastic 
slide.  Instead, they are encoded using their underlying values.
For example, the primitives of the text 
message of \tabref{tab:repair-short} are encoded as a sorted
list of their \eascii representations.

\figref{fig:repair-chi} shows how chiastic slide is used to 
enumerate the primitives and the first generation of phrases for 
the example of \tabref{tab:repair-short}.  On the left is a list
of all possible left components and along the bottom is a list
of all possible right components.  

\sidefig
{
\begin{tabular}{cccc@{}cccccccccccccccc}
\LCC
& & & & \lightgray & \lightgray & \lightgray & \lightgray & \lightgray & \lightgray & \lightgray & \lightgray & \lightgray & \lightgray & \lightgray & \gray & \gray & \gray & \gray & \gray \\
& oo & 15 & {\vline width 1pt} & 9 & 19 & 29 & 39 & 49 & 59 & 69 & 79 & 89 & 99 & 109 & 118 & 125 & 130 & 133 & 134 \\
& ch & 14 & {\vline width 1pt} & 8 & 18 & 28 & 38 & 48 & 58 & 68 & 78 & 88 & \framebox{98} & 108 & 117 & 124 & 129 & 132 & 131 \\
& ck & 13 & {\vline width 1pt} & 7 & 17 & 27 & 37 & 47 & 57 & 67 & 77 & 87 & 97 & 107 & 116 & 123 & 128 & 127 & 126 \\
& d{\tvs} & 12 & {\vline width 1pt} & 6 & 16 & 26 & 36 & 46 & 56 & 66 & 76 & 86 & 96 & 106 & 115 & 122 & 121 & 120 & 119 \\
& {\tvs}w & 11 & {\vline width 1pt} & 5 & 15 & 25 & 35 & 45 & 55 & 65 & 75 & 85 & 95 & 105 & 114 & 113 & 112 & 111 & \framebox{110} \\
\ECC
\LCC
& & &  &  &  &  &  &  &  &  &  &  &  &  & \lightgray & \lightgray & \lightgray & \lightgray & \lightgray \\
\multirow{1}{1mm}{\begin{sideways}{left component ($l$)}\end{sideways}}& w & 10 & {\vline width 1pt} & 20 & 39 & 56 & 71 & 84 & 95 & 104 & 111 & 116 & 119 & 120 & 104 & 103 & 102 & 101 & 100 \\
& u & 9 & {\vline width 1pt} & 19 & 38 & 55 & 70 & 83 & 94 & 103 & 110 & 115 & 118 & 117 & 94 & 93 & 92 & 91 & 90 \\
 & o & 8 & {\vline width 1pt} & 18 & 37 & 54 & 69 & 82 & 93 & 102 & 109 & \framebox{114} & 113 & 112 & 84 & 83 & 82 & 81 & 80 \\
& m & 7 & {\vline width 1pt} & 17 & 36 & 53 & 68 & 81 & 92 & 101 & 108 & 107 & 106 & 105 & 74 & 73 & 72 & 71 & 70 \\
& l & 6 & {\vline width 1pt} & 16 & 35 & 52 & 67 & 80 & 91 & 100 & 99 & 98 & 97 & 96 & 64 & 63 & 62 & 61 & 60 \\
& k & 5 & {\vline width 1pt} & 15 & 34 & 51 & 66 & 79 & 90 & 89 & 88 & 87 & 86 & 85 & 54 & 53 & 52 & 51 & 50 \\
& h & 4 & {\vline width 1pt} & 14 & 33 & 50 & 65 & 78 & 77 & 76 & 75 & 74 & 73 & 72 & 44 & 43 & 42 & 41 & 40 \\
& d & 3 & {\vline width 1pt} & \framebox{13} & 32 & 49 & 64 & 63 & 62 & 61 & 60 & 59 & 58 & 57 & 34 & 33 & 32 & 31 & 30 \\
& c & 2 & {\vline width 1pt} & 12 & 31 & 48 & 47 & \framebox{46} & \framebox{45} & 44 & 43 & 42 & 41 & 40 & 24 & 23 & 22 & 21 & 20 \\
& a & 1 & {\vline width 1pt} & 11 & 30 & 29 & 28 & 27 & 26 & 25 & 24 & 23 & 22 & 21 & 14 & 13 & 12 & 11 & 10 \\
& {\tvs} & 0 & {\vline width 1pt} & 10 & 9 & 8 & 7 & 6 & 5 & 4 & 3 & 2 & 1 & \framebox{0} & 4 & 3 & 2 & 1 & 0 \\ \cline{5-20}
\ECC
&  &  & & 0 & 1 & 2 & 3 & 4 & 5 & 6 & 7 & 8 & 9 & 10 & 11 & 12 & 13 & 14 & 15 \\
&  &  & & {\tvs} & a & c & d & h & k & l & m & o & u & w & {\tvs}w & d{\tvs} & ck & ch & oo \\
& & & & \multicolumn{16}{c}{right component ($r$)}\\
\end{tabular}}
{Example of chiastic slide for the example of 
\tabref{tab:repair-short}.  The chiastic slide values for the first 
two generations of phrases are shown in boxes.}
{Chiastic slide example}{fig:repair-chi}



The set of all possible first generation phrases (unshaded region)
can be calculated once all of the primitives have been sorted
and encoded.  The five boxed numbers in this region represent the
five phrases from the first generation.  This sorted list 
$[0, 13, 45, 46, 114]$ is then interpolative coded.

The encoding of the first generation of phrases is a special
case since only one previous generation (the primitives) 
exists.  The encoding of subsequent generations all
resembles the encoding of the second generation (shaded region).
Phrases in the second generation must have either both
components in the first generation (dark grey region) or one
component in an earlier generation (light grey regions).
Since it is more likely that one of the two components of a
phrase is from a different generation, the chiastic slide assigns 
values starting from where one of the components is zero up to
the top right corner of the dark grey region.  The sorted list
$[98, 110]$ is used to represent the second generation of 
phrases; they are assigned symbol numbers 16 and 17.

In order to assign chiastic slide values for the current generation, 
$i$, the size of the shaded region is required.  This region is bounded
by the total number of primitives and phrases up to and including 
the immediately preceding generation, $g_{i-1}$, and up to but
excluding it, $g_{i-2}$.  The size of the shaded region is thus
$g^2_{i-1} - g^2_{i-2}$.  In the example, to encode the second
generation of phrases, $i = 2$, $g_{1} = 16$, $g_{0} = 11$, and
$g^2_{1} - g^2_{0} = 134$.


The formula used to calculate the chiastic slide, adapted from 
\citeauthor{lm00:procieee}, is shown in \eqnref{eqn:repair-chi}.  The first 
two cases are the light grey regions, while the last 
two are used for the dark grey region.  While calculating
the chiastic slide requires multiplication and squaring operations
(and division and square roots for decoding),
many of the terms are calculated per generation
and not per phrase.

\neweqn{\chi(l,r) = 
\begin{cases}
2l (g_{i-1} - g_{i-2}) + g_{i-1} - 1 - r & \text{for $l < g_{i-2}$} \\
(2r + 1) (g_{i-1} - g_{i-2}) + l - g_{i-2} & \text{for $r < g_{i-2}$}\\
(l (2g_{i-1} - l)) + g_{i-1} - 1 - r - g_{i-2}^2 & \text{for $g_{i-2} \leq l \leq r$} \\
(r (2g_{i-1} - r - 2)) + g_{i-1} - 1 + l - g_{i-2}^2 & \text{for $g_{i-2} \leq r < l$} \\
\end{cases}}{eqn:repair-chi}

In order to use interpolative coding with the chiastic slide, the
phrase hierarchy needs to be sorted twice.  First, the phrases
are sorted by generation number, since they were added to the phrase
hierarchy by decreasing frequency and not increasing generation.
Second, for each generation, the phrases are assigned values 
using the chiastic slide, 
and are sorted by these values before being interpolative coded.
As shown in \figref{fig:repair-chi}, after the primitives and 
phrases have been assigned chiastic slide values and sorted
by them, they are enumerated starting from 0 (shown along the two
axes).  A final pass through the sequence is also required
to ensure it uses the re-numbered phrase hierarchy, rather 
than the one that corresponds to decreasing pair frequency.

The encoding of the phrase hierarchy 
resembles a context-free grammar in Chomsky normal form (CNF) 
\citep{chomsky59:ic}.  In CNF, every production rule consists of a 
non-terminal which generates either two other non-terminal, or a 
single terminal.  The encoding of phrase ``ch'' represents
\tc{14} $\rightarrow$ \tc{2}\tc{4} while the primitive ``c''
is represented as \tc{2} $\rightarrow$ ``c''.  It is
for this reason that each entry in the phrase hierarchy can
be considered to be a rule.  The significant differences 
between a \repair phrase hierarchy and a more general CNF is 
that the phrase hierarchy is used to generate a single message
using the sequence as a starting rule whereas a context-free 
grammar in CNF is  usually expected to generate 
a language with more than one string in it.

\newsubsection{The Reduced Sequence}{subsec:repair-seq}

The other output of the \repair process is the reduced sequence.  
As the primitives and phrases are enumerated from 0 at the end
of the chiastic slide encoding, references in the reduced
sequence are just integers which refer to a rule in the phrase hierarchy.
Since the purpose of \repair is to remove redundancy 
in the message, no pair of symbols appears more than once in
the reduced sequence.  An entropy coder using a zero-order model on the
symbols in the sequence captures the great majority of 
the remaining structure.  Certainly, using a higher 
order model would be counter productive.
The semi-static Huffman coder and the arithmetic coder described
in the last chapter 
can be used without further modification to the sequence.

\newsection{Experiments}{sec:repair-expt}


Experiments were conducted on the Large Canterbury Corpus 
and \wsja on a \vipe machine, described in the last chapter.  
Larger test data could not be 
compressed because of the amount of memory used by \repair.  
The next chapter demonstrates how
this limitation can be overcome by partitioning the message
into blocks.  For all experiments described in this chapter,
each message is processed as a single block.

Some statistics from applying \repair
on these files are shown in \tabref{tab:repair-stats}.  As
an example of the phrases formed by \repair, the beginning
of \wsja is presented in \figref{fig:repair-wsjhead}.

%%  Note:  \Seq length includes the terminating 0 symbol.
\tab{lccccc}
{\multirow{2}*{Filename} & \multirow{2}*{$|\Sigma|$} & \multirow{2}*{$|\rho|$}  & \multirow{2}*{$|\Seq|$}   & Number of & Longest phrase \\
                         &            &          &   & generations & (in primitives) \\}
{ 
{\texttt {E.coli}}       & \D4           & \D67,368 & \C652,665        & 20 & 1,800 \\
{\texttt {world192.txt}} & 94            & \D55,473 & \C212,648        & 19 & \D432 \\
{\texttt {bible.txt}}    & 63            & \D81,229 & \C386,095        & 19 & \D548 \\
\wsja                   & 90            & 310,486  & 1,904,118 & 21 & 1,257 \\
}{Statistics from experiments with \repair.}
{Statistics from \repair experiments (\largecc, \wsja)}{tab:repair-stats}

\fig{
\begin{tabular}{l}
\greyboxtt{{\tl}}
\greyboxtt{DOC{\tg}{\tb}{\tl}DOCNO{\tg}{\tvs}WSJ}
\greyboxtt{870}
\greyboxtt{324}
\greyboxtt{-0001{\tvs}{\tl}/DOCNO{\tg}{\tb}{\tl}HL{\tg}{\tvs}} \\
\greyboxtt{John{\tvs}Blair{\tvs}}
\greyboxtt{Is{\tvs}}
\greyboxtt{Near{\tvs}}
\greyboxtt{Accord{\tb}To{\tvs}Sell{\tvs}}
\greyboxtt{Unit,{\tvs}}
\greyboxtt{Sources{\tvs}Say} \\
\greyboxtt{{\tl}/HL{\tg}{\tb}{\tl}DD{\tg}{\tvs}03/2} \\
\greyboxtt{4/87{\tl}/DD{\tg}{\tb}{\tl}SO{\tg}{\tvs}WALL{\tvs}STREET{\tvs}JOURNAL{\tvs}(J){\tl}/SO{\tg}{\tb}{\tl}IN{\tg}{\tvs}} \\
\greyboxtt{REL}
\greyboxtt{{\tb}TENDER{\tvs}OFFERS,{\tvs}MERGERS,{\tvs}ACQUISITIONS{\tvs}(TNM){\tb}}{\crarrow} \\
\greyboxtt{MARKETING,{\tvs}ADVERTISING{\tvs}(MKT} 
\greyboxtt{){\tb}TELECOMMUNICATIONS,{\tvs}}{\crarrow} \\
\greyboxtt{BROADCASTING,{\tvs}TELEPHONE,{\tvs}TELEGRAPH{\tvs}(TEL} \\
\greyboxtt{){\tvs}{\tl}/IN{\tg}{\tb}{\tl}DATELINE{\tg}{\tvs}NEW{\tvs}YORK{\tvs}{\tl}/DATELINE{\tg}{\tb}{\tl}TEXT{\tg}{\tb}{\tvs}{\tvs}{\tvs}} \\
\greyboxtt{John{\tvs}Blair{\tvs}\&amp;{\tvs}Co.{\tvs}}
\greyboxtt{is{\tvs}close{\tvs}to{\tvs}}
\greyboxtt{an{\tvs}agreement{\tvs}to{\tvs}} \\
  
\end{tabular}
}{Expanding the first 20 symbols of the \wsja sequence into 
primitives.  For the purposes of display, {\tb} 
indicate linefeeds and long phrases are broken at the locations
indicated by {\crarrow}.}{Expanding the beginning of the  
sequence (\wsja)}{fig:repair-wsjhead}

%%%%  Removed as no value can be gained from such a figure.
%% The longest phrase in \wsja, with 1,257 primitives 
%% and occurring twice in the whole message, is the following:
%% \fig{
%% \begin{tabular}{l}
%% \input{./repair-wsjlong.tex}
%% \end{tabular}
%% }{(long)}{(short)}{fig:repair-wsjlong}


The compression ratios achieved by \repair on these four
files are shown in \tabref{tab:repair-bpc}.  The last three
columns of this table represent compression ratios for \bzip,
\gzip, and \ppmd, taken from \tabref{tab:tc-bpc} on \pgref{tab:tc-bpc}.
Recall that \gzip and \bzip were executed with the {\texttt {-9}}
option, while a seventh order model for \ppmd was used to ensure 
the best possible compression.  
Execution of \ppmd was limited to \mib{255} of memory, so as
to effectively remove
the restriction on its memory usage and reduce the
chance of requiring the model to be re-constructed.  For the results
reported in \tabref{tab:repair-bpc}, the model was not re-built
by \ppmd.  While \repair has the potential to use more memory 
than any of these compression systems, they are reasonable 
benchmarks due to their widespread popularity.

\tab{lcccccc}
{\multirow{3}*{Filename} & \multicolumn{3}{c}{\repair} & \multirow{3}*{\gzip} & \multirow{3}*{\bzip} &  \\ \cline{2-4}
& Phrase & Total using & Total using & & & \ppmd \\
& hierarchy & \shuff & \uint & & & \\}
{
{\texttt {E.coli}}        & 0.108 & 2.086 & 2.081   & 2.240 & 2.158 & 2.034 \\
{\texttt {world192.txt}}  & 0.312 & 1.624 & 1.597   & 2.333 & 1.584 & 1.454 \\
{\texttt {bible.txt}}     & 0.290 & 1.763 & 1.749   & 2.326 & 1.671 & 1.564 \\
\wsja                    & 0.223 & 1.774 & 1.770   & 2.907 & 2.078 & 1.656 \\
}{Compression ratios for the Large Canterbury Corpus and \wsja 
using \repair.  Total compression ratios for \repair include
the phrase hierarchy and the coded sequence with either \shuff
or \uint.  Compression results from the last three columns have been
taken from \tabref{tab:tc-bpc}.}{Compression ratios
using \repair (\largecc, \wsja)}{tab:repair-bpc}

Compression effectiveness using \repair is better than \gzip in 
all cases and varies with \bzip, depending on the file.  This
is an expected result given the differences in the amount
of memory used.  \ppmd achieves better compression than \repair, 
but as mentioned in the last chapter, statistical models 
can be expected to outperform dictionary-based ones.  The 
arithmetic coder gives better compression than
the Huffman coder.  Note also that the 
phrase hierarchy is small compared to the original message
and the reduced sequence, occupying less than 20\% of the final
message for all of the files.

Compression and decompression times for \repair are shown in 
\tabref{tab:repair-time} in CPU seconds, averaged
over three trials.  Even though the coding of the phrase 
hierarchy is logically separate from the \repair process, in the 
implementation of \repair used,
the coding of the phrase hierarchy is tightly coupled with \repair,
and so, the execution times of \repair and \despair both include the
encoding and decoding times of the phrase hierarchy.

\tab{lccccccc}
{\multirow{2}*{Filename} & \multicolumn{3}{c}{Encoding} & & \multicolumn{3}{c}{Decoding} \\ \cline{2-4} \cline{6-8}
                  & \repair & \shuff & \uint & & \despair & \shuff & \uint \\}
{
{\texttt {E.coli}}            & 16.2   & 0.3  & 1.1  & & 0.4  & \hphantom{{\tl}}0.1  & 1.2  \\
{\texttt {world192.txt}}      & \D9.1  & 0.1  & 0.4  & & 0.3  & {\tl}0.1  & 0.4  \\
{\texttt {bible.txt}}         & 15.0   & 0.3  & 0.9  & & 0.4  & \hphantom{{\tl}}0.1  & 1.0  \\
\wsja                        & 88.5   & 1.6  & 6.7  & & 2.5  & \hphantom{{\tl}}0.4  & 7.5  \\
}
{Compression and decompression times for the Large Canterbury 
Corpus and \wsja using \repair and \despair in seconds.  
Times have been averaged over three trials.}{Compression 
times using \repair (\largecc, \wsja)}{tab:repair-time}

Comparing the compression times of \repair with the other 
programs (\tabref{tab:tc-time}), \repair takes more time
even if entropy coding is not taken into account.  However, 
decoding with the two dictionary-based systems, \despair 
and \gzip, are fast.  Of the two, \gzip is consistently 
faster.  Furthermore, the encoding and decoding times are 
higher for arithmetic coding compared to Huffman coding.  
Because of this speed difference and the very slight difference 
in compression, the entropy coder used by experiments in later 
chapters is the Huffman coder.  The downside of this decision is 
the slightly worse compression, as shown in \tabref{tab:repair-bpc}.

\newsection{Related Work}{sec:repair-related}

Other than operating off-line, several other differences exist
between \repair and the dictionary-based algorithms of 
\chapref{chap:tc}.  First, \repair explicitly separates the
dictionary from the sequence of references, producing two
streams that together, make up the compressed message.
Second, every primitive and phrase must be used at least
once in order to appear in the \repair phrase hierarchy.
In contrast, phrases may be added to the dictionaries
formed by \lza or \lzb, but never
referred to.

\citet{ss82:jacm} showed that, in a dictionary-based 
compression algorithm,
determining the phrases that would achieve the best compression for
a given message is NP-hard.  Therefore, like \lza and \lzb, many 
dictionary-based compression algorithms use heuristics for phrase
construction.  \citet{bm01:is} considered a 
pre-processor to compression which identifies long repeated 
strings appearing far apart.

\citet{wolff75:bjp} described a program called \mka 
for identifying
breaks in text in the absence of any whitespaces.  The
system operates similar to \repair by identifying
and recursively replacing pairs of symbols.  However, unlike
\repair, the message is scanned from left to right and
replacements are made after a pre-determined threshold has
been met (for example, 10 occurrences).  Later, \citet{wolff77:bjp}
extended this work by selecting the most frequent pair of
symbols after one complete scan of the message.  However,
since the primary goal of \citeauthor{wolff75:bjp} was the
segmentation of text, little attention was given to the compression
effectiveness or efficiency of \mka.

The incremental encoding of \citet{rubin76:cacm}, like \repair,
uses frequency of digrams (pairs of symbols) and operates
through recursive pairing for compression.  However, more 
emphasis is placed on conditions for program termination 
rather than a compact method of storing the dictionary.  Some
of these conditions include phrase frequency and phrase length.
There were also few details given on an efficient 
implementation with respect to time or memory space.
However, as an indication of the amount of time and space 
that such an algorithm would require, 
\citet{rubin76:cacm} noted that the methods
used have ``fairly high computation time and large storage 
requirements''.

\citet{manber97:tis} used digrams for searching in compressed
text.  Even though \eascii is a character set of size 256, half
of it is unused by most English documents.  The remaining 128
values were assigned to frequently occurring pairs of primitives
and used to compress a message.  The search pattern was similarly
compressed so that a general-purpose string matching algorithm
could be used to locate occurrences of the pattern within the
compressed message.  The problem of overlapping pairs is avoided
by ensuring that a symbol which is a left component of one pair of
symbols cannot be a right component of another pair.  Earlier,
\citet{gage94:cusers} devised a technique called Byte-Pair
Encoding (\bpe) which resembles the algorithm
by \citeauthor{manber97:tis}, except that searching was not 
considered.

More recently, the \offline mechanism of \citet{al00:procieee} 
showed how an annotated trie for a message could be used to
locate candidates for a dictionary.  The phrases in the
dictionary were created directly from substrings of the message
without referring to other phrases in the dictionary.
Central to \offline is the computation of a
{\emph {gain measure}}.  The gain measure determines
the next best phrase to add to the dictionary using criteria
such as substring length, substring frequency, and pointer costs 
to all occurrences.  Three methods of dictionary encoding
were employed, one of which stored the phrases as literal text
in an external dictionary.  \citet{ts02:acsc} extended this work
by looking at other methods of computing the gain measure.

The \ray system \citep{cw01:jasist} also compresses a message
off-line, but through multiple passes so that the message itself
does not need to be stored in memory.  It builds a dictionary by 
recursively replacing diagrams, similar to \repair.  Each pass 
over the message is roughly equivalent to a generation in 
\repair.  The \ray algorithm consists of three steps:  
accumulation of frequencies, the identification of candidates, 
and replacement of symbol pairs.  The first step is only 
performed once since the sequence's statistics are updated as 
replacements are made.  The remaining two steps are repeated 
until either no pair of symbols appears twice or more, or a preset 
limit on generations has been reached.  Pairs of symbols of 
equal frequency were handled differently than \repair.  If the 
three symbols ${\alpha}$${\beta}$${\delta}$ appear in the sequence 
and step 2 is being performed, ${\alpha}{\beta}$ is identified 
as a candidate for replacement only if ${\alpha}$ occurs with 
an equal or higher frequency than ${\beta}$.  If this is not 
the case, then this pair of symbols is skipped, and the 
next pair, ${\beta}{\delta}$, is considered.  The \ray dictionary
is encoded using Elias and Huffman codes.


Finally, operating on-line instead of off-line is the \sequitur 
algorithm of \citet{nw97:cj}.  The \sequitur algorithm 
builds a dictionary resembling a grammar, but unlike \repair,
a rule can generate more than two symbols.  During 
compression, as each message symbol is read, the grammar is 
updated according to two rules:  {\emph {digram uniqueness}} and 
{\emph {rule utility}}.  Digram uniqueness ensures that no pair
of adjacent symbols appear more than once in the grammar.
Digram uniqueness effectively adds phrases to the dictionary 
as symbols appear in the sequence (rule $\Seq$).  The creation
of rules can cascade from rule $\Seq$, and results in a 
hierarchical grammar.  Rule utility prohibits any rule
from being used only once.  When a rule created earlier ends
up appearing once in the right-hand sides of all of the rules,
it is expanded at the location it is
used and then removed from the grammar.
Adherence to rule utility results in the right-hand sides of
rules becoming longer than just digrams.  Rule S is transmitted
to the receiver while the rest of the grammar is sent implicitly
using back-pointers, similar to a \lza implementation.  Also, 
\citet{nw00:procieee} compared on-line and off-line techniques for 
\sequitur by using phrase heuristics such as most frequent, longest, 
and most compressive.

An example of \sequitur is shown in \tabref{tab:repair-sequitur} 
using the same ``Woodchuck'' message as the one from the
example of \tabref{tab:repair-short}.
Each row of the table indicates the current rule $\Seq$, the \sequitur
rule that was violated, and the action required to fix the grammar.
The first row shows the first time a rule violation has occurred.
If a rule already 
exists when digram uniqueness is enforced, 
no new rule is added to the dictionary, as shown with the third
occurrence of ``ch''.  After phrase 
\tc{3} is added to the dictionary, phrase \tc{2} is used only once.
Phrase \tc{2} is expanded in phrase \tc{3} and removed from the
dictionary because of rule utility.  At the end of this example,
the rules sent to the receiver are $\Seq$, \tc{1}, \tc{3}, 
\tc{4}, \tc{5}, and \tc{8}.

\newcommand{\digram}{digram uniqueness}
\newcommand{\utility}[1]{rule \tc{#1}}
\newcommand{\addseq}[2]{Add \tc{#1} $\rightarrow$ #2}
\newcommand{\removeseq}[1]{Remove rule \tc{#1}}
\newcommand{\useseq}[1]{Use rule \tc{#1}}

\tab{lll}
{
\multicolumn{1}{c}{\multirow{2}*{Rule $\Seq$}} & \multicolumn{1}{c}{Rule} & \multicolumn{1}{c}{Action applied} \\
& \multicolumn{1}{c}{enforced} & \multicolumn{1}{c}{to grammar} \\
}
{
how{\tvs}much{\tvs}wood{\tvs}could{\tvs}  & \digram & \addseq{1}{d{\tvs}} \\
how{\tvs}much{\tvs}woo\tc{1}coul\tc{1}a{\tvs}w & \digram & \addseq{2}{{\tvs}w} \\
how{\tvs}much\tc{2}oo\tc{1}coul\tc{1}a\tc{2}o & \digram & \addseq{3}{\tc{2}o} \\
how{\tvs}much\tc{3}o\tc{1}coul\tc{1}a\tc{3} & rule utility of & Expand \tc{2}: \tc{3} $\rightarrow$ {\tvs}wo \\
& \utility{2} & \removeseq{2} \\
how{\tvs}much\tc{3}o\tc{1}coul\tc{1}a\tc{3}o & \digram & \addseq{4}{\tc{3}o} \\
how{\tvs}much\tc{4}\tc{1}coul\tc{1}a\tc{4}dch & \digram & \addseq{5}{ch} \\
%how{\tvs}mu\tc{5}\tc{4}\tc{1}coul\tc{1}a\tc{4}d\tc{5} & \digram & \addseq{5}{ch} \\
how{\tvs}mu\tc{5}\tc{4}\tc{1}coul\tc{1}a\tc{4}d\tc{5}uck{\tvs}ch & \digram & \useseq{5} \\
how{\tvs}mu\tc{5}\tc{4}\tc{1}coul\tc{1}a\tc{4}d\tc{5}uck{\tvs}\tc{5}u & \digram & \addseq{6}{\tc{5}u} \\
how{\tvs}mu\tc{5}\tc{4}\tc{1}coul\tc{1}a\tc{4}d\tc{6}ck{\tvs}\tc{6}c & \digram & \addseq{7}{\tc{6}c} \\
how{\tvs}mu\tc{5}\tc{4}\tc{1}coul\tc{1}a\tc{4}d\tc{7}k{\tvs}\tc{7} & rule utility of & Expand \tc{6}: \tc{7} $\rightarrow$ \tc{5}uc \\
& \utility{6} & \removeseq{6} \\
how{\tvs}mu\tc{5}\tc{4}\tc{1}coul\tc{1}a\tc{4}d\tc{7}k{\tvs}\tc{7}k & \digram & \addseq{8}{\tc{7}k} \\
how{\tvs}mu\tc{5}\tc{4}\tc{1}coul\tc{1}a\tc{4}d\tc{8}{\tvs}\tc{8} & rule utility of & Expand \tc{7}:  \tc{8} $\rightarrow$ \tc{5}uck \\
& \utility{7} & \removeseq{7} \\
}{A sample application of the \sequitur algorithm on the message
used in \tabref{tab:repair-short}.  Message symbols are read 
one at a time and appended to rule $\Seq$, shown in the first
column.  Each row represents the moment when one of the
two rules of \sequitur (digram uniqueness or rule utility) need
to be enforced.  The \sequitur rule that is violated and the 
action performed to enforce it are shown in columns 2 and 3, 
respectively.}{Sample application of \sequitur}{tab:repair-sequitur}

While the \sequitur algorithm differs from \repair in the above
points, a system for phrase browsing has also been developed called
\phind \citep{nwp97:acmdl}.  The methods used by \phind, discussed 
later in \chapref{chap:rephine},
are similar to the ones for \repair, but differences emerge due
to their underlying compression mechanisms.  

\newsection{Retrieval with \repair}{sec:repair-retrieve}

In \chapref{chap:tc}, a wide range of algorithms for modelling
were presented.  This chapter adds another dictionary-based
model to the list, which is more suited for both retrieval
and browsing.  

The most important characteristic of the \repair algorithm is that
recursive pairing based on frequency is more appropriately implemented
in a semi-static manner, so that the entire message is available.  As 
a consequence of this, the compressed message is composed of two
separate streams, which each has its own advantages.
The phrase hierarchy is small and occupies only 3\% of the space
taken by the original message.  Moreover, the phrase hierarchy
contains a significant amount of information about the document.
The hierarchical structure of the phrase hierarchy means that it
describes the many relationships symbols form with lower and 
higher generation symbols.  The phrase hierarchy is not only crucial
for decompression, but appears suitable for phrase browsing, as well.
The second stream is the reduced message of references to the
dictionary.  

Searching the reduced sequence has been examined by 
\citet{manber97:tis} and \citet{skftssa99:tech}, who based their
work on algorithms similar to \repair.  
While \citeauthor{manber97:tis} modified the pattern prior to
searching, \citeauthor{skftssa99:tech}\ presented two approaches.
First, they expanded all corresponding 
encodings of the pattern before applying a pattern matching 
algorithm.  Second, they used the Knuth-Morris-Pratt 
automaton \citep{kmp77:siam} for the pattern to aid the search.
In addition to searching, the reduced sequence can be decoded
from any reference, provided the phrase hierarchy has been processed
already.

At the conclusion of the last chapter, three requirements
were mentioned which a compression algorithm should satisfy in
order to be adapted for retrieval.  First, the algorithm should
allow decoding from an arbitrary point in the compressed message,
which is satisfied by the structure of the two streams from \repair.
In contrast, since algorithms like \ppm and block sorting rely on 
symbol contexts, a random point in the compressed representation
cannot be decoded without examining everything before.  Alternatively,
a compressed message made with block sorting can be 
pre-processed in-memory so that an indexing structure is available
for searching \citep{fm00:focs}.
The \lza and \lzb algorithms are both dictionary-based like 
\repair, but they were designed to be adaptive and so,
combine the dictionary and the references into
a single stream.

The second requirement is that decoding should be fast so that
the amount of time that a user of a retrieval system waits 
is minimised.  While \repair is slower than \gzip for 
decoding, a comparison between the results
of \tabref{tab:repair-time} and \tabref{tab:tc-time} show that
\repair is noticeably faster than \bzip and \ppmd.

Finally, the alignment of codewords to bits rather than bytes
would facilitate more efficient decoding.  Experiments in this
chapter compressed the reduced sequence with \shuff and \uint.
Even though Huffman coding was advocated after
examining the decoding times of \shuff and \uint, an alternative coding
mechanism which is able to decode the sequence faster may be
possible which sacrifices some compression effectiveness.  
As for the phrase hierarchy, the good compression effectiveness 
attained with interpolative coding make the 
lack of any byte-alignment acceptable.

There are, however, four changes to \repair before it can be
included into a retrieval system.
First, the use of 
frequency alone to select phrases for the phrase hierarchy
presents problems for retrieval.  
In a retrieval system, it is expected that the query terms match
the primitives or phrases in the phrase hierarchy.
The solution to this problem requires the 
attention to shift to how \repair can be forced to select
words and phrases which resemble the ones displayed to the
user in the interface.

The second change deals with the problem of memory
space required by \repair.
While \repair operates in space linear in the 
length of the message, the constant factor is quite high.  
Assuming four bytes per word, compressing
\wsja requires \mib{240} for the sequence nodes alone.
In order for a retrieval system to be useful, files
one or even two orders of magnitude larger 
may need to be processed using reasonable 
amounts of memory.

The third improvement is with the entropy coding applied to the 
sequence. While phrase browsing is performed solely with the 
phrase hierarchy, searching for those phrases involves the 
reduced sequence.  By using an entropy coder, if appearances 
of a phrase are being sought, the reduced sequence 
needs to be sequentially decoded.  Instead, a
coding mechanism is required which allows efficient searching
through the reduced sequence.

Finally, an interface is required which allows users
to browse phrases and then locate and examine the contexts in 
which they occur.

Solutions to each of these
four problems are investigated separately in the next
four chapters, in the order they have been listed.  
\chapref{chap:prepair} describes alternatives to \repair's
method of selecting phrases by first aligning phrases to
whitespaces, and then by using a word-based pre-processing scheme
using punctuation as a guide.  \chapref{chap:remerge} shows how
a message can be partitioned into blocks by \repair so that 
larger messages can be compressed.  Compressed blocks are merged
in order to produce a single phrase hierarchy for browsing.
\chapref{chap:review} examines alternatives to
entropy coding the entire reduced message to allow fast 
searching for phrases.  And finally, \chapref{chap:rephine}
describes phrase browsing and an interface which ties the
above components together.

